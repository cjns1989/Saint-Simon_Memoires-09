\PassOptionsToPackage{unicode=true}{hyperref} % options for packages loaded elsewhere
\PassOptionsToPackage{hyphens}{url}
%
\documentclass[oneside,12pt,french,]{extbook} % cjns1989 - 27112019 - added the oneside option: so that the text jumps left & right when reading on a tablet/ereader
\usepackage{lmodern}
\usepackage{amssymb,amsmath}
\usepackage{ifxetex,ifluatex}
\usepackage{fixltx2e} % provides \textsubscript
\ifnum 0\ifxetex 1\fi\ifluatex 1\fi=0 % if pdftex
  \usepackage[T1]{fontenc}
  \usepackage[utf8]{inputenc}
  \usepackage{textcomp} % provides euro and other symbols
\else % if luatex or xelatex
  \usepackage{unicode-math}
  \defaultfontfeatures{Ligatures=TeX,Scale=MatchLowercase}
%   \setmainfont[]{EBGaramond-Regular}
    \setmainfont[Numbers={OldStyle,Proportional}]{EBGaramond-Regular}      % cjns1989 - 20191129 - old style numbers 
\fi
% use upquote if available, for straight quotes in verbatim environments
\IfFileExists{upquote.sty}{\usepackage{upquote}}{}
% use microtype if available
\IfFileExists{microtype.sty}{%
\usepackage[]{microtype}
\UseMicrotypeSet[protrusion]{basicmath} % disable protrusion for tt fonts
}{}
\usepackage{hyperref}
\hypersetup{
            pdftitle={SAINT-SIMON},
            pdfauthor={Mémoires\_IX},
            pdfborder={0 0 0},
            breaklinks=true}
\urlstyle{same}  % don't use monospace font for urls
\usepackage[papersize={4.80 in, 6.40  in},left=.5 in,right=.5 in]{geometry}
\setlength{\emergencystretch}{3em}  % prevent overfull lines
\providecommand{\tightlist}{%
  \setlength{\itemsep}{0pt}\setlength{\parskip}{0pt}}
\setcounter{secnumdepth}{0}

% set default figure placement to htbp
\makeatletter
\def\fps@figure{htbp}
\makeatother

\usepackage{ragged2e}
\usepackage{epigraph}
\renewcommand{\textflush}{flushepinormal}

\usepackage{indentfirst}
\usepackage{relsize}

\usepackage{fancyhdr}
\pagestyle{fancy}
\fancyhf{}
\fancyhead[R]{\thepage}
\renewcommand{\headrulewidth}{0pt}
\usepackage{quoting}
\usepackage{ragged2e}

\newlength\mylen
\settowidth\mylen{...................}

\usepackage{stackengine}
\usepackage{graphicx}
\def\asterism{\par\vspace{1em}{\centering\scalebox{.9}{%
  \stackon[-0.6pt]{\bfseries*~*}{\bfseries*}}\par}\vspace{.8em}\par}

\usepackage{titlesec}
\titleformat{\chapter}[display]
  {\normalfont\bfseries\filcenter}{}{0pt}{\Large}
\titleformat{\section}[display]
  {\normalfont\bfseries\filcenter}{}{0pt}{\Large}
\titleformat{\subsection}[display]
  {\normalfont\bfseries\filcenter}{}{0pt}{\Large}

\setcounter{secnumdepth}{1}
\ifnum 0\ifxetex 1\fi\ifluatex 1\fi=0 % if pdftex
  \usepackage[shorthands=off,main=french]{babel}
\else
  % load polyglossia as late as possible as it *could* call bidi if RTL lang (e.g. Hebrew or Arabic)
%   \usepackage{polyglossia}
%   \setmainlanguage[]{french}
%   \usepackage[french]{babel} % cjns1989 - 1.43 version of polyglossia on this system does not allow disabling the autospacing feature
\fi

\title{SAINT-SIMON}
\author{Mémoires\_IX}
\date{}

\begin{document}
\maketitle

\hypertarget{chapitre-premier.}{%
\chapter{CHAPITRE PREMIER.}\label{chapitre-premier.}}

1710

~

{\textsc{Prince de Lorraine coadjuteur de Trêves.}} {\textsc{- Mort et
caractère du cardinal Grimani.}} {\textsc{- Mort et famille de la
duchesse de Modène\,; son deuil.}} {\textsc{- Mort et fortune du prince
de Salm.}} {\textsc{- Mort du comte de Noailles.}} {\textsc{- Mort et
caractère de M\textsuperscript{me} de Ravetot\,; sa famille et celle de
son mari.}} {\textsc{- Mort, famille et singularité de l'abbé de
Pompadour.}} {\textsc{- Dixième denier.}} {\textsc{- P. Tellier persuade
au roi que tous les biens de ses sujets sont à lui.}} {\textsc{-
Explication du conseil des finances.}} {\textsc{- Monseigneur et Mgr le
duc de Bourgogne fâchés du dixième.}} {\textsc{- Sortie de Mgr le duc de
Bourgogne contre les financiers.}} {\textsc{- Du Mont m'avertit de la
plus folle calomnie persuadée contre moi à Monseigneur.}} {\textsc{-
Crédulité inconcevable de ce prince.}} {\textsc{- M\textsuperscript{me}
de Saint-Simon s'adresse à M\textsuperscript{me} la duchesse de
Bourgogne, qui détrompe pleinement Monseigneur et me tire d'affaire.}}

~

M. de Lorraine, par la protection de l'empereur, avait forcé le chapitre
de Trêves de souffrir que son frère y entrât\,; je dis forcé, parce que
ce chapitre et celui de Mayence faits sages, et en cela appuyés de toute
la noblesse de l'empire, par l'exemple de celui de Cologne qui n'a plus
d'archevêque, il y a longtemps, que de la maison de Bavière, depuis que
ces princes se sont introduits dans le chapitre, ne veulent plus
souffrir de princes dans les leurs\,; ce que celui de Trêves craignait
du frère du duc de Lorraine et qui lui arriva. Les prières et les
menaces furent employées par la cour de Vienne\,; M. de Lorraine traita
et répandit l'argent à pleines mains. L'archevêque, qui était un baron
d'Orgbreicht, et qui avait soixante-quinze ans, fut gagné\,; la brigue
emporta les chanoines, et le frère du duc de Lorraine fut élu coadjuteur
sur la fin de septembre.

L'empereur fit incontinent après une perte d'un de ses plus effrénés
partisans, en la personne du cardinal Grimani, qui n'eut de dieu que son
service, à qui les crimes ne coûtaient rien, et qui en fut
singulièrement récompensé de la vice-royauté de Naples, où il mourut à
la grande satisfaction de ce royaume, qu'il tyrannisait fort, et du pape
et de tout Rome, qu'il maîtrisait sans ménagement d'une étrange sorte.
Ce prince perdit aussi sa belle-sœur, la duchesse de Modène\,; elle
n'avait que trente-neuf ans, et avait deux ans plus que l'impératrice\,;
toutes deux filles de la duchesse d'Hanovre, desquelles j'ai parlé à
l'occasion de ce qui les fit sortir de France, et de la feue princesse
de Salm, dont le mari mourut aussi fort peu après. Il avait eu les
premiers emplois à la cour de Vienne\,; il avait été gouverneur de la
personne de l'empereur, et avait fait son mariage avec sa nièce\,; des
mécontentements l'avaient fait renoncer à toutes ses charges et à la
cour depuis quelques années\,; il s'était retiré chez lui, et il mourut
à Aix-la-Chapelle. M\textsuperscript{me} la Princesse était sœur de sa
femme et de la duchesse d'Hanovre. Le roi prit le deuil quatre ou cinq
jours de M\textsuperscript{me} de Modène. M. de Modène avait l'honneur
d'être son parent.

Le jeune comte de Noailles mourut de la petite vérole à Perpignan. De
beaucoup de frères qu'avait eus le duc de Noailles, c'était le seul qui
restait. Il lui avait donné son régiment de cavalerie, et il était aussi
lieutenant général au gouvernement d'Auvergne. Cela ne vaut que huit
mille livres de rentes. Le roi donna l'un et l'autre au duc de Noailles.

M\textsuperscript{me} de Ravetot\footnote{Saint-Simon écrit Ravetot\,;
  mais la véritable forme est Raffetot, nom d'un village de la
  Seine-Inférieure.} mourut aussi. Ce fut une perte pour ses amis, dont
elle avait beaucoup, des deux sexes, et la plupart de haut parage\,:
c'en fut aussi une pour le monde, dont elle était fort et avec
considération. On l'appelait belle et bonne, et elle était l'une et
l'autre, avec de l'esprit, des grâces et rien de recherché ni d'affecté.
Elle avait été fort de la cour de Monsieur. Elle était fille de Pertuis,
autrefois capitaine des gardes de M. de Turenne, qui s'était fait
estimer et considérer, et était mort gouverneur de Menin. Le nom de son
mari était Canouville, gentilshommes riches, anciens et bien alliés de
haute Normandie. Le maréchal de Grammont avait une fille aînée
borgnesse, boiteuse et fort laide, qui ne voulut point être religieuse.
Ne sachant qu'en faire, il la maria à Ravetot presque pour rien, après
la mort duquel elle se ravisa et se fit carmélite. C'est la belle-mère
de celle dont je parle. Le mari était un fort brave homme, qui buvait
bien, fort bête et fort débauché, qui s'est ruiné et est mort lieutenant
général, et qui n'a laissé qu'une fille, son seul fils étant mort
longtemps devant lui, sans avoir été marié, après avoir perdu sa fortune
par une prison de douze ou quinze ans, pour s'être battu avec
Armentières, mort depuis premier gentilhomme de la chambre de M. le duc
d'Orléans.

L'abbé de Pompadour mourut en même temps et emporta moins de regrets.
C'était un petit homme qui, à quatre-vingt-cinq ou six ans, courait
encore la ville, et qui n'avait jamais fait la moindre figure. Son père
et son frère étaient chevaliers de l'ordre en 1633 et en 1661. Son père
s'était bien différemment marié, d'abord à une Montgommery, après à une
Rohan-Guéméné, sans enfants d'aucune\,; enfin à une Fabri, dont il en
eut. Son fils aîné fut père de M\textsuperscript{me}s de Saint-Luc et
d'Hautefort, et cet abbé, leur oncle paternel, a fini cette branche, qui
était l'aînée. Il avait un laquais presque aussi vieux que lui, à qui il
donnait, outre ses gages, tant par jour pour dire son bréviaire en sa
place, et qui le barbotait dans un coin des antichambres où son maître
allait. Il s'en croyait quitte de la sorte, apparemment sur l'exemple
des chanoines qui payent des chantres pour aller chanter au chœur pour
eux. Il avait un autre frère de qui le fils n'a laissé que
M\textsuperscript{me} de Courcillon, dont la fille unique, veuve d'un
fils du maréchal-duc de Chaulnes, s'est remariée au prince de Rohan, et
n'a point d'enfants de l'un ni de l'autre. L'impossibilité, trop
bassement éprouvée, d'obtenir la paix, et l'épuisement où était le
royaume, jetèrent le roi dans les plus cruelles angoisses, et Desmarets
dans le plus funeste embarras. Les papiers de toutes les espèces dont le
commerce se trouvait inondé, et qui tous avaient plus ou moins perdu
crédit, faisaient un chaos dont on n'apercevait point le remède\,:
billets d'État, billets de monnaie, billets des receveurs généraux,
billets sur les tailles, billets d'ustensile, étaient la ruine des
particuliers que le roi forçait de prendre en payement de lui, qui
perdaient moitié, deux tiers et plus, et avec le roi comme avec les
autres. Ces escomptes enrichissaient les gens d'argent et de finance aux
dépens du public, et la circulation de l'argent ne se faisait plus,
parce que l'espèce manquait, parce que le roi ne payait plus personne et
qu'il tirait toujours, et que ce qu'il y avait d'espèces hors de ses
mains était bien enfermé dans les coffres des partisans. La capitation
doublée et triplée à volonté arbitraire des intendants des provinces,
les marchandises et les denrées de toute espèce imposées en droit au
quadruple de leur valeur, taxes d'aisés et autres de toute nature et sur
toutes sortes de choses, tout cela écrasait nobles et roturiers,
seigneurs et gens d'Église, sans que ce qu'il en revenait au roi pût
suffire, qui tirait le sang de tous ses sujets sans distinction, qui en
exprimait jusqu'au pus, et qui enrichissait une armée infinie de
traitants et d'employés à ces divers genres d'impôts, entre les mains de
qui en demeurait la plus grande et la plus claire partie.

Desmarets, en qui enfin le roi avait été forcé de mettre toute sa
confiance pour les finances, imagina d'établir, en sus de tant d'impôts,
cette dîme royale sur tous les biens de chaque communauté et de chaque
particulier du royaume, que le maréchal de Vauban d'une façon, et que
Boisguilbert de l'autre, avaient autrefois proposée, ainsi que je l'ai
rapporté alors, comme une taxe unique, simple, qui suffirait à tout, qui
entrerait tout entière dans les coffres du roi, au moyen de laquelle
tout autre impôt serait aboli, même la taille et jusque son nom. On a vu
au même lieu et avec quel succès, que les financiers en frémirent, que
les ministres en rugirent, avec quel anathème cela fut rejeté, et à quel
point ces deux excellents et habiles citoyens en demeurèrent perdus.
C'est dont il faut se souvenir ici, puisque Desmarets, qui n'avait pas
perdu de vue ce système, non comme soulagement et remède, crime
irrémissible dans la doctrine financière, mais comme surcroît, y eut
maintenant recours.

Sans dire mot à personne, il fit son projet qu'il donna à examiner et à
limer à un bureau qu'il composa exprès et uniquement de Bouville,
conseiller d'État, mari de sa sœur\,; Nointel, conseiller d'État, frère
de sa femme\,; Vaubourg, conseiller d'État, son frère\,; Bercy,
intendant des finances, son gendre\,; Harlay-Cœli, maître des requêtes,
son affidé, mort depuis conseiller d'État et intendant de Paris, et de
trois maîtres financiers. Ce fut donc à ces gens si bien triés à digérer
l'affaire, à en diriger l'exécution, et à en dresser l'édit. Nointel,
seul d'entre eux, eut horreur d'une exaction si monstrueuse, et, sous
prétexte du travail du bureau qu'il avait des vivres des armées, il
s'excusa d'entrer en celui-ci, et fut imité par un des trois traitants,
à qui apparemment il restait encore quelque sorte d'âme. On fut étonné
que Vaubourg ne s'en fût point retiré, lui qui avait beaucoup de probité
et de piété, et qui s'était retiré des intendances par scrupule, où il
avait longtemps et bien servi.

Ces commissaires travaillèrent donc avec assiduité et grand'peine à
surmonter les difficultés qui se présentaient de toutes parts. Il
fallait d'abord tirer de chacun une confession de bonne foi, nette et
précise, de son bien, de ses dettes actives et passives, de la nature de
tout cela. Il en fallait exiger des preuves certaines, et trouver les
moyens de n'y être pas trompé. Sur ces points roulèrent toutes les
difficultés. On compta pour rien la désolation de l'impôt même dans une
multitude d'hommes de tous les états si prodigieuse, et leur désespoir
d'être forcés à révéler eux-mêmes le secret de leurs familles, la
turpitude d'un si grand nombre, le manque de bien suppléé par la
réputation et le crédit, dont la cessation allait jeter dans une ruine
inévitable, la discussion des facultés de chacun, la combustion des
familles par ces cruelles manifestations et par cette lampe portée sur
leurs parties les plus honteuses\,; en un mot, plus que le cousin
germain de ces dénombrements impies qui ont toujours indigné le créateur
et appesanti sa main sur ceux qui les ont fait faire, et presque
toujours attiré d'éclatants châtiments.

Moins d'un mois suffit à la pénétration de ces humains commissaires pour
rendre bon compte de ce doux projet au cyclope qui les en avait chargés.
Il revit avec eux l'édit qu'ils en avaient dressé tout hérissé de
foudres contre les délinquants qui seraient convaincus, mais qui n'avait
aucun égard aux charges que les biens portent par leur nature, et dès
lors il ne fut plus question que de le faire passer.

Alors Desmarets proposa au roi cette affaire dont il sut bien faire sa
cour\,; mais le roi, quelque accoutumé qu'il fût aux impôts les plus
énormes, ne laissa pas de s'épouvanter de celui-ci. Depuis longtemps il
n'entendait parler que des plus extrêmes misères\,; ce surcroît
l'inquiéta jusqu'à l'attrister d'une manière si sensible, que ses valets
intérieurs s'en aperçurent dans les cabinets plusieurs jours de suite,
et assez pour en être si en peine, que Maréchal, qui m'a conté toute
cette curieuse anecdote, se hasarda de lui parler de cette tristesse
qu'il remarquait, et qui était telle depuis plusieurs jours, qu'il
craignait pour sa santé. Le roi lui avoua qu'il sentait des peines
infinies, et se jeta vaguement sur la situation des affaires. Huit ou
dix jours après, et toujours la même mélancolie, le roi reprit son calme
accoutumé. Il appela Maréchal, et seul avec lui, il lui dit que,
maintenant qu'il se sentait au large, il voulait bien lui dire ce qui
l'avait si vivement peiné, et ce qui avait mis fin à ses peines.

Alors il lui conta que l'extrême besoin de ses affaires l'avait forcé à
de furieux impôts\,; que l'état où elles se trouvaient réduites le
mettait dans la nécessité de les augmenter très-considérablement\,; que,
outre la compassion, les scrupules de prendre ainsi les biens de tout le
monde l'avaient fort tourmenté\,; qu'à la fin il s'en était ouvert au P.
Tellier, qui lui avait demandé quelques jours à y penser, et qu'il était
revenu avec une consultation des plus habiles docteurs de Sorbonne qui
décidait nettement que tous les biens de ses sujets étaient à lui en
propre, et que, quand il les prenait, il ne prenait que ce qui lui
appartenait\footnote{Louis XIV disait lui-même à son fils (\emph{OEuvres
  de Louis XIV}, t. I\^{}er, p.~67)\,: «\,Vous devez être persuadé que
  les rois ont naturellement la disposition pleine et libre de tous les
  biens qui sont possédés aussi bien par les gens d'Église que par les
  séculiers, pour en user en tous temps comme de sages économes,
  c'est-à-dire suivant le besoin général de leur État.\,»}\,; qu'il
avouait que cette décision l'avait mis fort au large, ôté tous ses
scrupules, et lui avait rendu le calme et la tranquillité qu'il avait
perdue. Maréchal fut si étonné, si éperdu d'entendre ce récit, qu'il ne
put proférer un seul mot. Heureusement pour lui le roi le quitta dès
qu'il le lui eut fait, et Maréchal resta quelque temps seul en même
place, ne sachant presque où il en était. Cette anecdote, qu'il me conta
peu de jours après, et dont il était presque encore dans le premier
effroi, n'a pas besoin de commentaire\,; elle montre, sans qu'on ait
besoin de le dire, ce qu'est un roi livré à un pareil confesseur, et qui
ne parle qu'à lui, et ce que devient un État livré en de telles mains.

Maintenant il faut dire ce que c'était que le conseil des finances, et
ce qui s'y faisait, et qui est de même encore aujourd'hui. Le roi le
tenait tous les mardis matin et les samedis matin encore\,; mais celui
des samedis était supprimé toujours à Marly. Outre Monseigneur, et Mgr
le duc de Bourgogne qui entraient en tous, il était composé du
chancelier, parce qu'il avait été contrôleur général\,; du duc de
Beauvilliers, comme chef du conseil des finances, de Desmarets, comme
contrôleur général, et de deux conseillers d'État, comme conseillers du
conseil royal des finances, qui étaient lors Pelletier de Sousy, et
d'Aguesseau, père du chancelier d'aujourd'hui. Il faut se souvenir ici
de ce qui a été rapporté ailleurs de la création de l'inutile charge de
chef de ce conseil, lorsque Colbert, pour perdre Fouquet et se rendre
maître des finances, persuada au roi d'en supprimer le surintendant et
d'en faire la fonction lui-même. Ainsi ce conseil se passait presque
entier en signatures et en bons, que le roi mettait et faisait au lieu
du surintendant, en jugements d'affaires entre particuliers, que leur
nature ou la volonté du ministre y portait, et en appel du jugement du
conseil des prises des vaisseaux ennemis, mais marchands, que tenait
chez lui M. le comte de Toulouse, dont l'appel venait au conseil des
finances, que Pontchartrain y rapportait, et où pour ces affaires
seulement le comte de Toulouse entrait avec voie délibérative. Toutes
les autres y étaient rapportées par le contrôleur général, où le comte
de Toulouse et Pontchartrain n'entraient pas. Rien autre n'y était agité
ni délibéré. Tout ce qui s'appelle affaires des finances, taxes, impôts,
droits, impositions de toute espèce, nouveaux, augmentation des anciens,
régies de toutes les sortes, tout cela est fait par le contrôleur
général seul chez lui, avec un intendant des finances dont la fonction
est d'être son commis, quelquefois avec le traitant seul. Si la chose
est considérable à un certain point, elle est rapportée au roi par le
contrôleur général seul, dans son travail avec lui tête à tête,
tellement qu'il sort des arrêts du conseil en finance qui n'ont jamais
vu que le cabinet du contrôleur général, et des édits bursaux les plus
ruineux qui de même n'ont pas été portés ailleurs\,; que le secrétaire
d'État ne peut refuser de signer, ni le chancelier de viser et sceller
sans voir, sur la simple signature du contrôleur général\,; et ceux qui
entrent au conseil des finances n'en apprennent rien que par
l'impression de ces pièces devenues publiques, comme tous les
particuliers les plus éloignés des affaires. Cela se passait ainsi
alors, et s'est toujours continué de même depuis jusqu'à aujourd'hui.

L'établissement de la capitation fut proposé, et passa sans examen au
conseil des finances, comme je l'ai raconté en son lieu, singularité
donnée à l'énormité de cette espèce de dénombrement. La même énormitë
redoublée engagea Desmarets à la même cérémonie, ou plutôt au même jeu.
Le roi, mis au large par le P. Tellier et sa consultation de Sorbonne,
ne douta plus que tous les biens de ses sujets ne fussent siens, et que
ce qu'il n'en prenait pas et qu'il leur laissait était pure grâce. Ainsi
il ne fit plus de difficulté de les prendre à toutes mains et en toutes
les sortes\,; il goûta donc le dixième en sus de tous les autres droits,
impôts et affaires extraordinaires, et Desmarets n'eut plus qu'à
exécuter\,; Ainsi le mardi matin, 30 septembre, Desmarets entra au
conseil des finances avec l'édit du dixième dans son sac.

Il y avait déjà quelques jours que chacun savait la bombe en l'air, et
en frémissait avec ce reste d'espérance qui n'est fondé que sur le
désir, et toute la cour ainsi que Paris attendait dans une morne
tristesse ce qui en allait arriver. On s'en parlait à l'oreille, et bien
que ce projet prêt d'éclore fût déjà exprès rendu public, personne n'en
osait parler tout haut. Ceux du conseil des finances y entrèrent ce
jour-là sans en savoir davantage que le public, ni même si l'affaire
baiserait ou non le bureau de ce conseil.

Tout le monde assis, et Desmarets tirant un gros cahier de son sac, le
roi prit la parole et dit que l'impossibilité d'avoir la paix, et
l'extrême difficulté de soutenir la guerre, avaient fait travailler
Desmarets à trouver des moyens extraordinaires qui lui paraissaient
bons\,; qu'il lui en avait rendu compte, et qu'il avait été du même avis
quoique bien fâché d'être réduit à ces discours\,; qu'il ne doutait pas
qu'ils ne fussent d'avis semblable après que Desmarets le leur aurait
expliqué.

Après une préface si décisive et si contraire à la coutume du roi,
Desmarets fit un discours pathétique sur l'opiniâtreté des ennemis et
l'épuisement des finances, court et plein d'autorité, qu'il conclut par
dire qu'entre laisser le royaume en proie à leurs armes ou se servir des
seuls expédients qui restaient, lui n'en sachant aucuns autres, il
croyait encore moins dur de les mettre en usage que de souffrir l'entrée
des ennemis dans toutes les provinces de France\,; qu'il s'agissait de
l'imposition du dixième denier sans exception de personne\,; qu'outre la
raison d'impossibilité susdite, chacun encore y trouverait son compte,
parce que cette levée qui serait modique pour chacun en comparaison de
ce qu'il avait sur le roi en rentes et en bienfaits (mais outre cette
iniquité criante à ceux-là, combien de gens qui n'avaient rien du roi ni
sur le roi\,!) en procurerait le payement régulier désormais, et par là
un recouvrement de moyens pour tous les particuliers, et une circulation
pour le général qui remettrait une sorte de petite abondance et de
mouvement d'argent\,; qu'il avait tâché de prévenir tous les
inconvénients tant pour le roi que pour ses sujets, et que ces messieurs
en jugeraient mieux par la lecture de l'édit même qu'il allait faire,
que par tout ce qu'il en pourrait dire de plus. Aussitôt, et sans
attendre de réponse, il se mit à lire l'édit, et il le lut d'un bout à
l'autre tout de suite sans aucune interruption, puis il se tut.

Personne ne prenant la parole, le roi demanda l'avis à d'Aguesseau, à
qui comme le dernier du conseil c'était à parler le premier. Ce digne
magistrat répondit que l'affaire lui paraissait d'une si grande
importance qu'il n'en pouvait dire ainsi son avis sur-le-champ\,; qu'il
lui faudrait pour le former lire longtemps chez lui l'édit, tant sur la
chose même que sur la forme, partant qu'il suppliait le roi de le
dispenser d'opiner là-dessus. Le roi dit que d'Aguesseau avait raison\,;
que l'examen qu'il demandait était même inutile, puisqu'il ne pouvait
être travaillé plus que ce qu'avait fait Desmarets, qui était d'avis de
faire cet édit, et tel qu'ils le venaient d'entendre\,; que c'était
aussi son sentiment à lui à qui Desmarets en avait rendu compte, et
qu'ainsi ce ne serait que perdre le temps que de le discuter davantage.

Tous se turent, hormis le duc de Beauvilliers, qui, séduit par le neveu
de Colbert son beau-père, qu'il croyait un oracle en finances, et touché
de la réduction à l'impossible, dit en peu de mots que, tout fâcheux
qu'il reconnût ce secours, il ne pouvait ne le pas préférer à voir les
ennemis ravager la France, ni trouver que ce parti ne fût plus salutaire
à ceux-là mêmes qui en souffriraient le plus.

Ainsi fut bâclée cette sanglante affaire, et immédiatement après signée,
scellée, enregistrée parmi les sanglots suffoqués, et publiée parmi les
plus douces mais les plus pitoyables plaintes. La levée ni le produit
n'en furent pas tels à beaucoup près qu'on se l'était figuré dans ce
bureau d'anthropophages, et le roi ne paya non plus un seul denier à
personne qu'il faisait auparavant. Ainsi tourna en fumée ce beau
soulagement, cette sorte de petite abondance, cette circulation et ce
mouvement d'argent, lénitif unique du beau discours de Desmarets. Je sus
dès le lendemain tout le détail que je viens de rapporter, par le
chancelier. Quelques jours après la publication de l'édit, il se
répandit qu'il s'y était opposé avec vigueur au conseil des finances\,;
cela lui fit grand honneur, mais il s'en fit un bien plus véritable en
rejetant hautement le faux. Il avoua à quiconque lui en parla qu'il
s'était tu absolument, qu'il n'avait pas été mis à-portée de dire un
seul mot là-dessus, qu'il en était même bien aise, parce que tout ce
qu'il aurait pu dire n'aurait rien changé à une résolution de ce poids,
absolument prise, dont on ne leur avait parlé que par forme, cérémonie
qui l'avait même surpris. D'ailleurs il ne se cacha pas de blâmer cette
invention affreuse avec toute l'amertume que méritait un remède tourné
en poison.

Le maréchal de Vauban était mort de douleur du succès de son zèle et de
son livre, comme je l'ai raconté en son lieu. Le pauvre Boisguilbert,
qui avait survécu à l'exil que le sien lui avait coûté, conçut une
affliction extrême de ce que, par n'avoir songé qu'au bien de l'État et
au soulagement universel de tous ses membres, il se trouvait l'innocent
donneur d'avis d'un si exécrable monopole, lui qui n'avait imaginé et
proposé le dixième denier qu'en haine et pour la destruction totale de
la taille et de tout monopole, et soutint constamment que ce dixième
denier en sus des monopoles ne produirait presque rien, par le défaut de
circulation et de débit qui formait l'impuissance, et l'événement fit
voir en bref qu'il ne se trompait pas. Ainsi tout homme, sans aucun
excepter, se vit en proie aux exacteurs, réduit à supputer et à discuter
avec eux son propre patrimoine, à recevoir leur attache et leur
protection sous les peines les plus terribles, à montrer en public tous
les secrets de sa famille, à produire lui-même au grand jour les
turpitudes domestiques enveloppées jusqu'alors sous les replis des
précautions les plus sages et les plus multipliées\,; la plupart à
convaincre, et vainement, qu'eux-mêmes propriétaires ne jouissaient pas
de la dixième partie de leurs fonds. Le Languedoc entier, quoique sous
le joug du comite \footnote{Les comites et non \emph{comités}, comme on
  l'a imprimé dans plusieurs des éditions antérieures, étaient préposés
  aux travaux des galériens.} Bâville, offrit en corps d'abandonner au
roi tous ses biens sans réserve, moyennant assurance d'en pouvoir
conserver quitte et franche la dixième partie, et le demanda comme une
grâce. La proposition non-seulement ne fut pas écoutée, mais réputée à
injure et rudement tancée. Il ne fut donc que trop manifeste que la
plupart payèrent le quint\footnote{Cinquième partie. On appelait
  spécialement \emph{quint} un droit féodal que percevait le seigneur
  suzerain toutes les fois qu'une terre relevant de ses domaines passait
  à un nouveau propriétaire.}, le quart, le tiers de leurs biens pour
cette dîme seule, et que par conséquent ils furent réduits aux dernières
extrémités. Les seuls financiers s'en sauvèrent par leurs portefeuilles
inconnus, et par la protection de leurs semblables devenus les maîtres
de tous les biens des François de tous les ordres. Les protecteurs du
dixième denier virent clairement toutes ces horreurs sans être capables
d'en être touchés.

Quelques jours après là publication de l'édit, Monseigneur, par grand
extraordinaire, alla dîner à la Ménagerie avec les princes ses enfants
et leurs épouses, et des dames en petit nombre. Là, Mgr le duc de
Bourgogne, moins gêné que d'ordinaire, se mit sur les partisans, dit
qu'il fallait qu'il en parlât, parce qu'il en avait jusqu'à la gorge,
déclama contre le dixième denier et contre cette multitude d'autres
impôts, s'expliqua avec plus que de la dureté sur les financiers et les
traitants, même sur les gens de finances, et par cette juste et sainte
colère, rappela le souvenir de saint Louis, de Louis XII, Père du
peuple, et de Louis le Juste. Monseigneur, ému par cette sorte
d'emportement de son fils qui lui était si peu ordinaire, y entra aussi
un peu avec lui, et montra de la colère de tant d'exactions aussi
nuisibles que barbares, et de tant de gens de néant si monstrueusement
enrichis de ce sang\,; et tous deux surprirent infiniment ce peu de
témoins qui les entendirent, et les consolèrent un peu dans l'espérance
en eux de quelque ressource.

Mais le décret en était porté\,; le vrai successeur de Louis XIV était
le fils d'un rat de cave, qui ajouta dans son long et funeste
gouvernement à tout ce qui s'était auparavant inventé en ce genre, et
qui mit les publicains et leurs vastes armées en effroi, et, s'il était
possible, en honneur par la vénération qu'il leur porta, la puissance et
le crédit sans bornes qu'il leur donna, le respect odieux qu'il leur fit
porter par les plus grands et par tout le monde, et les grâces et les
distinctions de la cour, de l'Église et de la guerre qu'ils partagèrent
avec les seigneurs, même avec préférence, jusqu'à pas une desquelles
jusqu'alors aucun d'eux n'avait osé lever les yeux.

Il faut maintenant parler d'une nouvelle bombe qui me tomba sur la tête,
et rapporter ce que je n'ai fait qu'indiquer ailleurs de l'incroyable
crédulité de Monseigneur.

Il faut se souvenir de ce que j'ai dit de du Mont, de la confiance de
Monseigneur pour lui, et de son constant souvenir de ce que mon père
avait fait pour le sien. Il faut encore remarquer que le roi déclara, le
lundi 2 juin, à Marly, le mariage de M. le duc de Berry, et qu'il alla
le même jour faire à Madame la demande de Mademoiselle\,; que le
dimanche 15 juin, M\textsuperscript{me} de Saint-Simon fut nommée dame
d'honneur de la future duchesse de Berry, de la manière qui a été
rapportée, dans le cabinet du roi à Versailles\,; que le dimanche 6
juillet, le mariage se fit dans la chapelle de Versailles\,; que le
mercredi suivant 9 juillet, le roi alla à Marly jusqu'au samedi 2
août\,; qu'il y retourna le mercredi 20 août jusqu'au samedi 13
septembre\,; qu'il y retourna encore le mercredi 8 octobre jusqu'au
samedi 18 du même mois\footnote{Nous avons reproduit exactement les
  dates de Saint-Simon, qui ont été changées dans les précédentes
  éditions.}\,; enfin qu'il y retourna le lundi 3 novembre jusqu'au
samedi 15 du même mois, et qu'il n'alla point à Fontainebleau cette
année, retenu par les fâcheuses affaires et par la dépense de ce voyage.
Ce sont quatre voyages de Marly depuis le mariage de
M\textsuperscript{me} la duchesse de Berry, et il n'y en eut plus après
de cette année.

Quelques jours après, le second voyage de Marly, commencé, revenant avec
le roi de la messe, du Mont, dans le resserré de la porte du petit salon
de la chapelle, prit son temps de n'être pas aperçu, me tira par mon
habit, et comme je me tournai, mit un doigt sur sa bouche, et me montra
les jardins qui sont au bas de la rivière, c'est-à-dire de cette superbe
cascade que le cardinal Fleury a détruite, et qui était en face derrière
le château. En même temps du Mont me glissa dans l'oreille\,: «\,Aux
berceaux.\,» Cette partie du jardin en était entourée avec des
palissades qui ôtaient la vue de ce qui était dans ces berceaux\,;
c'était le lieu le moins fréquenté de Marly, qui ne conduisait à rien,
et où l'après-dînée même et les soirs il était rare qu'on se promenât.

Inquiet de ce que me voulait du Mont avec tant de mystère, je gagnai
doucement l'entrée des berceaux, où, sans être vu, je regardai par une
des ouvertures que je le visse paraître. Il s'y glissa par le coin de la
chapelle, et j'allai au-devant de lui. En me joignant il me pria de
retourner vers la rivière, afin d'être encore plus écartés, et nous nous
y mîmes contre la palissade la plus épaisse, et dans l'éloignement des
ouvertures, pour être encore plus cachés sous ces berceaux. Tant de
façons me surprirent et m'effrayèrent\,; je le fus bien autrement quand
j'appris de quoi il était question.

Après quelques compliments de reconnaissance sur mon père et d'amitié
pour moi, du Mont me dit qu'il venait me donner la plus grande marque de
l'une et de l'autre, mais à deux conditions\,: la première, que je ne
ferais pas en la moindre chose du monde aucun semblant de savoir rien de
ce qu'il m'allait apprendre\,; l'autre, que je n'en ferais aucun usage
que lorsqu'il me le dirait\,; et que de concert avec lui, et je lui
donnai parole de l'un et de l'autre. Alors il me dit que deux jours
après le mariage de M. le duc de Berry, étant entré sur la fin de la
matinée dans le cabinet de Monseigneur, où il était tout seul avec l'air
fort sérieux, il l'avait suivi tout seul encore par le jardin, où il
entrait par les fenêtres de ses cabinets chez M\textsuperscript{me} la
princesse de Conti, chez laquelle il entrait aussi de la terrasse de
l'Orangerie de Versailles, par les fenêtres de son appartement, laquelle
aussi il trouva seule dans son cabinet\,; que tout en entrant,
Monseigneur lui avait dit d'un air contre son naturel fort enflammé, et
comme par interrogation, qu'elle était là bien tranquille\,; ce qui la
surprit à tel point, qu'elle lui demanda avec frayeur s'il y avait des
nouvelles de Flandre, et qu'est-ce qui était arrivé. Monseigneur
répondit avec un air de dépit qu'il n'y avait point de nouvelles, sinon
que j'avais dit que maintenant que le mariage du duc de Berry était
fait, il fallait faire chasser M\textsuperscript{me} la Duchesse et
elle, et qu'après cela nous gouvernerions tout à notre aise ce bon
imbécile, en parlant de soi\,; qu'elle ne devait donc pas être si
assurée ni si en repos. Puis tout à coup, et comme se battant les flancs
pour s'irriter davantage, il tint tous les propos qu'eût mérités ce
discours, ajouta des menaces, et dit qu'il avertirait bien le duc de
Bourgogne de me craindre, de m'écarter, et de s'éloigner tout à fait de
moi. Cette manière de soliloque dura assez longtemps sans que j'aie su
ce que M\textsuperscript{me} la princesse de Conti dit là dessus\,; mais
par le silence de du Mont à cet égard, par le dépit qu'elle montra du
mariage, et par presque tout ce qui l'environnait, je n'eus pas lieu de
croire qu'elle cherchât à rien adoucir. Du Mont seul en tiers, collé à
la muraille, frémissait sans oser dire une parole, et la scène ne finit
qu'à l'arrivée de Sainte-Maure, qui fit tout court changer de discours.

On ne peut comprendre l'effet que fit sur moi ce récit. Entre plusieurs
l'étonnement l'emporta\,; je regardai du Mont, je lui demandai comment
un pareil rapport se pouvait concevoir, comment il osait se faire, et
comment il pouvait être cru, et je le priai de me dire par quel biais et
par quel moyen proposer au roi, et réussir à lui faire chasser ses deux
filles, princesses du sang, qu'il aimait, et Monseigneur encore mieux\,;
et s'il ne fallait pas être plus fou que les plus enfermés pour
concevoir un projet si radicalement insensé et si parfaitement
impossible\,; plus fou encore de s'en vanter et de le dire, et plus que
démon pour l'inventer et en affubler quelqu'un qui au moins n'avait
jamais passé pour fou ni pour visionnaire. Je lui demandai encore ce
qu'il lui semblait de celui qui s'en était si aisément persuadé. Du Mont
m'avoua que tout ce que je disais était véritable et d'une évidence
parfaite\,; mais que la calomnie n'en était pas moins faite et reçue. Je
n'osai enfoncer sur la crédulité de Monseigneur, content que du Mont, en
haussant les épaules, et par quelques mots échappés, me laissât entendre
qu'il en pensait tout comme moi.

Après la première surprise, qui fut en moi le sentiment le plus fort, je
vis l'abîme qu'on avait creusé sous mes pieds, et je demandai à du Mont
qu'y faire. «\,Rien du tout pour le présent, me dit-il\,; je n'ai osé
vous avertir plus tôt, parce que, ayant été le seul témoin de la scène
avec M\textsuperscript{me} la princesse de Conti, j'ai voulu laisser
éloigner le temps\,; il n'est pas encore venu de rien faire. Attendez
que je vous avertisse, et je le ferai soigneusement. --- Mais, monsieur,
lui répondis-je, qui suis-je, moi, vis-à-vis de Monseigneur en fureur,
et toujours dans les mêmes lieux que lui, hors à Meudon\,? Que devenir
ici dans le salon en sa présence\,? Comment oser lui faire ma cour chez
lui, et comment oser ne la lui pas faire en attendant que vous
m'avertissiez et que nous ayons trouvé moyen de lui faire entendre
raison, avec tous les démons qui l'obsèdent et qui l'entretiendront dans
cette humeur, ceux surtout qui ont osé abuser de lui jusqu'à lui faire
accroire une absurdité, trop forte même pour un enfant de six ans\,? ---
Tout cela est très-embarrassant, me répliqua du Mont\,; ne demandez
point pour Meudon, ne vous approchez guère ici de Monseigneur dans le
salon, allez chez lui de loin à loin, mais allez-y\,; vous ne vous êtes
aperçu de rien de lui jusqu'à cette heure\,; en vivant de la sorte à son
égard, il ne s'échappera à rien avec vous, c'est tout ce que je puis
vous dire.\,» Il me recommanda après tant et plus l'observation exacte
des deux conditions qu'il m'avait fait promettre, reçut mes remercîments
à la hâte, et s'enfuit par où il était venu, dans la frayeur d'être
avisé par quelqu'un.

Je demeurai assez longtemps à me promener sous ces berceaux, à rêver à
l'excès de scélératesse, à l'opinion que ceux qui l'avaient conçue
pouvaient avoir d'un prince à qui ils avaient osé espérer de la lui
faire croire, et à qui ils l'avaient si bien persuadée, et à m'abîmer
dans les réflexions de ce qu'on pourrait devenir sous un roi gouverné
par de pareils démons, et incapable de ne pas gober les absurdités les
plus grossières et les plus palpables. Revenant à moi, je ne savais ni
comment me tirer de celle-ci, bien moins encore parer toutes celles
qu'il plairait aux mêmes gens d'inventer, et d'en coiffer ce pauvre
prince. Je me retirai chez moi dans tout le malaise qu'il est aisé de
s'imaginer, et que je ne confiai qu'à M\textsuperscript{me} de
Saint-Simon, qui n'en fut pas moins étonnée que moi, ni moins
épouvantée. Je suivis exactement la conduite que du Mont m'avait
prescrite.

J'allais assez médiocrement chez Monseigneur, et même à Marly fort
rarement autour de lui, parce que cette cabale qui le gouvernait, et
dont j'ai plus d'une fois parlé, était toute composée de gens qui me
haïssaient parfaitement. Je n'avais donc aucune familiarité avec
Monseigneur\,; j'allais assez rarement à Meudon\,; ainsi la conduite que
j'eus à garder fut imperceptible au monde. Je n'ai jamais su, et j'en
loue Dieu encore, qui avait fait accroire à Monseigneur cette ineptie si
cruelle, et parmi cette troupe mâle et femelle de cette cabale, je n'ai
pu démêler ni asseoir aucun soupçon sur personne de distinct. Les choses
de rang pour les deux Lislebonne et leur oncle de Vaudemont, Rome à
l'égard de d'Antin, ce qui s'était passé avec feu M. le duc et
M\textsuperscript{me} la Duchesse, les choses de Flandre sur le tout les
avaient tous rendus mes ennemis personnels. Ils m'avaient vu, malgré
toutes leurs menées, ressusciter auprès du roi\,; ils frémissaient de ce
que je n'étais pas resté perdu\,; ma liaison intime avec M. {[}le duc{]}
et M\textsuperscript{me} la duchesse d'Orléans aigrissait leur haine\,;
enfin le mariage de M. le duc de Berry en avait comblé la mesure.
Quoique les détails en demeurassent ignorés, il n'avait que trop
transpiré que je l'avais fait, et la démarche que je fis par Bignon
auprès de la Choin, si proche de la déclaration du mariage, acheva de
les en persuader, quoique je me fusse bien gardé d'en rien laisser
imaginer dans tout ce qui se passa entre Bignon et moi. Mes liaisons si
intimes avec le chancelier, les ducs de Chevreuse et de Beauvilliers,
ces deux derniers qu'ils haïssaient parfaitement, et tant d'autres
principaux personnages des deux sexes, leur faisaient peur, et plus que
tout, comme je le sentis par ce qu'en dit Monseigneur, ce qui commençait
à se former d'intime entre Mgr le duc de Bourgogne et moi, que des yeux
si perçants et si attentifs commençaient à apercevoir parmi les
ténèbres, leur faisait frayeur et les déterminait à tout oser et à tout
entreprendre.

Dans une situation d'autant plus violente, dans la contrainte de son
secret, que l'avenir en était plus terrible que le présent n'en était
fâcheux et embarrassant à quelque point qu'il le fût, je pris du Mont
dans le salon, un matin, tout à la fin de ce même voyage. Après force
répétitions de l'absurdité de la calomnie, de respects pour Monseigneur,
je lui proposai de lui dire qu'ayant appris ce qui m'était imputé auprès
de lui, et le regardant comme étant déjà roi par avance, je ne pouvais
demeurer dans cet état, et que j'avais prié du Mont d'obtenir de lui la
grâce de le pouvoir entretenir un quart d'heure, ou de recevoir comme un
sacrifice fait à son injuste colère de me retirer en Guyenne jusqu'à ce
qu'il me permît de lui démontrer l'absurdité d'une si noire calomnie. Du
Mont ne put désapprouver mon impatience de sortir de cette étrange
affaire, ni le respect avec lequel je m'y prenais. Il me promit de
parler à Monseigneur avec étendue, mais il le fit avec un air beaucoup
moins ouvert, et en homme que cela embarrassait pour avoir été témoin de
la scène. C'était un homme de fort peu d'esprit, timide et fort mesuré,
qui craignait tout et qui s'embarrassait de tout. Il me dit qu'il
n'était pas temps encore, qu'il le prendrait dès qu'il le verrait à
propos, et se rabattit à m'exhorter à la patience et au secret, et à la
conduite que je lui avais promise.

Monseigneur traversa le salon et me vit parler à du Mont tête à tête.
J'en fus bien aise dans l'espérance qu'il lui demanderait ce que je lui
disais, et qu'il en pourrait profiter pour ce que je désirais. La messe
du roi finit notre conversation.

Ce Marly, comme je l'ai dit, était le second depuis le mariage.
J'espérais peu des mesures et de la faiblesse de du Mont\,; nous
songeâmes donc, M\textsuperscript{me} de Saint-Simon et moi, à nous
aider d'ailleurs, dès que du Mont m'en laisserait libre, mais comme ce
que nous résolûmes ne s'exécutait pas aisément par la mécanique si
principale en toutes les choses de la cour, fatigués d'ailleurs d'une
situation si pénible, et dans le dessein de ne laisser pas refroidir les
promesses de liberté pour y accoutumer de bonne heure, et s'établir sur
le pied d'en prendre, un peu avant le troisième Marly,
M\textsuperscript{me} de Saint-Simon eut une audience de
M\textsuperscript{me} la duchesse de Bourgogne, qui depuis le mariage ne
pouvait plus être remarquée.

Elle la supplia d'obtenir la permission du roi pour elle d'aller passer
ce voyage de Marly, qui devait être court, à la Ferté pour se trouver au
retour à Versailles. Cela ne fit aucune difficulté, mais grand bruit, et
grande envie par la distinction. Aucune dame d'honneur, pas même celles
des bâtardes du roi, n'avait eu liberté de s'absenter deux jours
seulement, et cet esclavage était passé en loi par l'habitude.
M\textsuperscript{me} de Saint-Simon usa sagement de cette liberté, mais
elle en usa plusieurs fois, et fut la seule à qui elle fut accordée,
laquelle même lui tourna à bien\,; nous allâmes donc nous reposer et
réfléchir à la Ferté, et nous y prîmes la résolution dont je parlerai
tout à l'heure.

De retour à Versailles, le roi fit le troisième voyage à Marly depuis le
mariage. Vers le milieu du voyage, du Mont, comme la première fois, me
tira en revenant de la messe du roi et me montra les berceaux. J'allai
aussitôt l'y attendre. Là il me dit qu'il croyait maintenant que je
pouvais faire parler à Monseigneur, parce qu'il y avait assez longtemps
de ce dont il m'avait averti pour que j'eusse pu l'être d'ailleurs, et
le laisser hors de soupçon de l'avoir fait\,; que néanmoins, après y
avoir bien réfléchi, il n'avait pas cru pouvoir hasarder de parler à
Monseigneur, parce qu'il avait été témoin de la scène, mais que si
Monseigneur, plein de ce qu'on lui aurait dit pour moi là-dessus, lui en
parlait, il saisirait l'occasion et dirait merveilles. Je lui fis valoir
l'exactitude si pénible avec laquelle je lui avais tenu les deux
conditions qu'il m'avait demandées\,; je ne fis pas semblant de sentir
sa faiblesse et sa timidité, parce qu'on ne peut tirer des gens plus que
ce qui est en eux, et que le service de l'avis n'en était pas moins
grand, et pour accomplir toute fidélité avec lui, je lui proposai de
faire parler à Monseigneur par M\textsuperscript{me} la duchesse de
Bourgogne\,; il l'approuva fort. Je ne laissai pas pourtant de lui
demander si ce canal serait agréable, et il m'en assura. Je lui promis
de l'instruire du succès, et nous nous séparâmes de la sorte avec force
amitiés et recommandations de sa part de continuer ma même conduite à
l'égard de Monseigneur, jusqu'à ce qu'il pût être pleinement détrompé.

L'impossibilité de trouver personne assez de nos amis et assez avant
dans la privance de Monseigneur pour lui faire parler, nous avait
tournés vers M\textsuperscript{me} la duchesse de Bourgogne.
M\textsuperscript{me} de Saint-Simon en eut une audience dans laquelle
elle lui conta ce qui vient d'être rapporté, sans lui nommer du Mont,
l'excita sur le mariage imputé à crime auquel elle avait eu une si
principale part, lui fit sentir jusque pour elle-même et pour Mgr le duc
de Bourgogne en quel danger chacun était par l'incroyable crédulité de
Monseigneur, livré sans réserve à de tels scélérats.
M\textsuperscript{me} la duchesse de Bourgogne en fut vivement
touchée\,; elle en sentit tout le péril, entra pleinement en tout ce que
M\textsuperscript{me} de Saint-Simon lui dit, lui parla avec toute sorte
d'intérêt et d'amitié, reçut avec mille bontés la prière qu'elle lui fit
de parler à Monseigneur, et lui promit de prendre son temps pour le
faire, avec l'étendue que la chose méritait, et en soi, et à mon égard.
Quinze ou vingt jours après, elle eut l'attention de dire à
M\textsuperscript{me} de Saint-Simon, qui ne lui en avait point reparlé,
de ne s'impatienter pas\,; qu'elle n'avait pu trouver encore occasion de
pouvoir parler avec étendue, mais qu'elle pouvait compter qu'elle la
cherchait et ne la manquerait pas. Cela dura jusqu'après le quatrième et
dernier voyage de Marly, d'où le roi revint le samedi 15 novembre.

Le lendemain dimanche, Monseigneur s'en alla à Meudon pour plusieurs
jours. Il vint à Versailles le mercredi suivant, 19 novembre, pour le
conseil d'État, au sortir duquel il retourna dîner à Meudon, et y mena
tête à tête avec lui M\textsuperscript{me} la duchesse de Bourgogne. Ce
fut là qu'elle lui parla, sûre du temps, d'être seule, et de ne pouvoir
être interrompue. Elle entama sur M\textsuperscript{me} de Saint-Simon,
qui allait aussi dîner à Meudon avec Mgrs ses fils et
M\textsuperscript{me} la duchesse de Berry. Sur ce que Monseigneur la
loua fort, la princesse lui dit qu'il la mettait pourtant au désespoir.
Il fut très-surpris, et demanda comment. Alors elle lui parla
franchement de l'affaire qu'on m'avait faite auprès de lui. Il l'avoua
et s'en irrita de nouveau. Elle lui laissa tout dire, et puis lui
demanda si bien sérieusement il en était persuadé\,; de là, lui dit avec
adresse qu'elle aimait fort M\textsuperscript{me} de Saint-Simon, que de
moi elle ne s'en souciait point, mais que pour lui-même elle ne pouvait
souffrir de le voir la dupe d'une invention si grossière\,; qu'il
n'était pas imaginable qu'un homme avec la moindre teinture de la cour,
combien moins un homme qu'on lui avait dépeint comme si remuant, si
plein d'esprit et de connaissances, si dangereux, pût se mettre dans la
tête un projet aussi insensé que celui de faire chasser de la cour deux
veuves de princes du sang, si aimées de lui et du roi qui était leur
père, bien moins encore de le dire, et qu'à la première vue de la chose,
nul homme du moindre sens n'y pouvait ajouter foi. II n'en fallut pas
davantage à ce pauvre prince pour lui persuader l'ineptie d'une
supposition qu'il avait si aisément gobée, et tout d'un coup pour lui
faire naître la honte d'avoir si pleinement donné dans un panneau si
grossièrement tendu. Il l'avoua à l'instant de bonne foi, convint de
tout avec elle, et dit qu'il n'avait pas tant fait de réflexion, parce
que la colère l'avait surpris.

Elle en prit occasion de lui donner des soupçons contre des personnes
qui avaient eu assez peu de respect pour lui pour l'exposer à une colère
si peu fondée et si fort à leur gré, et pour lui représenter qu'étant ce
qu'il était, il ne pouvait être trop en garde contre les faux rapports,
et contre les gens qu'il y aurait surpris, et si grossiers encore. Elle
n'osa lui demander qui c'était, et se contenta de lui dire que tout ce
qui l'approchait me haïssait, les uns par rang, les autres par d'autres
raisons. Elle le laissa changer de discours, dont il eut hâte, après
qu'elle lui eut fait suffisamment sentir combien ce rapport était peu
respectueux, hardi, scélérat et incroyable, et combien honteux et
dangereux pour lui d'y avoir donné sans y faire la moindre attention.

Elle ne voulut faire semblant de rien à M\textsuperscript{me} de
Saint-Simon à Meudon\,; mais à Versailles, le soir même, elle lui rendit
toute cette conversation, dont M\textsuperscript{me} de Saint-Simon lui
rendit les grâces que méritait ce service, rendu avec tant de force,
d'esprit\,; de bonté et de succès. Dès que je pus voir du Mont, je lui
dis, mais sans détail, que M\textsuperscript{me} la duchesse de
Bourgogne avait parlé à merveilles, et réussi à détromper Monseigneur,
dont il me parut fort aise. M. de Beauvilliers et le chancelier, qui
étaient en grande peine de me savoir dans ce bourbier, se réjouirent
fort de m'en savoir dehors, et {[}furent{]} fort d'avis du parti que je
m'étais proposé, de continuer à l'égard de Monseigneur, avec qui je
n'avais qu'à perdre par ses entours infernaux et rien à gagner, la même
conduite que je gardais depuis cette aventure, et de laisser croire
ainsi aux honnêtes gens qui m'y avaient mis que j'y étais encore, pour
ne leur pas donner envie de quelque autre invention qui me perdrait
peut-être auprès d'un prince si facile à croire, et si fort entre leurs
mains, sans que j'en pusse être averti.

\hypertarget{chapitre-ii.}{%
\chapter{CHAPITRE II.}\label{chapitre-ii.}}

1710

~

{\textsc{Abbé de Vaubrun rappelé après dix ans d'exil.}} {\textsc{- Sa
famille, son caractère.}} {\textsc{- Bulle qui condamne les jésuites sur
les usages chinois.}} {\textsc{- Cinq hommes d'augmentation par
compagnie d'infanterie.}} {\textsc{- Taxe d'usuriers.}} {\textsc{-
Refonte et profit de la monnaie.}} {\textsc{- Pont de Moulins tombé.}}
{\textsc{- Ravages de la Loire.}} {\textsc{- Grand prieur enlevé par une
espèce de partisan impérial.}} {\textsc{- Apanage et maison de M. {[}le
duc{]} et de M\textsuperscript{me} la duchesse de Berry.}} {\textsc{-
Rare méprise.}} {\textsc{- Benoist, contrôleur de la bouche, homme
dangereux.}} {\textsc{- Scrupule du roi sur la vénalité des charges de
ses aumôniers.}} {\textsc{- M\textsuperscript{me} de La Rochepot fort
étrangement admise, comme femme du chancelier de M. le duc de Berry, à
Marly, à {[}la{]} table et dans les carrosses de M\textsuperscript{me}
la duchesse de Bourgogne.}} {\textsc{- M\textsuperscript{me} la duchesse
de Bourgogne seule maîtresse indépendante de sa maison.}} {\textsc{-
Retour des généraux.}} {\textsc{- Fervaques quitte le service.}}
{\textsc{- Mort du lord Greffin.}} {\textsc{- Mort de Spanheim.}}
{\textsc{- Mort et deuil de la duchesse de Mantoue.}} {\textsc{-
Prétendu faiseur d'or.}} {\textsc{- Boudin\,; son état et son
caractère.}} {\textsc{- Bals, fêtes et plaisirs à la cour tout
l'hiver.}}

~

L'abbé de Vaubrun, depuis dix années en exil, et les dernières avec
permission d'être à Paris, sans approcher plus près de la cour, eut
enfin permission de venir saluer le roi, le jour du retour à Versailles
du dernier voyage de Marly de cette année. Son nom était Bautru, de la
plus petite et nouvelle bourgeoisie de Tours.

Vaubrun, son père, était frère de Nogent, tué maître de la garde-robe,
au passage du Rhin, qui avait épousé la sœur de M. de Lauzun, du
chevalier de Nogent, et de la Montauban, cette fausse princesse dont
j'ai parlé quelquefois. Leur père avait fait sa fortune par beaucoup
d'esprit et de souplesse, sur la fin de Louis XIII, et surtout dans la
minorité de Louis XIV, et était devenu capitaine de la porte.

Nogent eut sa charge à sa mort, et après celle de maître de la
garde-robe, pour épouser pour rien la sœur de M. de Lauzun, qui était
fille de la reine mère. Vaubrun avait épousé la fille de Serrant, frère
de son père, qui était très-riche et avait été maître des requêtes, qui
vivait encore à quatre-vingt-cinq ou six ans, retiré à Serrant en Anjou,
où l'abbé de Vaubrun avait passé son exil. Vaubrun fut tué lieutenant
général au combat d'Altenheim, à cette belle et fameuse retraite que mon
beau-père fit à la mort de M. de Turenne.

Il laissa deux filles, dont l'aînée fut, en 1688, seconde femme du duc
d'Estrées, et une autre, dont j'ai parlé à l'occasion de son enlèvement,
et qui fut depuis enfermée aux Annonciades de Saint-Denis, où elle a
fait profession, et un fils unique, mais absolument nain, extrêmement
boiteux, qui par ces défauts naturels se fit d'Église. Avec ses jambe
torses et une tête à faire peur, il ne laissait pas d'être fort
audacieux avec les femmes, pour lesquelles il se croyait de grands
talents. Il avait du savoir, beaucoup d'esprit, peu ou point de
jugement, une grande hardiesse, la science du monde où il voulait tout
savoir, être de tout, se mêler de tout, frappant à toutes les portes,
obséquieux, respectueux, bassement valet de tous gens en place souvent
ennemis, toujours dès qu'ils y arrivaient, et se fourrant chez tout ce
qui figurait. Une folle ambition et la passion du grand monde lui firent
acheter une charge de lecteur pour s'introduire à la cour. L'intrigue
était son élément, mais dangereux, imprudent, peu sûr d'ailleurs, et
comme tel, craint, évité, méprisé. Il se dévoua au cardinal de Bouillon
dont les intrigues le firent chasser, et les siennes avec les jésuites
le firent revenir. Il finit par se faire l'âme damnée de M. et de
M\textsuperscript{me} du Maine, qui ne le menèrent à rien. Toute sa vie
il eut la rage d'être évêque.

En ce temps-ci parut une bulle du pape, qui décida très-nettement toutes
les disputes des missionnaires et des jésuites de la Chine sur les
cérémonies chinoises de Confucius, des ancêtres et autres, qui les
déclara idolâtriques, les proscrivit, condamna les jésuites dans leur
tolérance et leur pratique là-dessus, approuva la conduite du feu
cardinal de Tournon, dont les souffrances, la constance et la mort y
étaient fort louées, et les menées et la désobéissance des jésuites fort
tancées. Cette bulle les mortifia moins qu'elle ne les mit en furie\,;
ils l'éludèrent, puis à découvert la sautèrent à joints pieds. On a tant
écrit sur ces matières que je n'en dirai pas davantage. Je fais
seulement mention de cette bulle comme de la source de tout le fracas
qui arriva bientôt après, et dont la persécution dure encore, et n'a
fait que croître en fureur. Je parlerai en son temps de son chef-d'œuvre
du démon et des jésuites, et en particulier du P. Tellier.

Le dixième établi donna lieu à augmenter toute l'infanterie de cinq
hommes par compagnie. On fit aussi une taxe sur les usuriers, qui
avaient gagné gros à trafiquer les papiers du roi, c'est-à-dire à
profiter du besoin de ceux à qui le roi les donnait en payement. On
appelait ces gens-là agioteurs, et leur manége, suivant la presse où
étaient les porteurs de billets, de donner par exemple trois ou quatre
cents livres, et souvent encore la plupart en denrées, pour un billet de
mille francs, ce manége, dis-je, s'appelait agio. On prétendit tirer une
trentaine de millions de cette taxe. Bien des gens y gagnèrent gros, je
ne sais si le roi y fut le mieux traité. Bientôt après on refondit la
monnaie, ce qui fit un grand profit au roi et un extrême tort aux
particuliers et au commerce. On a dans tous les temps regardé comme un
très-grand malheur, et comme quelque chose de plus, de toucher aux blés
et aux monnaies. Desmarets a accoutumé au manége de la monnaie\,; M. le
Duc et le cardinal Fleury, à celui des blés et de la famine factice.

Le pont que Mansart avait bâti à Moulins sur l'Allier avait été emporté
aussitôt qu'achevé, comme je l'ai rapporté en son lieu. Il y en avait
rebâti un autre, qu'il avait assuré devoir durer jusqu'à la postérité la
plus reculée. Il avait coûté plus de huit cent mille livres. Il fut
emporté aux premiers commencements de cet hiver, par l'inondation de la
Loire, qui par ses ravages coûta plus de dix millions au royaume, qui,
comme il a été expliqué ailleurs, en fut redevable au crédit du duc de
La Feuillade. Le grand prieur, encore sorti du royaume, comme il a été
rapporté en son lieu, s'était, à force d'errer, établi à Venise. Ne se
trouvant bien nulle part, il alla promener ses inquiétudes tout à la fin
d'octobre, et se mit en chemin pour Lausanne, en Suisse. Une manière de
bandit nommé Massenar, ayant pourtant une commission de l'empereur, et
dont le fils avait été pris depuis quelques mois, et mis à Pierre-Encise
pour les crimes de son père et pour les siens, attrapa le grand prieur
dans son chemin, lui fit passer diligemment le Rhin, l'enferma dans un
château de l'empereur, et lui déclara qu'il le traiterait tout
pareillement que son fils serait traité. Il eut permission d'en envoyer
avertir le comte du Luc, ambassadeur du roi en Suisse, qui en donna avis
par un courrier. Il ne parut pas que le roi fût fort ému de cette
nouvelle, ni que personne y prît grande part.

L'emprunt continuel où M. le duc et M\textsuperscript{me} la duchesse de
Berry étaient sans cesse réduits d'officiers de chambre, et de gardes du
roi, et de table de M\textsuperscript{me} la duchesse de Bourgogne,
lassa enfin par l'importunité, tellement qu'au lieu d'attendre la paix
qui paraissait encore si éloignée, le roi, contre sa première
résolution, se porta à donner un apanage à son petit-fils. Les pensions
furent accordées sur le pied de celles qu'avaient eues Monsieur et
Madame, mais l'apanage fut fort différent. La reine mère, qui aimait
tendrement Monsieur et qui était régente, régla le sien et n'y garda
point de mesure\,; on tomba pour celui-ci dans l'extrémité contraire. Le
revenu ne suffit pas à la dépense du pied de la maison\,; les
extraordinaires, si souvent indispensables, se trouvèrent sans fonds\,;
on ne donna pas le moindre meuble, ni aucune maison de ville ni de
campagne\,; et ce ne fut que du temps après que le palais de Luxembourg
ou d'Orléans leur fut donné à Paris. Cet apanage fut des duchés
d'Angoulême et d'Alençon, avec quelque extension légère, et du pays de
Ponthieu, avec la collation de tous les bénéfices de nomination royale,
excepté les évêchés comme à feu Monsieur, mais qui s'y trouvèrent rares
et petits.

Tout cela fait et passé\,; MM. d'Abbeville, qui par leur ancienne
fidélité et service ont obtenu et conservé le privilège de garder
eux-mêmes le roi lorsqu'il passe par leur ville, et de n'y recevoir
aucunes troupes, députèrent pour demander en cette considération que
leur ville fût détachée de l'apanage, et réservée immédiatement à la
couronne. La Vrillière, secrétaire d'État, qui l'avait dans son
département, en rendit compte au roi, dont la surprise fut extrême
d'apprendre qu'Abbeville fût de l'apanage, et demanda pourquoi. La
question parut étrange\,; mais l'étonnement le devint quand, à la
réponse, il dit qu'il ne savait pas que le Ponthieu fût là, ni
qu'Abbeville en fût la capitale. Il ajouta que ce pays sentait trop la
poudre à canon pour être donné en apanage, et le fit retirer.

Le Berry en la place, et même tout d'abord, convenait mieux qu'aucune
autre pièce, puisque le prince en portait le nom. Mais, en examinant, on
trouva que tout le domaine en était engagé à la maison de Condé. On eut
donc recours au comté de Gisors et à quelques environs pour remplacer le
Ponthieu\,; et, comme les noms d'Angoulême et d'Alençon avaient été
profanés par la bâtardise de Charles IX, et par le fils mort enfant du
dernier duc de Guise, le roi fit expédier des lettres patentes à son
petit-fils pour porter le nom de duc de Berry, qui lui avait été imposé
en naissant, quoiqu'il n'y eût aucune propriété. L'affaire de l'apanage
consommée, on mit en vente les charges de la maison de M. {[}le duc{]}
et de M\textsuperscript{me} la duchesse de Berry. Comme ils y désirèrent
des noms, la chose fila assez lentement. Son peu d'importance n'en fera
pas ici à deux fois.

Le duc de Beauvilliers, qui, comme ayant été gouverneur de M. le duc de
Berry, était seul de droit premier gentilhomme de sa chambre, eut la
disposition de cette charge. Comme tout se réglait sur le premier pied
de la maison de feu Monsieur pour le nombre des charges et de leurs
appointements, M. de Beauvilliers fit deux charges de la sienne. Il fit
présent en plein de l'une au duc de Saint-Aignan, son frère, dont la
naissance et encore plus la dignité flattèrent extrêmement M. {[}le
duc{]} et M\textsuperscript{me} la duchesse de Berry, et vendit l'autre
au marquis de Béthune, gendre de Desmarets, devenu depuis duc de Sully.
Le chevalier de Roye acheta une des deux charges de capitaine des
gardes. Clermont-d'Amboise, gendre d'O, prit l'autre\,; Montendre, celle
de capitaine des Cent-Suisses.

Rasilly, porté par le duc de Beauvilliers, qui l'avait fait
sous-gouverneur des princes, et qui depuis la fin de cet emploi n'avait
pas quitté M. le duc de Berry d'un pas, avec des fatigues de courses, de
chasses et de veilles incroyables, par ordre du roi et sans
appointements, en fut récompensé par le beau présent de la charge de
premier écuyer, demandée pour un prix fort haut par des gens de la
première qualité.

Toute la cour applaudit à cette grâce, parce qu'il la méritait, et qu'il
s'était fait universellement aimer, estimer et considérer.
M\textsuperscript{me} la duchesse de Berry, qui y voulait de plus grands
noms, en pleura amèrement et n'en cacha son dépit à personne. Il est
pourtant vrai que Rasilly était gentilhomme ancien, de fort bon lieu,
bien allié, lieutenant général de sa province, et que ses pères
l'avaient été quand ne l'était pas qui voulait, ni pour de l'argent.

Cette princesse ne fut pas si délicate pour La Haye, écuyer de M. le duc
de Berry, à qui elle fit donner pour rien la charge de premier veneur,
et bientôt après lui fit acheter par M. le duc de Berry celle de premier
chambellan, qui lui donnait place dans son carrosse, et à sa table,
quand il mangeait avec des hommes. Il s'en redressa et s'en regarda au
miroir avec plus de complaisance. Il était bien fait, mais avec une
taille haute de planche contrainte et un visage écorché qui d'ailleurs
n'avait rien de beau. Il fut heureux en plus d'une sorte, et plus
attaché à sa nouvelle maîtresse qu'à son maître. Le roi fut fort en
colère quand il sut que M. le duc de Berry avait emprunté ce présent.

De Pons et Monchy, gens de bonne maison, achetèrent les deux charges de
maîtres de la garde-robe. Champignelle, gentilhomme de bon lieu, et
gendre de feu Denonville, premier sous-gouverneur des princes, prit
celle de premier maître d'hôtel, et la fit très-honorablement. Le fils
du baron de Beauvais et de cette M\textsuperscript{me} de Beauvais,
première femme de chambre si confidente de la reine mère, desquels j'ai
parlé ailleurs, acheta celle de capitaine de la porte. Le roi l'avait
fait défaire de la capitainerie de Grenelle, Montrouge, etc., en faveur
de Bontems par une noire malice de Benoist, contrôleur de la bouche.

C'était un gros brutal qui servait toute l'année, fils d'un cuisinier de
Louis XIII. Il s'était rendu si familier avec le roi, par son assiduité
et son attention à ses mets, qu'il s'était fait craindre à toute la
cour, à Livry même, et ménager jusque par M. le Prince et M. le Duc. Il
traita souvent fort mal ce petit Beauvais sur du gibier assez mal à
propos, qui se rebéqua. Benoist fît languir le gibier, vanta les autres
capitaines des chasses qui en envoyaient de bonne heure, et quantité, se
plaignit qu'il n'en pouvait tirer de celui-ci, l'accusa de le vendre, et
fit si bien qu'il mit le roi en colère, et qu'il le perdit. Je sens bien
qu'en soi c'est la dernière des bagatelles pour être rapportée\,; mais
elle caractérise et dépeint.

L'abbé Turgot, aumônier du roi, venait d'être sacré évêque de Séez, et
cherchait à vendre sa charge. Il n'y avait plus que lui et l'abbé Morel
qui les eussent achetées\,; le roi les avait toutes retirées peu à peu
par scrupule de simonie\footnote{Trafic des choses saintes et
  spécialement des dignités ecclésiastiques. Fleury (\emph{Instit. au
  droit écclés}., IIIe part., chap.~xi) détermine les cas de simonie\,:
  vendre ou acheter la prédication ou l'administration des sacrements,
  en sorte que l'on refuse d'instruire, de baptiser, de donner
  l'absolution des péchés, sinon à certain prix\,; vendre l'ordination
  des évêques, des prêtres, des diacres ou des autres ministres de
  l'Église, et par conséquent la collation des offices ecclésiastiques
  et des revenus qui y sont attachés. «\,Les canons traitent encore de
  simonie d'exiger quelque chose pour la permission d'enseigner (il
  s'agit ici d'enseignement religieux), pour l'entrée dans les
  monastères qui ne doit avoir pour but que la pénitence et la
  perfection chrétienne, pour la consécration des églises, etc.\,»}. Il
croyait avec raison que ces charges s'achetaient pour se frayer et
s'abréger le chemin aux abbayes et à l'épiscopat, et que c'était
indirectement les acheter. Cette considération fit l'évêque de Séez
premier aumônier de M. le duc Berry, pour la plupart du prix de sa
charge, dont le roi lui paya le surplus. C'était un très-bon et honnête
homme.

Je procurai à Coettenfao, mon ami de tout temps, la charge de chevalier
d'honneur de M\textsuperscript{me} la duchesse de Berry, la plus belle
sans comparaison, et la plus commode de toutes à faire, et qui portait
naturellement à être chevalier de l'ordre. II était lieutenant général,
et des bons, et premier officier des chevau-légers, qu'il vendit. Pour
m'être trop pressé, il n'eut point là diminution que la difficulté de
vendre introduisit quelque temps après qu'il fut pourvu. Le chevalier
d'Hautefort acheta la charge de premier écuyer\,; le frère de son père
l'était de la reine. Il fut curieux de voir en même temps lui avec cette
charge chez une fille de France, et son frère écuyer de M. le comte de
Toulouse, lequel encore faisait l'important. Saumery, frère du
sous-gouverneur des princes, mais homme droit, simple et d'honneur, qui
s'ennuyait de sa retraite après avoir longtemps servi, acheta la charge
de premier maître d'hôtel, et la remplit très-honnêtement.

Celle de premier aumônier demeura longtemps à vendre, ainsi qu'une
infinité de petites. À la fin, l'abbé de castries, frère du chevalier
d'honneur de M\textsuperscript{me} la duchesse d'Orléans, maintenant
archevêque d'Alby et commandeur de l'ordre du Saint-Esprit, le fut
très-longtemps après.

Voysin, profitant de sa faveur, et ne sachant que faire de sa fille
aînée qu'il aimait fort, et qui était exclue de tout pour avoir épousé
un homme de robe, La Rochepot, fils de La Berchère, fort riche, lui fit
acheter la charge de chancelier de M. le duc de Berry, et fit accroire
au roi qu'avec cela il pouvait lui faire la grâce de l'admettre dans les
carrosses et à la table de M\textsuperscript{me} la duchesse de
Bourgogne, et par là la mener à Marly, ce qui fut très-extraordinaire.

En même temps le roi fit pour M\textsuperscript{me} la duchesse de
Bourgogne ce qu'il n'avait accordé ni a la reine ni à
M\textsuperscript{me} la Dauphine. Il lui laissa l'entier gouvernement
des affaires de sa maison, et la disposition de toutes les charges et
places, même sans lui rendre compte de rien\,: en un mot maîtresse
absolue. Il s'en expliqua ainsi tout haut, dit qu'il se fiait assez en
elle pour cela, et qu'elle serait capable de choses plus difficiles et
plus importantes. Cette faveur très-signalée vint de lui-même.
M\textsuperscript{me} la duchesse de Bourgogne se serait perdue avec lui
pour toujours, si elle avait fait la moindre tentative pour l'obtenir.
On peut croire qu'elle sut ménager une faveur si distinguée\,; et que,
pour peu que ce dont elle eut à disposer ne fût pas tout à fait dans le
petit, elle con-noissoit trop bien le roi pour rien faire sans lui, mais
sûre alors de son approbation et du gré de cette déférence.

Berwick, chassé par les neiges, revint le premier après avoir détaché
une partie de ses troupes pour le Roussillon. Harcourt revint ensuite,
Besons après, et tous les officiers de leurs armées entrées en quartiers
d'hiver. Villars aussi arriva des eaux de Bourbonne. Goesbriant fut reçu
en gendre de ministre, et eut avec l'ordre une pension de vingt mille
livres, en attendant le premier gouvernement.

Fervaques, colonel du régiment de Piémont, et brigadier d'infanterie
avec réputation, quitta le service. J'ai parlé ailleurs de ces Bullion à
l'occasion du carrosse de Madame où M\textsuperscript{me} de Bullion sa
mère entra une fois pour de l'argent qu'elle donna à
M\textsuperscript{me} de Ventadour, mais sans que cela ait été plus
loin. C'était une femme fort impérieuse, qui fit quitter son fils,
piquée qu'il ne fût pas maréchal de camp au sortir de Douai, quoique
brigadier seulement de l'hiver. Le roi en fut fort blessé. Qui lui
aurait dit que ce même Fervaques serait fait officier général comme s'il
n'eût point quitté, et chevalier de l'ordre en 1724\,: il aurait été
étrangement étonné et scandalisé, comme le fut aussi toute la France. Le
roi le punit par la bourse. Piémont lui avait coûté cent mille livres,
il le fixa à soixante-quinze mille. Ils purent être fâchés de ce petit
coup de houssine, mais trop riches pour se soucier de vingt-cinq mille
livres.

Lord Greffin, pris avec le marquis de Lévi, en mer, lors de la tentative
d'Écosse, dont il a été fait à cette occasion mention honorable, mourut
à Londres, dans un grand âge, de sa mort naturelle, ayant eu des répits
de sa condamnation de temps en temps, et sûreté qu'il en aurait
toujours. Il a été parlé alors assez de lui pour n'avoir rien à y
ajouter.

Spanheim, si connu dans la république des lettres, et qui ne l'a pas
moins été par ses négociations et ses emplois, mourut en ce même temps à
Londres, à quatre-vingt-quatre ans, avec une aussi bonne tête que
jamais, et une santé parfaite jusqu'à la fin. Il avait été longtemps à
Paris envoyé de l'électeur de Brandebourg, et il passa en la même
qualité à Londres lorsque les affaires se brouillèrent sur la succession
d'Espagne.

La duchesse de Mantoue mourut aussi à Paris, à la fleur de son âge, et
d'une beauté qui promettait une grande santé, le 16 décembre. Sa maladie
fut longue, dont elle sut heureusement profiter. Depuis son bizarre
mariage sa vie avait été fort triste\,; aucun des beaux projets de la
duchesse d'Elbœuf ni de ses grandes prétentions pour elle n'avait pu
réussir. Elle avait depuis son retour mené à Paris une vie fort triste.
Elle n'avait point d'enfants et n'eut rien de son mari. Il avait
l'honneur d'appartenir au roi, qui prit le deuil en noir pour cinq ou
six jours.

Il se produisit en ces derniers jours de l'année un de ces aventuriers
escrocs, qui prétendait avoir le grand secret de faire de l'or. Boudin,
premier médecin de Monseigneur, le fit travailler chez lui, sous ses
yeux et sous clef. On le verra dans quelque temps un hardi et dangereux
personnage pour un homme de son espèce. Il est bon d'en dire un mot
puisqu'il se trouve naturellement ici sous la main. Il était boudin de
figure comme de nom, fils d'un apothicaire du roi dont personne n'avait
jamais fait cas. Il étudia en médecine, fut laborieux, curieux, savant.
S'il fût demeuré dans l'application et le sérieux, c'eût été un bel et
bon esprit. Il l'avait d'ailleurs extrêmement orné de littérature et
d'histoire, et en avait infiniment d'un tour naturel, plein d'agrément,
de vivacité, de reparties, et si naïvement plaisant que personne n'était
plus continuellement divertissant, sans jamais vouloir l'être. Il fut
doyen de la faculté de Paris, médecin du roi, et enfin premier médecin
de Monseigneur, avec lequel il était au mieux. Il subjugua M. Fagon, le
tyran de la médecine et le maître absolu des médecins, au point d'en
faire tout ce qu'il voulait, et d'entrer chez lui à toute heure, lui
toujours sous cent verrous. Il haïssait le tabac jusqu'à le croire un
poison\,; Boudin lui dédia une thèse de médecine contre le tabac, et la
soutint toute en sa présence, se crevant de tabac, dont il eut toujours
les doigts pleins, sa tabatière à la main, et le visage barbouillé. Cela
eût mis Fagon en fureur d'un autre\,; de lui tout passait. Un homme de
si bonne compagnie réussit bientôt dans une cour où il ne pouvait faire
envie à personne. Il fut des soupers familiers de M. le Duc, de ceux de
M. le prince de Conti. C'était à qui l'aurait, hommes et femmes du plus
haut parage et de la meilleure compagnie, et ne l'avait pas qui voulait,
vieux à dîner, jeunes dans leurs parties\,; libertin et débauché à
l'excès, gourmand à faire plaisir à table, et tout cela avec une vérité
et un sel qui ravissait. De cette façon, Boudin fut bientôt gâté.
D'ailleurs c'était un compagnon hardi, audacieux, qui se refusait peu de
choses, et qui n'en ménageait aucune quand il n'en craignait point les
retours ou quand il était poussé, et devenu fort familier, et de là fort
tôt très-impertinent. Initié de cette sorte dans le monde le plus
choisi, il se mit dans l'intrigue, et il sut et fut de bien des choses
secrètes et importantes de la cour.

Le maréchal de Villeroy, durant sa brillante faveur, se mit à le
plaisanter devant Monseigneur, un matin qu'il prenait médecine. Ses
grands airs déplurent à Boudin, qui répondit sec.~Le maréchal
continua\,; l'autre n'en fit pas à deux fois\,; il l'insolenta si net
que la compagnie en demeura confondue et le maréchal muet et outré.
Monseigneur, qui n'aimait pas le maréchal et qui se divertissait de son
médecin, fort bien avec lui et avec tout ce qui l'environnait, ne dit
mot. Après un peu de silence, le maréchal s'en alla, et Monseigneur se
mit à rire. L'histoire courut incontinent et il n'en fut autre chose.

Quoique Boudin aimât son métier, il s'y rouilla tout à fait parce qu'il
ne prenait plus la peine de voir les malades\,; mais sa curiosité pour
toutes sortes de remèdes et de secrets ne l'abandonna point. Il était
sur cela de la meilleure foi du monde, et tombait sur la Faculté qui
n'en veut point, et qui laisse mourir les gens dans ses règles. Il
aimait la chimie, il y était savant et aussi bon artiste, mais il alla
plus loin, il souffla. Il se mit dans la tête que la pierre philosophale
n'était pas impossible à trouver, et avec toute sa science et son esprit
il y fut cent fois dupé. Il lui en coûta beaucoup d'argent, et quoiqu'il
l'aimât beaucoup, rien ne lui coûtait pour cela, et il quittait les
parties et les meilleures compagnies pour ses alambics et pour les
fripons qui l'escroquaient. Mille fois attrapé, mille autres il s'y
laissait reprendre. Il s'en moquait lui-même et de ses frayeurs, car il
avait peur de tout et en faisait les contes les plus comiques. Ce
faiseur d'or-ci l'amusa et le trompa enfin comme les autres, et lui
coûta bien de l'argent qu'il regretta fort, car il ne négligeait pour en
amasser aucun des moyens que sa faveur lui pouvait fournir. Seigneurs et
ministres le comptaient et le ménageaient comme un homme fort dangereux,
et lui aussi, pourvu qu'il ne fût pas poussé, connaissoit à qui il avait
affaire, et ne laissait pas de se ménager aussi avec eux. Il tenait fort
à la cabale de Meudon et assez à celle des seigneurs.

Dès le commencement de décembre, le roi déclara qu'il voulait qu'il y
eût à Versailles des comédies et des appartements, même lorsque
Monseigneur serait à Meudon, contre l'ordinaire. Il crut apparemment
devoir tenir sa cour en divertissements pour cacher mieux au dehors, et
au dedans s'il l'eût pu, le désordre et l'extrémité des affaires. La
même raison fit qu'on ouvrit de bonne heure le carnaval, et qu'il y eut
tout l'hiver force bals à la cour de toutes les sortes, où les femmes
des ministres en donnèrent de fort magnifiques, et comme des espèces de
fêtes, à M\textsuperscript{me} la duchesse de Bourgogne et à toute la
cour\,; mais Paris n'en demeura pas moins triste, ni les provinces moins
désolées.

\hypertarget{chapitre-iii.}{%
\chapter{CHAPITRE III.}\label{chapitre-iii.}}

1711

~

{\textsc{Prince de Conti, Médavy, du Bourg, Albergotti, Goesbriant reçus
chevaliers de l'ordre.}} {\textsc{- Singularités sur le prince de
Conti.}} {\textsc{- Goesbriant gouverneur de Verdun.}} {\textsc{-
Mariage de Châtillon avec une fille de Voysin.}} {\textsc{- Électeur de
Cologne, à Paris et à la cour, dit la messe à M\textsuperscript{me} la
duchesse de Bourgogne.}} {\textsc{- Son étrange poisson d'avril.}}
{\textsc{- Mort de l'électeur de Trêves.}} {\textsc{- La Porte déclare
la guerre à la Russie.}} {\textsc{- Nangis colonel du régiment du roi.}}
{\textsc{- Mort, famille et caractère de Feuquières.}} {\textsc{-
Réflexion sur les vilains.}} {\textsc{- Mort et caractère d'Estrades\,;
sa naissance.}} {\textsc{- Prétention et procès de d'Antin sur la
dignité de duc et pair d'Épernon.}} {\textsc{- D'Antin obtient
permission du roi d'intenter son procès.}} {\textsc{- Ruse et artifice
de son discours.}} {\textsc{- Appartement du roi à Marly.}} {\textsc{-
Ferme et nombreuse résolution de défense.}} {\textsc{- Avis sensé et
hardi d'Harcourt.}} {\textsc{- Causes de fermeté.}} {\textsc{- Mesures
prises.}} {\textsc{- Je refuse la direction de l'affaire, dont je fais
charger les ducs de Charost et d'Humières.}} {\textsc{- Opposition à
d'Antin signée.}} {\textsc{- Étrange procédé du duc de Mortemart.}}
{\textsc{- Souplesse de d'Antin.}} {\textsc{- Partialité du roi pour
d'Antin, inutile.}} {\textsc{- Misérable procédé de La Feuillade.}}
{\textsc{- Ducs dyscoles.}} {\textsc{- Aiguillon.}} {\textsc{- Le roi
fait déclarer son impartialité au parlement.}} {\textsc{- Inquiétude
singulière du duc de Beauvilliers à la réception du duc de Saint-Aignan,
son frère.}}

~

Cette année commença par la cérémonie de faire chevaliers de l'ordre M.
le prince de Conti, Médavy et du Bourg, longtemps depuis maréchaux de
France, Albergotti et Goesbriant.

M. le prince de Conti n'avait pas quinze ans. M\textsuperscript{me} sa
mère ne laissait pas de demander l'ordre pour lui depuis longtemps avec
le dernier empressement. L'âge des princes du sang pour l'avoir est
vingt-cinq ans\,; mais le roi, qui l'avait donné au comte de Toulouse
avant quatorze ans, ne sut que répondre à cet exemple que M. du Maine
fit valoir, dans la liaison intime où les affaires de la succession de
M. le Prince l'avaient mis avec M\textsuperscript{me} la princesse de
Conti. Aussi, moyennant les bâtards qui peu à peu renversèrent tout et
défigurèrent tout, les princes du sang eurent l'ordre sans âge comme les
fils de France, c'est-à-dire que, les fils de la couronne et ceux de
l'adultère y étant traités pour l'âge en toute égalité, les princes du
sang ne purent demeurer exclus du même avantage.

La présentation de M. le prince de Conti fut une autre nouveauté tout
aussi étrange. Les parrains doivent être de même rang que le présenté.
Lorsque les chevaliers manquent, comme en 1661 et en 1688, on n'y
regarde point par l'impossibilité, et les fils de France sont parrains
indifféremment de tous les chevaliers novices, à leur tour\,; mais quand
il y a des chevaliers suffisamment on revient à la règle toujours
observée. C'était donc à deux princes du sang à présenter le prince de
Conti, mais il n'y avait de prince du sang que M. le Duc qui fût
chevalier de l'ordre. La raison voulait donc que, pour le second
parrain, on en approchât au plus près, et que M. du Maine, ou, si sa
jambe boiteuse l'en empêchait, le comte de Toulouse le fût, puisqu'il ne
leur manquait rien, nulle part en France, du rang de prince du sang que
des bagatelles au parlement imperceptibles, et que les enfants mêmes de
M. du Maine y étaient pareillement montés. Néanmoins, avec la pique
d'entre M\textsuperscript{me} la Duchesse et M. du Maine, qui était dès
lors très-vive, sur la succession de M. le Prince, le roi hésita à
coupler M. du Maine avec M. le Duc. On pouvait, pour honorer les princes
du sang, coupler M. le Duc avec M. le duc d'Orléans\,; mais le rang de
petit-fils de France, si récent et si distingué de celui des princes du
sang, s'accommoda encore moins de cela que M. le Duc de M. du Maine.
Pour couper court, on remonta au faîte, afin que tout y fût sans
proportion\,; on ne s'arrêta point aux fils de France, quoiqu'il n'y en
pût avoir d'un prince du sang avec eux, et la présentation se fit par
Monseigneur et Mgr le duc de Bourgogne.

Les quatre autres, on a vu à quelle occasion ils furent nommés, et
jusqu'à quel point la décoration de la cour, des plus hautes dignités,
de la première naissance, devint de plus en plus, depuis Louvois et sa
promotion de 1688, récompense militaire. Les deux premiers portaient
l'ordre depuis longtemps jusqu'à ce qu'ils pussent être reçus. À cette
occasion, ils furent mandés pour l'être\,: l'un de Strasbourg, où il
commandait sur toute la frontière du Rhin\,; l'autre de Grenoble, où il
commandait sur toute la frontière de Savoie. Les deux autres venaient
d'être nommés, et ne portèrent l'ordre qu'après avoir été reçus. Les
deux premiers retournèrent bientôt après à leur commandement\,; et
Goesbriant s'en alla commander à Saint-Omer. Le roi lui donna une
pension de vingt mille livres en attendant le premier gouvernement
vacant. C'était bien le moins pour le gendre de celui qui les payait.
Goesbriant n'attendit pas longtemps le gouvernement de Verdun, que la
mort de Feuquières lui procura.

Voysin maria sa seconde fille au comte de Châtillon, fils et neveu des
deux premiers gentilshommes de la chambre de Monsieur et de M. le duc
d'Orléans, qui sûrement n'auraient pas cru à son horoscope, si elle leur
eût dit la fortune dans laquelle il est aujourd'hui, et que son oncle,
le favori de Monsieur, a eu le loisir de voir quelques années avant sa
mort à quatre-vingt-sept ou huit ans, retiré depuis longtemps dans sa
province. Voysin, au lieu des deux cent mille livres que le roi, avant
cette dernière guerre, donnait aux filles de ses ministres, eut, comme
ils ont eu depuis, dix mille livres de pension pour sa fille.

L'électeur de Cologne, qui était venu de Valenciennes voir l'électeur de
Bavière à Compiègne, arriva à Paris les deux ou trois premiers jours de
cette année. Il eut incontinent après une audience du roi incognito, et
alla de même tout de suite chez M\textsuperscript{me} la duchesse de
Bourgogne, où Mgr le duc de Bourgogne se trouva. L'électeur s'amusa
quelques semaines à Paris, et vint après dîner à Meudon. Monseigneur se
mit à table dans son fauteuil à sa place ordinaire, sans cadenas, parce
qu'à Meudon il n'en avait jamais, et comme à l'ordinaire une serviette
plissée sur la nappe sous son couvert, et servi par du Mont, avec une
soucoupe pour boire. L'électeur de Cologne se mit vis-à-vis de
Monseigneur, parmi les courtisans, sur un siége pareil à eux\,; et cette
place vis-à-vis de Monseigneur n'était point celle des princes du sang,
ni distinguée en rien. Il n'eut point de serviette sous son couvert, ni
de couvert distingué, mais fut servi par un officier de la bouche, et
sans soucoupe pour boire, comme tous le-autres courtisans. Il fut par
toute la maison avec Monseigneur, qui aux portes étroites passait devant
lui sans aucun compliment, et l'électeur s'arrêtait et se rangeait avec
un air de respect, et parlant à lui l'appela toujours Monseigneur, usage
qui avait tellement prévalu que le roi ne lui parlait jamais autrement,
et que, parlant de lui, il le nommait plus ordinairement Monseigneur
qu'il ne disait mon fils\,; mais M. le Dauphin, il ne le disait jamais.

Deux jours après, qui fut le mardi 3 février, il vit l'électeur dans son
cabinet, lequel en sortant de là s'en alla dire la messe à
M\textsuperscript{me} la duchesse de Bourgogne. Il aimait à la dire, et
basse et haute, et à faire toutes sortes de fonctions. Il avait fort
prié M\textsuperscript{me} la duchesse de Bourgogne de l'entendre. Il la
dit au grand autel de la chapelle, basse, et comme un évêque ordinaire.
M\textsuperscript{me} la duchesse de Bourgogne-était en haut dans la
tribune, pour éviter le corporal que le prêtre lui apportait à baiser à
la fin de la messe quand elle était en bas, et pour que cette messe eût
l'air d'une messe ordinaire\,; mais l'électeur la salua profondément en
entrant et en sortant de l'autel, et s'inclina comme un chapelain
ordinaire aux \emph{Dominus vobiscum} et à la bénidiction. En entrant et
en sortant de l'autel, M\textsuperscript{me} la duchesse de Bourgogne
reçut debout son inclination profonde, et lui fit une révérence fort
marquée. Madame fut outrée de cette messe, et se garda bien de s'y
trouver. L'électeur en effet aurait pu s'en passer\,; mais non-seulement
ce fut lui qui la proposa, mais qui en pressa, et qui témoigna que
M\textsuperscript{me} la duchesse de Bourgogne le désobligerait si elle
l'en refusait. Il n'y avait point de cérémonies qu'il n'aimât à faire.
Enfin il aimait même à prêcher, et on peut juger comment il prêchait. Il
s'avisa un premier jour d'avril de monter en chaire\,; il y avait envoyé
inviter tout ce qui était à Valenciennes, et l'église était toute
remplie. L'électeur parut en chaire, regarda la compagnie de tous côtés,
puis tout à coup se prit à crier\,: «\,Poisson d'avril\,! poisson
d'avril\,!» et sa musique avec force trompettes et timbales à lui
répondre. Lui cependant fit le plongeon et s'en alla. Voilà des
plaisanteries allemandes, et de prince, dont l'assistance, qui en rit
fort, ne laissa pas bien d'être étonnée.

Après avoir dit la messe à M\textsuperscript{me} la duchesse de
Bourgogne, il dîna chez le duc de Villeroy, et fut ensuite voir
M\textsuperscript{me} de Maintenoh à Saint-Cyr, qui lui donna
M\textsuperscript{me} de Dangeau pour le conduire à voir toutes les
classes de demoiselles, et l'accompagner par toute la maison. Il avait
pris congé du roi le matin, qui lui fit donner beaucoup d'argent et le
renvoya fort content. Deux jours après, il apprit la vacance d'un
canonicat de Liége, dont il était aussi évêque\,; il l'envoya offrir
galamment à M\textsuperscript{me} de Dangeau, pour le comte de
Lowenstein, son frère, chanoine de Cologne et, grand doyen de
Strasbourg, mort longtemps depuis évêque de Tournai\,; et le canonicat
fut accepté avec l'agrément du roi. L'électeur de Cologne s'en alla le 7
février à Compiègne, d'où il s'en retourna à Valenciennes.

On apprit quelques jours après la mort de l'électeur de Trèves. Ainsi le
frère de M. de Lorraine ne fut pas longtemps coadjuteur\,; et ces
chapitres de Mayence et de Trèves, si résolus, par l'exemple de celui de
Cologne, à se faire sages contre l'ambition des princes, et à n'en point
recevoir parmi eux, tombèrent dans le même inconvénient, Trèves dès lors
et Mayence ensuite, dont le coadjuteur était le grand maître de l'ordre
Teutonique, frère de l'électeur palatin et de l'impératrice douairière.

Le roi de Suède, de son asile de Bender, sut si bien remuer la Porte en
sa faveur qu'on sut par des Alleurs, qui avait succédé à Fériol dans
l'ambassade de Constantinople, que le Grand Seigneur déclarait la
guerre, et prétendait, avec une armée de trois cent mille Turcs,
Tartares ou Cosaques, chasser les Moscovites et les Saxons de Pologne,
et rétablir le roi de-Suède et le roi Stanislas. Cette nouvelle, qui
pouvait influer sur les affaires de l'empereur, fit un peu de
soulagement.

Le roi, lassé de voir son régiment d'infanterie dans un assez mauvais
état, donna le gouvernement de Landrecies à du Barail et le fit maréchal
de camp. Il était lieutenant-colonel lorsque le roi l'ôta, comme on l'a
dit, à Surville, et le donna à du Barail à qui il le reprit et le donna
à Nangis. Cela parut un grand commencement de fortune à tous les détails
que le colonel de ce régiment avait fréquemment tête à tête avec le roi,
qui se croyait le colonel particulier de ce régiment, avec le même goût
qu'un jeune homme qui sort des mousquetaires.

Feuquières mourut en ce temps-ci. Il était ancien lieutenant général,
d'une grande et froide valeur, de beaucoup plus d'esprit qu'on n'en a
d'ordinaire, orné et instruit, et d'une science à la guerre qui l'aurait
porté à tout, pour peu que sa méchanceté suprême lui eût permis de
cacher au moins un peu qu'il n'avait ni cœur ni âme. On en a vu quelques
traits ici répandus, dont sa vie ne fut qu'un tissu. C'était un homme
qui ne servait jamais dans une armée qu'à dessein de la commander, de
s'emparer du général, de s'approprier tout, de se jouer de tous les
officiers généraux et particuliers\,; et, comme il ne trouva point de
général d'armée qui s'accommodât de son joug, il devenait son ennemi, et
encore celui de l'État, en lui faisant, tant qu'il pouvait, manquer
toutes ses entreprises. On ferait un livre de ces sortes de crimes\,;
aussi ne servait-il plus, il y avait très-longtemps, parce que aucun
général ne le voulait dans son armée, pour en avoir tous tâté. Il a
laissé des Mémoires sur la guerre, qui seraient un chef-d'œuvre en ce
genre, et savamment, clairement, précisément et noblement écrits, si,
comme un chien enragé, il n'avait pas déchiré, et souvent mal à propos,
tous les généraux sous lesquels il a servi. Aussi mourut-il pauvre, sans
récompense et sans amis. Il n'avait qu'une pension de six mille livres,
que le roi laissa à sa famille. Leur nom est Pas, bonne et ancienne
noblesse de Picardie. Son père fut tué approchant fort du bâton, vers
lequel il avait rapidement et vertueusement couru\,; et son grand-père
s'était signalé dans les plus importantes négociations de son temps, sur
les traces duquel Rebenac, frère de celui-ci, commençait à marcher quand
il mourut, et avec cela ils n'ont jamais pu rien obtenir de la fortune
que le gouvernement de Verdun, qui fut donné à Goesbriant. Son fils
mourut bientôt après lui sans enfants\,; et sa fille unique, dont la
mère était fille du marquis d'Hocquincourt, chevalier de l'ordre, fils
du maréchal, laquelle hérita de tous ses frères, porta tous ses biens à
un Seiglière, dont la vie honteuse a même déshonoré jusqu'à la bassesse
de sa naissance, et dont la mère, fille du marquis de Soyecourt,
chevalier de l'ordre et grand veneur, avait aussi hérité de ses deux
frères, tués sans alliance tous deux à la bataille de Pleurus\,; et
voilà comme on donne des filles de qualité à des vilains, parce qu'ils
les prennent pour rien, desquelles après ils ont tous les biens de leurs
maisons\,! Ce fameux Soyecourt est mort fugitif à Venise, sa femme
bientôt après\,; et leur fils a eu un régiment, tandis que les gens les
plus qualifiés n'en peuvent obtenir du cardinal Fleury\,: \emph{Similis
simili gaudet}. Cela se retrouve en tout. Il n'y a plus d'Hocquincourt,
qui est Monchy, ni de Pas. Rebenac n'a laissé que M\textsuperscript{me}
de Souvré, mascarade de Tellier\,; et leur troisième frère est mort fort
vieux, sans enfants de la fille de Mignard, ce peintre fameux qui, pour
sa beauté, l'a peinte en plusieurs endroits de la galerie de Versailles
et dans plusieurs autres de ses ouvrages.

Estrades mourut presque en même temps. Il était fils aîné de ce maréchal
d'Estrades, si capable dans son métier, et si célèbre par le nombre,
l'importance et le succès de ses négociations, et qui mourut, en 1686,
en février, à soixante-dix-neuf ans, gouverneur de M. le duc de
Chartres. Il venait de conclure et de signer la paix à Nimègue en 1678.

Il dépêcha ce fils au roi sur-le-champ. Il s'amusa à Bruxelles à une
maîtresse, et donna ainsi le temps au prince d'Orange, qui était au
désespoir d'une paix qui mettait des bornes à sa puissance en Hollande,
de donner la bataille de Saint-Denis à M. de Luxembourg, qui ne
s'attendait à rien moins, comptant la paix faite, et qui en reçut la
nouvelle du roi le lendemain. Le prince d'Orange l'avait dans sa poche
avant le combat, mais il espéra la rompre par une victoire, et s'il ne
la remportait pas, profiter de la paix.

Estrades fit dire vrai encore à ce proverbe\,: \emph{Filii heroum noxæ}.
Il mena toujours une vie, obscure, avec peu de commerce, peu d'amis et
moins de considération. Celle de son père, qui sut faire le marché si
important du secours maritime des États généraux pour prendre Dunkerque,
dont il eut le gouvernement après le maréchal de Rantzau, le lui valut
après lui, et la mairie perpétuelle de Bordeaux. Son fils, devenu
lieutenant général, voulut bien accompagner les enfants de M. du Maine
en Hongrie, où il fut tué devant Belgrade en 1717, et a laissé des
enfants qui n'ont pas percé dans le monde.

Le maréchal d'Estrades avait deux fils qui valaient mieux que l'aîné. Le
chevalier d'Estrades, attaché à M. le duc de Chartres d'alors, qui fut
tué à la tête de son régiment à Steinkerque en 1692, et qui serait
devenu digne de son père\,; et l'abbé d'Estrades, dont il sera parlé
ailleurs.

On ne connaît rien au delà du grand-père du maréchal d'Estrades. Son
père, qui était brave et sage, et qui avait servi Henri IV contre la
Ligue, fut successivement gouverneur du comte de Moret, bâtard d'Henri
IV, et des ducs de Mercœur et de Beaufort, enfin des ducs de Nemours, de
Guise et d'Aumale. La mère de celui-là était fille d'un conseiller au
parlement de Bordeaux et d'une Jeanne, dite de Mendoze, qui était de
race juive d'Espagne. On a parlé ailleurs de la ridicule coutume de ce
pays-là, de donner aux juifs qui se convertissent, et dont on est
parrain, non-seulement son nom de baptême comme partout, mais encore son
nom de maison et de ses armes, qui deviennent le nom et les armes du
juif filleul et de sa postérité. Le père ou le grand-père de cette
Jeanne Mendoze eut ainsi le nom et les armes de Mendoze de son parrain,
et M. d'Estrades en décora ses armes et sa postérité après lui. Il y a
d'excellents Mémoires du maréchal d'Estrades.

Maintenant il est temps de venir au procès que d'Antin intenta sur des
chimères aussi folles que rances de l'ancienne duché-pairie d'Épernon,
et aux adresses incomparables par lesquelles il sembla faire grâce au
roi et aux ducs de le devenir, et à l'édit qui, à cette occasion, sous
prétexte de grâces et de bienfaits, donna comme le dernier coup à une
dignité que le roi voulut sans cesse abattre, et dont le sort était d'en
recevoir des coups de massue à chaque occasion de procès de préséance
que des chimères et l'ambition intentaient aux ducs. Ce récit, qui ne
saurait être court, et qui pourra même avoir des parties ennuyeuses,
sert si fort à peindre les ruses d'un courtisan, la jalousie des autres,
les artifices des bâtards, un intérieur de cour et de seigneurs peu
connu, et à montrer à découvert les pierres d'attente et la préparation
de grands événements de cour et d'intérieur d'État, qu'il ne sera pas un
des moins curieux de ce genre.

On a vu, lors du procès de préséance de feu M. de Luxembourg, la
tentative que firent les Estrées en faveur de M\textsuperscript{lle} de
Rouillac, pour ce duché d'Épernon en sa personne, et que le comte
d'Estrées devait épouser en cas de succès, et qui fut depuis gendre de
M. de Noailles. Ce coup manqué, feu M. de Montespan avait passé avec
elle tous les actes nécessaires pour succéder après elle à sa terre
d'Epernon et à ses prétentions, et n'avait rien oublié pour les tenir
secrets, quoiqu'il n'eût pu se tenir d'essayer de prendre dans ses
terres de Guyenne, où il demeurait, le nom de duc d'Épernon, et de s'y
faire moquer de lui. Il était mort dans ses idées, et d'Antin s'en était
toujours nourri.

Arrivé enfin à la faveur et aux privances avec le funeste appui de la
coupable fécondité de sa mère, il sentit ses forces, et il se crut en
état de se faire écouter du roi, et craindre de ceux qu'il avait à
attaquer. Il choisit Marly comme un lieu qui lui était encore plus
favorable. Il épia son moment dans les cabinets, et le trouva le samedi
10 janvier de cette année. Là il dit au roi que, comblé de ses grâces,
il lui siérait mal de l'importuner pour de nouvelles, mais qu'étant le
plus juste des rois, il croyait devoir à Sa Majesté et à soi-même de lui
représenter qu'il souffrait une injustice de sa part, qu'il ne pouvait
se persuader qui fût dans sa mémoire, puisque, comblé de ses bienfaits,
il ne pouvait croire qu'il la voulût faire au plus inconnu de ses
sujets. Après ce bel exorde, il dit au roi que sa coutume était de
laisser à chacun le libre cours de la justice, et entre particuliers de
ne se mêler point de leurs affaires\,; que néanmoins il en avait une où
il allait de toute sa fortune, qui ne touchait le roi en rien, et qui
était arrêtée par sa seule autorité\,; que cette affaire était la
prétention à la dignité de duc et pair d'Épernon, que le dernier marquis
de Rouillac avait poursuivie après son père, et que le crédit des ducs
prêts à la perdre avait suspendue par un coup d'autorité du roi, que
depuis il avait eu la bonté de permettre à M\textsuperscript{lle} de
Rouillac de reprendre cette instance dont le succès aurait fait son
établissement\,; que les difficultés toujours plus fâcheuses à ce sexe,
et la grande piété de M\textsuperscript{lle} de Rouillac lui avaient
fait prendre le parti d'un saint repos, dans lequel elle était morte\,:
qu'il avait recueilli ses droits avec sa succession dans des temps où il
n'avait pas trop osé demander justice\,; que maintenant qu'il se croyait
assez heureux pour que ces temps fussent changés, il ne demandait pour
toute grâce que celle qu'il ne refusait à personne, et de lui permettre
de faire valoir son droit\,; qu'il ne serait importuné de rien\,; que ce
serait un procès à l'ordinaire à la grand'chambre\,; qu'il avait
extrêmement examiné et fait examiner la question\,; qu'elle était
indubitable, et que de plus, quoiqu'il dût s'attendre à des oppositions,
il tâcherait de mériter, par sa conduite, de s'en attirer une dont il
n'eût pas lieu de se plaindre\,; que d'ailleurs c'était si peu de chose
pour chacun des ducs de reculer d'un pas, et pour lui une si grande
fortune que de se trouver leur confrère, et du même coup à leur tête,
qu'il ne savait si beaucoup s'opposeraient bien sérieusement à lui\,;
que par là devenu duc et pair sans grâce, personne ne serait en droit
d'exemple d'importuner Sa Majesté\,; qu'il espérait assez de ses bontés
pour oser se flatter qu'il ne serait point fâché de le voir en ce rang,
sans qu'il lui en coûtât rien.

C'était là toucher le roi par l'endroit sensible, après lui avoir menti
de point en point sur tous les faits qu'il avait avancés, et avoir mis
dans son discours tout l'art du plus délié et du plus expérimenté
courtisan. Il était vrai que, le roi subjugué par lui, il était hors de
portée du refus. Mais la prostitution des dignités et l'outrecuidance
française y portait des gens que le roi ne voulait ni faire ni
mécontenter. Mais la raison intime, et que d'Antin avait bien sentie,
était la jalousie du roi contre ses favoris, dont il redoutait autant
l'apparence d'être gouverné, comme il leur en abandonnait la réalité de
bonne grâce. La faveur si éclatante de d'Antin n'avait pas besoin d'un
nouvel accroissement aux yeux du monde\,; et il sut mettre le roi si
avant dans ses intérêts, par ce tour adroit et si ajusté à son goût, que
la partialité du roi eut peine à demeurer en quelques bornes. Parler
donc en ce sens et obtenir ne fut qu'une même chose, laquelle fut plus
tôt faite qu'éventée.

Le lendemain dimanche j'entrai dans le salon vers l'heure que le roi
allait sortir pour la messe. Je m'approchai d'abord d'une des cheminées,
où La Vrillière se chauffait avec je ne sais plus qui. À peine les
eus-je joints que La Vrillière m'apprit la nouvelle. Je baissai la tête
et haussai les épaules. Il me demanda ce que j'en pensais. Je lui dis
que je croyais que le triomphe ne coûterait guère sur des victimes comme
nous. Un moment après, je vis de l'autre côté du salon les ducs de
Villeroy, de Berwick et de La Rocheguyon, qui parlaient tous trois
ensemble, et qui dès qu'ils m'aperçurent m'appelèrent. Non-seulement ils
savaient la chose, mais tout le propos de d'Antin que j'ai rapporté.

Le roi, à Marly, n'avait que deux cabinets, encore le second était-il
retranché en deux pour une chaise percée, dont le lieu était assez
grand, aux dépens du reste du cabinet qui lui donnait le jour, pour que
ce fût là que le roi se tînt après son souper avec sa famille. Ainsi les
valets intérieurs dont ces cabinets étaient pleins, et dont les portes
étaient toujours toutes ouvertes, voyaient tout ce qui s'y passait, et
entendaient tout. Bloin, qui n'aimait pas d'Antin, n'avait pas perdu un
mot de son discours, et l'avait rendu aux ducs de Villeroy et de La
Rocheguyon ses intimes, et qui soupaient chez lui presque tous les
soirs.

Dès que je fus à eux, ils me le rendirent et me demandèrent mon avis. Je
leur répondis comme je venais de faire à la Vrillière. Ma surprise fut
grande de les voir tous trois s'en irriter, et me demander si j'avais
résolu de ne me point défendre. Je dis languissamment que je ferais
comme les autres\,; et dans la vérité c'était bien ma résolution de
laisser tout aller, par les expériences que j'avais de ces choses et ce
qui m'en était arrivé, et qui se trouve ici en plusieurs endroits. Mais
je trouvai une vigueur qui ranima un peu la mienne, mais sans me faire
sortir des bornes que je crus ne devoir pas outre-passer.

Ils me dirent qu'ils venaient de parler aux maréchaux de Boufflers et
d'Harcourt, qui pensaient comme eux à une juste et verte défense\,; que
d'Antin, sorti exprès des cabinets, leur venait de dire ce qu'il avait
obtenu\,; qu'il y avait ajouté des respects infinis, entre autres que,
s'il lui était possible de détacher l'ancienneté de sa prétention, il
s'estimerait trop honoré d'être le dernier de nous, et toutes sortes de
déférences et de beaux propos sur les procédés dans l'affaire, que je
supprime ici\,; qu'ils lui avaient répondu, avec la politesse que
demandait son compliment, mais avec la fermeté la plus nette, sur la
défense, qu'ils y étaient résolus\,; qu'il y aurait de la honte à
marquer de la crainte de sa faveur et de la défiance du droit\,; que
j'étais celui qui entendait le mieux ces sortes d'affaires, pour avoir
défendu celle contre M. de Luxembourg, et empêché celle d'Aiguillon\,;
que, ne doutant pas de mon courage, ils venaient à moi me prier de me
joindre à eux, et de leur dire ce qu'il y avait à faire. Ils ajoutèrent
qu'il ne fallait pas douter que le roi ne fût pour d'Antin\,; que
l'espérance de celui-ci était qu'il ne se trouverait personne qui osât
le traverser, chose dont sûrement le roi serait bien aise, mais que ce
serait la dernière lâcheté\,; qu'il fallait tous nous bien entendre et
marcher d'un pas égal\,; que, cela fait, le roi n'oserait nous en
montrer du mécontentement, ni, pour d'Antin seul, fâcher tout ce qui
l'environnait dans les principales charges, qui, réunis, feraient au
favori la moitié de la peur\,; qu'il fallait commencer par rassembler ce
qui était à Marly, et que cet exemple serait puissant sur les autres. La
Rocheguyon surtout insista que céder serait abandonner la cause pendante
contre M. de Luxembourg, ouvrir la porte à toutes les prétentions du
monde\,; et prit avidement ce hameçon de l'affaire de M. de Luxembourg
que je lâchai froidement dans le discours. Ils insistèrent donc vivement
pour savoir mon sentiment, et surtout comment il s'y fallait prendre
pour se bien et fermement défendre.

À ce qu'ils venaient de dire sur le roi, je sentis qu'ils parlaient de
bonne foi sur tout le reste. Je leur dis donc, mais sans sortir du
flegme, que j'étais bien aise de les voir dans des sentiments que
l'expérience de toute ma vie les devait empêcher de douter qu'ils ne
fussent les miens\,; mais que je leur avouais aussi que mon expérience
particulière me rendait leur ardeur nécessaire pour rallumer la
mienne\,; que, puisqu'ils voulaient savoir ce qu'il fallait faire, et ne
pas perdre un moment, la première démarche nécessaire était de signer
une opposition à ce que nul ne fût reçu duc et pair à la dignité
d'Épernon, et de la faire signifier au procureur général et au greffier
en chef du parlement, moyennant quoi il n'y avait plus de surprise à
craindre\,; la seconde, de nous former un conseil, que le meilleur, à
mon avis, était de prendre ce qui restait du nôtre contre M. de
Luxembourg\,; et que je m'offrais de pourvoir à ces deux préliminaires.
Ils m'en conjurèrent avec mille protestations de courage et d'union.

Aussitôt j'exécutai par une lettre chez moi l'engagement que je venais
de prendre. Rentrant au château, je trouvai M. de Beauvilliers, qui se
jeta dans mon oreille, et me dit de ne me point séparer des autres ducs,
de faire même tout ce que je pourrais contre d'Antin, mais de me
contenir dans l'extérieur en des mesures d'honnêteté et de modération,
et qu'il en avait dit autant à son frère et à son gendre. C'était bien
mon projet\,; mais je ne laissai pas d'être surpris et encouragé de cet
avis d'un homme si mesuré, surtout en ces sortes d'affaires.

Arrivant dans le salon, les trois qui m'avaient parlé, et que j'y avais
laissés, m'avertirent de me trouver chez le maréchal de Boufflers dans
une demi-heure, où ils se devaient rendre. Les ducs de Tresmes et
d'Harcourt y vinrent. Je leur rendis compte de ce que je venais de
faire, et je les réjouis fort de leur apprendre que les ducs de
Mortemart et de Saint-Aignan seraient des nôtres, de l'aveu du duc de
Beauvilliers, d'autant que le duc de Mortemart avait répondu au duc de
Villeroy qui lui avait parlé, à ce qu'il nous dit là, qu'il consulterait
son beau-père. Nous raisonnâmes sur une liste de ducs sur lesquels on
pourrait compter ou non. Chacun se chargea d'écrire à ses amis, excepté
à ceux qui avaient des duchés femelles, quoique l'exemple de M. de
Richelieu contre M. de Luxembourg les dût rassurer. On parla ensuite de
notre conduite de cour.

Il fut résolu, M. d'Harcourt menant la parole, que nous payerions
d'Antin de compliments\,; que nous déclarerions notre union et notre
attachement à notre défense\,; que nous ne ferions pas semblant de nous
douter que le roi, quoi qu'il fît, pût souhaiter contre nous, afin de
l'obliger par cette surdité volontaire à des démarches plus marquées,
que nous savions bien que d'Antin avec toute sa faveur n'arracherait pas
contre des personnes, desquelles plusieurs l'approchaient de si près
dans ses affaires, ou autour de sa personne, outre sa conduite ordinaire
en ces sortes d'affaires de se piquer de neutralité. On discuta ensuite
les démarches du palais. Il fut question de donner une forme à la
conduite de l'affaire.

Je rendis compte de celle du procès contre M. de Luxembourg. Il fut jugé
à propos de l'imiter en tout pour celui-ci. M. d'Harcourt appuya fort
sur la nécessité d'en choisir un ou deux parmi nous qui eussent la
direction de l'affaire, qui y donnassent le mouvement par leur soin et
leur présence\,; et qui eussent le pouvoir d'agir et de signer pour
tous, quand il serait nécessaire, pour ne point perdre de temps aux
occasions pressées\,; puis proposa de me prier de vouloir bien m'en
charger. Je n'avais pas eu peine à reconnaître que la chose avait été
agitée entre eux, auparavant l'assemblée, et résolue. Tous applaudirent,
et joignirent à l'invitation la plus empressée toute l'adresse, et la
plus flatteuse politesse pour piquer mon courage. Je répondis avec
modestie, bien résolu à ne pas accepter un emploi dont j'avais bien
prévu la nécessité et les inconvénients, et qu'il me serait présenté. Je
fus pressé avec éloquence. Je représentai que mon assiduité à la cour ne
m'en pouvait permettre assez à Paris pour suivre l'affaire d'aussi près
qu'il était nécessaire. Comme je vis que rien ne les satisfaisait, je
leur dis que ces affaires communes ne m'avaient pas personnellement
assez bien réussi pour m'engager de nouveau à les conduire\,; que,
d'ailleurs, les raisons particulières qui m'avaient plus d'une fois
commis avec M. d'Antin ne me permettaient pas de m'exposer
volontairement à une occasion nouvelle\,; que je les suppliais de
n'imputer point mes excuses à paresse ni à mollesse, mais à une
nécessité qui ne pouvait se surmonter. Nous nous séparâmes de la sorte,
contents de nos mesures prises en si peu de moments, mais ces messieurs
fort peu de mon refus à travers toutes les honnêtetés possibles.

Tant de fermeté, dans un temps de si misérable faiblesse, et parmi des
courtisans si rampants qui voyaient clairement le roi contre eux, eut
des raisons que dans ma surprise je découvris sans peine. Les ducs de
Villeroy et de La Rocheguyon avaient de tout temps vécu dans un parfait
mépris pour d'Antin, et si marqué, que d'Antin, dont la politique avait
toujours été de ne s'aliéner personne, s'en était souvent plaint à eux
par des tiers, et quelquefois par lui-même\,; et comme ç'avait été sans
succès, il s'en était formé une inimitié, même assez peu voilée, que la
jalousie de la cour intérieure de Monseigneur avait fomentée, et que la
faveur déclarée de d'Antin auprès du roi avait comblée dans les deux
beaux-frères, qui avant de l'être, et de toute leur vie, n'avaient
jamais été qu'un, et M. de Liancourt avec eux. Harcourt extrêmement leur
ami, et plus encore du premier écuyer qui haïssait sournaisement
d'Antin, et qui de plus ne lui pouvait pardonner les bâtiments sur
lesquels il avait eu lieu de compter, avait épousé leurs sentiments avec
d'autant plus de facilité qu'il regardait d'Antin comme un dangereux
rival pour le conseil, et comme un obstacle à entrer. Bouffiers, si
droit, et si touché de la dignité, n'avait pas oublié les mauvais
offices de d'Antin lors de la bataille de Malplaquet\,; et Villars lié à
d'Antin, par la raison contraire, n'osa jamais abandonner une communauté
d'intérêts qui lui faisait un si prodigieux honneur. Tresmes, né noble,
je ne sais pas pourquoi, ayait de plus Harcourt pour boussole\,; et
Berwick fort anglais, ne pouvait souffrir l'interversion des rangs.

Notre conseil fut formé en vingt-quatre heures, et notre opposition
dressée me fut renvoyée\,: Il fut singulier que le hasard fit que celui
de d'Antin fut celui de M\textsuperscript{me} la Duchesse pour la
succession de M. le Prince, et le nôtre, le même qui lui fut opposé par
ses belles-sœurs. Je dis à ces messieurs, en arrivant pour la messe du
roi, que j'avais l'opposition. Le roi au sortir de sa messe étant entré
chez M\textsuperscript{me} de Maintenon, MM. de Tresmes et d'Harcourt
firent sortir tout ce qui se trouva dans l'antichambre, et en firent
fermer les portes. Là je rendis compte aux mêmes de la veille, de la
formation de notre conseil, et des mesures prises, et il fut arrêté
qu'on proposerait l'opposition à signer aux ducs qui étaient à Marly. On
y dansait, et le roi y avait mené pour cela de jeunes gens, entre autres
le duc de Brissac. Je fis observer qu'à son âge, sa signature de plus ou
de moins n'aurait pas grand poids, et qu'il embarrasserait fort au
contraire s'il s'avisait de consulter auparavant son oncle Desmarets, et
celui-ci le roi, et qu'après il refusât sa signature. Cela fit qu'on ne
lui en parla point.

On reprit après l'article, qui était demeuré indécis la veille, de la
conduite de l'affaire, dont je fus pressé de me charger, sans
comparaison plus fortement que je ne l'avais été. Plus j'y avais pensé
depuis vingt-quatre heures, plus je m'étais fortifié dans ma résolution,
mais de faire en sorte d'en tenir les rênes de derrière la tapisserie.
Ainsi, après avoir fait valoir les excuses que j'avais déjà apportées,
je leur dis que ce n'était pas pour refuser mon temps ni mes soins\,;
que je me rendrais même le plus souvent que je le pourrais aux
assemblées de notre conseil\,; mais que, ne pouvant me livrer à ce
qu'ils désiraient de moi, j'estimais qu'il y avait deux de nos confrères
très-capables d'y suppléer, et assez de mes amis pour vouloir bien user
de mes conseils dans le cours de l'emploi dont j'étais d'avis qu'ils
fussent priés de se charger\,; et je leur proposai les ducs de Charost
et d'Humières, par qui je comptais bien gouverner l'affaire comme si
j'en avais accepté le soin. J'ajoutai que, d'Antin attaquant tous les
ducs, les vérifiés n'avaient pas un moins juste sujet de défense que les
pairs\,; que les vérifiés se trouveraient flattés d'avoir part en la
direction de l'affaire\,; et, après avoir dit ce que je crus convenable
sur ceux que je proposais, je les assurai que, encore que M. d'Humières
fût l'ancien de M. de Charost, il lui céderait sans difficulté partout
en une cause de pairie. Ces raisons, et, s'il faut l'avouer, celle de
l'influence que j'aurais avec ces messieurs sur la conduite de
l'affaire, déterminèrent à s'y arrêter. Ils n'étaient ni l'un ni l'autre
à Marly\,; on remit à le leur proposer au retour à Versailles, et on
résolut de signer ce jour même l'opposition.

Elle fut datée de Paris, en faveur de ceux qui y étaient et qui la
voudraient signer le lendemain avant qu'elle fût signifiée, comme elle
le fut ce lendemain-là même à d'Aguesseau, procureur général, et au
greffier en chef du parlement. Ceux qui la signèrent furent\,: les ducs
de La Trémoille, Sully, Saint-Simon, Louvigny, Villeroy, Mortemart,
Tresmes, Aumont, Charost, Boufflers, Villars, Harcourt et Berwick,
pairs\,; La Rocheguyon, pour soi et pour M. de La Rochefoucauld pair et
aveugle\,; Humières et Lauzun vérifiés. On ne jugea pas à propos d'en
faire signer davantage pour en réserver en adjonction.

Je fus averti par le duc de Villeroy de me trouver le soir de ce même
jour chez le duc de La Rocheguyon, pour y discuter encore je ne sais
quoi. Comme j'y entrais on proposa d'attendre le duc de Mortemart. Je le
connaissois trop, depuis mon aventure avec lui sur M\textsuperscript{me}
de Soubise, pour parler de rien devant lui\,; je le dis à la compagnie,
avec ménagement toutefois pour le gendre du duc de Beauvilliers\,; et je
me contentai de les avertir que ce n'était pas un homme sûr.

La Rocheguyon et Villeroy qui pourtant en savait davantage là-dessus que
son beau-frère, traitèrent cela de fantaisie, et soutinrent que, tout
fou et léger qu'était Mortemart, il ne ferait rien de mal à propos dans
une affaire où il avait même intérêt, et dans laquelle il était entré de
bonne grâce. Là-dessus il entra. Ces messieurs lui firent signer
l'opposition, et la lui donnèrent pour la faire signer à Villars, et me
la remettre après, le soir même, dans le salon, sans qu'on pût s'en
apercevoir, et lui recommandèrent fortement le secret de l'opposition
même. Je me défendis de la reprendre en lieu si public. Toutefois cela
passa brusquement, et ils renvoyèrent aussitôt le duc de Mortemart, sous
prétexte de diligenter la signature dont il s'était chargé, et en effet
pour me laisser la parole libre. Quand nous eûmes achevé je retournai au
salon. Bientôt après j'y aperçus M. de Mortemart au milieu d'un tas de
jeunes gens, qui parlait d'un air fort sérieux à M. de Gondrin, fils
aîné de d'Antin. Je m'approchai doucement par derrière, j'entendis des
compliments et je me retirai. Un peu après le duc de Mortemart vint à
moi, son papier à la main, qui tout haut, en plein salon et devant tout
le monde, me dit qu'il n'avait pu trouver le maréchal de Villars, et
qu'il me le rendait. Le trait était complet. Nous ne voulions pas qu'il
parût d'autre mesure que de simple raisonnement entre nous, moins encore
que d'Antin sût qu'il y avait des opposants, quels, ni combien que par
la signification. Tout cela avait été bien expliqué au duc de Mortemart,
et le secret fort recommandé\,; et moi qui plus que nul des autres
craignais d'y paraître, je m'y vis affiché dans le salon, et tout auprès
du lansquenet. Je me battis en retraite, et le Mortemart après moi,
disant\,: «\,Tenez, tenez\,!» son papier à découvert en main, jusque
dans le petit salon de la Perspective plein de gens et de valets. Là je
le lui pris rudement sans lui dire un seul mot\,; je m'en allai chez
moi, et j'eus encore la peine de le faire signer à Villars ce même soir.

Une heure après, Gondrin donna au public notre opposition avec les
compliments que lui avait faits le duc de Mortemart. Le duc de Villeroy
en fut outré de colère plus que pas un de nous, avec plus de raison
qu'aucun, parce qu'il en avait davantage de se défier de lui après ce
qu'il en avait su de moi. Chacun de nous s'expliqua sur lui sans
ménagement\,; et il fut résolu de se défier de lui comme de d'Antin
même, et de l'exclure de toutes nos assemblées, en pas une desquelles
aussi il n'osa se présenter depuis, ni même s'informer de l'affaire.
D'Antin, de son côté, pouilla son fils d'importance d'avoir compromis
leur cousin, comme si la chose se fût passée tête à tête. Il apprit donc
par là qu'il y avait une opposition, et quoiqu'il ne pût savoir que le
petit nombre de ceux de Marly qui avaient signé, il ne laissa pas d'être
étonné que quelqu'un osât lui résister, et de trouver des charges et du
crédit déclarés contre lui. Ce n'était pas qu'il n'eût affecté de
publier que, s'il avait un fils honoré de cette dignité, il l'obligerait
à s'opposer à lui\,; mais le gascon parlait au plus loin de sa pensée.
Il jetait ce propos à tout événement comme un sentiment de douceur et
d'équité, pour voir comment il serait reçu dans le monde, et pour
décorer sa cause si la lâcheté se trouvait telle qu'il espérait par un
silence unanime, ou rompu seulement par un si petit nombre, et de
considération si légère, qu'il en pût encore plus triompher.

Ce début si peu attendu lui fit juger à propos de tâcher à ralentir ce
premier feu par des marques de partialité du roi, qui effrayassent et
qui empêchassent de pousser contre lui les mesures qu'il voyait prises.

Je fus pressé par mes amis de faire une honnêteté à d'Antin, à l'exemple
des autres, en même intérêt\,; j'eus peine à m'y rendre, mais je le fis.
Je n'ai point pénétré quel put être son objet, mais si j'eusse été le
favori il ne m'eût pas accablé de plus de respects ni de plus profonds,
et de remercîments plus excessifs de l'honnêteté que je lui voulais bien
faire\,; non content de cela, il vint chez moi les redoubler quoique je
n'eusse point été chez lui. Il affecta de publier ma politesse à son
égard, et la satisfaction qu'il en ressentait\,; il s'en vanta au roi,
et cela me revint aussitôt\,: j'en fus extrêmement surpris, et beaucoup
de gens aussi le furent. Cependant notre opposition signifiée avait eu
le temps de lui revenir\,; les seize noms qu'il y trouva achevèrent de
le presser de faire usage de son crédit. Le roi, à la promenade, parla
de l'absence de d'Antin, et à ce propos de l'affaire qui le rendait
absent. Il choisit le duc de Villeroy, qu'il compta apparemment
embarrasser davantage, et lui demanda d'un air et d'un ton mal satisfait
s'il serait des opposants, ce n'était pas sans doute qu'il ignorât ce
qui en était. Il répondit qu'il y en avait déjà nombre, que la chose lui
importait trop pour n'en être pas, et qu'il croyait qu'il y en aurait
encore d'autres. Le roi reprit que d'Antin avait fort consulté son
affaire, et qu'il la croyait indubitable\,; et sans plus adresser
particulièrement la parole, il tâcha en prolongeant le propos d'engager
des réponses auxquelles il pût répliquer. Mais Villeroy, content de
n'avoir point molli, s'en tint à ce qui avait été arrêté entre nous, et
fut sourd et muet.

Le lendemain le duc de Tresmes essuya la même question et fit la même
réponse. Le roi dit qu'au moins ne se fallait-il point fonder en
longueurs, et aller de bon pied au jugement. Une troisième fois le roi
parla vaguement de l'affaire, et s'adressant encore au duc de Villeroy,
lui dit qu'il ne comprenait pas que personne se pût opposer à d'Antin,
que sa prétention ne faisait rien à personne, hormis quelques anciens
devant lesquels il se trouverait, ce qui serait imperceptible à tous les
autres, et qu'il n'y avait point d'intérêt à être avancé ou reculé d'un
rang. Villeroy répondit que chacun y était fort intéressé, puisque ce
pas de plus ou de moins était ce qui de tout temps était le plus cher
aux hommes\,; qu'il retombait sur les nouveaux comme sur les anciens\,;
que d'ailleurs la prétention de d'Antin ouvrirait la porte à quantité
d'autres\,; que chacun disputait bien une mouvance, à plus forte raison
ce qui appartenait à la première dignité du royaume. Le roi, qui ne
s'attendait qu'à étourdir son homme, et de là sans doute à étonner et
ralentir les opposants, ne répliqua rien à une si digne réponse. Il
cessa même de plus rien témoigner sur ce procès, non qu'il pût se tenir
d'en parler encore quelquefois, mais vaguement, et sans plus rien
témoigner de partial. Nous reconnûmes bien à quel point il l'était, et
combien salutaire la résolution que nous avions prise à cet égard,
puisque, si on eût molli et parlé en vils courtisans qui veulent faire
leur cour, nous étions désarmés sans ressource, au lieu que, nous
conduisant comme nous l'avions arrêté, le roi rebuté de ses tentatives,
et en garde contre la réputation d'être gouverné, n'osa jamais passer
outre dans cette crainte, et par le même esprit professa bientôt la
neutralité. Maintenant il est juste de montrer tout de suite quels
furent les ducs qui surent se respecter, quels les lâches, quels enfin
les déserteurs. Les ducs de Ventadour, Montbazon, Lesdiguières, Brissac,
La Rochefoucauld, La Force, Valentinois, Saint-Aignan et Foix, pairs\,;
La Feuillade et Lorges vérifiés, se joignirent à nous. Notre surprise
fut grande d'apprendre que M. de Luxembourg, qui avait été envoyé en
Normandie pour quelque émeute qui le retenait à Rouen, trouvait la
prétention de d'Antin si étrange, malgré la sienne qui ne l'était guère
moins, qu'il s'unit à nous contre lui, mais en même temps se mit en état
de recommencer son procès de préséance.

La Feuillade, moins uni et plus semblable à lui-même, s'était joint à
nous, et il avait paru que c'était de bonne foi. Séduit tôt après par
l'abbé de Lignerac, détaché par d'Antin, il chercha à se retirer. Il
prit pour prétexte que les pairs, moins anciens qu'il n'était duc, le
précéderaient dans les actes et les énoncés d'un procès de pairie. Cette
fantaisie, qui aurait dû guérir, si elle avait été réelle, l'exemple des
autres ducs vérifiés joints à nous, ne put être soutenue. Quelques jours
après s'être rendu là-dessus, il allégua au duc de Charost une
prétention de pairie et d'ancienneté de Roannais, qu'il inventa parce
qu'elle était sans apparence. Le bon Charost, qui goba ce leurre, eut la
facilité de lui répondre que nous ne prétendions pas lui faire tort en
rien, et que c'était à lui à voir son intérêt. J'avais su le manége de
l'abbé de Lignerac, et que d'Antin s'en vantait. J'en parlai vivement
chez moi à M\textsuperscript{me} Dreux, et du peu de succès que ce
procédé trouvait dans le monde\,; et je me moquai un peu de ce qu'il
songeait, dans l'état où il était plongé depuis Turin, à faire valoir ce
que son père avait oublié dans sa longue faveur. J'ajoutai qu'il était
plaisant de voir un homme de plus de quarante ans, qui dans sa courte
prospérité avait à propos de rien insulté d'Antin à Meudon de la façon
la plus cruelle, qui depuis ses infortunes avait abdiqué la cour avec
éclat, n'oublier rien pour s'y raccrocher jusqu'à l'infamie d'agir
contre sa signature qui était entre nos mains, pour acheter la
protection du même d'Antin, qui ne ferait, avec l'ancienne rancune que
le mépriser et en rire après en avoir fait ce qu'il aurait voulu.
J'étendis ces choses avec peu de ménagement pour La Feuillade et peu de
souci, de notre part, de lui de plus ou de moins, mais par amitié pour
Chamillart qui serait très-affligé des suites. Je lui appris en même
temps qu'étant informés de l'usage juridique que d'Antin se proposait de
faire de la désertion, la résolution était prise et arrêtée entre nous
de faire énoncer par nos avocats en plaidant, et la chose était vraie,
les raisons et les motifs de chacun des déserteurs, sans ménagement
aucun pour des gens qui en avaient si peu pour nous et pour eux-mêmes.
Deux jours après, La Feuillade se plaignit qu'il avait été mal entendu
et rigoureusement traité. Sans s'expliquer mieux, il protesta qu'il
n'avait jamais eu dessein de se séparer de nous, et nous le fit dire, en
forme. Peu de jours après, je le trouvai chez Chamillart, que je voyais
régulièrement tous les jours que j'étais à Paris. La Feuillade m'y
demanda un entretien tête à tête. Il s'entortilla dans un long
éclaircissement, dans des protestations inutiles, dans des compliments
personnels sans fin.

Je pris tout cela pour bon\,; la fin fut que la peur le tint joint à
nous, mais le premier payement fait, il n'en voulut plus ouïr parler, et
que nous ne le vîmes ni aux assemblées, ni aux sollicitations, ni en
aucunes des démarches sur ce procès. Avec cette conduite il s'attira
ceux que d'effet il abandonnait et qui ne s'en contraignirent pas dans
le monde, lequel leur fit écho sur un homme peu estimé et aimé pour
avoir abusé de sa faveur, et en être tombé par ses fautes avec une
grande brèche à l'État. Il n'apaisa pas l'ancienne haine de d'Antin,
bien loin de se concilier son secours, pour n'oser prendre son parti, et
il n'y eut pas jusqu'à l'entremetteur Lignerac qui fut trouvé fort
ridicule.

Les ducs, démis, destitués de qualité pour agir, ne purent que demeurer
dans l'inaction\,; les pairs ecclésiastiques furent réservés pour être
juges, quoique les trois ducs nous eussent offert leur jonction, et M.
de Metz n'était pas encore en situation de rien faire. Le duc de
Noailles ne répondit jamais un mot là-dessus aux maréchaux de Boufflers
et d'Harcourt qui lui en écrivirent plus d'une fois. Le cardinal son
oncle avait alors bien d'autres affaires à démêler. Le duc d'Uzès en usa
tout autrement, il manda franchement à d'Antin qu'étant son beau-frère
et alors en Languedoc, il se tairait sous prétexte d'ignorance, mais que
s'il s'avisait de le faire assigner comme il prétendait faire à tous
pour les obliger à une déclaration expresse, il ferait la sienne contre
lui, sur quoi d'Antin n'osa passer outre avec lui. M. d'Elbœuf,
au-dessus ou au-dessous de tous procédés, en avait eu un fort inégal
dans l'affaire de M. de Luxembourg, et fort différent de celui de son
père qui s'était porté vivement toujours, et de grand concert dans cette
affaire et dans les pareilles qui s'étaient offertes de son temps, et
qui n'intéressaient pas les prétentions de sa naissance. M. d'Elbœuf,
seul de tous les pairs de sa maison, ne s'était point fait recevoir au
parlement, et il n'eut point honte de chercher bassement à faire sa cour
en se déclarant verbalement pour d'Antin. Le duc de Chevreuse, toujours
arrêté par son idée de l'ancien Chevreuse, et par une nouvelle aussi peu
fondée pour le moins sur Chaulnes, se tint à part comme il avait fait
sur l'affaire de M. de Luxembourg, et en fit user de même au jeune duc
de Luynes son petit-fils. Les ducs de Richelieu et de Rohan, si vifs sur
M. de Luxembourg, ne jugèrent pas à propos d'entrer dans celle-ci.
Véritablement leurs procédés avaient été si pénibles à supporter en
cette affaire, leur crédit présent si peu de chose, qu'on fut aisément
consolé de n'avoir rien de commun avec eux. On les a vus dans le récit
de cette affaire. M. de Rohan prit feu d'abord, et se plaignit de
n'avoir pas été invité comme quelques autres le furent, à signer d'abord
l'opposition, et s'en était pris à moi. La vérité était que cela s'était
proposé à Marly chez le maréchal de Boufflers, et que ses disparates
m'engagèrent à en détourner, pour cette première signature. Je sus ses
plaintes, je dis mes raisons qui ne lui plurent pas, il demeura piqué et
spectateur, et nous y gagnâmes plus que nous n'y perdîmes. M. de Fronsac
suivit M. de Richelieu son père. M. de Bouillon, qui lors du procès de
M. de Luxembourg s'était si bien fait moquer de lui avec sa chimère de
l'ancien Albret et Château-Thierry, qui l'avait empêché de se joindre à
nous, laissa entendre la même excuse, sans pourtant oser l'énoncer. Nous
comprîmes que dans la situation critique où l'éclat du cardinal de
Bouillon l'avait mis, il comptait avoir besoin de tout et n'osait
choquer d'Antin de la faveur duquel il pouvait espérer et craindre. Le
duc d'Estrées, fidèle au cabaret et au tripot, y attendit paisiblement
les événements, si toutefois il sut l'affaire. M. Mazarin absent, et
toujours au troisième ciel, ne se détourna point aux choses de la terre.
Le duc de La Meilleraye, son fils, de vie et de mœurs si opposées, mais
qui ne mettait jamais le pied à la cour, se rangea du côté de d'Antin
sans qu'il sût lui-même pourquoi, et s'attira la risée. Le duc de Duras,
qui depuis son mariage ne connaissoit plus que les Noailles, si liés à
d'Antin, n'osa se déclarer contre lui. Il s'était attaché au comte de
Toulouse, et avait demandé à servir en Catalogne, sous le duc de
Noailles, qui l'avait envoyé peu décemment porter la nouvelle de la
prise de Girone. Il était avec eux sur le pied de ces sortes d'amis
qu'on souffre pour en abuser. Cela m'avait impatienté souvent d'un homme
de sa naissance, de sa dignité et si proche de M\textsuperscript{me} de
Saint-Simon. Cette conduite sur d'Antin acheva de me choquer tellement,
qu'il m'échappa qu'il n'en fallait pas attendre une autre du
portemanteau de M. le comte de Toulouse, et du courrier de M. le duc de
Noailles. Ils le surent, et en furent désolés. Le duc de Châtillon,
malgré la démarche du duc de Luxembourg son frère, prétexta son procès
contre nous pour ne pas entrer dans celui-ci. Le duc de Noirmoutiers,
plus franchement, déclara qu'étant aveugle, sans enfants, ni espérance
d'en avoir, il n'avait aucun intérêt à prendre. On ne laissa pas de
tomber fortement de notre part sur ces messieurs, qui cependant se
trouvèrent fort embarrassés. MM. de Charost et d'Humières conduisirent
l'affaire avec une suite et un concert qui furent extrêmement utiles et
qui méritèrent toute la reconnaissance des intéressés.

Ce serait ici le lieu d'expliquer la prétention de d'Antin, et les
raisons contraires\,; cela serait long et peut-être ennuyeux. Cela
couperait trop aussi la suite des matières. Cette explication se
trouvera plus convenablement parmi les Pièces, ainsi que celle de la
prétention de Matignon au duché d'Estouteville\footnote{Voir les Pièces
  sur Épernon et sur Estouteville. \emph{\{Note de Saint-Simon.)} ---
  Les anciens éditeurs ont supprimé ce passage depuis \emph{Cette
  explication} jusqu'à \emph{d'engrossir les Pièces}. Outre toutes les
  raisons du fond, on verra dans les Pièces que la terre d'Épernon avoit
  été vendue à Armenonville\,; que d'Antin lui avoit fait parler si net
  par Monseigneur, qu'il la lui revendit\,; que ce manége avoit été
  couvert par toutes sortes d'artifices, jusqu'à avoir retiré des
  notaires les deux minutes des deux contrats de vente et les avoir
  brûlées, parce qu'une vente éteint de droit un duché, et qu'il ne peut
  être recueilli que par héritage par celui qui a le droit le plus clair
  à sa dignité. C'est ce que d'Antin s'étoit voulu ménager. Il fut bien
  étonné de la découverte des deux ventes, et lui, et plus encore
  Armenonville, effrayés du parti que nous résolûmes, et dont nous ne
  nous cachâmes pas de les faire jurer. Il se trouvera encore parmi les
  Pièces que l'érection d'Épernon portoit une clause par laquelle tout
  roturier en étoit exclu, c'est-à-dire la femelle en droit de
  recueillir la dignité épousant un roturier, ce roturier ni sa
  postérité ne pouvoient succéder à la dignité qui s'éteignoit par cette
  clause. La prétention de d'Antin venoit de sa grand'mère, Christine
  Zamet, mère de M. de Montespan, qui étoit fille du fameux Sébastien
  Zamet, si connu sous Henri IV, qui s'intituloit plaisamment seigneur
  de un million sept cent mille écus, somme alors prodigieuse pour un
  particulier. Ce riche partisan avoit épousé une Goth, sœur et tante
  des Rouillac, dont la mère étoit sœur du célèbre duc d'Épernon, et
  morte avant qu'il fut fait duc. Or, pour s'en tenir ici à la roture et
  renvoyer tout le reste aux Pièces, ces Zamet étoient du bas peuple de
  Lucques, que la banque avoit enrichis et qui ne s'étoient jamais
  prétendus autre chose. J'écrivis donc au cardinal Gualterio de faire
  chercher par ses amis, et par l'autorité du grand-duc avec lequel il
  étoit intimement, tout ce qui pouvoit prouver juridiquement cette
  roture, de le faire authentiquer par la république de Lucques et de me
  l'envoyer.}. Il perdit cette terre par un grand procès contre la
duchesse de Luynes, héritière de la duchesse de Nemours. Il la racheta
ensuite et forma sa prétention à la dignité. Je fis un mémoire sur cela,
que je donnai au chancelier\,; sur le compte qu'il en rendit au roi, la
permission de poursuivre fut refusée. On verra aux Pièces l'ineptie de
pareilles prétentions. J'y joindrais ce qui regarde celle d'Aiguillon
qui n'est pas mieux fondée\,; mais ayant été, depuis ce règne, portée au
parlement, malgré le refus du feu roi et l'édit sur les duchés dont il
sera parlé, le procès mal défendu de notre part et sollicité par
M\textsuperscript{me} la princesse de Conti, qui en fit publiquement son
affaire, réussit pour Aiguillon, comme fit, vers le même temps, la
czarine pour la Courlande, et par les mêmes raisons, que ni l'une ni
l'autre ne s'embarrassèrent pas de cacher. Ainsi les factums imprimés,
quoique mauvais, font assez connaître de quoi il s'agissait pour me
dispenser d'en grossir les Pièces.

Tout ce qui reste pour le présent à ajouter sur l'affaire de d'Antin,
c'est que nos sollicitations faites ensemble et en apparat contre lui
l'étonnèrent fort, et qu'il se sentit tout à fait déconcerté sur la
partialité du roi qu'il avait adroitement su persuader au parlement. Les
maréchaux de Bouffiers et d'Harcourt en parlèrent ensemble au roi en
gens de leur sorte, et si bien, que le roi ne fut pas fâché de s'en
trouver quitte pour une déclaration d'entière neutralité. Il la déclara
tout de suite au premier président, avec ordre de la rendre de sa part à
sa compagnie. Nous eûmes soin de nous assurer de son exécution MM. de
Charost, d'Humières et moi, en allant chez le premier président qui nous
la certifia, et de nous en procurer la dernière certitude par plusieurs
juges qui nous certifièrent que le premier président l'avait signifiée à
la compagnie de la part du roi, d'une manière nette et positive. Une
déclaration si précise et si contraire aux idées et beaucoup au delà que
d'Antin avait données au parlement, et dont il avait rempli le public,
qui fut incontinent informé du vrai, changea fort l'affaire de face. Les
noms de faveur, de grandes charges, de généraux d'armée, de gens de
privance et de réputation qui se trouvèrent parmi nous emportèrent la
balance sur d'Antin, dès que le roi se fut si nettement et si hautement
expliqué. Les fins de non-recevoir contre d'Antin ajoutèrent fort au
démérite du fond de ses prétentions. Le public revint de l'opinion qu'il
avait prise que la cause du favori était celle du roi, et le parlement
commença à trouver qu'il avait au moins la cause à juger, et non plus
uniquement les personnes.

Nous tînmes cela secret entre quatre ou cinq de nous autres, de peur que
le dessein transpirât, et que d'Antin ne le fît échouer par Torcy ou par
le roi même sans s'y montrer, et pour avoir aussi le plaisir de le
servir tout à coup de cette bombe en plein parlement. Les choses
n'allèrent pas jusqu'au jugement, comme on le verra ci-après. Il faut
maintenant terminer cette matière par une frayeur du duc de
Beauvilliers, qui ne fut pas sans fondement.

Il avait cédé son duché à son frère en le mariant, qui de ce moment
avait joui du rang et des honneurs, sans que personne se fût avisé même
d'en parler. Cette année il le fît recevoir pair au parlement le 22
janvier, et il voulut se trouver à la cérémonie avec sa famille dans la
lanterne. Comme j'entrais ce matin-là dans la grand'chambre, je fus
surpris de trouver le duc de Beauvilliers qui m'attendait derrière la
porte, qui, dès que je la débouchai, me prit par la main et me mena en
un coin. Là, il me dit qu'il m'attendait avec impatience, dans
l'inquiétude extrême où il était sur un avis qui ne lui était venu que
depuis qu'il était arrivé au palais, mais qu'on lui avait redoublé de
plusieurs endroits. On l'avait averti que plusieurs du parlement étaient
résolus à s'opposer à la réception de son frère, mais plusieurs pairs,
fondés sur ce que la duchesse de Beauvilliers pouvait mourir avant lui,
lui se remarier et avoir un fils\,; que ce fils exclurait son oncle de
droit, et pourtant se trouverait lui-même exclu par la réception de ce
même oncle dont la postérité prétendait succéder. M. de Beauvilliers,
fort alarmé d'une difficulté plausible, me demanda ce que je lui
conseillais.

Je pensai un moment, je lui dis ensuite que la cérémonie, commencée par
l'arrivée des pairs et par celle des princes du sang et du reste des
pairs qui allait suivre, ne se pouvait remettre ni interrompre\,; que je
n'avais pas ouï dire un mot de ce qu'il m'apprenait\,; que j'avais
grand'peine à croire qu'il y eût là-dessus plus que quelque raisonnement
de conversation, et point du tout du dessein ni de résolution prise sur
un futur contingent sans apparence, et qui ne blessait personne\,; que,
de plus, arrêter la réception en sa présence, étant ce qu'il était, et
d'un homme jouissant, par le consentement du roi, du rang et des
honneurs de sa dignité, me paraissait une démarche bien forte pour le
temps où nous étions, n'étant surtout excité par l'intérêt de personne.
«\, Mais néanmoins que faire si la chose arrive\,? interrompit le duc
fort peiné. --- Le voici, lui dis-je, et je réponds du succès\,; mais,
encore une fois, je ne croirai point qu'il y ait une seule voix qui
s'élève que je ne l'aie entendue\,; mais, si le cas arrive, je compterai
bien exactement les voix pour et contre, et je crois encore en ce cas
que les voix contre seront si rares que ce ne sera pas la peine de les
réfuter\,; que si à tout reste il le faut faire, j'attendrai mon tour à
parler. Alors je dirai que je suis surpris que quelqu'un dans la
compagnie puisse faire difficulté de recevoir celui que le roi en a si
publiquement jugé capable et digne, en lui permettant, et à vous de
céder et d'accepter le duché, en le faisant jouir du rang et des
honneurs, et en lui permettant de se faire recevoir\,; que le cas
possible qui sert de fondement à la difficulté proposée, est un cas
chimérique et reconnu tel par le roi, qui aurait dû arrêter sur la
démission, s'il en eût fait le moindre cas, sur lequel le parlement ne
devait pas montrer plus de délicatesse d'exécution que le roi n'en avait
eue pour la permission\,; qu'enfin, pour lever tout scrupule, la cour
avait dans ses registres un exemple tout semblable, non en sa cause,
mais en son effet, qui paraissait fait exprès pour servir d'exemple et
de modèle de ce qui se devrait faire si le cas proposé arrivait. Que la
duchesse d'Halluyn avait épousé le fils aîné du premier duc d'Épernon
qui, comme duc et pair d'Halluyn, avait été reçu au parlement\,; que
huit ans après ces époux s'étant brouillés, et n'ayant point d'enfants,
ils s'étaient accordés à faire casser leur mariage\,; qu'ensuite la
duchesse d'Halluyn s'était remariée au fils du maréchal de Schomberg,
depuis aussi maréchal de France, lequel, au titre de ce mariage, était
devenu aussi duc d'Halluyn et pair de France, et avait été reçu au
parlement en cette qualité, encore que l'autre mari l'eût conservée en
sa totalité, parce que les rangs et les honneurs acquis par titres ne se
perdent point\,; qu'à la cour, aux cérémonies, le premier mari précédait
le second\,; qu'au parlement, où on ne pouvait connaître qu'un seul
titulaire à la fois, celui des deux qui arrivait le premier prenait
place, et l'autre venant après trouvait le premier huissier qui
l'abordait dans la grand'chambre et lui disait que M. le duc d'Halluyn
était en place, et aussitôt celui-ci s'en retournerait\,; que le cas
prévu arrivant, l'âge de l'oncle et du neveu seraient trop différents
pour causer aucun embarras\,; mais qu'enfin leur leçon se trouverait
toute réglée tant à la cour qu'au parlement par l'exemple des deux ducs
d'Halluyn\,; qu'à l'égard de la succession, il n'était pas douteux que
le fils de l'oncle ne pourrait être duc au préjudice de son cousin et
par la teneur de l'érection, et parce qu'on ne peut être duc sans
posséder de droit la terre érigée, qui retournerait de droit à ce fils
qu'on imaginait, dont la naissance ferait tomber et annulerait seule
toutes les donations de père.\,» Cet exemple ignoré du duc de
Beauvilliers, et je crois de bien d'autres, le soulagea extrêmement. Il
regagna sa lanterne et je me mis en place.

Peu après que j'y fus, je remarquai quelque chose, des gens qui se
parlaient bas\,; et, comme les pairs qui arrivent successivement coupent
ceux qui sont placés pour se mettre en leurs rangs, je me trouvai
d'abord voisin des ducs de La Meilleraye et de Villeroy, qui en effet,
sifflés apparemment par quelques-uns me firent la difficulté. Je la
rejetai comme ridicule\,; je leur fis peur du roi à qui on voudrait
apprendre la leçon, enfin j'alléguai MM. d'Halluyn, qui leur firent
ouvrir les oreilles. Je ne sais si, en attendant et pendant le rapport,
cela courut par les bancs\,; mais quoi qu'il en soit, nulle voix ne
s'éleva. Le duc de Saint-Aignan fut reçu tout à l'ordinaire, et M. de
Beauvilliers sortit de là fort aise et fort content.

\hypertarget{chapitre-iv.}{%
\chapter{CHAPITRE IV.}\label{chapitre-iv.}}

1711

~

{\textsc{Prise de Girone.}} {\textsc{- Brancas en est fait gouverneur.}}
{\textsc{- Estaires et Beaufremont chevaliers de la Toison d'Or, et le
duc de Noailles grand d'Espagne de la première classe, qui passe en
Espagne, dont l'armée ne peut s'assembler qu'en août.}} {\textsc{- Dix
mille livres de pension du roi d'Espagne à M\textsuperscript{me} de
Rupelmonde, dont le mari avait été tué à Brihuega.}} {\textsc{- Mort du
duc de Medina-Celi.}} {\textsc{- Mort du marquis de Legañez}} {\textsc{-
Mort du prince de Médicis, auparavant cardinal.}} {\textsc{- Bergheyck à
Paris, passe en Espagne, d'où il est bientôt renvoyé par la princesse
des Ursins.}} {\textsc{- Premier mariage du duc de Fronsac, peu après
mis en correction à la Bastille.}} {\textsc{- Fortune de
M\textsuperscript{me} de Villefort.}} {\textsc{- Fortune de
M\textsuperscript{lle} de Pincré, qui épouse le fils de
M\textsuperscript{me} de Villefort.}} {\textsc{- Mariage d'un cadet de
Nassau-Siegen avec la sœur du marquis de Nesle.}} {\textsc{- Famille et
mariage de Saint-Germain-Beaupré avec la fille de Doublet, qui se fourre
de tout.}} {\textsc{- Mot cruel du premier président Harlay aux deux
frères Doublet.}} {\textsc{- Mouvements du procès de la succession de M.
le Prince.}} {\textsc{- M. le Duc perd en plein son procès contre
M\textsuperscript{me}s ses tantes, et avec des queues fâcheuses.}}
{\textsc{- Mort et court éloge du maréchal de Choiseul.}} {\textsc{-
Chevalier de Luxembourg gouverneur de Valenciennes.}} {\textsc{- Mort de
Boileau-Despréaux.}} {\textsc{- Mort du fils aîné du maréchal de
Boufflers, dont la survivance passe au cadet.}}

~

On a vu, dans les derniers jours de l'année précédente, le siège de
Girone formé par le duc de Noailles après la bataille de Villaviciosa,
et que, les neiges ayant fini la campagne de Savoie, il avait reçu un
grand renfort de l'armée du maréchal de Berwick. Ce siège commençait à
s'avancer lorsqu'un furieux ouragan, suivi d'un grand débordement
d'eaux, renversa le camp et les travaux, mit l'armée en état de mourir
de faim, et pensa sauver la place. L'activité fut grande à réparer un
inconvénient si fâcheux, qui donna une grande inquiétude au roi, et
retarda fort le siége. La basse ville fut emportée l'épée à la main\,;
le 23 février la haute ville capitula à condition de se rendre le 30
avec les deux forts, s'ils n'étaient pas secourus. Staremberg n'y songea
pas\,; la garnison sortit avec les honneurs de la guerre. Planque, qui
en apporta la première nouvelle, en fut fait brigadier\,; et le duc de
Duras apporta celle de l'évacuation de la place, dont le gouvernement
fut donné aussitôt au marquis de Brancas, au grand scandale des
Espagnols.

Le comte d'Estaires porta la nouvelle de cette conquête au roi
d'Espagne, il en eut la Toison\,; et en même temps Beaufremont eut celle
que la mort de Listenais, son frère, avait laissée vacante dans Aire où
il fut tué. En même temps aussi le duc de Noailles fut fait grand
d'Espagne de la première classe. On le sut aussitôt à la cour. La
maréchale de Noailles, ravie de cette nouvelle élévation de son fils, en
reçut les compliments\,; mais le roi trouva les compliments et la
grandesse fort mauvais. Il était convenu avec le roi d'Espagne, depuis
que les affaires tournaient mal et qu'on se voyait forcé de désirer la
paix en l'abandonnant, qu'il ne donnerait plus de grandesses ni de
Toison à des François\,; il fut donc fort choqué des trois grâces qui
viennent d'être rapportées, et il le témoigna. La maréchale de Noailles
et les siens en furent transis, revomirent les compliments reçus, et ne
savaient plus où ils en étaient, lorsqu'enfin le roi, apaisé par
M\textsuperscript{me} de Maintenon, sans la participation de qui
M\textsuperscript{me} des Ursins ne l'eût pas hasardé, consentit enfin,
et les compliments furent de nouveau faits et reçus.

Le duc de Noailles pourvut Girone, sépara son armée, alla passer un mois
à Perpignan, et de là à Saragosse, et à la suite de la cour d'Espagne,
où il demeura plusieurs mois. On y envoya bientôt après vingt-six
bataillons et trente-six escadrons, que le duc de Noailles y devait
commander à part, mais aux ordres de M. de Vendôme, et le roi d'Espagne
se mettre de bonne heure à la tête de l'armée. Mais tout manqua
tellement en Espagne, par les désastres et les efforts précédents, que
les troupes ne purent être mises en mouvement avant la fin d'août, et
que le duc de Noailles, au lieu d'être un peu général en Espagne, n'y
fut que courtisan.

Malgré l'étrange détresse des affaires de ce pays-là,
M\textsuperscript{me} de Rupelmonde, dont le triste mari avait été tué à
Brihuega dans les troupes d'Espagne, et lui avait laissé un fils, sut si
bien intriguer dans les deux cours, faire pitié à M\textsuperscript{me}
de Maintenon, et s'aider de Desmarets beau-père de sa sœur, qu'elle
obtint du roi d'Espagne une pension de dix mille livres.

Le duc de Médina-Celi mourut prisonnier à Bayonne bientôt après y avoir
été transféré\,; ce fut les premiers jours de février. En lui finit la
seconde race de ce titre sortie d'un bâtard de Gaston-Phœbus, comte de
Foix, qui épousa l'héritière de Lacerda. Le marquis de Priego, déjà plus
d'une fois grand d'Espagne, fils de la sœur aînée du duc de Medina-Celi,
en prit le titre et succéda à ses biens et à ses grandesses. Son nom est
Figueroa\,; il y ajoute celui de Cordoue.

Peu de jours après mourut à Paris, dans un honnête exil, après la prison
de Vincennes, le marquis de Leganez, à qui M\textsuperscript{me} des
Ursins fit accroire qu'on avait trouvé un grand amas d'armes au
Buen-Retiro, dont il était gouverneur, et le fit arrêter et paqueter en
France, comme il a été dit en son lieu. Il n'y eut jamais d'informations
contre lui, beaucoup moins de preuves, et il fit à Paris, entre les
mains du duc d'Albe, ambassadeur d'Espagne, les serments qu'on voulut.
Il avait été vice-roi de Catalogne et gouverneur du Milanais, capitaine
général de l'artillerie d'Espagne et conseiller d'État, à la vérité fort
autrichien. On fut honteux enfin de le tenir à Vincennes, on y adoucit
sa prison, on lui permit enfin de demeurer à Paris, mais on ne voulut
pas le voir à la cour, et on n'osa le renvoyer en Espagne. Il était veuf
et sans enfants. Le comte d'Altamire hérita de ses grandesses et de ses
biens. Je ferais ici une digression trop longue sur la naissance et la
fortune de ces deux seigneurs\,; j'aurai lieu de parler d'eux lorsque je
m'étendrai sur l'Espagne, à l'occasion de mon ambassade à Madrid.

Le frère du grand-duc de Toscane mourut en ce même temps, celui qui
quitta le chapeau pour épouser une Guastalle dont il n'eut point
d'enfants, et dont il a été parlé à l'occasion du voyage du roi
d'Espagne à Naples. Il avait l'abbaye de Saint-Amand étant cardinal, et
lorsqu'il se maria il se réserva trente mille livres de rentes dessus.
Ce fut un deuil de noir de quelques jours.

Bergheyck, qui avait toujours servi le roi d'Espagne avec tant de
fidélité et de capacité à la tête de toutes ses affaires en Flandre, et
mandé par lui pour l'aller trouver, passa à Paris et eut plusieurs
audiences du roi. On croyait, et le roi l'aurait fort désiré, qu'il
aurait grande part aux affaires en Espagne, mais plus on en était
capable et moins on en était à portée, tant que la princesse des Ursins
y gouvernait, qui sut barrer et renvoyer bientôt Bergheyck, comme elle
en avait chassé, puis exclu tant d'autres.

Le duc de Fronsac épousa la fille unique de feu M. de Noailles, frère du
cardinal et de la troisième femme du duc de Richelieu, son père, qui en
se mariant avaient arrêté cette affaire entre leurs enfants. Ce petit
duc de Fronsac, qui n'avait guère alors que seize ans, était la plus
jolie créature de corps et d'esprit qu'on pût voir. Son père l'avait
présenté déjà à la cour, où M\textsuperscript{me} de Maintenon, ancienne
amie de M. de Richelieu, comme je l'ai dit ailleurs, en fit comme de son
fils, et par conséquent M\textsuperscript{me} la duchesse de Bourgogne
et tout le monde lui fit merveilles, jusqu'au roi. Il y sut répondre
avec tant de grâce, et se démêler avec tant d'esprit, de finesse, de
liberté, de politesse, qu'il devint bientôt la coqueluche de la cour.
Son père lui laissa la bride sur le cou\,; sa figure enchanta les dames.
Celle de sa femme, qui n'avait pourtant rien de désagréable, ne le
charma pas. Livré au monde avec tout ce qu'il fallait pour plaire et ne
rien valoir, il fit force sottises qui firent faire, moins de trois mois
après son mariage, celle à son père de le faire mettre à la Bastille. Ce
fut un lieu avec lequel il fit si bonne connaissance qu'on l'y verra
plus d'une fois.

Il se fit un petit mariage qui semblerait devoir être omis ici, mais
dont les singularités méritent d'y trouver place, c'est celui de
Villefort avec Jeannette. Cela ne promet pas, et toutefois cela va
rendre. Il faut expliquer les personnages\,: la mère de Villefort était
belle, de grand air, de belle taille\,; elle perdit son mari
officier-major de je ne sais plus quelle place\,; elle n'avait rien que
des enfants, ou fort peu à partager avec eux. Elle avait de l'esprit et
de l'intrigue, mais sans galanterie, et de la vertu. Elle eut quelque
recommandation particulière auprès de M\textsuperscript{me} de
Maintenon, à qui par là elle parvint à être présentée.
M\textsuperscript{me} de Maintenon, ainsi que le roi, était la personne
du monde qui se prenait le plus par les figures. L'air modeste, affligé,
malheureux de celle-ci la toucha. Elle lui fit donner une pension, la
prit en protection singulière, lui trouva de l'esprit\,; la figure la
soutint. Son mari était bien gentilhomme, et elle demoiselle.
M\textsuperscript{me} de Maintenon ne l'appelait que sa belle veuve, et
la fit une des deux sous-gouvernantes des enfants de France.

Jeannette était une demoiselle de Bretagne dont le nom est Pincré\,; son
père mourut et laissa sa femme sans pain avec un tas d'enfants tous
petits. Réduite à la mendicité, elle s'en vint avec eux, comme elle put,
se jeter à genoux au carrosse dans lequel M\textsuperscript{me} de
Maintenon s'en allait à Saint-Cyr. Elle était charitable, se fit
informer de cette malheureuse famille, leur donna quelque chose, plaça
les enfants, selon leur âge, où elle put, et prit une petite fille tout
enfant chez elle, qu'elle mit avec ses femmes en attendant que ses
preuves fussent faites, et elle en âge d'entrer à Saint-Cyr. Cette
enfant était très-jolie\,; elle amusa les femmes de
M\textsuperscript{me} de Maintenon par son petit caquet, et bientôt elle
l'amusa elle-même. Le roi la trouva quelquefois comme on la renvoyait,
il la caressa, elle ne s'effaroucha point de lui, il fut ravi de trouver
une jolie petite enfant à qui il ne faisait point peur, il s'accoutuma à
badiner avec elle, et si bien que lorsqu'il fut question de la mettre à
Saint-Cyr, il ne le voulut pas. Devenue plus grandelette, elle devint
plus amusante et plus jolie, et montra de l'esprit et de la grâce, avec
une familiarité discrète et avisée qui n'importunait jamais. Elle
parlait au roi de tout, lui faisait des questions et des
plaisanteries\,; le tiraillait quand elle le voyait de bonne humeur, se
jouait même avec ses papiers quand il travaillait, mais tout cela avec
jugement et mesure. Elle en usait de même avec M\textsuperscript{me} de
Maintenon, et se fit aimer de tous ses gens. M\textsuperscript{me} la
duchesse de Bourgogne à la fin la ménageait, la craignait même, et la
soupçonnait d'aller redire au roi. Néanmoins elle n'a jamais fait mal à
personne. M\textsuperscript{me} de Maintenon elle-même commença à lui
trouver trop d'esprit et de jugement, et que le roi s'y attachait trop.
La crainte et la jalousie la déterminèrent à s'en défaire honnêtement
par un mariage\,; elle en proposa au roi qui trouva à tous quelque chose
à redire. Cela la pressa encore plus. Enfin elle fit celui du fils de sa
belle veuve. Le roi avait donné des fonds à Jeannette à diverses fois\,;
il lui en donna encore pour ce mariage, le gouvernement de Guérande en
Bretagne pour son mari, qui était capitaine de cavalerie, avec assurance
du premier régiment d'infanterie. M\textsuperscript{me} de Maintenon se
crut délivrée, elle s'y trompa. Tout conclu, le roi lui déclara bien
sérieusement qu'il n'agréait le mariage qu'à condition que Jeannette
demeurerait chez elle, après le mariage, tout comme elle y était devant,
et il en fallut passer {[}par{]} là. Croirait-on qu'un an après elle
devint la seule ressource des moments oisifs de leur particulier,
jusqu'à la fin de la vie du roi\,! Le mariage se fit la nuit dans la
chapelle, M\textsuperscript{me} Voysin donna le souper, les mariés
couchèrent chez M\textsuperscript{me} de Villefort, où
M\textsuperscript{me} la duchesse de Bourgogne donna la chemise à
M\textsuperscript{me} d'Ossy, c'est le nom que Jeannette porta. Son mari
fut dans la suite un des gentilshommes de la manche du roi
d'aujourd'hui, et se poussa à la guerre.

Le marquis de Nesle avait une sœur qui, moyennant la substitution des
vieux Mailly, avait fort peu de chose, et montait en graine sans vouloir
tâter du voile. Il trouva un arrière-cadet de Nassau-Siegen, qui n'avait
pas de chausses, et qui servait en petite charge subalterne en Flandre,
dans les gardes du roi d'Espagne. Le nom flatta les Mailly qui firent ce
mariage, où la faim épousa la soif, qui fut très-malheureux, et qui
donna force scènes au monde.

En même temps Saint-Germain-Beaupré maria son fils à la fille de Doublet
de Persan, conseiller au parlement, fort riche, qui avait un frère
conseiller aussi, qui s'appelait Doublet de Grouy\footnote{Voy. t. V,
  p.~384, où cette anecdote est racontée, Le second des frères y est
  appelé à tort Doublet de Croï.}. Ils se firent annoncer un jour au
premier président Harlay sous ces noms de seigneurie. Le premier
président leur fit d'abord de grandes révérences, les regarda après
depuis les pieds jusqu'à la tête, et faisant semblant de ne les avoir
pas connus auparavant\,: «\,Masques, je vous connais,\,» leur dit-il, et
leur tourna le dos, les laissant confondus devant toute son audience.
Cette Doublet, qui était riche, et qui aimait le monde, se mit à jouer
gros jeu, s'intrigua chez M\textsuperscript{me} la Duchesse, et fut plus
heureuse que sa belle-grand'mère, fille du président de Bailleul et sœur
de la mère du maréchal d'Huxelles. J'ai parlé ailleurs de ces deux
sœurs. Jamais la belle-grand'mère ne put parvenir par tous ses amis et
amies, dont elle avait beaucoup, à manger, ni à entrer dans les
carrosses. Sa belle-petite-fille l'obtint fort promptement et alla à
Marly. Le père était gouverneur de la Marche, qui n'avait jamais rien
fait qu'ennuyer le monde, où sa femme, qui était aussi de robe, n'avait
jamais paru ni guère vécu. Le roi permit au père de donner son
gouvernement à son fils, aussi ennuyeux que lui, mais bien plus obscur
et goutteux, qui n'a presque jamais paru nulle part. Le maréchal
Foucault était frère de son grand-père, c'est-à-dire du mari de la
Bailleul. Il porta le nom de du Doignon avant d'être maréchal de
France\,; il fut page du cardinal de Richelieu, qui le mit après, comme
un homme de confiance, auprès du duc de Fronsac qu'il avait fait amiral,
et du Doignon vice-amiral. Il était auprès de lui lorsqu'il fut tué, en
1646, devant Orbitello. Du Doignon s'en revint tout court s'emparer de
Brouage, et comme c'était la mode alors de faire la loi à la cour, il
s'y maintint et ne s'en démit que moyennant le bâton de maréchal de
France qu'il eut en mars 1652, et il mourut à Paris sans alliance, à
quarante-trois ans, en octobre 1649, sans avoir figuré depuis.

Le procès de la succession de M. le Prince, suspendu par la mort de M.
le Duc, n'avait pu être accommodé, et tous les soins de
M\textsuperscript{me} la Princesse, peu secourue de lumière et de
fermeté, avaient échoué à mettre la paix dans sa famille. Elle eut le
déplaisir de voir la seule fille qui lui restait lui échapper par un
mariage qui ne pouvait être de son goût, et qui, fait par M. et
M\textsuperscript{me} du Maine, la tira de chez elle et de la neutralité
pour prendre le parti de M\textsuperscript{me}s ses sœurs et de son
propre intérêt. M\textsuperscript{me} la Duchesse partagea son temps
entre Paris pour y vaquer à cette affaire, et la cour où le soin de se
rendre de plus en plus considérable en dominant Monseigneur, la tenait
attentive à tout, et où celui de l'amuser chez elle avait étrangement
mitigé les lois du deuil de sa première année.

On peut juger que les meilleurs avocats furent retenus de part et
d'autre, et que de chaque côté ils se firent un point d'honneur de
vaincre. Le roi avait défendu de part et d'autre de se faire
accompagner, comme on l'a dit, et de faire solliciter. Le premier fut
exécuté, le second écorné par les sollicitations secrètes, qui furent
recherchées des deux côtés. La bâtardise me répugnait, je ne pouvais
aussi souhaiter pour M\textsuperscript{me} la Duchesse après tout ce qui
a été rapporté. Je demeurai donc exactement spectateur à l'abri de
l'ordre du roi. M\textsuperscript{me} la Duchesse, en pauvre veuve vexée
par ses belles-sœurs, qui voulaient, disait-elle, ruiner ses enfants,
vit chez eux ses juges plusieurs fois, marchant modestement avec
M\textsuperscript{lle}s ses filles, sa dame d'honneur et la seule fille
de sa dame d'honneur pour suite des siennes, se rangeait aux heures de
trouver Messieurs, les complimentait, entrait peu dans son affaire, mais
s'étendait fort à exciter leur compassion par l'excès des demandes qui
étaient faites, et si elles avaient lieu, par la dissipation des grands
biens de M. le Prince, par l'autorité de sa dernière volonté, par le
nombre et le bas âge de ses enfants, par la dignité de l'aîné, par les
pertes qui la livraient sans appui aux vexations de ses belles-sœurs, au
mépris de son contrat de mariage, et du testament et de l'honneur du
père commun, qu'elle soutenait seule contre des attaques si dures. M. le
Duc, accompagné de M. le comte de Charolais, son frère, encore enfant et
le plus beau du monde, allait à part rendre les mêmes devoirs à
Messieurs, et les toucher moins par ses paroles, qu'il n'a jamais eues à
la main, que par l'état humilié devant eux de cette maison de Condé, qui
avait été si formidable au parlement et à l'État, et dont toute la
fortune se trouvait entre leurs mains. En revanche de tant de modestie,
la cour ne retentissait que du bon droit de M\textsuperscript{me} la
Duchesse, et de son autorité à le faire valoir. On y avait peine à
comprendre d'où pouvaient sortir de si hautes demandes contre la sœur si
fort la bien-aimée d'un Dauphin de cinquante ans, si près du trône, et
si déclaré pour elle. M\textsuperscript{me} la princesse de Conti y
passait pour une emportée sans raison, pour une princesse du sang de
Paris, à qui personne ne prenait la peine de parler, et ses enfants pour
ne pouvoir vivre qu'à l'ombre de la protection de ceux de
M\textsuperscript{me} la Duchesse, et qui, renfermés dans leur faubourg
Saint-Germain, croissaient obscurément sous une mère folle, dont la
conduite avec M\textsuperscript{me} la Duchesse ferait le malheur de
leur vie, s'ils n'obtenaient de sa générosité le pardon des fautes dont
leur âge les pouvait excuser en quelque sorte. M. du Maine, plus craint
et par là plus ménagé, était, disait-on, le complaisant forcé de
M\textsuperscript{me} sa femme sur cette affaire, comme dans tout le
reste, laquelle haïssait trop M\textsuperscript{me} la Duchesse pour
être capable de raison, et pour la laisser suivre à M. du Maine. La vie
de Sceaux, l'assemblage bizarre des commensaux, les fêtes, les
spectacles, les plaisirs de ce lieu, étaient chamarrés en ridicule, et
les brocards tombaient sur la vie à part de M\textsuperscript{me} de
Vendôme, et jusque sur sa figure.

Tel était l'air de la cour et de cette partie de la ville qui établit
tout son mérite sur l'imitation de la cour. Tout ce qui environnait
Monseigneur et tout ce qui se proposait de l'environner, même de s'en
approcher, le gros du monde qui suivait le torrent, parlait le même
langage\,; tous s'empressaient de servir M\textsuperscript{me} la
Duchesse et de se faire un mérite de {[}leurs{]} soins. Le formidable
triumvirat se remua solidement, et Monseigneur, tout asservi qu'il était
à suivre les moindres impulsions du roi, ne put refuser
M\textsuperscript{me} la Duchesse à ce coup de parti de laisser nommer
son auguste nom tout bas à l'oreille de ses juges.

Mais la robe du parlement est toute différente de celle du conseil. La
première est sans commerce avec la cour, comme elle vit sans espérance
d'elle. Elle n'a point de part aux intendances, aux places de conseiller
d'État, aux emplois brillants qui dévouent celle du conseil à la
fortune. La robe du parlement n'est pas insensible à se dédommager d'un
état fixe et borné par le mépris de ceux qui distribuent les grâces, et
les occasions lui en sont d'autant plus chères qu'elles se rencontrent
plus rarement.

Cet esprit parut dans celle-ci, où le parti des princesses ne négligea
pas de piquer le courage des juges par les propos et le triomphe
anticipé de celui de M\textsuperscript{me} la Duchesse. Ces princesses,
assidues à leur conseil et à leurs sollicitations, les firent avec
apparat, mais elles y ajoutèrent le solide en plaidant elles-mêmes leur
cause qu'elles possédaient fort bien. Elles demeuraient des heures
entières et souvent davantage avec chaque juge, et elles le ravissaient
de se montrer si instruites. M. du Maine les voyait à part et résumait
avec eux ce qui s'était dit aux visites des princesses. Lui-même
travaillait aux écritures, et procurait par de sourdes mais fortes
sollicitations le fruit à son travail. Son crédit auprès du roi n'était
pas ignoré au parlement, ni sa partialité effective pour ce fils
bien-aimé, qui fit impression sur ceux qui comptèrent le temps
présent\,; et dans la vérité, les dernières années surtout de M. le
Prince avaient tellement informé le public de presque toute sa vie qu'on
fut moins indigné que persuadé de tout ce qui fut plaidé sur l'état de
son esprit, avec une licence fort indécente. Il fut surprenant combien
peu de gens demeurèrent neutres. Le roi, qui le voulut paraître, ne put
souvent s'empêcher de laisser échapper des demi-mots, et peut-être à
dessein, qui ne gardaient pas ce caractère et qui ne purent empêcher
Monseigneur de se montrer de plus en plus partial de l'autre côté, à
mesure que l'affaire tendait à sa fin. Elle produisit plusieurs
contrastes qui augmentèrent l'aigreur. M\textsuperscript{me} la Duchesse
s'y prétendit lésée, et ne se contraignit pas en propos, tandis que ses
parties surent se taire et cheminer à leur but.

La cause solennellement plaidée et tant qu'il plut aux deux parties,
Joly de Fleury, avocat général, parla avec grand applaudissement et
conclut en faveur des princesses. Une heure après, car les opinions
furent longues et à huis clos, son avis fut confirmé\,; mais l'arrêt
alla plus loin encore. M. le Duc perdit tout ce qui lui était demandé,
de toutes les voix, excepté quatre dont le poids même passa pour fort
léger. Il est aisé de comprendre quelle fut la joie des victorieux et la
rage de M\textsuperscript{me} la Duchesse. Elle se jeta au lit à
l'instant à l'hôtel de Condé, et ne voulut voir qui que ce fût de toute
la journée.

D'Antin, qui, moins en frère commun qu'en courtisan habile, avait gardé
un parfait équilibre, s'était tenu au palais pour être plus à portée
d'être instruit à l'instant même du jugement. Il avait secrètement
dépêché trois courriers au roi pendant la séance, tellement que le roi
fut le premier averti\,; mais il n'en fit pas semblant, lorsque
Chambonnas lui porta la nouvelle de la part du duc du Maine. Le roi se
contint tant qu'il put mais quelque longue habitude qu'il eût contractée
d'être le maître de soi et de savoir se posséder et se masquer
parfaitement, sa joie le trahit et perça à travers des propos d'amitié
commune à tous.

Monseigneur, qui avait été en des inquiétudes qu'il ne prenait plus la
peine de dissimuler, montra son dépit dans toute l'étendue qu'il put
avoir. Il s'émerveilla de l'issue\,; demanda à tout ce qu'il vit ce
qu'il leur en semblait, se tourmenta des noms des principaux juges,
trouva l'arrêt mauvais, s'inquiéta fort du chagrin de
M\textsuperscript{me} la Duchesse et de l'état des affaires de ses
enfants, lui dépêcha un message, ne se contraignit pas le soir au
cabinet d'en montrer son dépit à M. du Maine, et de le laisser remarquer
à tout le monde plusieurs jours de suite.

M\textsuperscript{me} la duchesse d'Orléans, à qui M. du Maine avait
envoyé un courrier sur-le-champ, me le manda à l'instant même. L'arrêt
laissait des queues cruelles à démêler à M\textsuperscript{me} la
Duchesse, qui eurent de fortes suites.

M. du Maine consulta longtemps à l'hôtel de Conti leurs affaires
communes en conséquence de l'arrêt, et alla de là chez
M\textsuperscript{me} la Princesse. Il lui témoigna, avec cette vérité
qu'on connaissoit en lui, qu'il ne pouvait sentir de joie dans un
événement qui donnait du déplaisir à M\textsuperscript{me} la Duchesse,
avec tous les compliments si aisés à faire quand on a vaincu et qu'on
nage dans la joie. M\textsuperscript{me} la Princesse ne lui conseilla
pas de voir M\textsuperscript{me} la Duchesse dans ces premiers
instants, et se chargea des compliments. Il vint coucher à Versailles,
où il déclara qu'il n'en recevrait aucuns, avec une modestie qui ne
trompa personne.

M\textsuperscript{me} la Duchesse donna plusieurs jours à Paris à sa
douleur et à ses affaires. Elle fut longtemps à se remettre d'un revers
que le triumvirat et que Monseigneur qualifièrent d'affront. On chercha
à renouer un accommodement pour éviter une hydre de procès qui naissait
du jugement de celui-ci\,; mais le surcroît d'aigreur y fut un obstacle
invincible.

Les tenants de M\textsuperscript{me} la Duchesse se lâchèrent en propos
qui ne demeurèrent pas sans repartie, et sa consolation fut de se venger
un jour des injures du barreau par Monseigneur. M. du Maine me conta,
peu de jours après à Marly, que le parti de M\textsuperscript{me} la
Duchesse s'exhalait en injures contre lui, et publiait qu'il avait fait
agir maîtresses et confesseurs, qu'il avait soulevé jusqu'aux
jansénistes, en mémoire de l'ancien hôtel de Conti. Le parti victorieux
alla remercier les juges, et jusque chez les avocats de son conseil qui
triomphèrent de joie.

Je perdis le 15 mars un ami que je regretterai toute ma vie, et de ces
amis qui ne se trouvent plus, dont j'ai fait ici mention en diverses
occasions. Ce fut le maréchal de Choiseul, doyen des maréchaux de France
(et ils étaient encore dix-sept), chevalier de l'ordre et gouverneur de
Valenciennes. Quoique de la plus grande naissance, sans bien et sans
parents, il ne dut rien qu'à sa vertu et à son mérite, assez grands l'un
et l'autre pour s'être soutenus, malgré fort peu d'esprit, contre la
persécution de Louvois et de son fils, avec une hauteur qu'il n'eut
jamais pour personne, et un courage qu'il montra égal dans toutes les
occasions de sa vie. La vérité, l'équité, le désintéressement au milieu
des plus grands besoins, la dignité, l'honneur, l'égalité furent les
compagnes de toute sa vie, et lui acquirent beaucoup d'amis et la
vénération publique. Compté partout, quoique sans crédit\,; considéré du
roi, quoique sans distinctions et sans grâces\,; accueilli partout\,;
quoique peu amusant, il n'eut d'ennemis et de jaloux que ceux de la
vertu même qui n'osoient même le montrer, et des ministres qui
haïssaient et redoutaient également la capacité, le courage et la grande
naissance. On a vu en plus d'un endroit ci-dessus combien il était
capitaine, il avait aussi l'estime et l'affection des armées. Tout
pauvre qu'il était, il ne demandait rien. Il n'était jaloux de personne,
il ne parlait mal de qui que ce soit\,; et il savait trouver les deux
bouts de l'année sans dettes, avec un équipage et une table simple et
modeste, mais qui satisfaisait les plus honnêtes gens, et où ceux du
plus haut parage de la cour s'honoraient d'être conviés et de s'y
trouver. Il avait soixante-dix-sept ans, et ne se prostituait ni à la
cour, où il paraissait des moments rares par devoir, ni dans le monde,
où il se montrait avec la même rareté\,; mais il avait chez lui bonne
compagnie\,; et il se peut dire que, au milieu d'un monde corrompu, la
vertu triompha en lui de tous les agréments et de la faveur qu'il
recherche. Il mourut avec une grande fermeté, la tête entière toute sa
vie, et le corps sain, sans être presque malade, et reçut tous les
sacrements avec beaucoup de piété. M. le Prince, qu'il avait suivi en
Flandre comme tant d'autres, a toujours fait un cas très distingué de
lui. Il ne laissa point d'enfants de la sœur du marquis de Renti, qu'il
avait perdue, mais dont il était séparé de corps et de biens depuis un
grand nombre d'années.

Le chevalier de Luxembourg eut aussitôt après le gouvernement de
Valenciennes.

En même temps mourut Boileau-Despréaux si connu par son esprit, ses
ouvrages, et surtout par ses satires. Il se peut dire que c'est en ce
dernier genre qu'il a excellé, quoique ce fût un des meilleurs hommes du
monde. Il avait été chargé d'écrire l'histoire du roi\,; il ne se trouva
pas qu'il y eût presque travaillé.

Peu de jours après, il arriva un cruel malheur au maréchal de Boufflers.
Son fils aîné avait quatorze ans, joli, bien fait, qui promettait toutes
choses, et qui réussit à merveilles à la cour, lorsque son père l'y
présenta au roi pour le remercier de la survivance du gouvernement
général de Flandre et particulier de Lille, qu'il lui avait donnée. Il
retourna ensuite au collège des jésuites où il était pensionnaire. Je ne
sais quelle jeunesse il y fit avec les deux fils d'Argenson. Les
jésuites voulurent montrer qu'ils ne craignaient et ne considéraient
personne, et fouettèrent le petit garçon, parce qu'en effet ils
n'avaient rien à craindre du maréchal de Boufflers\,; mais ils {[}se{]}
gardèrent bien d'en faire autant aux deux autres quoique également
coupables, si cela se peut appeler ainsi, parce qu'ils avaient à compter
tous les jours avec Argenson, lieutenant de police, très-accrédité, sur
les livres, les jansénistes, et toutes sortes de choses et d'affaires
qui leur importaient beaucoup. Le petit Boufflers, plein de courage, et
qui n'en avait pas plus fait que les deux d'Argenson, et avec eux, fut
saisi d'un tel désespoir qu'il en tomba malade le jour même. On le porta
chez le maréchel où il fut impossible de le sauver. Le cœur était saisi,
le sang gâté\,; le pourpre parut, en quatre jours cela fut fini. On peut
juger de l'état du père et de la mère. Le roi qui en fut touché ne les
laissa ni demander ni attendre. Il leur envoya témoigner la part qu'il
prenait à leur perte par un gentilhomme ordinaire, et leur manda qu'il
donnait la même survivance au cadet qui leur restait. Pour les jésuites,
le cri universel fut prodigieux, mais il n'en fut autre chose.

\hypertarget{chapitre-v.}{%
\chapter{CHAPITRE V.}\label{chapitre-v.}}

1711

~

{\textsc{Commencement de l'affaire qui a produit la constitution
\emph{Unigenitus}.}} {\textsc{- Bagatelles d'Espagne.}} {\textsc{-
Maillebois resté otage à Lille, s'en sauve.}} {\textsc{- Étrange fin de
l'abbé de La Bourlie à Londres.}} {\textsc{- Mariage de Lassai\,; sa
famille.}} {\textsc{- Enfants de M. du Maine en princes du sang à la
chapelle.}} {\textsc{- Mort de la duchesse douairière d'Aumont\,; son
caractère.}} {\textsc{- Mort et famille de M\textsuperscript{me} de
Châteauneuf.}} {\textsc{- Mon embarras à l'égard de Monseigneur et de sa
cour intérieure.}}

~

Ce même mois de mars vit éclore les premiers commencements de l'affaire
qui produisit la constitution \emph{Unigenitus} si fatale à l'Église et
à l'État, si honteuse à Rome, si funeste à la religion, si avantageuse
aux jésuites, aux sulpiciens, aux ultramontains, aux ignorants, aux gens
de néant, et surtout à tout genre de fripons et de scélérats, dont les
suites, dirigées autant qu'il leur a été possible sur le modèle de celle
de la révocation de l'édit de Nantes, ont mis le désordre, l'ignorance,
la tromperie, la confusion partout, avec une violence qui dure encore,
sous l'oppression de laquelle tout le royaume tremble et gémit, et qui,
après plus de trente ans de la persécution la plus effrénée, en éprouve,
en tout genre et en toutes professions, un poids qui s'étend à tout, et
qui s'appesantit toujours. Je me garderai bien d'entreprendre une
histoire théologique, ni même celle qui serait bornée aux faits et aux
procédés\,; cette dernière partie seule composerait plusieurs volumes.
Il serait à désirer qu'il y en eût moins de donnés au public sur la
doctrine où bien des répétitions se trouvent multipliées, et qu'il y en
eût davantage sur l'historique de la naissance, du cours et des progrès
de cette terrible affaire\,; de ses suites, de ses branches, de la
conduite et des procédés des deux côtés\,; des fortunes, même
séculières, qui en sont nées, et qui en ont été ruinées\,; et des effets
si étendus et si prodigieux de l'ouverture de cette boîte de Pandore, si
fort au delà des espérances des uns et de l'étonnement des autres, qui
ont fait taire les lois, les tribunaux, les règles, pour faire place à
une inquisition militaire qui ne cesse point d'inonder la France de
lettres de cachet, et d'anéantir toute justice. Je me bornerai à ce peu
d'historique qui s'est passé sous mes yeux, et quelquefois par mes
mains, pour traiter cette matière comme j'ai tâché de traiter toutes les
autres, et laisser ce que je n'ai ni vu ni appris des acteurs à des
plumes plus instruites, meilleures et moins paresseuses.

Pour entendre ce peu qui de temps en temps sera rapporté d'une affaire
qui a si principalement occupé tout le reste du règne de Louis XIV, la
minorité de Louis XV et tout le règne, caché sous M. le Duc, et à
découvert depuis sa chute, du cardinal Fleury, il faut se souvenir de
bien des choses qui se trouvent éparses dans ces Mémoires, et qui
seraient trop longues et trop ennuyeuses à répéter ici, mais qu'il faut
remettre en deux mots sous les yeux, pour en donner le souvenir et le
moyen de se les rappeler aisément dans les lieux épars où elles se
trouvent rapportées. Il faut d'abord se remettre l'orage du quiétisme,
la disgrâce de M. de Cambrai\,; le danger des ducs de Chevreuse et de
Beauvilliers, qui fut extrême et qui n'a fait que resserrer les liens de
leur abandon à ce prélat\,; le triumvirat contre lui\,; la conduite
secrète des jésuites, dont le gros et le ministère public se déclara
contre lui, sans nuire, et le sanhédrin ténébreux et mystérieux le
servit de toutes ses forces, l'union qui en résulta\,; ce qui a été dit
de Saint-Sulpice, de Bissy, évêque de Toul, puis de Meaux et cardinal\,;
enfin du P. Tellier, conséquemment de l'état de l'épiscopat
soigneusement rempli de gens sans nom, sans lumières, de plusieurs sans
conscience et sans honneur, et de quelques-uns publiquement vendus à
l'ambition la plus déclarée, et à la servitude la plus parfaite du parti
qui les pouvait élever\,; l'affaire de la Chine, la situation si
fâcheuse des jésuites à cet égard, la part si personnelle que le P.
Tellier y prenait\,; la haine des jésuites et la sienne particulière
pour le cardinal de Noailles\,; l'usage si heureux qu'ils ont toujours
su faire du jansénisme\,; enfin le caractère du cardinal de Noailles, et
ce qu'on a vu de ceux du roi et de M\textsuperscript{me} de Maintenon.

Ces choses rappelées à l'esprit et à la mémoire, on se persuadera
aisément de l'extrême désir du P. Tellier de sauver les jésuites de
l'opprobre où leur condamnation sur la Chine les livrait, et d'abattre
le cardinal de Noailles. Pour frapper deux si puissants coups il fallait
une affaire éclatante, qui intéressât Rome en ce qu'elle a de plus
sensible, et sur laquelle elle ne pût espérer qu'en la protection du P.
Tellier. Il était sans cesse occupé d'en trouver les moyens et d'en
ménager la conjoncture. L'affaire de la Chine, qui ne lui laissait plus
le temps de différer, précipita son entreprise, dans laquelle il n'eut
pour conseil unique, à la totale exclusion de tous autres même jésuites,
que les PP. Doucin et Lallemant, aussi fins, aussi faux, aussi profonds
que lui, et dont les preuves étaient faites que les crimes ne leur
coûtaient rien, jésuites aussi furieux que lui, et aussi emportés contre
le cardinal de Noailles qui, pour quelques excès du P. Doucin, lui avait
fait ôter une pension du clergé, qu'il avait attrapée d'un temps de
faiblesse et de disgrâce des dernières années d'Harlay, archevêque de
Paris. Ces deux jésuites demeuraient à Paris en leur maison professe, où
le P. Tellier demeurait aussi\,; et tous trois par leur violence, leur
profondeur et leur méchanceté étaient secrètement la terreur de tous les
autres jésuites, jusqu'aux plus confits et les plus livrés aux vues, aux
sentiments et aux intérêts de la société.

Les conjonctures aussi parurent favorables au P. Tellier. Il avait par
M. de Cambrai les ducs de Chevreuse et de Beauvilliers\,; il avait
Pontchartrain par opposition à son père\,; et par basse politique il
avait d'Argenson\,; par ces deux hommes il était maître de faire revenir
au roi tout ce qui lui serait utile sans y paraître. L'alliance et la
liaison personnelle du cardinal de Noailles avec M\textsuperscript{me}
de Maintenon ne l'embarrassait plus. Elle était usée dans cet esprit
changeant. Trois hommes avaient succédé auprès d'elle à M. de
Chartres\,: l'évêque successeur et neveu à cause de Saint-Cyr, mais qui
à vingt-sept ou vingt-huit ans, en était pour ainsi dire à recevoir
encore du bonbon de sa main\,; La Chétardie, curé de Saint-Sulpice, son
confesseur, dont on a vu ailleurs l'extrême imbécillité, et Bissy,
évêque de Meaux, que feu M. de Chartres lui avait donné comme son
Élisée, qu'elle avait adopté sur le même pied, et qui, sans qu'elle s'en
aperçût, était à vendre et dépendre corps et âme, pour sa fortune, aux
jésuites, et plus particulièrement encore au P. Tellier et à ses deux
acolytes. C'était une suite de ses menées secrètes à Rome pour la
pourpre du temps qu'il était à Toul\,; et il s'était d'autant plus
attaché à eux, depuis sa translation à Meaux, que la confiance déclarée
de M\textsuperscript{me} de Maintenon en lui le leur rendait
très-considérable, comme eux à lui, en supplément à Rome des moyens
d'arriver, qui lui étaient retranchés par sa translation, qui faisait
cesser ses disputes avec M. de Lorraine. Quelque bien qu'il fût avec
M\textsuperscript{me} de Maintenon, le siége et l'alliance du cardinal
de Noailles avec elle, un reste de considération et de privance qu'elle
ne pouvait lui refuser, faisait toujours peur à l'évêque de Meaux, qui
par cet intérêt n'était pas moins ardent à la ruine du cardinal de
Noailles que le P. Tellier même. Tous ces côtés assurés, l'épiscopat ne
leur fit point de peur. Il faut se souvenir ici du crédit que feu M. de
Chartres avait emblé sur les nominations pendant les dernières années du
P. de La Chaise, et de quels misérables sujets il l'avait rempli, avec
les meilleures intentions du monde, et le P. Tellier avait renchéri par
art et dessein en pernicieux choix. Ainsi, ils méprisèrent le gros, et
ne doutèrent pas d'intimider et d'entraîner presque tous les autres.

Il ne faut pas oublier encore qu'avec toute l'aversion et la crainte de
ceux de Saint-Sulpice, des jésuites, et la jalousie et la haine de
ceux-ci pour ceux-là, ils convenaient entièrement sur tout ce qui
regardait jansénisme en détestation, et Rome en adoration\,: les uns par
le plus puissant intérêt, les autres par la plus grossière ignorance.
Ainsi, les jésuites menèrent en cette affaire Saint-Sulpice en laisse
tant qu'il leur plut, les yeux bandés, et s'en servirent à tous les
usages qu'ils voulurent.

Le plan dressé, et les mesures prises, il fut résolu d'exciter l'orage
sans y paraître, et de le faire tomber sur un livre intitulé
\emph{Réflexions morales sur le Nouveau Testament}, par le P. Quesnel,
et d'en choisir l'édition approuvée par le cardinal de Noailles, lors
évêque-comte de Châlons. Quel était le P. Quesnel, dont il a été
quelquefois mention dans ces Mémoires, et d'ailleurs si universellement
connu, ce serait chose superflue à expliquer. Ce livre avait été
approuvé par un grand nombre de prélats et de théologiens. Le célèbre
Vialart, prédécesseur à Châlons du cardinal de Noailles, en avait été
un. Son successeur, avec qui toute l'Église de France avait une grande
vénération pour un prélat d'une si grande réputation de piété et de
doctrine, ne balança pas, sur la même approbation, sans autre examen, et
à donner la sienne à une nouvelle édition qui s'en fit. Il y avait plus
de quarante ans que ce livre édifiait toute l'Église sans avoir reçu la
moindre contradiction. Bissy, évêque de Toul, qu'on a vu faire tant de
figure et de fortune à ses dépens, l'avait proposé à tout son diocèse\,;
et par un mandement publié, imprimé et fait exprès, avait recommandé à
tous ses curés d'en avoir chacun un exemplaire, en les assurant que,
dans l'impossibilité où leur peu de moyens les mettait d'avoir plusieurs
livres, celui-là seul leur suffirait pour y trouver, pour eux et pour
l'instruction de leurs peuples, toute la doctrine et toute la piété qui
leur étaient nécessaires. Le P. de La Chaise l'avait toujours sur sa
table\,; et sur ce qu'au nom de l'auteur quelques personnes lui en
parlèrent avec surprise, il leur répondit qu'il aimait le bien et le
bon, de quelque part qu'il vînt\,; que ses occupations lui ôtaient le
temps de faire des lectures\,; que ce livre était une mine de doctrine
et de piété excellente\,; que c'était pour suppléer à son peu de loisir
qu'il le voulait toujours sous sa main, parce que, dès qu'il avait
quelques moments, il l'ouvrait, et qu'il y trouvait toujours de quoi
s'édifier et s'instruire.

Il semblait qu'un livre si universellement lu et estimé, depuis un si
grand nombre d'années, et dont la bonté et la sûreté était annoncée dès
les premières pages par un si grand nombre d'approbateurs célèbres, eût
dû être à couvert de tout dessein de l'attaquer\,; mais l'exemple du
succès obtenu contre le livre \emph{de la fréquente Communion}, de M.
Arnauld, plus illustre encore par le nom de son auteur, le nombre, la
dignité, la réputation de ses approbateurs, l'applaudissement avec
lequel il fut reçu et lu, avait rassuré le P. Tellier contre de
pareilles craintes, et il ne douta point de le faire attaquer
coinjointement avec le cardinal de Noailles, comme l'ayant approuvé.

Pour un coup si hardi, il se servit de deux hommes les plus inconnus,
les plus isolés, les plus infimes, pour qu'ils pussent être moins
abordés, et plus dans sa parfaite dépendance. Champflour, évêque de la
Rochelle, était l'ignorance et la grossièreté même, qui ne savait
qu'être follement ultramontain, qui avait été exilé pour cela, lors des
propositions du clergé de 1682, et que Saint-Sulpice et les jésuites
réunis en faveur de ce martyr de leur cause favorite, avait à la fin
bombardé à la Rochelle. L'autre était Valderies de Lescure, moins
ignorant, mais aussi grossier et aussi ultramontain que l'autre, aussi
abandonné aux jésuites qui l'avaient fait évêque de Luçon, ardent,
impétueux et boute-feu par sa nature\,: celui-ci pauvre et petit
gentilhomme, l'autre le néant\,; et tous deux noyés dans la plus
parfaite obscurité et sans commerce avec personne.

Pour les dresser à ce qu'on leur voulut faire faire, on leur envoya un
prêtre nommé Chalmet, élève de Saint-Sulpice, perfectionné à Cambrai, et
bien instruit par le célèbre Fénelon, qui espérait son retour, et tout
ce qui le pouvait suivre de plus flatteur, de la chute de celui de ses
trois vainqueurs qui restait, et de l'appui du P. Tellier, appuyé
lui-même de ses anciens amis, mais qui ne pouvaient ouvrir la bouche en
sa faveur. Ce Chalmet avait de l'esprit et de la véhémence en pédant dur
et ferré, livré aux maximes ultramontaines de Saint-Sulpice, dévoué à M.
de Cambrai, et abandonné sans réserve aux jésuites, et en particulier au
P. Tellier. Il s'en alla donc secrètement en Saintonge, s'établit tantôt
à la Rochelle, tantôt à Luçon, et fort caché dans ces commencements, les
fit aboucher souvent tous deux en sa présence, les endoctrina, mais si
durement et si haut à la main qu'ils firent souvent leurs plaintes d'un
précepteur si absolu, et les ont depuis très-souvent renouvelées, avec
peu de jugement et de discrétion pour leur honneur.

Il leur fit faire un mandement en commun, portant condamnation du
\emph{Nouveau Testament} du P. Quesnel, de l'édition approuvée par le
cardinal de Noailles, lors évêque-comte de Châlons, avec une censure si
reconnaissable de ce prélat que personne ne l'y put méconnaître, comme
fauteur d'hérétiques, et avec les plus vives couleurs, sans aucune sorte
de ménagement. Cette pièce, qui était proprement un tocsin, n'était pas
faite pour demeurer ensevelie dans les diocèses de Luçon et de la
Rochelle. Elle fut non-seulement envoyée à Paris qu'on en inonda, mais,
contre toute règle ecclésiastique et de police, affichée partout, et
principalement aux portes de l'église et de l'archevêché de Paris, et ce
fut par où le cardinal de Noailles et tout Paris en eurent la-première
notion.

Ces deux évêques avaient chacun un neveu au séminaire de Saint-Sulpice,
fort sots enfants pour leur âge, et aussi peu capables que leurs oncles
de quoi que ce fût sans impulsion d'autrui, beaucoup moins d'une
publication de ce mandement si nerveuse, si prompte, si hardie, qui
marquait un concert entre plusieurs. Le cardinal de Noailles, si
étrangement outragé par deux évêques de campagne, commit la faute
capitale d'imiter le chien qui mord la pierre qu'on lui jette, et qui
laisse le bras qui l'a ruée. Il manda le supérieur du séminaire de
Saint-Sulpice, à qui il ordonna de mettre dehors de sa maison ces deux
jeunes gens, sitôt qu'il y serait retourné. Le supérieur représenta le
scandale d'un congé si subit, la vertu des deux ecclésiastiques, le tort
que cela ferait à leur réputation. Rien ne fut écouté. Le curé de
Saint-Sulpice, averti par le supérieur en arrivant de l'archevêché,
espéra mieux de son crédit. Sa piété et sa simplicité n'étaient pas à
l'abri de l'enflure que lui donnait la confiance entière de
M\textsuperscript{me} de Maintenon, et la considération mêlée de crainte
qui en résultait. Il courut à l'archevêché plein de cette confiance\,;
elle fut trompée. Il s'en revint plein d'indignation. Il fallut obéir
sur-le-champ. Mais il arriva que M\textsuperscript{me} de Maintenon fut
piquée du peu de considération que le cardinal de Noailles ayait montré
pour son cher directeur, dont Bissy, évêque de Meaux, sut bien profiter.

Cette expulsion fît grand vacarme. Le cardinal rendit compte au roi de
l'injure qu'il recevait, et lui en demanda justice. Le roi entra dans sa
peine, mais lui fit entendre qu'il avait commencé par se la faire\,; et
la chose traîna par la lenteur naturelle du cardinal, et par le délai de
ses audiences de huit jours en huit jours, qu'il ne crut pas devoir
prévenir.

Pendant ces intervalles on aigrissait le roi qui différait toujours,
mais qui aimait et respectait le cardinal. Le P. Tellier directement, et
le Meaux par M\textsuperscript{me} de Maintenon, retenaient le roi que
le cardinal ne pressait que mollement, et qui ne doutait pas d'obtenir
justice d'une chose si criante\,; tandis qu'on envoyait aux deux évoques
une lettre toute faite, pour le roi, à signer, qui la reçut par le P.
Tellier, à qui elle fut adressée comme au ministre naturel de tous les
évêques, et qui la présenta au roi comme une fonction de sa place qui ne
se pouvait refuser.

La lettre était également furieuse et adroite, et en commun des deux
évêques. Il ne fallait que jeter les yeux dessus, car elle devint
bientôt publique, pour voir que ces deux animaux mitrés n'y avaient eu
de part que leur signature, et qu'elle était du plus habile et du plus
délié courtisan, aussi bien que de l'écrivain le plus malicieusement
emporté. Après avoir comblé le roi d'éloges, et l'avoir comparé à
Constantin et à Théodose par son amour et sa protection pour l'Église,
ils la lui demandaient non pour eux-mêmes prosternés à ses pieds, ni
pour leurs neveux, mais pour l'Église, pour l'épiscopat, pour la liberté
de la bonne doctrine, et justice de l'attentat par lequel le cardinal de
Noailles prétendait l'opprimer, en montrant par l'exemple fait sur leurs
neveux ce que pouvait attendre tout homme soupçonné de défendre la bonne
cause, sans en être même convaincu, comme leurs neveux ne l'étaient pas
de la distribution ni de l'affiche de leur mandement. Après une longue
et forte prosopopée contre le P. Quesnel et ses \emph{Réflexions morales
sur le Nouveau Testament}, approuvées par le cardinal de Noailles, ils
le représentèrent, ce cardinal, comme un ennemi de l'Église, du pape et
du roi, tel que sous Constantin et ses premiers successeurs furent ces
évêques de la ville impériale qui faisaient tout trembler sous leur
autorité, et sous qui les évêques orthodoxes gémissaient. La lettre
était longue et se soutenait par tout le style, l'art qui perçait à
travers la ruse. Ce portrait si dissemblable au naturel, à la vie, aux
mœurs, à la conduite du cardinal de Noailles, l'emportement de toute la
pièce dévailait à nu le mystère d'iniquité, et découvrait à plein qu'une
lettre si hardie, si fine, si forte, n'avait pas été composée à la
Rochelle ni à Luçon, {[}mais{]} dans l'embarras de couvrir une attaque
faite de gaieté de cœur, avec l'éclat le plus irrégulier et le plus
injurieux, dont l'art était employé à profiter de l'expulsion des neveux
du séminaire de Saint-Sulpice, pour irriter un roi si jaloux de son
autorité et pour changer l'état de la question, se rendre agresseurs, et
réduire le cardinal à la défensive.

C'est ce qui lui arriva en effet. Il avait été bien reçu sur les
plaintes des injures du mandement\,; l'expulsion des neveux lui avait
été plutôt remise devant les yeux que reprochée\,; mais quand il voulut
porter ses plaintes de la lettre, le roi, qu'on avait eu le temps
d'aigrir et de préparer, revint sèchement aux neveux, avec un reproche
amer de s'être fait justice au lieu de l'attendre de lui. Néanmoins,
quoique pris à un hameçon si grossier, il demeura encore plus choqué de
l'insolence des deux évêques. Il laissa voir au cardinal qu'il sentait
que la querelle sur le livre était aussi peu nécessaire que peu
attendue, après un si long espace de la réputation non interrompue de
cet ouvrage, et qu'ils lui en voulaient moins qu'à sa personne.

Ce fut une seconde et très-lourde faute du cardinal de n'avoir pas porté
le mandement et la lettre à cette audience.

Pour peu qu'il en eût lu au roi quelques endroits principaux en injures
et en adresse, qu'il eût su les paraphraser, profiter de la disposition
du roi à cet égard, lui faire sentir la cabale, le désir de faire du
bruit, et combien deux plats évêques de campagne étaient peu capables
d'eux-mêmes d'enfanter ce dessein, et de l'exécuter avec tant d'art,
d'éclat et de hauteur, il aurait déterminé le roi à imposer de façon que
l'affaire aurait été dès là étouffée. Mais le cardinal lent, doux, peu
né pour la cour et pour les affaires, plein de confiance en sa
conscience et en ce qu'il était en soi et auprès du roi, se tint pour
content d'avoir remis les choses, à la fin de son audience, où elles en
étaient avant la lettre des deux évêques, et ne douta point de recevoir
une satisfaction convenable, telle que le roi la lui avait promise
lorsqu'il lui en avait parlé la première fois.

À son tour le P. Tellier eut son audience. Il y eut moyen de piquer le
roi de nouveau sur son autorité, et sur la protection due à des prélats
infimes et abandonnés, qui se trouvaient à la veille d'être persécutés
pour la bonne doctrine. L'évêque de Meaux avait de son côté travaillé
auprès de M\textsuperscript{me} de Maintenon, de manière que, lorsque
huit jours après le cardinal de Noailles revint à l'audience, il fut
bien étonné que le roi lui fermât la bouche sur cette affaire, et lui
déclarât que, puisque sans lui il s'était fait justice à lui-même, il
n'avait qu'à s'en tirer tout comme il voudrait sans l'y mêler davantage,
et que c'était tout ce qu'il pouvait de plus en sa faveur. C'était bien
là où on en voulait venir pour les deux évêques, qui ne s'étaient
plaints que pour se soustraire à ce que méritait l'injure qu'ils avaient
faite, et qui, ainsi mis hors de cour, se trouvaient après une calomnie
si publique, et sur la foi, égalés au cardinal de Noailles, malgré tant
et de si grandes disproportions.

Dans ce fâcheux état, le cardinal dit au roi que, puisqu'il
l'abandonnait à la calomnie et à l'insulte, sans même avoir pu mériter
ni deviner ce qui lui arrivait, il le suppliait au moins de trouver bon
qu'il se défendît\,; et il se retira avec la sèche permission de faire
tout ce qu'il jugerait à propos.

Deux jours après il publia un mandement court et fort, par lequel il
prétendit montrer diverses erreurs dans celui des deux évêques. Il l'y
traita de libelles fait sous leur nom, dont il disait assez peu à propos
qu'il les croyait incapables, s'éleva contre l'inquiétude du temps, sur
la doctrine et sur la licence de quelques évêques de s'ingérer dans la
moisson d'autrui, défendit sous les peines de droit la lecture de ce
mandement qu'il flétrit en plusieurs manières. Il semblait qu'il eût
droit d'en user de la sorte, par l'abandon et par la permission du roi,
et que c'était encore avec ménagement par rapport à la nature de la
chose. Néanmoins ce fût un nouveau crime, qui lui fit envoyer défense
d'aller à la cour s'il n'y était mandé.

Les deux évêques, c'est-à-dire ceux qui les mettaient en avant,
profitant du succès de leur trame, écrivirent de nouveau. Hébert, de la
congrégation de la Mission, avait acquis une grande et juste réputation
étant curé de Versailles. Le cardinal de Noailles lui avait fait donner
l'évêché d'Agen, nonobstant les constitutions de cette congrégation qui
excluent leurs membres de l'épiscopat. Il faisait merveilles dans son
diocèse, où il était conprovincial des deux évêques. Il leur écrivit une
excellente lettre, savante, fort pieuse, par laquelle il leur
représenta, avec beaucoup de modestie épiscopale, le tort extrême qu'ils
avaient de troubler l'Église, et d'attaquer personnellement le cardinal
de Noailles.

Cependant ses ennemis ne dormaient pas et travaillaient à lui en
susciter d'autres. Il parut un mandement de Berger de Malissoles, évêque
de Gap, moins grossier, mais aussi mordant, que le cardinal défendit par
un autre, comme il avait fait celui des deux évêques. Ensuite il écrivit
une belle lettre à l'évêque d'Agen, contenant l'histoire de tout ce qui
s'était passé, mais avec une mesure et une modestie qui la relevait
encore, et qui fut comme un manifeste de sa part qui fut distribué
partout. L'affaire en elle-même avait indigné tout ce qui n'était pas
dévoué aux jésuites ou à la fortune, ou aveuglé de l'abus qui se faisait
du jansénisme pour décrier et perdre qui on voulait. Ce manifeste acheva
d'enlever ce qui restait de gens neutres, et fit un tel effet que les
agresseurs, qui pensaient déjà avoir étourdi le cardinal de Noailles, en
furent effrayés, et ne songèrent que plus efficacement aux moyens de
profiter de tous leurs avantages, et de le pousser en si beau chemin.
J'en demeurerai là pour le présent, il est temps de rentrer en d'autres
matières.

L'Espagne, comme je l'ai dit d'avance, produisit peu de choses cette
année. Ses incroyables efforts l'avaient trop épuisée pour pouvoir
profiter, par de nouveaux succès, de ceux qu'ils avaient produits contre
toute espérance\,; et les ennemis, battus contre la leur, après un court
triomphe, n'étaient pas en état de se relever. Ils abandonnèrent
Balaguer, où ils n'avaient que deux ou trois cents hommes, sur le bruit
qu'il allait être assiégé. Bientôt après, Muret, lieutenant général,
prit la Seu-d'Urgel\,; mais peu après, le gouverneur de
Miranda-de-Duero, place importante sur la frontière de Portugal, se
laissa corrompre, et vendit pour une grosse somme d'argent aux Portugais
la place et mille hommes qu'il avait dedans. Bientôt après, en Sicile,
les Autrichiens se saisirent de Palerme.

Maillebois, fils de Desmarets, à qui sa femme et le cardinal Fleury ont
longtemps depuis fait faire un si grand et si triste personnage, était
toujours à Lille, depuis sa prise, demeuré par la capitulation en otage,
avec un commissaire des guerres, de ce qui était dû aux magistrats et
aux bourgeois de la ville. Ils surent que, pour en presser le payement,
on était sur le point de les enfermer dans la citadelle, contre la
teneur de la capitulation. Ils se sauvèrent, et gagnèrent Arras avec une
escorte que le maréchal de Montesquiou envoya à mi-chemin au-devant
d'eux. D'Arras, ils écrivirent au comte d'Albemarle, qui commandait en
Flandre pour les ennemis, et lui rendirent raison de leur conduite\,; et
de là Maillebois vint à la cour, où le roi l'entretint longtemps dans
son cabinet, Desmarest seul en tiers. Il avait rencontré en chemin
Surville, en otage aussi à Tournai, d'où il avait eu permission de faire
un tour chez lui, et qui s'en retournait à Tournai. Maillebois l'avertit
de son aventure, lui fît peur d'être mis dans la citadelle de Tournai,
tellement que Surville s'en retourna chez lui en Picardie, en attendant
les ordres du roi là-dessus.

J'ai parlé ailleurs de l'abbé de La Bourlie, frère de Guiscard, qui,
ayant plusieurs bénéfices et nul mécontentement, passa en Hollande et en
Angleterre, promit merveilles aux Cévennes qu'il ne tint pas, et publia
des libelles très-séditieux par le Languedoc. Traître à sa patrie, il ne
fut pas plus fidèle à ceux à qui il s'était donné. Je ne sais de quoi il
se mêla contre le ministère, mais à la fin de mars il fut arrêté à
Londres, dans le parc de Saint-James, par ordre de la reine, pour des
commerces suspects. Conduit chez Saint-Jean, secrétaire d'État, il se
saisit d'un canif qu'il trouva sur une table de l'antichambre, sans
qu'on s'en aperçût\,; il entra dans le cabinet où il était attendu par
les ducs d'Ormond, de Buckingham et d'Argyle, et par les deux
secrétaires d'État Harley et Saint-Jean. Le premier l'interrogea. Au
lieu de lui répondre, il lui donna deux coups de canif dans le ventre,
qui heureusement ne firent que glisser légèrement. On se jeta sur ce
galant homme, qui reçut trois coups d'épée\,; il fallut le lier pour le
panser à la prison de Newgate, où on le mena. Il demanda à parler en
particulier au duc d'Ormond, qui y fut. Ce malheureux y mourut peu de
jours après, sans avoir voulu prendre de nourriture ni parler, et des
blessures qu'il se fit.

Lassai maria, en ce temps-ci, son fils à sa sœur. Leur nom est
Madaillan, trop connu dans l'histoire de la vie du fameux duc d'Épernon
sur la fin. Lassai avait fait toutes sortes de métiers, dont
M\textsuperscript{me} la Duchesse a fait une chanson qui les décrit
d'une manière très-plaisante et peu flatteuse. Elle ne se doutait pas
alors de ce qui lui est arrivé depuis avec son fils.

Le père avait été marié plusieurs fois, et mal toutes. Il épousa en
secondes noces la fille d'un apothicaire, que le duc Charles IV de
Lorraine avait voulu épouser aussi, et dont il ne put être empêché que
par force. Lassai la perdit, et, dans le désespoir de son amour, il se
retira dans la plus grande solitude auprès des Incurables, et dans une
grande dévotion. Quelques années le consolèrent. L'ennui le prit, il
ajusta sa maison et chercha à se remettre dans le monde. Il avait de
l'esprit, de la lecture, de la valeur\,; il avait peu servi, et fait
après le noble de province, avant sa retraite. Le voyage des princes de
Conti en Hongrie lui parut propre pour en sortir tout à fait. Comme ils
y allèrent contre le gré du roi, ils étaient fort seuls. Tout leur fut
bon\,; Lassai les suivit. Au retour, l'un étant mort, l'autre exilé à
Chantilly, Lassai s'attacha à M. le Duc, se fourra dans ses parties
obscures, y fut acteur commode, s'intrigua vainement, mais tant qu'il
put. Il épousa une bâtarde de M. le Prince, qui mourut folle quelques
années après. Il fréquenta la cour sans avoir jamais pu en être.

Son fils servit et fut brigadier d'infanterie, non sans talent et avec
beaucoup d'esprit. Par son père il se trouva attaché à la maison de
Condé. Avec un visage de singe, il était parfaitement bien fait. Il plut
à M\textsuperscript{me} la Duchesse vers ce temps-ci de son mariage avec
sa tante\,; elle le trouva sous sa main\,; la liaison entre eux se fit
la plus intime, et la plus étrangement publique. Il devint à visage
découvert le maître de M\textsuperscript{me} la Duchesse et le directeur
de toutes ses affaires. Il y eut bien quelque voile de gaze là-dessus
pendant le reste de la vie du roi, qui ne laissa pas de le voir, mais
qui, dans ses fins, laissait aller bien des choses, de peur de se fâcher
et de se donner de la peine\,; mais après lui il n'y eut plus de mesure.
Cela se retrouvera en son temps.

C'est ce qui fit son père chevalier de l'ordre, en la promotion de 1724,
si abondante en étranges choix. Lassai père a vécu très-vieux, fade et
abandonné adulateur du cardinal Fleury, qui avalait ses louanges à longs
traits et lui en savait le meilleur gré du monde. Ce pauvre flatteur se
cramponnait au monde qu'il fatiguait, et mourut enfin en homme qui avait
quitté Dieu pour le monde. Il avait eu une fille de son premier mariage,
qui épousa le dernier de cette ancienne et illustre race des Coligny, de
laquelle il sera parlé dans la suite. De la fille de l'apothicaire il
eut son fils, et de la bâtarde de M. le Prince et de la Montalais, dont
M\textsuperscript{me} de Sévigné parle si plaisamment dans ses lettres,
il eut une fille qu'il maria au fils de M. d'O. Elle fut galante, et
après folle, et mourut à l'hôtel de Condé. Elle ne laissa qu'une fille,
belle comme le jour, à qui Lassai, plein de millions et sans enfants ni
parents, donna prodigieusement, pour épouser le fils du duc de
Villars-Brancas, dont la noce se fit chez M\textsuperscript{me} la
Duchesse, comme de sa petite-nièce bâtarde. C'est peut-être une des
moindres infamies où ce duc de Villars-Brancas soit tombé.

Les enfants de M. du Maine triomphèrent toute la semaine sainte en rang
de princes du sang. La joie de M. et de M\textsuperscript{me} du Maine
en fut grande, la complaisance que le roi en prit extrême, le scandale
encore plus fort.

La duchesse douairière d'Aumont mourut le jour de Pâques, assez
brusquement, à soixante et un ans, veuve depuis sept ans, et peu
regrettée dans sa famille. Elle était sœur aînée des duchesses de
Ventadour et de La Ferté, et n'eut d'enfants que le duc d'Humières.
C'était une grande et grosse femme, qui avait eu plus de grande mine que
de beauté\,; impérieuse, méchante, difficile à vivre, grande joueuse,
grande dévote à directeurs. Elle avait été fort du grand monde et de la
cour, où elle ne paraissait plus depuis beaucoup d'années\,; elle était
riche et fut très-attachée à son bien. Le roi lui donnait dix mille
livres de pension. Il envoya un gentilhomme ordinaire faire compliment
aux ducs d'Humières et d'Aumont, et aux duchesses de Ventadour, La
Ferté, Aumont et d'Humières. Monseigneur, Mgr {[}le duc{]} et
M\textsuperscript{me} la duchesse de Bourgogne, M. {[}le duc{]} et
M\textsuperscript{me} la duchesse de Berry, et Madame, allèrent voir la
duchesse de Ventadour. J'ai parlé ailleurs de la suppression de la
visite aux duchesses et princesses étrangères\,; celle-ci fut donnée à
la place de gouvernante des enfants de France, et de fille de la
maréchale de La Mothe qui avait été la leur. Madame y fut par amitié, et
comme ayant été sa dame d'honneur.

M\textsuperscript{me} de Châteauneuf mourut quelques semaines après, à
cinquante-cinq ans, à Versailles où elle n'avait presque bougé de sa
chambre, et y avait passé sa vie fort seule. Elle était d'une
prodigieuse grosseur, la meilleure femme du monde, et veuve depuis onze
ans du secrétaire d'État, et mère de La Vrillière. Elle était fille de
Fourcy, conseiller au grand conseil, et d'une sœur d'un premier lit
d'Armenonville, depuis garde des sceaux, qui avait plus de vingt ans
plus que lui, et qui se remaria à Pelletier, depuis ministre d'État et
contrôleur général des finances, qui fit la fortune d'Armenonville.

Cette année le dimanche de Pâques échut au 5 avril. Le mercredi suivant
8, Monseigneur, au sortir du conseil, alla dîner à Meudon en
\emph{parvulo}, et y mena M\textsuperscript{me} la duchesse de Bourgogne
tête à tête. On a expliqué ailleurs ce que c'était que ces
\emph{parvulo}. Les courtisans avaient demandé pour Meudon, où le voyage
devait être de huit jours, jusqu'à celui de Marly annoncé pour le
mercredi suivant Je m'en étais allé dès le lundi saint, pour me trouver
à Marly le même jour que le roi. Les Meudons m'embarrassaient
étrangement. Depuis cette rare crédulité de Monseigneur qui a été
rapportée, et que M\textsuperscript{me} la duchesse de Bourgogne l'avait
dépersuadé, jusqu'à lui en avoir fait honte, je n'avais osé me commettre
à Meudon\,: c'était pour moi un lieu infesté de démons.
M\textsuperscript{me} la Duchesse, délivrée des bienséances de sa
première année, y retournait régner, et y menait M\textsuperscript{lle}s
ses filles\,; d'Antin y gouvernait\,; M\textsuperscript{lle} de
Lislebonne et sa sœur y dominaient à découvert\,; c'étaient mes ennemis
personnels\,; ils gouvernaient Monseigneur\,; c'était bien certainement
à eux à qui je devais cet inepte et hardi \emph{godant} qu'ils avaient
donné à Monseigneur, et qui l'avait mis dans une si grande colère.
Capable de prendre à celui-là, et eux capables d'oser l'inventer, et y
réussir en plein, à quoi ne pouvais-je point m'attendre\,! tout ce qui
était là à leurs pieds ne songeant qu'à leur plaire, et ne pouvant
espérer que par eux\,; par conséquent moi tout à en craindre, dès qu'il
conviendrait à des ennemis si autorisés de me susciter quelque nouvelle
noirceur sur leur terrain\,; M\textsuperscript{lle} Choin, la vraie
tenante, en mesures extrêmes et en tous ménagements pour eux, fée
invisible dont on n'approchait point, et moi moins que personne, et qui
en étant inconnu ne pouvais rien espérer d'elle, et du Mont pour toute
ressource, lequel sans force et sans esprit. Je ne pouvais douter qu'ils
ne me voulussent perdre après l'échantillon que j'en avais éprouvé\,; et
ce qui les excitait contre moi n'était pas de nature à s'émousser,
beaucoup moins à pouvoir jamais me raccommoder avec eux. Ce qui s'était
passé à l'égard de feu M. le Duc et de M\textsuperscript{me} la
Duchesse, les choses de rang à l'égard des deux Lorraines et de leur
oncle le Vaudemont\,; l'affaire de Rome pour d'Antin, et de nouveau sa
prétention d'Épernon\,; les choses de Flandre, ma liaison intime avec ce
qu'ils ne songeaient qu'à anéantir, Mgr {[}le duc{]} et
M\textsuperscript{me} la duchesse de Bourgogne, M. {[}le duc{]} et
M\textsuperscript{me} la duchesse d'Orléans, les ducs de Chevreuse et de
Beauvilliers\,; la part qu'ils me donnaient au mariage de M. le duc de
Berry qui avait comblé leur rage, c'en était trop, et sans aucun
contre-poids, pour ne me pas faire regarder cette cour comme hérissée
pour moi de dangers et d'abîmes.

Je poussais donc le temps avec l'épaule sur les voyages de Meudon,
embarrassé de Monseigneur et du monde, en ne m'y présentant jamais,
beaucoup plus en peine d'y hasarder des voyages. Si ce continuel présent
me causait ces soucis, combien de réflexions plus fâcheuses\,: la
perspective d'un avenir qui s'avançait tous les jours, qui mettrait
Monseigneur sur le trône, et qui, à travers le chamaillis de ce qui le
gouvernait et le voudrait dominer alors à l'exclusion des autres,
porterait très-certainement sur le trône avec lui les uns ou les autres
de ces mêmes ennemis qui ne respiraient que ma perte, et à qui elle ne
coûterait alors que le vouloir\,! Faute de mieux, je me soutenais de
courage. Je me disais qu'on n'éprouvait jamais ni tout le bien ni tout
le mal qu'on avait, à ce qu'il semblait, le plus de raison de prévoir.
J'espérais ainsi, contre toute espérance, de l'incertitude attachée aux
choses de cette vie, et je coulais le temps ainsi à l'égard de l'avenir,
mais dans le dernier embarras sur le présent pour Meudon.

J'allai donc rêver et me délasser à mon aise, pendant cette quinzaine de
Pâques, loin du monde et de la cour, qui, à celle de Monseigneur près,
n'avait pour moi rien que de riant\,; mais cette épine, et sans remède,
m'était cruellement poignante, lorsqu'il plut à Dieu de m'en délivrer au
moment le plus inattendu. Je n'avais à la Ferté que M. de Saint-Louis,
vieux brigadier de cavalerie fort estimé du roi, de M. de Turenne et de
tout ce qui l'avait vu servir, retiré depuis trente ans dans l'abbatial
de la Trappe, où il menait une vie fort sainte\,; et un gentilhomme de
Normandie qui avait été capitaine dans mon régiment, et qui m'était fort
attaché. Je m'étais promené avec eux tout le matin du samedi 11, veille
de la Quasimodo, et j'étais entré seul dans mon cabinet un peu avant le
dîner, lorsqu'un courrier, que M\textsuperscript{me} de Saint-Simon
m'envoya, m'y rendit une lettre d'elle qui m'apprit la maladie de
Monseigneur.

\hypertarget{chapitre-vi.}{%
\chapter{CHAPITRE VI.}\label{chapitre-vi.}}

1711

~

{\textsc{Maladie de Monseigneur.}} {\textsc{- Le roi à Meudon.}}
{\textsc{- Le roi mal à son aise hors de ses maisons\,;
M\textsuperscript{me} de Maintenon encore plus.}} {\textsc{- Contrastes
dans Meudon.}} {\textsc{- Versailles.}} {\textsc{- Harengères à Meudon
bien reçues.}} {\textsc{- Singulière conversation avec
M\textsuperscript{me} la duchesse d'Orléans chez moi.}} {\textsc{-
Spectacle de Meudon.}} {\textsc{- Extrémité de Monseigneur.}} {\textsc{-
Mort de Monseigneur.}} {\textsc{- Le roi va à Marly.}} {\textsc{-
Spectacle de Versailles.}} {\textsc{- Surprenantes larmes de M. le duc
d'Orléans.}}

~

Ce prince, allant, comme je l'ai dit, à Meudon le lendemain des fêtes de
Pâques, rencontra à Chaville un prêtre qui portait Notre-Seigneur à un
malade, et mit pied à terre pour l'adorer à genoux, avec
M\textsuperscript{me} la duchesse de Bourgogne. Il demanda à quel malade
on le portait\,; il apprit que ce malade avait la petite vérole. Il y en
avait partout quantité. Il ne l'avait eue que légère volante, et
enfant\,; il la craignait fort. Il en fut frappé, et dit le soir à
Boudin, son premier médecin, qu'il ne serait pas surpris s'il l'avait.
La journée s'était cependant passée tout à fait à l'ordinaire.

Il se leva le lendemain jeudi, 9, pour aller courre le loup\,; mais, en
s'habillant, il lui prit une faiblesse qui le fit tomber dans sa chaise.
Boudin le fit remettre au lit. Toute la journée fut effrayante par
l'état du pouls. Le roi, qui en fut faiblement averti par Fagon, crut
que ce n'était rien, et s'alla promener à Marly après son dîner, où il
eut plusieurs fois des nouvelles de Meudon. Mgr {[}le duc{]} et
M\textsuperscript{me} la duchesse de Bourgogne y dînèrent, et ne
voulurent pas quitter Monseigneur d'un moment. La princesse ajouta aux
devoirs de belle-fille toutes les grâces qui étaient en elle, et
présenta tout de sa main à Monseigneur. Le cœur ne pouvait pas être
troublé de ce que l'esprit lui faisait envisager comme possible\,; mais
les soins et l'empressement n'en furent pas moins marqués, sans air
d'affectation ni de comédie. Mgr le duc de Bourgogne, tout simple, tout
saint, tout plein de ses devoirs, les remplit outre mesure\,; et,
quoiqu'il y eût déjà un grand soupçon de petite vérole, et que ce prince
ne l'eût jamais eue, ils ne voulurent pas s'éloigner un moment de
Monseigneur, et ne le quittèrent que pour le souper du roi.

À leur récit, le roi envoya le lendemain vendredi, 10, des ordres si
précis à Meudon qu'il apprit à son réveil le grand péril où on trouvait
Monseigneur. Il avait dit la veille, en revenant de Marly, qu'il irait
le lendemain matin à Meudon, pour y demeurer pendant toute la maladie de
Monseigneur, de quelque nature qu'elle pût être\,; et en effet il s'y en
alla au sortir de la messe. En partant, il défendit à ses enfants d'y
aller. Il le défendit en général à quiconque n'avait pas eu la petite
vérole, avec une réflexion de bonté, et permit à tous ceux qui l'avaient
eue de lui faire leur cour à Meudon, ou de n'y aller pas, suivant le
degré de leur peur ou de leur convenance.

Du Mont renvoya plusieurs de ceux qui étaient de ce voyage de Meudon,
pour y loger la suite du roi, qu'il borna à son service le plus étroit
et à ses ministres, excepté le chancelier, qui n'y coucha pas, pour y
travailler avec eux. M\textsuperscript{me} la Duchesse et
M\textsuperscript{me} la princesse de Conti, chacune uniquement avec sa
dame d'honneur\,; M\textsuperscript{lle} de Lislebonne,
M\textsuperscript{me} d'Espinoy et M\textsuperscript{lle} de Melun,
comme si particulièrement attachées à Monseigneur, et
M\textsuperscript{lle} de Bouillon, parce qu'elle ne quittait point son
père, qui suivit comme grand chambellan, y avaient devancé le roi, et
furent les seules dames qui y demeurèrent, et qui mangèrent les soirs
avec le roi, qui dîna seul comme à Marly. Je ne parle point de
M\textsuperscript{lle} Choin qui y dîna dès le mercredi, ni de
M\textsuperscript{me} de Maintenon, qui vint trouver le roi après dîner
avec M\textsuperscript{me} la duchesse de Bourgogne. Le roi ne voulut
point qu'elle approchât de l'appartement de Monseigneur et la renvoya
assez promptement. C'est où en étaient les choses lorsque
M\textsuperscript{me} de Saint-Simon m'envoya le courrier, les médecins
souhaitant la petite vérole, dont on était persuadé, quoiqu'elle ne fût
pas encore déclarée.

Je continuerai à parler de moi avec la même vérité dont {[}je{]} traite
les autres et les choses, avec toute l'exactitude qui m'est possible. À
la situation où j'étais à l'égard de Monseigneur et de son intime cour,
on sentira aisément quelle impression je reçus de cette nouvelle. Je
compris, par ce qui m'était mandé de l'état de Monseigneur, que la chose
en bien ou en mal serait promptement décidée\,; je me trouvais fort à
mon aise à la Ferté\,; je résolus d'y attendre des nouvelles de la
journée. Je renvoyai un courrier à M\textsuperscript{me} de Saint-Simon,
et je lui en demandai un pour le lendemain. Je passai la journée dans un
mouvement vague et de flux et de reflux qui gagne et qui perd du
terrain, tenant l'homme et le chrétien en garde contre l'homme et le
courtisan, avec cette foule de choses et d'objets qui se présentaient à
moi dans une conjoncture si critique, qui me faisait entrevoir une
délivrance inespérée, subite, sous les plus agréables apparences pour
les suites.

Le courrier que j'attendais impatiemment arriva le lendemain, dimanche
de Quasimodo, de bonne heure dans l'après-dînée. J'appris par lui que la
petite vérole était déclarée, et allait aussi bien qu'on le pouvait
souhaiter\,; et je le crus d'autant mieux que j'appris que la veille,
qui était celle du dimanche de Quasimodo, M\textsuperscript{me} de
Maintenon, qui à Meudon ne sortait point de sa chambre, et qui y avait
M\textsuperscript{me} de Dangeau pour toute compagnie, avec qui elle
mangeait, était allée dès le matin à Versailles, y avait dîné chez
M\textsuperscript{me} de Caylus où elle avait vu M\textsuperscript{me}
la duchesse de Bourgogne, et n'était pas retournée de fort bonne heure à
Meudon.

Je crus Monseigneur sauvé, et voulus demeurer chez moi\,; néanmoins je
crus conseil, et comme j'ai fait toute ma vie, et m'en suis toujours
bien trouvé. Je donnai ordre à regret pour mon départ le lendemain, qui
était celui de la Quasimodo, 13 avril, et je partis en effet de bon
matin. Arrivant à la Queue, à quatorze lieues de la Ferté et à six de
Versailles, un financier, qui se nommait La Fontaine, et que je
connaissois fort pour l'avoir vu toute ma vie à la Ferté chargé de
Senonches et des autres biens de feu M. le Prince de ce voisinage,
aborda ma chaise comme je relayais. Il venait de Paris et de Versailles
où il avait vu des gens de M\textsuperscript{me} la Duchesse\,; il me
dit Monseigneur le mieux du monde, et avec des détails qui le faisaient
compter hors de danger. J'arrivai à Versailles rempli de cette opinion,
qui me fut confirmée par M\textsuperscript{me} de Saint-Simon et tout ce
que je vis de gens, en sorte qu'on ne craignait plus que par la nature
traîtresse de cette sorte de maladie dans un homme de cinquante ans fort
épais.

Le roi tenait son conseil et travaillait le soir avec ses ministres,
comme à l'ordinaire. Il voyait Monseigneur les matins et les soirs, et
plusieurs fois l'après-dînée, et toujours longtemps dans la ruelle de
son lit. Ce lundi que j'arrivai, il avait dîné de bonne heure, et
s'était allé promener à Marly, où M\textsuperscript{me} la duchesse de
Bourgogne l'alla trouver. Il vit en passant au bord des jardins de
Versailles Mgrs ses petits-fils qui étaient venus l'y attendre, mais
qu'il ne laissa pas approcher, et leur cria bonjour.
M\textsuperscript{me} la duchesse de Bourgogne avait eu la petite
vérole, mais il n'y paraissait point.

Le roi ne se plaisait que dans ses maisons et n'aimait point être
ailleurs. C'est par ce goût que ses voyages à Meudon étaient rares et
courts, et de pure complaisance. M\textsuperscript{me} de Maintenon s'y
trouvait encore plus déplacée. Quoique sa chambre fût partout un
sanctuaire où il n'entrait que des femmes de la plus étroite privance,
il lui fallait partout une autre retraite entièrement inaccessible,
sinon à M\textsuperscript{me} la duchesse de Bourgogne, encore pour des
instants, et seule. Ainsi elle avait Saint-Cyr pour Versailles et pour
Marly, et à Marly encore ce repos dont j'ai parlé ailleurs\,; à
Fontainebleau sa maison à la ville. Voyant donc Monseigneur si bien, et
conséquemment un long séjour à Meudon, les tapissiers du roi eurent
l'ordre de meubler Chaville, maison du feu chancelier Le Tellier, que
Monseigneur avait achetée et mise dans le parc de Meudon\,; et ce fut à
Chaville où M\textsuperscript{me} de Maintenon destina ses retraites
pendant la journée.

Le roi avait commandé la revue des gens d'armes et des chevau-légers
pour le mercredi, tellement que tout semblait aller à souhait. J'écrivis
en arrivant à Versailles à M. de Beauvilliers, à Meudon, pour le prier
de dire au roi que j'étais revenu sur la maladie de Monseigneur\,; et
que je serais allé à Meudon si, n'ayant pas eu la petite vérole, je ne
me trouvais dans le cas de la défense. Il s'en acquitta, me manda que
mon retour avait été fort à propos, et me réitéra de la part du roi la
défense d'aller à Meudon, tant pour moi que pour M\textsuperscript{me}
de Saint-Simon qui n'avait point eu non plus la petite vérole. Cette
défense particulière ne m'affligea point du tout. M\textsuperscript{me}
la duchesse de Berry, qui l'avait eue, n'eut point le privilège de voir
le roi comme M\textsuperscript{me} la duchesse de Bourgogne\,; les deux
époux ne l'avaient point eue. La même raison exclut M. le duc d'Orléans
de voir le roi\,; mais M\textsuperscript{me} la duchesse d'Orléans, qui
n'était pas dans le même cas, eut permission de l'aller voir, dont elle
usa pourtant fort sobrement. Madame ne le vit point, quoiqu'il n'y eût
point pour elle des raisons d'exclusion, qui, excepté les deux fils de
France, par juste crainte pour eux, ne s'étendit dans la famille royale
que selon le goût du roi.

Meudon, pris en soi, avait aussi ses contrastes. La Choin y était dans
son grenier\,; M\textsuperscript{me} la Duchesse, M\textsuperscript{lle}
de Lislebonne et M\textsuperscript{me} d'Espinoy, ne bougeaient de la
chambre de Monseigneur, et la recluse n'y entrait que lorsque le roi n'y
était pas, et que M\textsuperscript{me} la princesse de Conti, qui y
était aussi fort assidue, était retirée. Cette princesse sentit bien
qu'elle contraindrait cruellement Monseigneur si elle ne le mettait en
liberté là-dessus, et elle le fit de fort bonne grâce. Dès le matin du
jour que le roi arriva (et elle y avait déjà couché), elle dit à
Monseigneur qu'il y avait longtemps qu'elle n'ignorait pas ce qui était
dans Meudon\,; qu'elle n'avait pu vivre hors de ce château dans
l'inquiétude où elle était, mais qu'il n'était pas juste que cette
amitié fût importune\,; qu'elle le priait d'en user très-librement, de
la renvoyer toutes les fois que cela lui conviendrait\,; et qu'elle
aurait soin, de son côté, de n'entrer jamais dans sa chambre sans savoir
si elle pouvait le voir sans l'embarrasser. Ce compliment plut
infiniment à Monseigneur. La princesse fut en effet fidèle à cette
conduite, et docile aux avis de M\textsuperscript{me} la Duchesse et des
deux Lorraines pour sortir quand il était à propos, sans air de chagrin
ni de contrainte. Elle revenait après quand cela se pouvait, sans la
plus mauvaise humeur, en quoi elle mérita de vraies louanges.

C'était M\textsuperscript{lle} Choin dont il était question, qui
figurait à Meudon, avec le P. Tellier, d'une façon tout à fait étrange.
Tous deux incognito, relégués chacun dans leur grenier, servis seuls
chacun dans leur chambre, vus des seuls indispensables, et sus pourtant
de chacun, avec cette différence que la demoiselle voyait Monseigneur
nuit et jour sans mettre le pied ailleurs, et que le confesseur allait
chez le roi et partout, excepté dans l'appartement de Monseigneur ni
dans tout ce qui en approchait. M\textsuperscript{me} d'Espinoy portait
et rapportait les compliments entre M\textsuperscript{me} de Maintenon
et M\textsuperscript{lle} Choin. Le roi ne la vit point. Il croyait que
M\textsuperscript{me} de Maintenon l'avait vue, il le lui demanda un peu
sur le tard. Il sut que non, et il ne l'approuva pas. Là-dessus
M\textsuperscript{me} de Maintenon chargea M\textsuperscript{me}
d'Espinoy d'en faire ses excuses à M\textsuperscript{lle} Choin, et de
lui dire qu'elle espérait qu'elles se verraient, compliment bizarre
d'une chambre à l'autre, sous le même toit. Elles ne se virent jamais
depuis.

Versailles présentait une autre scène\,: Mgr {[}le duc{]} et
M\textsuperscript{me} la duchesse de Bourgogne y tenaient ouvertement la
cour, et cette cour ressemblait à la première pointe de l'aurore. Toute
la cour était là rassemblée, tout Paris y abondait\,; et comme la
discrétion et la précaution ne furent jamais françaises, tout Meudon y
venait, et on en croyait les gens sur leur parole de n'être pas entrés
chez Monseigneur ce jour-là. Lever et coucher, dîner et souper avec les
dames, conversations publiques après les repas, promenades, étaient les
heures de faire sa cour, et les appartements ne pouvaient contenir la
foule. Courriers à tous quarts d'heure, qui rappelaient l'attention aux
nouvelles de Monseigneur, cours de maladie à souhait, et facilité
extrême d'espérance et de confiance\,; désir et empressement de tous de
plaire à la nouvelle cour, majesté et gravité gaie dans le jeune prince
et la jeune princesse, accueil obligeant à tous, attention continuelle à
parler à chacun, et complaisance dans cette foule, satisfaction
réciproque, duc et duchesse de Berry à peu près nuls. De cette sorte
s'écoulèrent cinq jours, chacun pensant sans cesse aux futurs
contingents, tâchant d'avance de s'accommoder à tout événement.

Le mardi 14 avril, lendemain de mon retour de la Ferté à Versailles, le
roi, qui, comme j'ai dit, s'ennuyait à Meudon, donna à l'ordinaire
conseil des finances le matin, et contre sa coutume conseil de dépêches
l'après-dinée pour en remplir le vide. J'allai voir le chancelier à son
retour de ce dernier conseil, et je m'informai beaucoup à lui de l'état
de Monseigneur. Il me l'assura bon, et me dit que Fagon lui avait dit
ces mêmes mots\,: «\,que les choses allaient selon leurs souhaits, et au
delà de leurs espérances.\,» Le chancelier me parut dans une grande
confiance\,; et j'y ajoutai foi d'autant plus aisément qu'il était
extrêmement bien avec Monseigneur, et qu'il ne bannissait pas toute
crainte, mais sans en avoir d'autre que celle de la nature propre à
cette sorte de maladie.

Les harengères de Paris, amies fidèles de Monseigneur, qui s'étaient
déjà signalées à cette forte indigestion qui fut prise pour apoplexie,
donnèrent ici le second tome de leur zèle. Ce même matin, elles
arrivèrent en plusieurs carrosses de louage à Meudon. Monseigneur les
voulut voir. Elles se jetèrent au pied de son lit qu'elles baisèrent
plusieurs fois\,; et, ravies d'apprendre de si bonnes nouvelles, elles
s'écrièrent dans leur joie qu'elles allaient réjouir tout Paris, et
faire chanter le \emph{Te Deum}. Monseigneur, qui n'était pas insensible
à ces marques d'amour du peuple, leur dit qu'il n'était pas encore
temps\,; et, après les avoir remerciées, il ordonna qu'on leur fît voir
sa maison, qu'on les traitât à dîner, et qu'on les renvoyât avec de
l'argent.

Revenant chez moi, de chez le chancelier, par les cours, je vis
M\textsuperscript{me} la duchesse d'Orléans se promenant sur la terrasse
de l'aile neuve, qui m'appela, et que je ne fis semblant de voir ni
d'entendre, parce que la Montauban était avec elle, et je gagnai mon
appartement l'esprit fort rempli de ces bonnes nouvelles de Meudon. Ce
logement était dans la galerie haute de l'aile neuve, qu'il n'y avait
presque qu'à traverser pour être dans l'appartement de M. {[}le duc{]}
et de M\textsuperscript{me} la duchesse de Berry, qui ce soir-là
devaient donner à souper chez eux à M. {[}le duc{]} et à
M\textsuperscript{me} la duchesse d'Orléans et à quelques dames, dont
M\textsuperscript{me} de Saint-Simon se dispensa sur ce qu'elle avait
été un peu incommodée.

Il y avait peu que j'étais dans mon cabinet seul avec Coettenfao, qu'on
m'annonça M\textsuperscript{me} la duchesse d'Orléans, qui venait causer
en attendant l'heure du souper. J'allai la recevoir dans l'appartement
de M\textsuperscript{me} de Saint-Simon, qui était sortie, et qui revint
bientôt après se mettre en tiers avec nous. La princesse et moi étions,
comme on dit, gros de nous voir et de nous entretenir dans cette
conjoncture, sur laquelle elle et moi nous pensions si pareillement. Il
n'y avait guère qu'une heure qu'elle était revenue de Meudon, où elle
avait vu le roi, et il en était alors huit du soir de ce même mardi 14
avril.

Elle me dit la même expression dont Fagon s'était servi, que j'avais
apprise du chancelier. Elle me rendit la confiance qui régnait dans
Meudon\,; elle me vanta les soins et la capacité des médecins, qui ne
négligeaient pas jusqu'aux plus petits remèdes, qu'ils ont coutume de
mépriser le plus\,: elle nous en exagéra le succès\,; et, pour en parler
franchement et en avouer la honte, elle et moi nous lamentâmes ensemble
de voir Monseigneur échapper, à son âge et à sa graisse, d'un mal si
dangereux. Elle réfléchissait tristement, mais avec ce sel et ces tons à
la Mortemart, qu'après une dépuration de cette sorte il ne restait plus
la moindre pauvre petite apparence aux apoplexies\,; que celle des
indigestions était ruinée sans ressource depuis la peur que Monseigneur
en avait prise, et l'empire qu'il avait donné sur sa santé aux médecins,
et nous conclûmes plus que langoureusement qu'il fallait désormais
compter que ce prince vivrait et régnerait longtemps. De là, des
raisonnements sans fin sur les funestes accompagnements de son règne,
sur la vanité des apparences les mieux fondées d'une vie qui promettait
si peu, et qui trouvait son salut et sa durée au sein du péril et de la
mort. En un mot, nous nous lâchâmes, non sans quelque scrupule qui
interrompait de fois à autre cette rare conversation, mais qu'avec un
tour languissamment plaisant elle ramenait toujours à son point.
M\textsuperscript{me} de Saint-Simon, tout dévotement, enrayait tant
qu'elle pouvait ces propos étranges\,; mais l'enrayure cassait, et
entretenait ainsi un combat très-singulier entre la liberté des
sentiments, humainement pour nous très-raisonnables, mais qui ne
laissait pas de nous faire sentir qui n'étaient pas selon la religion.

Deux heures s'écoulèrent de la sorte entre nous trois, qui nous parurent
courtes, mais que l'heure du souper termina. M\textsuperscript{me} la
duchesse d'Orléans s'en alla chez M\textsuperscript{me} sa fille, et
nous passâmes dans ma chambre, où bonne compagnie s'était cependant
assemblée, qui soupa avec nous.

Tandis qu'on était si tranquille à Versailles, et même à Meudon, tout y
changeait de face. Le roi avait vu Monseigneur plusieurs fois dans la
journée, qui était sensible à ces marques d'amitié et de considération.
Dans la visite de l'après-dînée, avant le conseil des dépêches, le roi
fut si frappé de l'enflure extraordinaire du visage et de la tête, qu'il
abrégea, et qu'il laissa échapper quelques larmes en sortant de la
chambre. On le rassura tant qu'on put\,; et après le conseil des
dépêches, il se promena dans les jardins.

Cependant Monseigneur avait déjà méconnu M\textsuperscript{me} la
princesse de Conti, et Boudin en avait été alarmé. Ce prince l'avait
toujours été. Les courtisans le voyaient tous les uns après les autres,
les plus familiers n'en bougeaient jour et nuit. Il s'informait sans
cesse à eux si on avait coutume d'être dans cette maladie dans l'état où
il se sentait. Dans les temps où ce qu'on lui disait pour le rassurer
lui faisait le plus d'impression, il fondait sur cette dépuration
désespérances de vie et de santé\,; et en une de ces occasions, il lui
échappa d'avouer à M\textsuperscript{me} la princesse de Conti qu'il y
avait longtemps qu'il se sentait fort mal sans en avoir voulu rien
témoigner, et dans un tel état de faiblesse que, le jeudi saint dernier,
il n'avait pu durant l'office tenir sa \emph{Semaine sainte} dans ses
mains.

Il se trouva plus mal vers quatre heures après midi, pendant le conseil
des dépêches, tellement que Boudin proposa à Fagon d'envoyer querir du
conseil, lui représenta qu'eux, médecins de la cour qui ne voyaient
jamais aucune maladie de venin, n'en pouvaient avoir d'expérience, et le
pressa de mander promptement des médecins de Paris\,; mais Fagon se mit
en colère, ne se paya d'aucunes raisons, s'opiniâtra au refus d'appeler
personne, à dire qu'il était inutile de se commettre à des disputes et à
des contrariétés, soutint qu'ils feraient aussi bien et mieux que tout
le secours qu'ils pourraient faire venir, voulut enfin tenir secret
l'état de Monseigneur, quoiqu'il empirât d'heure en heure, et que sur
les sept heures du soir quelques valets et quelques courtisans même
commençassent à s'en apercevoir. Mais tout en ce genre tremblait sous
Fagon. Il était là, et personne n'osait ouvrir la bouche pour avertir le
roi ni M\textsuperscript{me} de Maintenon. M\textsuperscript{me} la
Duchesse et M\textsuperscript{me} la princesse de Conti, dans la même
impuissance, cherchaient à se rassurer. Le rare fut qu'on voulut laisser
mettre le roi à table pour souper avant d'effrayer par de grands
remèdes, et laisser achever son souper sans l'interrompre et sans
l'avertir de rien, qui sur la foi de Fagon et le silence public croyait
Monseigneur en bon état, quoiqu'il l'eût trouvé enflé et changé dans
l'après-dînée, et qu'il en eût été fort peiné.

Pendant que le roi soupait ainsi tranquillement, la tête commença à
tourner à ceux qui étaient dans la chambre de Monseigneur. Fagon et les
autres entassèrent remèdes sur remèdes sans en attendre l'effet. Le
curé, qui tous les soirs avant de se retirer chez lui allait savoir des
nouvelles, trouva, contre l'ordinaire, toutes les portes ouvertes et les
valets éperdus. Il entra dans la chambre, où, voyant de quoi il n'était
que trop tardivement question, il courut au lit, prit la main de
Monseigneur, lui parla de Dieu\,; et, le voyant plein de connaissance,
mais presque hors d'état de parler, il en tira ce qu'il put pour une
confession, dont qui que ce soit ne s'était avisé, lui suggéra des actes
de contrition. Le pauvre prince en répéta distinctement quelques mots,
confusément les autres, se frappa la poitrine, serra la main au curé,
parut pénétré des meilleurs sentiments, et reçut d'un air contrit et
désireux l'absolution du curé.

Cependant le roi sortait de table, et pensa tomber à la renverse lorsque
Fagon se présentant à lui lui cria, tout troublé, que tout était perdu.
On peut juger quelle horreur saisit tout le monde en ce passage si subit
d'une sécurité entière à la plus désespérée extrémité.

Le roi, à peine à lui-même, prit à l'instant le chemin de l'appartement
de Monseigneur, et réprima très-sèchement l'indiscret empressement de
quelques courtisans à le retenir, disant qu'il voulait voir encore son
fils, et s'il n'y avait plus de remède. Comme il était près d'entrer
dans la chambre, M\textsuperscript{me} la princesse de Conti, qui avait
eu le temps d'accourir chez Monseigneur dans ce court intervalle de la
sortie de table, se présenta pour l'empêcher d'entrer. Elle le repoussa
même des mains, et lui dit qu'il ne fallait plus désormais penser qu'à
lui-même. Alors le roi, presque en faiblesse d'un renversement si subit
et si entier, se laissa aller sur un canapé qui se trouva à l'entrée de
la porte du cabinet par lequel il était entré, qui donnait dans la
chambre. Il demandait des nouvelles à tout ce qui en sortait, sans que
presque personne osât lui répondre. En descendant chez Monseigneur, car
il logeait au-dessus de lui, il avait envoyé chercher le P. Tellier, qui
venait de se mettre au lit\,; il fut bientôt habillé et arrivé dans la
chambre\,; mais il n'était plus temps, à ce qu'ont dit depuis tous les
domestiques, quoique le jésuite, peut-être pour consoler le roi, lui eût
assuré qu'il avait donné une absolution bien fondée.
M\textsuperscript{me} de Maintenon, accourue auprès du roi, et assise
sur le même canapé, tâchait de pleurer. Elle essayait d'emmener le roi,
dont les carrosses étaient déjà prêts dans la cour, mais il n'y eut pas
moyen de l'y faire résoudre que Monseigneur ne fût expiré.

Cette agonie sans connaissance dura près d'une heure depuis que le roi
fut dans le cabinet. M\textsuperscript{me} la Duchesse et
M\textsuperscript{me} la princesse de Conti se partageaient entre les
soins du mourant et ceux du roi, près duquel elles revenaient souvent,
tandis que la Faculté confondue, les valets éperdus, le courtisan
bourdonnant, se poussaient les uns les autres, et cheminaient sans cesse
sans presque changer de lieu. Enfin le moment fatal arriva. Fagon sortit
qui le laissa entendre.

Le roi, fort affligé, et très-peiné du défaut de confession, maltraita
un peu ce premier médecin, puis sortit emmené par M\textsuperscript{me}
de Maintenon et par les deux princesses. L'appartement était de
plain-pied à la cour\,; et comme il se présenta pour monter en carrosse,
il trouva devant lui la berline de Monseigneur. Il fit signe de la main
qu'on lui amenât un autre carrosse, par la peine que lui faisait
celui-là. Il n'en fut pas néanmoins tellement occupé que, voyant
Pontchartrain, il ne l'appelât pour lui dire d'avertir son père et les
autres ministres de se trouver le lendemain matin un peu tard à Marly
pour le conseil d'État ordinaire du mercredi. Sans commenter ce
sang-froid, je me contenterai de rapporter la surprise extrême de tous
les témoins et de tous ceux qui l'apprirent. Pontchartrain répondit que,
ne s'agissant que d'affaires courantes, il vaudrait mieux remettre le
conseil d'un jour que de l'en importuner. Le roi y consentit. Il monta
avec peine en carrosse appuyé des deux côtés, M\textsuperscript{me} de
Maintenon tout de suite après qui se mit à côté de lui\,;
M\textsuperscript{me} la Duchesse et M\textsuperscript{me} la princesse
de Conti montèrent après elle, et se mirent sur le devant. Une foule
d'officiers de Monseigneur se jetèrent à genoux tout le long de la cour,
des deux côtés, sur le passage du roi, lui criant avec des hurlements
étranges d'avoir compassion d'eux, qui avaient tout perdu et qui
mouraient de faim.

Tandis que Meudon était rempli d'horreur, tout était tranquille à
Versailles, sans en avoir le moindre soupçon. Nous avions soupé. La
compagnie quelque temps après s'était retirée, et je causais avec
M\textsuperscript{me} de Saint-Simon qui achevait de se déshabiller pour
se mettre au lit, lorsqu'un ancien valet de chambre, à qui elle avait
donné une charge de garçon de la chambre de M\textsuperscript{me} la
duchesse de Berry, et qui y servait à table, entra tout effarouché. Il
nous dit qu'il fallait qu'il y eût de mauvaises nouvelles de Meudon\,;
que Mgr le duc de Bourgogne venait d'envoyer parler à l'oreille à M. le
duc de Berry, à qui les yeux avaient rougi à l'instant\,; qu'aussitôt il
était sorti de table, et que, sur un second message fort prompt, la
table où la compagnie était restée s'était levée avec précipitation, et
que tout le monde était passé dans le cabinet. Un changement si subit
rendit ma surprise extrême. Je courus chez M\textsuperscript{me} la
duchesse de Berry aussitôt\,; il n'y avait plus personne\,; ils étaient
tous allés chez M\textsuperscript{me} la duchesse de Bourgogne\,; j'y
poussai tout de suite.

J'y trouvai tout Versailles rassemblé, ou y arrivant\,; toutes les dames
en déshabillé, la plupart prêtes à se mettre au lit, toutes les portes
ouvertes, et tout en trouble. J'appris que Monseigneur avait reçu
l'extrême-onction, qu'il était sans connaissance et hors de toute
espérance, et que le roi avait mandé à M\textsuperscript{me} la duchesse
de Bourgogne qu'il s'en allait à Marly, et de le venir attendre dans
l'avenue entre les deux écuries, pour le voir en passant.

Le spectacle attira toute l'attention que j'y pus donner parmi les
divers mouvements de mon âme, et ce qui tout à la fois se présenta à mon
esprit. Les deux princes et les deux princesses étaient dans le petit
cabinet derrière la ruelle du lit. La toilette pour le coucher était à
l'ordinaire dans la chambre de M\textsuperscript{me} la duchesse de
Bourgogne, remplie de toute la cour en confusion. Elle allait et venait
du cabinet dans la chambre, en attendant le moment d'aller au passage du
roi\,; et son maintien, toujours avec ses mêmes grâces, était un
maintien de trouble et de compassion que celui de chacun semblait
prendre pour douleur. Elle disait ou répondait en passant devant les uns
et les autres quelques mots rares. Tous les assistants étaient des
personnages vraiment expressifs, il ne fallait qu'avoir des yeux, sans
aucune connaissance de la cour, pour distinguer les intérêts peints sur
les visages, ou le néant de ceux qui n'étaient de rien\,: ceux-ci
tranquilles à eux-mêmes, les autres pénétrés de douleur ou de gravité et
d'attention sur eux-mêmes, pour cacher leur élargissement et leur joie.

Mon premier mouvement fut de m'informer à plus d'une fois, de ne croire
qu'à peine au spectacle et aux paroles\,; ensuite de craindre trop peu
de cause pour tant d'alarme, enfin de retour sur soi-même par la
considération de la misère commune à tous les hommes, et que moi-même je
me trouverais un jour aux portes de la mort. La joie néanmoins perçait à
travers les réflexions momentanées de religion et d'humanité par
lesquelles j'essayais de me rappeler. Ma délivrance particulière me
semblait si grande et si inespérée qu'il me semblait, avec une évidence
encore plus parfaite que la vérité, que l'État gagnait tout en une telle
perte. Parmi ces pensées, je sentais malgré moi un reste de crainte que
le malade en réchappât, et j'en avais une extrême honte.

Enfoncé de la sorte en moi-même, je ne laissai pas de mander à
M\textsuperscript{me} de Saint-Simon qu'il était à propos qu'elle vînt,
et de percer de mes regards clandestins chaque visage, chaque maintien,
chaque mouvement, d'y délecter ma curiosité, d'y nourrir les idées que
je m'étais formées de chaque personnage, qui ne m'ont jamais guère
trompé, et de tirer de justes conjectures de la vérité de ces premiers
élans dont on est si rarement maître, et qui par là, à qui connaît la
carte et les gens, deviennent des indictions\footnote{Saint-Simon a
  écrit \emph{indictions}, probablement pour \emph{indications}. Nous
  n'avons pas cru devoir remplacer ce mot par celui \emph{d'inductions},
  comme l'ont fait les précédents éditeurs.} sûres des liaisons et des
sentiments les moins visibles en tous autres temps rassis.

Je vis arriver M\textsuperscript{me} la duchesse d'Orléans dont la
contenance majestueuse et compassée ne disait rien. Elle entra dans le
petit cabinet, d'où bientôt après elle sortit avec M. le duc d'Orléans,
duquel l'activité et l'air turbulent marquaient plus l'émotion du
spectacle que tout autre sentiment. Ils s'en allèrent, et je le remarque
exprès, par ce qui bientôt après arriva en ma présence.

Quelques moments après, je vis de loin, vers la porte du petit cabinet,
Mgr le duc de Bourgogne avec un air fort ému et peiné\,; mais le coup
d'œil que j'assénai vivement sur lui ne m'y rendit rien de tendre, et ne
me rendit que l'occupation profonde d'un esprit saisi.

Valets et femmes de chambre criaient déjà indiscrètement, et leur
douleur prouva bien tout ce que cette espèce de gens allait perdre. Vers
minuit et demi, on eut des nouvelles du roi\,; et aussitôt je vis
M\textsuperscript{me} la duchesse de Bourgogne sortir du petit cabinet
avec Mgr le duc de Bourgogne, l'air alors plus touché qu'il ne m'avait
paru la première fois, et qui rentra aussitôt dans le cabinet. La
princesse prit à sa toilette son écharpe et ses coiffes, debout et d'un
air délibéré, traversa la chambre, les yeux à peine mouillés, mais
trahie par de curieux regards lancés de part et d'autre à la dérobée,
et, suivie seulement de ses dames, gagna son carrosse par le grand
escalier.

Comme elle sortit de sa chambre, je pris mon temps pour aller chez
M\textsuperscript{me} la duchesse d'Orléans avec qui je grillais d'être.
Entrant chez elle, j'appris qu'ils étaient chez Madame. Je poussai
jusque-là à travers leurs appartements. Je trouvai M\textsuperscript{me}
la duchesse d'Orléans qui retournait chez elle, et qui, d'un air fort
sérieux, me dit de revenir avec elle. M. le duc d'Orléans était demeuré.
Elle s'assit dans sa chambre, et auprès d'elle la duchesse de Villeroy,
la maréchale de Rochefort et cinq ou six dames familières. Je petillais
cependant de tant de compagnie\,; M\textsuperscript{me} la duchesse
d'Orléans, qui n'en était pas moins importunée, prit une bougie et passa
derrière sa chambre. J'allai alors dire un mot à l'oreille à la duchesse
de Villeroy\,; elle et moi pensions de même sur l'événement présent.
Elle me poussa et me dit tout bas de me bien contenir. J'étouffais de
silence parmi les plaintes et les surprises narratives de ces dames,
lorsque M. le duc d'Orléans parut à la porte du cabinet et m'appela.

Je le suivis dans son arrière-cabinet en bas sur la galerie, lui près de
se trouver mal, et moi les jambes tremblantes de tout ce qui se passait
sous mes yeux et au dedans de moi. Nous nous assîmes par hasard
vis-à-vis l'un de l'autre\,; mais quel fut mon étonnement lorsque
incontinent après je vis les larmes lui tomber des yeux\,:
«\,Monsieur\,!» m'écriai-je en me levant dans l'excès de ma surprise. Il
me comprit aussitôt et me répondit d'une voix coupée et pleurant
véritablement\,: «\,Vous avez raison d'être surpris, et je le suis
moi-même\,; mais le spectacle touche. C'est un bon homme avec qui j'ai
passé ma vie\,; il m'a bien traité et avec amitié tant qu'on l'a laissé
faire et qu'il a agi de lui-même. Je sens bien que l'affliction ne peut
pas être longue\,; mais ce sera dans quelques jours que je trouverai
tous les motifs de me consoler dans l'état où on m'avait mis avec lui\,;
mais présentement le sang, la proximité, l'humanité, tout touche, et les
entrailles s'émeuvent.\,» Je louai ce sentiment, mais j'en avouai mon
extrême surprise par la façon dont il était avec Monseigneur. Il se
leva, se mit la tête dans un coin, le nez dedans, et pleura amèrement et
à sanglots, chose que, si je n'avais vue, je n'eusse jamais crue. Après
quelque peu de silence, je l'exhortai à se calmer. Je lui représentai
qu'incessamment il faudrait retourner chez M\textsuperscript{me} la
duchesse de Bourgogne, et que si on l'y voyait avec des yeux pleureux,
il n'y avait personne qui ne s'en moquât comme d'une comédie
très-déplacée, à la façon dont toute la cour savait qu'il était avec
Monseigneur. Il fit donc ce qu'il put pour arrêter ses larmes, et pour
bien essuyer et retaper ses yeux. Il y travaillait encore, lorsqu'il fut
averti que M\textsuperscript{me} la duchesse de Bourgogne arrivait, et
que M\textsuperscript{me} la duchesse d'Orléans allait retourner chez
elle. Il la fut joindre et je les y suivis.

\hypertarget{chapitre-vii.}{%
\chapter{CHAPITRE VII.}\label{chapitre-vii.}}

1711

~

{\textsc{Continuation du spectacle de Versailles.}} {\textsc{- Plaisante
aventure d'un Suisse.}} {\textsc{- Horreur de Meudon.}} {\textsc{-
Confusion de Marly.}} {\textsc{- Caractère de Monseigneur}} {\textsc{-
Problème si Monseigneur avait épousé M\textsuperscript{lle} Choin.}}
{\textsc{- Monseigneur sans agrément, sans liberté, sans crédit avec le
roi.}} {\textsc{- Monsieur et Monseigneur morts outrés contre le roi}}
{\textsc{-Monseigneur peu à Versailles.}} {\textsc{- Complaisant aux
choses du sacre.}} {\textsc{- Monseigneur et M\textsuperscript{me} de
Maintenon fort éloignés.}} {\textsc{- Cour intime de Monseigneur.}}
{\textsc{- Monseigneur, plus que sec avec Mgr {[}le duc{]} et
M\textsuperscript{me} la duchesse de Bourgogne, aime M. le duc de Berry
et traite bien M\textsuperscript{me} la duchesse de Berry.}} {\textsc{-
Monseigneur favorable aux ducs contre les princes.}} {\textsc{-
Monseigneur fort vrai\,; M\textsuperscript{lle} Choin aussi.}}
{\textsc{- Opposition de Monseigneur à l'alliance du sang bâtard
prétendue.}} {\textsc{- Désintéressement de M\textsuperscript{lle}
Choin.}} {\textsc{- Monseigneur attaché à la mémoire et à la famille du
duc de Montausier.}} {\textsc{- Amours de Monseigneur.}} {\textsc{-
Ridicule aventure.}} {\textsc{- Monseigneur n'aime point M. du Maine et
traite bien le comte de Toulouse.}} {\textsc{- Cour plus ou moins
particulière de Monseigneur.}} {\textsc{- Infamies du maréchal
d'Huxelles.}} {\textsc{- Aversions de Monseigneur.}} {\textsc{-
Éloignement de Mgr {[}le duc{]} et de M\textsuperscript{me} la duchesse
de Bourgogne.}} {\textsc{- M. {[}le duc{]} et M\textsuperscript{me} la
duchesse de Berry bien avec Monseigneur.}} {\textsc{- Crayon et projets
de M\textsuperscript{me} la duchesse de Berry.}} {\textsc{- Affection de
Monseigneur pour le roi d'Espagne.}} {\textsc{- Portrait raccourci de
Monseigneur.}}

~

M\textsuperscript{me} la duchesse de Bourgogne, arrêtée dans l'avenue
entre les deux écuries, n'avait attendu le roi que fort peu de temps.
Dès qu'il approcha, elle mit pied à terre et alla à sa portière.
M\textsuperscript{me} de Maintenon, qui était de ce même côté, lui
cria\,: «\,Où allez-vous, madame\,? N'approchez pas\,; nous sommes
pestiférés.\,» Je n'ai point su quel mouvement fit le roi, qui ne
l'embrassa point à cause du mauvais air. La princesse à l'instant
regagna son carrosse et s'en revint. Le beau secret que Fagon avait
imposé sur l'état de Monseigneur avait si bien trompé tout le monde, que
le duc de Beauvilliers était revenu à Versailles après le conseil de
dépêches, et qu'il y coucha contre son ordinaire depuis la maladie de
Monseigneur. Comme il se levait fort matin, il se couchait toujours sur
les dix heures, et il s'était mis au lit sans se défier de rien. Il ne
fut pas longtemps sans être réveillé par un message de
M\textsuperscript{me} la duchesse de Bourgogne, qui l'envoya chercher,
et il arriva dans son appartement peu avant son retour du passage du
roi. Elle retrouva les deux princes et M\textsuperscript{me} la duchesse
de Berry avec le duc de Beauvilliers, dans ce petit cabinet où elle les
avait laissés.

Après les premiers embrassements d'un retour qui signifiait tout, le duc
de Beauvilliers, qui les vit étouffant dans ce petit lieu, les fit
passer par la chambre dans le salon qui la sépare de la galerie, dont,
depuis quelque temps, on avait fermé ce salon d'une porte pour en faire
un grand cabinet. On y ouvrit des fenêtres, et les deux princes, ayant
chacun sa princesse à son côté, s'assirent sur un même canapé près des
fenêtres, le dos à la galerie\,; tout le monde épars, assis et debout,
et en confusion dans ce salon, et les dames les plus familières par
terre aux pieds ou proche du canapé des princes.

Là, dans la chambre et par tout l'appartement, on lisait apertement sur
les visages. Monseigneur n'était plus\,; on le savait, on le disait,
nulle contrainte ne retenait plus à son égard, et ces premiers moments
étaient ceux des premiers mouvements peints au naturel et pour lors
affranchis de toute politique, quoique avec sagesse, par le trouble,
l'agitation, la surprise, la foule, le spectacle confus de cette nuit si
rassemblée.

Les premières pièces offraient les mugissements contenus des valets,
desespérés de là perte d'un maître si fait exprès pour eux, et pour les
consoler d'une autre qu'ils ne prévoyaient qu'avec transissement, et qui
par celle-ci devenait la leur propre. Parmi eux s'en remarquaient
d'autres des plus éveillés de gens principaux de la cour, qui étaient
accourus aux nouvelles, et qui montraient bien à leur air de quelle
boutique ils étaient balayeurs.

Plus avant commençait la foule des courtisans de toute espèce. Le plus
grand nombre, c'est-à-dire les sots, tiraient des soupirs de leurs
talons, et, avec des yeux égarés et secs, louaient Monseigneur, mais
toujours de la même louange, c'est-à-dire de bonté, et plaignaient le
roi de la perte d'un si bon fils. Les plus fins d'entre eux, ou les plus
considérables, s'inquiétaient déjà de la santé du roi\,; ils se savaient
bon gré de conserver tant de jugement parmi ce trouble, et n'en
laissaient pas douter par la fréquence de leurs répétitions. D'autres,
vraiment affligés, et de cabale frappée, pleuraient amèrement, ou se
contenaient avec un effort aussi aisé à remarquer que les sanglots. Les
plus forts de ceux-là, ou les plus politiques, les yeux fichés à terre,
et reclus en des coins, méditaient profondément aux suites d'un
événement si peu attendu, et bien davantage sur eux-mêmes. Parmi ces
diverses sortes d'affligés, point ou peu de propos, de conversation
nulle, quelque exclamation parfois échappée à la douleur et parfois
répondue par une douleur voisine, un mot en un quart-d'heure, des yeux
sombres ou hagards, des mouvements de mains moins rares
qu'involontaires, immobilité du reste presque entière\,; les simples
curieux et peu soucieux presque nuls, hors les sots qui avaient le
caquet en partage, les questions, et le redoublement du désespoir des
affligés, et l'importunité pour les autres. Ceux qui déjà regardaient
cet événement comme favorable avaient beau pousser la gravité jusqu'au
maintien chagrin et austère, le tout n'était qu'un voile clair, qui
n'empêchait pas de bons yeux de remarquer et de distinguer tous leurs
traits. Ceux-ci se tenaient aussi tenaces en place que les plus touchés,
en garde contre l'opinion, contre la curiosité, contre leur
satisfaction, contre leurs mouvements\,; mais leurs yeux suppléaient au
peu d'agitation de leur corps. Des changements de posture, comme des
gens peu assis ou mal debout\,; un certain soin de s'éviter les uns les
autres, même de se rencontrer des yeux\,; les accidents momentanés qui
arrivaient de ces rencontres\,; un je ne sais quoi de plus libre en
toute la personne, à travers le soin de se tenir et de se composer\,; un
vif, une sorte d'étincelant autour d'eux les distinguait malgré qu'ils
en eussent.

Les deux princes, et les deux princesses assises à leurs côtés, prenant
soin d'eux, étaient les plus exposés à la pleine vue. Mgr le duc de
Bourgogne pleurait d'attendrissement et de bonne foi, avec un air de
douceur, des larmes de nature, de religion, de patience. M. le duc de
Berry tout d'aussi bonne foi en versait en abondance, mais des larmes
pour ainsi dire sanglantes, tant l'amertume en paraissait grande, et
poussait non des sanglots, mais des cris, mais des hurlements. Il se
taisait parfois, mais de suffocation, puis éclatait, mais avec un tel
bruit, et un bruit si fort la trompette forcée du désespoir, que la
plupart éclataient aussi à ces redoublements si douloureux, ou par un
aiguillon d'amertume, ou par un aiguillon de bienséance. Cela fut au
point qu'il fallut le déshabiller là même, et se précautionner de
remèdes et de gens de la Faculté. M\textsuperscript{me} la duchesse de
Berry était hors d'elle, on verra bientôt pourquoi. Le désespoir le plus
amer était peint avec horreur sur son visage. On y voyait comme écrite
une rage de douleur, non d'amitié mais d'intérêt\,; des intervalles secs
mais profonds et farouches, puis un torrent de larmes et de gestes
involontaires, et cependant retenus, qui montraient une amertume d'âme
extrême, fruit de la méditation profonde qui venait de précéder. Souvent
réveillée par les cris de son époux, prompte à le secourir, à le
soutenir, à l'embrasser, à lui présenter quelque chose à sentir, on
voyait un soin vif pour lui, mais tôt après une chute profonde en
elle-même, puis un torrent de larmes qui lui aidaient à suffoquer ses
cris. M\textsuperscript{me} la duchesse de Bourgogne consolait aussi son
époux, et y avait moins de peine qu'à acquérir le besoin d'être
elle-même consolée, à quoi pourtant, sans rien montrer de faux, on
voyait bien qu'elle faisait de son mieux pour s'acquitter d'un devoir
pressant de bienséance sentie, mais qui se refuse au plus grand besoin.
Le fréquent moucher répondait aux cris du prince son beau-frère.
Quelques larmes amenées du spectacle, et souvent entretenues avec soin,
fournissaient à l'art du mouchoir pour rougir et grossir les yeux et
barbouiller le visage, et cependant le coup d'œil fréquemment dérobé se
promenait sur l'assistance et sur la contenance de chacun.

Le duc de Beauvilliers, debout auprès d'eux, l'air tranquille et froid,
comme à chose non avenue ou à spectacle ordinaire, donnait ses ordres
pour le soulagement des princes, pour que peu de gens entrassent,
quoique les portes fussent ouvertes à chacun, en un mot pour tout ce
qu'il était besoin, sans empressement, sans se méprendre en quoi que ce
soit ni aux gens ni aux choses\,; vous l'auriez cru au lever ou au petit
couvert servant à l'ordinaire. Ce flegme dura sans la moindre
altération, également éloigné d'être aise par la religion, et de cacher
aussi le peu d'affliction qu'il ressentait, pour conserver toujours la
vérité.

Madame, rhabillée, en grand habit, arriva hurlante, ne sachant bonnement
pourquoi ni l'un ni l'autre, les inonda tous de ses larmes en les
embrassant, fit retentir le château d'un renouvellement de cris, et
fournit un spectacle bizarre d'une princesse qui se remet en cérémonie,
en pleine nuit, pour venir pleurer et crier parmi une foule de femmes en
déshabillé de nuit, presque en mascarades.

M\textsuperscript{me} la duchesse d'Orléans s'était éloignée des
princes, et s'était assise le dos à la galerie, vers la cheminée, avec
quelques dames. Tout étant fort silencieux autour d'elle, ces dames peu
à peu se retirèrent d'auprès d'elle, et lui firent grand plaisir. Il n'y
resta que la duchesse Sforce, la duchesse de Villeroy,
M\textsuperscript{me} de Castries, sa dame d'atours, et
M\textsuperscript{me} de Saint-Simon. Ravies de leur liberté, elles
s'approchèrent en un tas, tout le long d'un lit de veille à pavillon et
le joignant\,; et comme elles étaient toutes affectées de même à l'égard
de l'événement qui rassemblait là tant de monde, elles se mirent à en
deviser tout bas ensemble dans ce groupe avec liberté.

Dans la galerie et dans ce salon il y avait plusieurs lits de veille,
comme dans tout le grand appartement, pour la sûreté, où couchaient des
Suisses de l'appartement et des trotteurs, et ils avaient été mis à
l'ordinaire avant les mauvaises nouvelles de Meudon. Au fort de la
conversation de ces dames, M\textsuperscript{me} de Castries qui
touchait au lit le sentit remuer et en fut fort effrayée, car elle
l'était de tout quoique avec beaucoup d'esprit. Un moment après elles
virent un gros bras presque nu relever tout à coup le pavillon, qui leur
montra un bon gros Suisse entre deux draps, demi-éveillé et tout ébahi,
très-long à reconnaître son monde qu'il regardait fixement l'un après
l'autre, et qui enfin, ne jugeant pas à propos de se lever en si grande
compagnie, se renfonça dans son lit et ferma son pavillon. Le bonhomme
s'était apparemment couché avant que personne eût rien appris, et avait
assez profondément dormi depuis pour ne s'être réveillé qu'alors. Les
plus tristes spectacles sont assez souvent sujets aux contrastes les
plus ridicules. Celui-ci fit rire quelque dame de là autour, et
{[}fit{]} quelque peur à M\textsuperscript{me} la duchesse d'Orléans et
à ce qui causait avec elle d'avoir été entendues. Mais, réflexion faite,
le sommeil et la grossièreté du personnage les rassura.

La duchesse de Villeroy, qui ne faisait presque que les joindre, s'était
fourrée un peu auparavant dans le petit cabinet avec la comtesse de
Roucy et quelques dames du palais, dont M\textsuperscript{me} de Lévi
n'avait osé approcher, par penser trop conformément à la duchesse de
Villeroy. Elles y étaient quand j'arrivai.

Je voulais douter encore, quoique tout me montrât ce qui était, mais je
ne pus me résoudre à m'abandonner à le croire que le mot ne m'en fût
prononcé par quelqu'un à qui on pût ajouter foi. Le hasard me fit
rencontrer M. d'O, à qui je le demandai, et qui me le dit nettement.
Cela su, je tâchai de n'en être pas bien aise. Je ne sais pas trop si
j'y réussis bien, mais au moins est-il vrai que ni joie ni douleur
n'émoussèrent ma curiosité, et qu'en prenant bien garde à conserver
toute bienséance, je ne me crus pas engagé par rien au personnage
douloureux. Je ne craignais plus les retours du feu de la citadelle de
Meudon, ni les cruelles courses de son implacable garnison, et je me
contraignis moins qu'avant le passage du roi pour Marly de considérer
plus librement toute cette nombreuse compagnie, d'arrêter mes yeux sur
les plus touchés et sur ceux qui l'étaient moins avec une affection
différente, de suivre les uns et les autres de mes regards et de les en
percer tous à la dérobée.

Il faut avouer que, pour qui est bien au fait de la carte intime d'une
cour, les premiers spectacles d'événements rares de cette nature, si
intéressante à tant de divers égards, sont d'une satisfaction extrême.
Chaque visage vous rappelle les soins, les intrigues, les sueurs,
employés à l'avancement des fortunes, à la formation, à la force des
cabales\,; les adresses à se maintenir et en écarter d'autres, les
moyens de toute espèce mis en œuvre pour cela\,; les liaisons plus ou
moins avancées, les éloignements, les froideurs, les haines, les mauvais
offices, les manéges, les avances, les ménagements, les petitesses, les
bassesses de chacun\,; le déconcertement des uns au milieu de leur
chemin, au milieu ou au comble de leurs espérances\,; la stupeur de ceux
qui en jouissaient en plein, le poids donné du même coup à leurs
contraires et à la cabale opposée\,; la vertu de ressort qui pousse dans
cet instant leurs menées et leurs concerts à bien, la satisfaction
extrême et inespérée de ceux-là, et j'en étais des plus avant, la rage
qu'en conçoivent les autres, leur embarras et leur dépit à le cacher. La
promptitude des yeux à voler partout en sondant les âmes, à la faveur de
ce premier trouble de surprise et de dérangement subit, la combinaison
de tout ce qu'on y remarque, l'étonnement de ne pas trouver ce qu'on
avait cru de quelques-uns faute de cœur et d'assez d'esprit en eux, et
plus en d'autres qu'on avait pensé, tout cet amas d'objets vifs et de
choses si importantes forme un plaisir à qui le sait prendre qui, tout
peu solide qu'il devient, est un des plus grands dont on puisse jouir
dans une cour.

Ce fut donc à celui-là que je me livrai tout entier en moi-même, avec
d'autant plus d'abandon que, dans une délivrance bien réelle, je me
trouvais étroitement lié et embarqué avec les têtes principales qui
n'avaient point de larmes à donner à leurs yeux. Je jouissais de leur
avantage sans contre-poids, et de leur satisfaction qui augmentait la
mienne, qui consolidait mes espérances, qui me les élevait, qui
m'assurait un repos, auquel sans cet événement je voyais si peu
d'apparence que je ne cessais point de m'inquiéter d'un triste avenir,
et que, d'autre part, ennemi de liaison, et presque personnel des
principaux personnages que cette perte accablait, je vis, du premier
coup d'œil vivement porté, tout ce qui leur échappait et tout ce qui les
accablerait, avec un plaisir qui ne se peut rendre. J'avais si fort
imprimé dans ma tête les différentes cabales, leurs subdivisions, leurs
replis, leurs divers personnages et leurs degrés, la connaissance de
leurs chemins, de leurs ressorts, de leurs divers intérêts, que la
méditation de plusieurs jours ne m'aurait pas développé et représenté
toutes ces choses plus nettement que ce premier aspect de tous ces
visages, qui me rappelaient encore ceux que je ne voyais pas, et qui
n'étaient pas les moins friands à s'en repaître.

Je m'arrêtai donc un peu à considérer le spectacle de ces différentes
pièces de ce vaste et tumultueux appartement. Cette sorte de désordre
dura bien une heure, où la duchesse du Lude ne parut point, retenue au
lit par la goutte. À la fin M. de Beauvilliers s'avisa qu'il était temps
de délivrer les deux princes d'un si fâcheux public. Il leur proposa
donc que M. {[}le duc{]} et M\textsuperscript{me} la duchesse de Berry
se retirassent dans leur appartement\,; et le monde, de celui de
M\textsuperscript{me} la duchesse de Bourgogne. Cet avis fut aussitôt
embrassé. M. le duc de Berry s'achemina donc partie seul et quelquefois
appuyé sur son épouse, M\textsuperscript{me} de Saint-Simon avec eux et
une poignée de gens. Je les suivis de loin pour ne pas exposer ma
curiosité plus longtemps. Ce prince voulait coucher chez lui, mais
M\textsuperscript{me} la duchesse de Berry ne le voulut pas quitter\,;
il était si suffoqué et elle aussi qu'on fit demeurer auprès d'eux une
Faculté complète et munie.

Toute leur nuit se passa en larmes et en cris. De fois à autre M. le duc
de Berry demandait des nouvelles de Meudon, sans vouloir comprendre la
cause de la retraite du roi à Marly. Quelquefois il s'informait s'il n'y
avait plus d'espérance, il voulait envoyer aux nouvelles\,; et ce ne fut
qu'assez avant dans la matinée que le funeste rideau fut tiré de devant
ses yeux, tant la nature et l'intérêt ont de peine à se persuader des
maux extrêmes sans remède. On ne peut rendre l'état où il fut quand il
le sentit enfin dans toute son étendue. Celui de M\textsuperscript{me}
la duchesse de Berry ne fut guère meilleur, mais qui ne l'empêcha pas de
prendre de lui tous les soins possibles.

La nuit de Mgr {[}le duc{]} et de M\textsuperscript{me} la duchesse de
Bourgogne fut plus tranquille\,; ils se couchèrent assez paisiblement.
M\textsuperscript{me} de Lévi dit tout bas à la princesse que, n'ayant
pas lieu d'être affligée, il serait horrible de lui voir jouer la
comédie. Elle répondit bien naturellement que, sans comédie, la pitié et
le spectacle la touchaient, et la bienséance la conte-noit, et rien de
plus\,; et en effet elle se tint dans ces bornes-là avec vérité et avec
décence. Ils voulurent que quelques-unes des dames du palais passassent
la nuit dans leur chambre dans des fauteuils. Le rideau demeura ouvert,
et cette chambre devint aussitôt le palais de Morphée. Le prince et la
princesse s'endormirent promptement, s'éveillèrent une fois ou deux un
instant\,; à la vérité ils se levèrent d'assez bonne heure, et assez
doucement. Le réservoir d'eau était tari chez eux, les larmes ne
revinrent plus depuis que rares et faibles à force d'occasion. Les dames
qui avaient veillé et dormi dans cette chambre contèrent à leurs amis ce
qui s'y était passé. Personne n'en fut surpris\,; et comme il n'y avait
plus de Monseigneur, personne aussi n'en fut scandalisé.

M\textsuperscript{me} de Saint-Simon et moi, au sortir de chez M. {[}le
duc{]} et M\textsuperscript{me} la duchesse de Berry, nous fûmes encore
deux heures ensemble. La raison plutôt que le besoin nous fit coucher,
mais avec si peu de sommeil qu'à sept heures du matin j'étais debout\,;
mais, il faut l'avouer, de telles insomnies sont douces, et de tels
réveils savoureux.

L'horreur régnait à Meudon. Dès que le roi en fut parti, tout ce qu'il y
avait de gens de la cour le suivirent, et s'entassèrent dans ce qui se
trouva de carrosses, et dans ce qu'il en vint aussitôt après. En un
instant Meudon se trouva vide. M\textsuperscript{lle} de Lislebonne et
M\textsuperscript{lle} de Melun montèrent chez M\textsuperscript{lle}
Choin, qui, recluse dans son grenier, ne faisait que commencer à entrer
dans des transes funestes. Elle avait tout ignoré, personne n'avait pris
soin de lui apprendre de tristes nouvelles. Elle ne fut instruite de son
malheur que par les cris. Ces deux amies la jetèrent dans un carrosse de
louage qui se trouva encore là par hasard, y montèrent avec elle, et la
menèrent à Paris.

Pontchartrain, avant partir, monta chez Voysin. Il trouva ses gens
difficiles à ouvrir et lui profondément endormi\,; il s'était couché
sans aucun soupçon sinistre, et fut étrangement surpris à ce réveil. Le
comte de Brionne le fut bien davantage. Lui et ses gens s'étaient
couchés dans la même confiance, personne ne songea à eux. Lorsqu'en se
levant il sentit ce grand silence, il voulut aller aux nouvelles et ne
trouva personne, jusqu'à ce que, dans cette surprise, il apprit enfin ce
qui était arrivé.

Cette foule de bas officiers de Monseigneur, et bien d'au tres, errèrent
toute la nuit dans les jardins. Plusieurs courtisans étaient partis
épars à pied. La dissipation fut entière et la dispersion générale. Un
ou deux valets au plus demeurèrent auprès du corps\,; et, ce qui est
très-digne de louange, La Vallière fut le seul des courtisans qui, ne
l'ayant point abandonné pendant sa vie, ne l'abandonna point après sa
mort. Il eut peine à trouver quelqu'un pour aller chercher des capucins
pour venir prier Dieu auprès du corps. L'infection en devint si prompte
et si grande que l'ouverture des fenêtres qui donnaient en portes sur la
terrasse ne suffit pas, et que La Vallière, les capucins et ce très-peu
de bas étage qui était demeuré, passèrent la nuit dehors. Du Mont et
Casau son neveu, navrés de la plus extrême douleur, y étaient ensevelis
dans la capitainerie. Ils perdaient tout après une longue vie toute de
petits soins, d'assiduité, de travail, soutenue par les plus flatteuses
et les plus raisonnables espérances, et les plus longuement prolongées,
qui leur échappaient en un moment. À peine sur le matin du Mont put-il
donner quelques ordres. Je plaignis celui-là avec amitié.

On s'était reposé sur une telle confiance que personne n'avait songé que
le roi pût aller à Marly. Aussi n'y trouva-t-il rien de prêt\,; point de
clefs des appartements, à peine quelques bouts de bougie, et même de
chandelle. Le roi fut plus d'une heure dans cet état avec
M\textsuperscript{me} de Maintenon dans son antichambre à elle,
M\textsuperscript{me} la Duchesse, M\textsuperscript{me} la princesse de
Conti, M\textsuperscript{me}s de Dangeau et de Caylus, celle-ci accourue
de Versailles auprès de sa tante. Mais ces deux dames ne se tinrent que
peu, par-ci par-là, dans cette antichambre par discrétion\,; ce qui
avait suivi et qui arrivait à la file était dans le salon en même
désarroi et sans savoir où gîter. On fut longtemps à tâtons, et toujours
sans feu, et toujours les clefs mêlées, égarées par l'égarement des
valets. Les plus hardis de ce qui était dans le salon montrèrent peu à
peu le nez dans l'antichambre, où M\textsuperscript{me} d'Espinoy ne fut
pas des dernières\,; et de l'un à l'autre tout ce qui était venu s'y
présenta, poussés de curiosité et de désir de tâcher que leur
empressement fût remarqué. Le roi, reculé en un coin, assis entre
M\textsuperscript{me} de Maintenon et les deux princesses, pleurait à
longues reprises. Enfin la chambre de M\textsuperscript{me} de Maintenon
fut ouverte, qui le délivra de cette importunité. Il y entra seul avec
elle, et y demeura encore une heure. Il alla ensuite se coucher qu'il
était près de quatre heures du matin, et la laissa en liberté de
respirer et de se rendre à elle-même. Le roi couché, chacun sut enfin où
loger\,; et Bloin eut ordre de répandre que les gens qui désireraient
des logements à Marly s'adressassent à lui, pour qu'il en rendît compte
au roi et qu'il avertît les élus.

Monseigneur était plutôt grand que petit, fort gros, mais sans être trop
entassé, l'air fort haut et fort noble, sans rien de rude, et il aurait
eu le visage fort agréable si M. le prince de Conti, le dernier mort, ne
lui avait pas cassé le nez par malheur en jouant étant tous deux
enfants. Il était d'un fort beau blond, il avait le visage fort rouge de
hâle partout et fort plein, mais sans aucune physionomie\,; les plus
belles jambes du monde, les pieds singulièrement petits et maigres. Il
tâtonnait toujours en marchant, et mettait le pied à deux fois\,; il
avait toujours peur de tomber, et il se faisait aider pour peu que le
chemin ne fut pas parfaitement droit et uni. Il était fort bien à cheval
et y avait grande mine, mais il n'y était pas hardi. Casau courait
devant lui à la chasse\,; s'il le perdait de vue il croyait tout
perdu\,; il n'allait guère qu'au petit galop, et attendait souvent sous
un arbre ce que devenait la chasse, la cherchait lentement et s'en
revenait. Il avait fort aimé la table, mais toujours sans indécence.
Depuis cette grande indigestion qui fut prise d'abord pour apoplexie, il
ne faisait guère qu'un vrai repas, et se contenait fort, quoique grand
mangeur comme toute la maison royale. Presque tous ses portraits lui
ressemblent bien.

De caractère, il n'en avait aucun\,; du sens assez, sans aucune sorte
d'esprit, comme il parut dans l'affaire du testament du roi d'Espagne\,;
de la hauteur, de la dignité par nature, par prestance, par imitation du
roi\,; de l'opiniâtreté sans mesure, et un tissu de petitesses arrangées
qui formaient tout le tissu de sa vie\,; doux par paresse et par une
sorte de stupidité\,; dur au fond, avec un extérieur de bonté qui ne
portait que sur des subalternes et sur des valets, et qui ne s'exprimait
que par des questions basses. Il était avec eux d'une familiarité
prodigieuse, d'ailleurs insensible à la misère et à la douleur des
autres, en cela peut-être plutôt en proie à l'incurie et à l'imitation
qu'à un mauvais naturel\,; silencieux à l'incroyable, conséquemmént fort
secret, jusque-là qu'on a cru qu'il n'avait jamais parlé d'affaires
d'État à la Choin, peut-être parce que tous {[}deux{]} n'y entendaient
guère. L'épaisseur d'une part, la crainte de l'autre, formaient en ce
prince une retenue qui a peu d'exemples\,; en même temps glorieux à
l'excès, ce qui est plaisant à dire d'un Dauphin jaloux de respect, et
presque uniquement attentif et sensible à tout ce qui lut était dû, et
partout. Il dit une fois à M\textsuperscript{lle} Choin, sur ce silence
dont elle lui parlait, que les paroles de gens comme lui portant un
grand poids, et obligeant aussi à de grandes réparations quand elles
n'étaient pas mesurées, il aimait mieux très-souvent garder le silence
que de parler. C'était aussi plus tôt fait pour sa paresse et sa
parfaite incurie\,; et cette maxime excellente, mais qu'il outrait,
était apparemment une des leçons du roi ou du duc de Montausier qu'il
avait le mieux retenue.

Son arrangement était extrême pour ses affaires particulières\,; il
écrivit lui-même toutes ses dépenses prises sur lui. Il savait ce que
lui coûtaient les moindres choses quoiqu'il dépensât infiniment en
bâtiments, en meubles, en joyaux de toute espèce, en voyages de Meudon,
et à l'équipage du loup dont il s'était laissé accroire qu'il aimait la
chasse. Il avait fort aimé toute sorte de gros jeu, mais depuis qu'il
s'était mis à bâtir il s'était réduit à des jeux médiocres. Du reste
avare au delà de toute bienséance, excepté de très-rares occasions qui
se bornaient à quelques pensions à des valets, ou à quelques médiocres
domestiques\,; mais assez d'aumônes au curé et aux capucins de Meudon.

Il est inconcevable le peu qu'il donnait à la Choin, si fort sa
bien-aimée. Cela ne passait point quatre cents louis par quartier, en
or, quoi qu'ils valussent, faisant pour tout seize cents louis par an.
Il les lui donnait lui-même, de la main à la main, sans y ajouter ni s'y
méprendre jamais d'une pistole, et tout au plus une boîte ou deux par
an, encore y regardait-il de fort près.

Il faut rendre justice à cette fille et convenir aussi qu'il est
difficile d'être plus désintéressée qu'elle l'était, soit qu'elle en
connût la nécessité avec ce prince, soit plutôt que cela lui fût
naturel, comme il a paru dans tout le tissu de sa vie. C'est encore un
problème si elle était mariée. Tout ce qui a été le plus intimement
initié dans leurs mystères s'est toujours fortement récrié qu'il n'y a
jamais eu de mariage. Ce n'a jamais été qu'une grosse camarde brune,
qui, avec toute la physionomie d'esprit et aussi de jeu, n'avait l'air
que d'une servante, et qui longtemps avant cet événement-ci était
devenue excessivement grasse et encore vieille et puante. Mais de la
voir aux \emph{parvulo }de Meudon, dans un fauteuil devant Monseigneur,
en présence de tout ce qui y était admis, M\textsuperscript{me} la
duchesse de Bourgogne et M\textsuperscript{me} la duchesse de Berry, qui
y fut tôt introduite, chacune sur un tabouret, dire devant Monseigneur
et tout cet intérieur «\,la duchesse de Bourgogne\,» et «\,la duchesse
de Berry\,» et «\,le duc de Berry,\,» en parlant d'eux, répondre souvent
sèchement aux deux filles de la maison, les reprendre, trouver à redire
à leur ajustement, et quelquefois à leur air et à leur conduite, et le
leur dire, on a peine à tout cela à ne pas reconnaître la belle-mère et
la parité avec M\textsuperscript{me} de Maintenon. À la vérité, elle ne
disait pas \emph{mignonne} en parlant à M\textsuperscript{me} la
duchesse de Bourgogne, qui l'appelait \emph{mademoiselle}, et non
\emph{ma tante}\,; mais aussi c'était toute la différence d'avec
M\textsuperscript{me} de Maintenon. D'ailleurs encore, cela n'avait
jamais pris de même entre elles. M\textsuperscript{me} la Duchesse, les
deux Lislebonne et tout cet intérieur y était un obstacle\,; et
M\textsuperscript{me} la duchesse de Bourgogne, qui le sentait et qui
était timide, se trouvait toujours gênée et en brassière à Meudon,
tandis qu'entre le roi et M\textsuperscript{me} de Maintenon elle
jouissait de toute aisance et de toute liberté. De voir encore
M\textsuperscript{lle} Choin à Meudon, pendant une maladie si
périlleuse, voir Monseigneur plusieurs fois le jour, le roi
non-seulement le savoir, mais demander à M\textsuperscript{me} de
Maintenon, qui, à Meudon non plus qu'ailleurs, ne voyait personne, et
qui n'entra peut-être pas deux fois chez Monseigneur, lui demander,
dis-je, si elle avait vu la Choin, et trouver mauvais qu'elle ne l'eût
pas vue, bien loin de la faire sortir du château, comme on le fait
toujours en ces occasions, c'est encore une preuve du mariage d'autant
plus grande que M\textsuperscript{me} de Maintenon, mariée elle-même, et
qui affichait si fort la pruderie et la dévotion, n'avait, ni le roi non
plus, aucun intérêt d'exemple et de ménagement à garder là-dessus, s'il
n'y avait point de sacrement, et on ne voit point qu'en aucun temps, la
présence de M\textsuperscript{lle} Choin ait causé le plus léger
embarras. Cet attachement incompréhensible, et si semblable en tout à
celui du roi, à la figure près de la personne chérie, est peut-être
l'unique endroit par où le fils ait ressemblé au père.

Monseigneur, tel pour l'esprit qu'il vient d'être représenté, n'avait pu
profiter de l'excellente culture qu'il reçut du duc de Montausier, et de
Bossuet et de Fléchier, évêques de Meaux et de Nîmes. Son peu de
lumières, s'il en eut jamais, s'éteignit au contraire sous la rigueur
d'une éducation dure et austère, qui donna le premier poids à sa
timidité naturelle, et le dernier degré d'aversion pour toute espèce,
non pas de travail et d'étude, mais d'amusement d'esprit, en sorte que,
de son aveu, depuis qu'il avait été affranchi des maîtres, il n'avait de
sa vie lu que l'article de Paris de la \emph{Gazette de France}, pour y
voir les morts et les mariages.

Tout contribua donc en lui, timidité naturelle, dur joug d'éducation,
ignorance parfaite et défaut de lumière, à le faire trembler devant le
roi, qui, de son côté, n'omit rien pour entretenir et prolonger cette
terreur toute sa vie. Toujours roi, presque jamais père avec lui, ou,
s'il lui en échappa bien rarement quelques traits, ils ne furent jamais
purs et sans mélange de royauté, non pas même dans les moments les plus
particuliers et les plus intérieurs. Ces moments mêmes étaient rares
tête à tête, et n'étaient que des moments presque toujours en présence
des bâtards et des valets intérieurs, sans liberté, sans aisance,
toujours en contrainte et en respect, sans jamais oser rien hasarder ni
usurper, tandis que tous les jours il voyait faire l'un et l'autre au
duc du Maine avec succès, et M\textsuperscript{me} la duchesse de
Bourgogne dans une habitude de tous les temps particuliers, des plus
familiers badinages, et des privautés avec le roi quelquefois les plus
outrées. Il en sentait contre eux une secrète jalousie, mais qui ne
l'élargissait pas. L'esprit ne lui fournissait rien comme à M. du Maine,
fils d'ailleurs de la personne et non de la royauté, et en telle
disproportion, qu'elle n'était point en garde. Il n'était plus de l'âge
de M\textsuperscript{me} la duchesse de Bourgogne, à qui on passait
encore les enfances par habitude et par la grâce qu'elle y mettait. Il
ne lui restait donc que la qualité de fils et de successeur, qui était
précisément ce qui tenait le roi en garde, et lui sous le joug. Il
n'avait donc pas l'ombre seulement de crédit auprès du roi. Il suffisait
même que son goût se marquât pour quelqu'un pour que ce quelqu'un en
sentît un contre-coup nuisible\,; et le roi était si jaloux de montrer
qu'il ne pouvait rien qu'il n'a rien fait pour aucun de ceux qui se sont
attachés à lui faire une cour plus particulière, non pas même pour aucun
de ses menins, quoique choisis et nommés par le roi, qui même eût trouvé
très-mauvais qu'ils n'eussent pas suivi Monseigneur avec grande
assiduité. J'en excepte d'Antin qui a été sans comparaison de personne,
et Dangeau qui ne l'a été que de nom, qui tenait au roi d'ailleurs, et
dont la femme était dans la parfaite intimité de M\textsuperscript{me}
de Maintenon. Les ministres n'osoient s'approcher de Monseigneur, qui
aussi ne se commettait comme jamais à leur rien demander, et si
quelqu'un d'eux ou des courtisans considérables étaient bien avec lui,
comme le chancelier, le Premier, Harcourt, le maréchal d'Huxelles, ils
s'en cachaient avec un soin extrême, et Monseigneur s'y prêtait. Si le
roi le découvrait, il traitait cela de cabale. On lui devenait suspect
et on se perdait. Ce fut la cause de l'éloignement si marqué pour M. de
Luxembourg, que ni la privance de sa charge, ni la nécessité de s'en
servir à la tête des armées, ni les succès qu'il y eut, ni toutes les
flatteries et les bassesses qu'il employa, ne purent jamais
rapprocher\,; aussi Monseigneur, pressé de s'intéresser pour quelqu'un,
répondait franchement que ce serait le moyen de tout gâter pour lui.

Il lui est quelquefois échappé des monosyllabes de plaintes amères
là-dessus, quelquefois après avoir été refusé du roi et toujours avec
sécheresse\,; et la dernière fois de sa vie qu'il alla à Meudon, d'où il
ne revint plus, il y arriva si outré d'un refus de fort peu de chose
qu'il avait demandé au roi pour Casau, qui me l'a conté, qu'il lui
protesta qu'il ne lui arriverait jamais plus de s'exposer pour personne,
et de dépit le consola par les espérances d'un temps plus favorable,
lorsque la nature l'ordonnerait, qui était pour lui dire comme par
prodige. Ainsi on remarquera en passant, que Monsieur et Monseigneur
moururent tous deux dans des moments où ils étaient outrés contre le
roi.

La part entière que Monseigneur avait à tous les secrets de l'État,
depuis bien des années, n'avait jamais eu aucune influence aux affaires,
il les savait et c'était tout. Cette sécheresse, peut-être aussi son peu
d'intelligence, l'en faisait retirer tant qu'il pouvait. Il était
cependant assidu aux conseils d'État\,; mais, quoiqu'il eût la même
entrée en ceux de finance et de dépêches, il n'y allait presque jamais.
Quant au travail particulier du roi, il n'en fut pas question pour lui,
et hors de grandes nouvelles, pas un ministre n'allait jamais lui rendre
compte de rien\,; beaucoup moins les généraux d'armée, ni ceux qui
revenaient d'être employés au dehors.

Ce peu d'onction et de considération, cette dépendance, jusqu'à la mort,
de n'oser faire un pas hors de la cour sans le dire au roi, équivalent
de permission, y mettait Monseigneur en malaise. Il y remplissait les
devoirs de fils et de courtisan avec la régularité la plus exacte, mais
toujours la même, sans y rien ajouter, et avec un air plus respectueux
et plus mesuré qu'aucun sujet. Tout cela ensemble lui faisait trouver
Meudon et la liberté qu'il y goûtait délicieuse\,; et bien qu'il ne tînt
qu'à lui de s'apercevoir souvent que le roi était peiné de ces
fréquentes séparations et par la séparation même, et par celle de la
cour, surtout les étés qu'elle n'était pas nombreuse à cause de la
guerre, il n'en fit jamais semblant, et ne changea rien en ses voyages,
ni pour leur nombre ni pour leur durée. Il était fort peu à Versailles,
et rompait souvent par des Meudons de plusieurs jours les Marlys quand
ils s'allongeaient trop. De tout cela, on peut juger quelle pouvait être
la tendresse de cœur\,; mais le respect, la vénération, l'admiration,
l'imitation en tout ce qui était de sa portée était visible, et ne se
démentit jamais, non plus que la crainte, la frayeur, et la conduite.

On a prétendu qu'il avait une appréhension extrême de perdre le roi. Il
n'est pas douteux qu'il n'ait montré ce sentiment\,; mais d'en concilier
la vérité avec celles qui viennent d'être rapportées, c'est ce qui ne
paraît pas aisé. Toujours est-il certain que, quelques mois avant sa
mort, M\textsuperscript{me} la duchesse de Bourgogne l'étant allée voir
à Meudon, elle monta dans le sanctuaire de son entre-sol, suivie de
M\textsuperscript{me} de Nogaret, qui par Biron et par elle-même encore
en avait la privance, et qu'elles y trouvèrent Monseigneur, avec
M\textsuperscript{lle} Choin, M\textsuperscript{me} la duchesse et les
deux Lislebonne, fort occupés à une table sur laquelle était un grand
livre d'estampes du sacre, et Monseigneur fort appliqué à les
considérer, à les expliquer à la compagnie, et recevant avec
complaisance les propos qui le regardaient là-dessus, jusqu'à lui
dire\,: «\,Voilà donc celui qui vous mettra les éperons, cet autre le
manteau royal, les pairs qui vous mettront la couronne sur la tête,\,»
et ainsi du reste, et que cela dura fort longtemps. Je le sus deux jours
après de M\textsuperscript{me} de Nogaret, qui en fut fort étonnée, et
que l'arrivée de M\textsuperscript{me} la duchesse de Bourgogne n'eût
pas interrompu cet amusement singulier, qui ne marquait pas une si
grande appréhension de perdre le roi et de le devenir lui-même.

Il n'avait jamais pu aimer M\textsuperscript{me} de Maintenon, ni se
ployer à obtenir rien par son entremise. Il l'allait voir un moment au
retour du peu de campagnes qu'il a faites, ou aux occasions
très-rares\,; jamais de particulier\,; quelquefois il entrait chez elle
un instant avant le souper, pour y suivre le roi. Elle aussi avait à son
égard une conduite fort sèche, et qui lui faisait sentir qu'elle le
comptait pour rien. La haine commune des deux sultanes contre
Chamillart, et le besoin de tout pour le renverser, les rapprocha comme
il a été dit, et fit le miracle d'y faire entrer puissamment
Monseigneur\,; mais qui ne l'eût jamais osé sans l'impulsion
toute-puissante de la sienne, la sûreté de l'appui de l'autre, et tout
ce qui s'en mêla. Aussi ce rapprochement ne fit depuis que se refroidir
et s'éloigner peu à peu.

Avec M\textsuperscript{lle} Choin, sa vraie confiance était en
M\textsuperscript{lle} de Lislebonne, et par l'intime union des deux
sœurs, avec M\textsuperscript{me} d'Espinoy. Presque tous les matins, il
allait prendre du chocolat chez la première. C'était l'heure des
secrets, qui était inaccessible sans réserve, excepté à l'unique
M\textsuperscript{me} d'Espinoy. Par elles plus que par soi-même, tenait
le reste de considération et de commerce avec M\textsuperscript{me} la
princesse de Conti et même l'amitié avec M\textsuperscript{me} la
Duchesse, que soutenaient les amusements qu'il trouvait chez elle. Par
là encore, cette préférence du duc de Vendôme sur le prince de Conti, à
la mort duquel il fut si indécemment insensible. Un tel mérite si
reconnu dans un prince du sang, joint à la privance de l'éducation
presque commune, et à l'habitude de toute la vie, aurait eu trop de
poids sur Monseigneur devenu roi, si l'amitié première s'était
conservée, et les sœurs, qui voulaient gouverner, écartèrent doucement
ce prince. Cette même raison fut, comme on l'a dit, le fondement de
cette terrible cabale, dont les effets éclatèrent dans la campagne de
Lille, et furent soigneusement entretenus depuis dans l'esprit de
Monseigneur, naturellement éloigné de la contrainte et de l'austérité
des mœurs de Mgr le duc de Bourgogne, {[}éloignement{]} que la haine de
M\textsuperscript{me} la Duchesse pour M\textsuperscript{me} la duchesse
de Bourgogne entretenait pour tous les deux. Par les raisons contraires,
il aimait M. le duc de Berry, que cette cabale protégeait pour le
diviser d'avec Mgr {[}le duc{]} et M\textsuperscript{me} la duchesse de
Bourgogne, tellement, qu'après toute leur opposition et leur dépit à
tous de son mariage, M\textsuperscript{me} la duchesse de Berry ne
laissa pas d'être admise aussitôt après au \emph{parvulo}, sans même
l'avoir demandé, et d'y être fort bien traitée.

Avec tout cet ascendant des deux Lislebonne sur Monseigneur, il est
pourtant vrai qu'il n'épousait pas toutes leurs fantaisies, soit par la
Choin, qui, tout en les ménageant, les connaissoit bien et ne s'y fiait
point, comme Bignon me l'avait dit, soit par M\textsuperscript{me} la
Duchesse, qui sûrement ne s'y fiait pas davantage, et qui n'était rien
moins que coiffée de leurs prétentions. Inquiet à cet égard pour le
futur, j'employai l'évêque de Laon pour découvrir par la Choin les
sentiments de Monseigneur entre les ducs et les princes. Il était frère
de Clermont, qui avait été perdu pour elle, lorsque
M\textsuperscript{me} la princesse de Conti la chassa, et les deux
frères étaient demeurés dans la plus intime liaison avec elle. Je sus
par lui qu'il était échappé quelquefois, quoique rarement, des choses à
Monseigneur, qui montraient que tout l'empire que ces deux sœurs avaient
sur lui n'allait pas à le rendre aussi favorable à leur rang qu'elles
eussent voulu, et que M\textsuperscript{lle} Choin l'ayant plus
particulièrement sondé là-dessus, à la prière de l'évêque, il s'était
expliqué fort favorablement pour le rang des ducs, et contre les
injustices qu'il était persuadé qu'ils avaient souffertes. Il était
incapable non-seulement de mensonge, mais de déguisement, et la Choin
tout aussi peu capable, surtout avec l'évêque, duquel elle ne se cachait
pas non plus qu'à Bignon, de ses secrets sentiments sur
M\textsuperscript{lle} de Lislebonne et M\textsuperscript{me} d'Espinoy.

Celte réponse de M. de Laon me fît souvenir de celle que Monseigneur fit
au roi, qui le trouva, comme je l'ai raconté, dans ses arrière-cabinets,
au sortir de cette audience que je lui avais emblée dans son cabinet sur
l'affaire de la quête, et le roi en ayant parlé à Monseigneur avec
satisfaction, ce prince à qui j'étais au moins très-indifférent, et
qu'on n'avait point instruit de notre part, lui dit qu'il savait bien
que j'avais raison.

M\textsuperscript{lle} Choin a prétendu et soutenu depuis sa mort (car
pendant sa vie il ne sortait rien d'elle) qu'il avait autant
d'opposition au mariage de M\textsuperscript{lle} de Bourbon qu'à celui
de Mademoiselle, parce qu'il ne pouvait soutfrir le mélange du sang
bâtard au sien. Peut-être était-il vrai. Il a toujours montré une
aversion constante à tous leurs avantages, et il ne lui est rien échappé
de marqué en faveur de M\textsuperscript{lle} de Bourbon pour le mariage
de M. le duc de Berry. Mais l'autorité de M\textsuperscript{me} la
Duchesse était si entière sur lui, et si solidement appuyée de celle de
tout ce qui le gouvernait, et la réunion de toute la cabale était si
grande en faveur de M\textsuperscript{lle} de Bourbon, et se montrait si
assurée là-dessus, qu'elle l'y eût sans doute amené s'il ne l'était
déjà, comme on eut tant de raisons de le croire, opinion qui servit si
utilement Mademoiselle. La Choin a même avoué depuis qu'elle-même était
contraire à tous les deux par cette raison de bâtardise. De celui de
Mademoiselle, cela n'est pas douteux. On a vu, par ce qui se passa entre
Bignon et moi, à quel point elle était éloignée de M. le duc d'Orléans.
De l'autre, il se pouvait bien que les vues de l'avenir lui faisaient
craindre d'ajouter ce poids d'union et de crédit à M\textsuperscript{me}
la Duchesse\,; mais ses liaisons présentes avec elle, par ce
qu'elle-même en avoua à Bignon, et qu'il me rendit, étaient si
nécessaires, si grandes, si intimes, qu'il y a fort à douter qu'elle eût
pu éviter d'y être entraînée, et que, éclairée surtout d'aussi près
qu'elle l'était par un aussi grand intérêt et de M\textsuperscript{me}
la Duchesse, et des deux Lislebonne qui en prenaient pour les leurs
autant que M\textsuperscript{me} la Duchesse elle-même, et par d'Antin,
tout elles là-dessus, M\textsuperscript{lle} Choin eût osé se laisser
apercevoir contraire, et qu'avec un prince aussi faible et aussi
puissamment environné, elle eût osé hasarder de soutenir contre ce
torrent toujours présent, elle si souvent absente.

Il ne faut pas taire un beau trait de cette fille ou femme si
singulière. Monseigneur, sur le point d'aller commander l'armée de
Flandre la campagne d'après celle de Lille, où pourtant il n'alla pas,
fit un testament, et dans ce testament un bien fort considérable à
M\textsuperscript{lle} Choin. Il le lui dit, et lui montra une lettre
cachetée pour elle qui en faisait mention, pour lui être rendue s'il
mésarrivait de lui. Elle fut extrêmement sensible, comme il est aisé de
le juger à une marque d'affection de cette prévoyance, mais elle n'eut
point de repos qu'elle ne lui eût fait mettre devant elle le testament
et la lettre au feu\,; et protesta que si elle avait le malheur de lui
survivre, mille écus de rente qu'elle avait amassés seraient encore trop
pour elle. Après cela, il est surprenant qu'il ne se soit trouvé aucune
disposition dans les papiers de Monseigneur.

Quelque dure qu'ait été son éducation, il avait conservé de l'amitié et
de la considération pour le célèbre évêque de Meaux, et un vrai respect
pour la mémoire du duc de Montausier, tant il est vrai que la vertu se
fait honorer des hommes malgré leur goût et leur amour de l'indépendance
et de la liberté. Monseigneur n'était pas même insensible au plaisir de
la marquer à tout ce qui était de sa famille, et jusqu'aux anciens
domestiques qu'il lui avait connus. C'est peut-être une des choses qui a
le plus soutenu d'Antin auprès de lui dans les diverses aventures de sa
vie, dont la femme était fille de la duchesse d'Uzès, fille unique du
duc de Montausier, et qu'il aimait passionnément. Il le marqua encore à
Sainte-Maure, qui, embarrassé dans ses affaires sur le point de se
marier, reçut une pension de Monseigneur sans l'avoir demandée, avec ces
obligeantes paroles, mais qui faisaient tant d'honneur au prince\,:
«\,qu'il ne manquerait jamais au nom et au neveu de M. de Montausier.\,»
Sainte-Maure se montra digne de cette grâce. Son mariage se rompit, et
il ne s'est jamais marié. Il remit la pension qui n'était donnée qu'en
faveur du mariage. Monseigneur la reprit\,; je ne dirai pas qu'il eût
mieux fait de la lui laisser.

C'était peut-être le seul homme de qualité qu'il aida de sa poche. Aussi
tenait-il à lui par des confidences, tandis qu'il eut des maîtresses\,;
que le roi ne lui souffrit guère. En leur place, il eut plutôt des
soulagements passagers et obscurs que des galanteries dont il était peu
capable, et que du Mont et Francine, gendre de Lulli, et qui eurent si
longtemps ensemble l'Opéra, lui fournirent.

À ce propos, je ne puis m'empêcher de rapporter un échantillon de sa
délicatesse. Il avait eu envie d'une de ces créatures fort jolie. À jour
pris, elle fut introduite à Versailles dans un premier cabinet avec une
autre, vilaine, pour l'accompagner. Monseigneur, averti qu'elles étaient
là, ouvrit la porte, et prenant celle qui était la plus proche, la tira
après lui. Elle se défendit\,; c'était la vilaine qui vit bien qu'il se
méprenait\,; lui, au contraire, crut qu'elle faisait des façons, la
poussa dedans et ferma sa porte\,; l'autre cependant riait de la méprise
et de l'affront qu'elle s'attendait qu'allait avoir sa compagne d'être
renvoyée, et elle appelée. Fort peu après, du Mont entra, qui, fort
étonné de la voir là et seule, lui demanda ce qu'elle faisait là, et
qu'était devenue son amie. Elle lui conta l'aventure. Voilà du Mont à
frapper à la porte, et à crier\,: «\,Ce n'est pas celle-là\,; vous vous
méprenez.\,» Point de réponse. Du Mont redouble encore sans succès.
Enfin Monseigneur ouvre sa porte et pousse sa créature dehors. Du Mont
s'y présente avec l'autre, en disant\,: «\,Tenez donc, la voilà. ---
L'affaire est faite, dit Monseigneur\,; ce sera pour une autre fois,\,»
et referma sa porte. Qui fut honteuse et outrée\,? ce fut celle qui
avait ri, et plus qu'elle du Mont encore. La laide avait profité de la
méprise, mais elle n'osa se moquer d'eux\,; la jolie fut si piquée
qu'elle le conta à ses amis, tellement qu'en bref toute la cour en sut
l'histoire.

La Raisin, fameuse comédienne et fort belle, fut la seule de celles-là
qui dura et figura dans son obscurité. On la ménageait. Et le maréchal
de Noailles, à son âge et avec sa dévotion, n'était pas honteux de
l'aller voir, et de lui fournir, à Fontainebleau, de sa table tout ce
qu'il y avait de meilleur. Il n'eut d'enfants de toutes ces sortes de
créatures, qu'une seule fille de celle-ci, assez médiocrement
entretenue, à Chaillot, chez les Augustines. Cette fille fut mariée
depuis sa mort par M\textsuperscript{me} la princesse de Conti, qui en
prit soin, à un gentilhomme qui la perdit bientôt après. Celte
indigestion qu'on prit pour une apoplexie mit fin à tous ces commerces.
À son éloignement de la bâtardise, il y a apparence qu'il n'eût jamais
reconnu aucun de ces sortes d'enfants. Il n'avait jamais pu souffrir M.
du Maine, qui l'avait peu ménagé dans les premiers temps, et qui en
était bien en peine et en transe dans les derniers, il traitait le comte
de Toulouse avec assez d'amitié, qui avait toute sa vie eu pour lui de
grandes attentions à lui plaire et de grands respects. Ce qui était ou
le mieux ou le plus familièrement avec lui parmi les courtisans étaient
d'Antin et le comte de Mailly, mari de la dame d'atours, mais mort il y
avait longtemps. C'étaient en petit les deux rivaux de faveur, comme en
grand M. le prince de Conti et M. de Vendôme. Les ducs de Luxembourg,
Villeroy et de La Rocheguyon, et ceux-là sur un pied de considération et
de quelque confiance\,; Sainte-Maure, le comte de Roucy, Biron et
Albergotti, voilà les distingués et les marqués. De vieux seigneurs,
cela l'était moins, et qui le voyaient très-peu chez lui\,: M. de La
Rochefoucauld, les maréchaux de Boufflers, de Duras, de Lorges, Catinat,
il les traitait avec plus d'affabilité et de familiarité\,; feu M. de
Luxembourg et Clermont, frère de M. de Laon, c'était l'intimité, j'en ai
parlé ailleurs\,; le maréchal de Choiseul encore avec considération\,;
sur les fins, le maréchal d'Huxelles, mais qui s'en cachait comme
Harcourt, le chancelier et le premier écuyer, qui l'avait initié auprès
de M\textsuperscript{lle} Choin, qui s'en était entêtée, et avait
persuadé à Monseigneur que c'était le plus capable homme du monde pour
tout. Elle avait une chienne dont elle était folle, à qui tous les jours
le maréchal d'Huxelles, de la porte Gaillon où il logeait, envoyait des
têtes de lapin rôties attenant le Petit-Saint-Antoine où elle logeait,
et où le maréchal allait souvent et était reçu et regardé comme un
oracle. Le lendemain de la mort de Monseigneur, l'envoi des têtes de
lapins cessa, et oncques depuis M\textsuperscript{lle} Choin ne le revit
ni n'en ouït parler. À la fin, lorsqu'elle fut revenue à elle-même, elle
s'en aperçut, elle s'en plaignit même comme d'un homme sur qui elle
avait eu lieu de compter, et qu'elle avait fort avancé dans l'estime et
la confiance de Monseigneur. Le maréchal d'Huxelles le sut\,; il n'en
fut point embarrassé, et répondit froidement qu'il ne savait pas ce
qu'elle voulait dire, qu'il ne l'avait jamais vue que fort rarement et
fort généralement, et que pour Monseigneur à peine en était-il connu.
C'était un homme qui courait en cachette, mais plus bassement et plus
avidement que personne, à tout ce qui le pouvait conduire, et qui
n'aimait pas à se charger de reconnaissance inutile. Néanmoins cela fut
su, et ne lui fit pas honneur.

Monseigneur n'eut que deux hommes d'aversion dans toute la cour, et
cette aversion ne lui était pas inspirée comme celle de Chamillart et de
quelques autres\,: ces deux hommes étaient le maréchal de Villeroy et M.
de Lauzun\,; il était ravi dès qu'il y avait quelque bon conte sur eux.
Le maréchal était plus ménagé, mais pas assez pour que lui-même n'en fût
pas souvent embarrassé. Pour l'autre, Monseigneur ne s'en pouvait
contraindre\,; et M. de Lauzun, au contraire du maréchal, ne s'en
embarrassait point. Je n'ai point démêlé où il avait pris son aversion.
Il en avait une fort marquée pour les ducs de Chevreuse et de
Beauvilliers, mais c'était l'effet de la cabale aidée de l'entière
disparité des mœurs.

À ce qui a été rapporté de l'incompréhensible crédulité de Monseigneur
sur ce qui me regarde, et de la facilité avec laquelle
M\textsuperscript{me} la duchesse de Bourgogne l'en fit revenir, jusqu'à
lui en donner de la honte, on reconnaît aisément de quelle trempe était
son esprit et son discernement\,; aussi ceux qui l'avaient englobé, et
qui avaient si beau jeu à l'infatuer de tout ce qu'ils voulaient,
n'eurent-ils aucune peine à le tenir éloigné de Mgr le duc de Bourgogne,
et de l'en éloigner de plus en plus, par le grand intérêt qui a été mis
au net plus d'une fois. On peut juger aussi ce qu'eût été le règne d'un
tel prince livré en de telles mains. La division entre les deux princes
était remarquée de toute la cour. Les mœurs du fils, sa piété, son
application à s'instruire, ses talents, son esprit, toutes choses si
satisfaisantes pour un père, étaient autant de démérites, parce que
c'étaient autant de motifs de craindre qu'il eût part au gouvernement,
sous un père qui en eût connu le prix. La réputation qui en naissait
était un autre sujet de crainte. La façon dont le roi commençait à le
traiter en fut un de jalousie, et tout cela fut mis en œuvre de plus en
plus. Le jeune prince glissait, avec un respect et une douceur qui
aurait ramené tout autre qu'un père qui ne voyait et ne sentait que par
autrui. M\textsuperscript{me} la duchesse de Bourgogne partageait les
mauvaises grâces de son époux, et si elle usurpait plus de liberté et de
familiarité que lui, elle essuyait aussi des sécheresses et quelquefois
des duretés dont la circonspection du jeune prince le garantissait. Il
voyait Monseigneur plus en courtisan qu'en fils, sans particulier, sans
entretien tête a tête\,; et on s'apercevait aisément que, le devoir
rempli, il ne cherchait pas Monseigneur, et se trouvait mieux partout
ailleurs qu'auprès de lui. M\textsuperscript{me} la duchesse avait fort
augmenté cette séparation, surtout depuis le mariage de M. le duc de
Berry\,; et quoique dès auparavant Monseigneur commençât à traiter moins
bien M\textsuperscript{me} la duchesse de Bourgogne, plus durement
pendant la campagne de Lille, et surtout après l'expulsion du duc de
Vendôme de Marly et de Meudon, les mesures s'étaient moins gardées
depuis le mariage. Ce n'était pas que l'adroite princesse ne ramât
contre le fil de l'eau avec une application et des grâces capables de
désarmer un ressentiment fondé, et que souvent elle ne réussît à ramener
Monseigneur par intervalles\,; mais les personnes qui l'obsédaient
regardaient la fonte de ses glaces comme trop dangereuse pour leurs
projets, pour souffrir que la fille de la maison se remît en grâces,
tellement que M. le duc de Bourgogne, privé des secours qu'il avait
auparavant de ce côté-là par elle, tous deux se trouvaient de jour en
jour plus éloignés, et moins en état de se rapprocher. Les choses se
poussèrent\,: même si loin là-dessus peu avant la mort de Monseigneur,
sur une partie acceptée par lui à la Ménagerie et qui fut rompue, que
M\textsuperscript{me} la duchesse de Bourgogne voulut enfin essayer
d'autres moyens que ceux de la patience et de la complaisance qu'elle
avait seuls employés jusqu'alors, et qu'elle fit sentir aux deux
Lislebqnne qu'elle se prendrait à elles des contre-temps qui lui
arriveraient de la part de Monseigneur. Toute la cabale trembla de la
menace, moins pour l'avenir que pour le temps présent, que la santé du
roi promettait encore durable. Ils n'avaient garde de quitter prise,
leur avenir si projeté en dépendait\,; mais la conduite pour le présent
leur devenait épineuse par ce petit trait d'impatience et de vigueur.
Les deux sœurs recherchèrent une explication qui leur fut refusée.
M\textsuperscript{me} la Duchesse s'alarma pour elle-même, et d'Antin en
passa de mauvais quarts d'heure. Monseigneur essaya de raccommoder ce
qui s'était passé par des honnêtetés, qu'on sentit exigées, mais ils
tinrent bon sur la partie qui ne s'exécuta point\,; et après quelque
temps de bonace peu naturelle, les choses reprirent leur cours,
toutefois avec un peu plus de ménagement, mais qui servit moins à
montrer les remèdes qu'à découvrir le danger de plus en plus.

On a vu, à propos des choses de Flandre, que là même cabale qui
travaillait avec tant d'ardeur, d'audace et de suite, à perdre,
M\textsuperscript{me} la duchesse de Bourgogne auprès de Monseigneur, et
à anéantir Mgr le duc de Bourgogne, ne s'était pas moins appliquée à
augmenter l'amitié que la conformité de mœurs et de goût nourrissait en
Monseigneur pour M. le duc de Berry, duquel rien n'était à craindre pour
les vues de, l'avenir\,; et on a vu depuis que, quelque rage qu'ils
eussent tous de son mariage, ils avaient fait bien traiter
M\textsuperscript{me} la duchesse de Berry par Monseigneur, jusqu'à la
faire admettre tout de suite, et sans qu'elle l'eût demandé, dans ce
sanctuaire du \emph{parvulo}. Ils voulaient ainsi ôter le soupçon qu'ils
eussent dessein d'éloigner tous les enfants de la maison, et tâcher de
diviser les deux frères si unis, et semer entre eux la jalousie. La
moitié leur réussit par la voie la plus inattendue, mais le principal
leur manqua. Jamais l'union intime des frères ne put recevoir, de part
ni d'autre, l'altération la plus légère, quelques machines, même
domestiques, qui s'y pussent employer. Mais M\textsuperscript{me} la
duchesse de Berry se trouva aussi méchante qu'eux, et aussi pleine de
vues. M. le duc d'Orléans appelait souvent M\textsuperscript{me} la
duchesse d'Orléans M\textsuperscript{me} Lucifer\,; et elle en souriait
avec complaisance. Il avait raison, elle eût été un prodige d'orgueil si
elle n'eût pas eu une fille\,; mais cette fille la surpassa de beaucoup.
Il n'est pas temps ici de faire le portrait ni de l'une ni de l'autre\,;
je me contenterai sur M\textsuperscript{me} la duchesse de Berry de ce
qu'il est nécessaire d'expliquer sur ce dont il s'agit, en deux mots.

C'était un prodige d'esprit, d'orgueil, d'ingratitude et de folie, et
c'en fut un aussi de débauche et d'entêtement. À peine fut-elle huit
jours mariée qu'elle commença à se développer sur tous ces points, que
la fausseté suprême qui était en elle, et dont même elle se piquait
comme d'un excellent talent, ne laissa pas d'envelopper un temps, quand
l'humeur la laissait libre, mais qui la dominait souvent. On s'aperçut
bientôt de son dépit d'être née d'une mère bâtarde, et d'en avoir été
contrainte, quoique avec des ménagements infinis\,; de son mépris pour
la faiblesse de M. le duc d'Orléans, et de sa confiance en l'empire
qu'elle avait pris sur lui\,; de l'aversion qu'elle avait conçue contre
toutes les personnes qui avaient eu part à son mariage, parce qu'elle
était indignée de penser qu'elle pût avoir obligation à quelqu'un, et
elle eut bientôt après la folie non-seulement de l'avouer, mais de s'en
vanter. Ainsi elle ne tarda pas d'agir en conséquence. Et voilà comme on
travaille en ce monde la tête dans un sac, et que la prudence et la
sagesse humaine sont confondues jusque dans les succès le plus
raisonnablement désirés, et qui se trouvent après les plus
détestables\,! Toutes les machines de ce mariage avaient porté sur deux
points d'objets principaux\,: l'un d'empêcher celui de
M\textsuperscript{lle} de Bourbon, par tant de raisons et si
essentielles qu'on en a vues\,; l'autre d'assurer cette union si
heureuse, si désirable, si bien cimentée, entre les deux frères et
M\textsuperscript{me} la duchesse de Bourgogne, qui faisait le bonheur
solide et la grandeur de l'État, la paix et la félicité de la famille
royale, la joie et la tranquillité de la cour, et qui mettait, autant
qu'il était possible, un frein à tout ce qu'on avait à craindre du règne
de Monseigneur. Il se trouve, par ce qui a été remarqué de
M\textsuperscript{lle} Choin, que peut-être le mariage de
M\textsuperscript{lle} de Bourbon ne se serait point fait, et qu'on lui
substitue une furie qui ne songe qu'à perdre tout ce qui l'a établie, à
brouiller les frères, à perdre sa bienfaitrice parce qu'elle l'est, à se
livrer à ses ennemis parce qu'ils sont ceux de Mgr {[}le duc{]} et de
M\textsuperscript{me} la duchesse de Bourgogne, et à se promettre de
gouverner Monseigneur Dauphin et roi par des personnes outrées contre
son mariage, et pleines de haine contre M. {[}le duc{]} et
M\textsuperscript{me} la duchesse d'Orléans, qui ont attenté et
attentaient sans cesse à l'anéantissement de Mgr {[}le duc{]} et de
M\textsuperscript{me} la duchesse de Bourgogne, pour gouverner seuls
Monseigneur et l'État quand il en serait devenu le maître, et qui
n'étaient pas sûrement pour abandonner à M\textsuperscript{me} la
duchesse de Berry le fruit de leurs sueurs, de leurs travaux si longs et
si suivis, et de tant de ce qui se peut appeler crimes, pour arriver au
timon et le gouverner sans concurrence. Tel fut pourtant le sage, le
facile, l'honnête projet que M\textsuperscript{me} la duchesse de Berry
se mit dans la tête aussitôt après qu'elle fut mariée. On a vu que,
pendant tout le cours des menées de son mariage, M. le duc d'Orléans ne
lui en avait rien caché. Elle connut ainsi le tableau intérieur de la
cour, la cabale qui gouvernait Monseigneur, et la triste situation de
Mgr le {[}duc{]} et de M\textsuperscript{me} la duchesse de Bourgogne
avec lui. La différence si marquée de celle de M. le duc de Berry
qu'elle aperçut dès qu'elle fut mariée, et incontinent après de la
sienne même, les caresses qu'elle reçut de toute la cabale, les
agréments qu'elle éprouvait aux \emph{parvulo} où elle était témoin de
l'embarras, des sécheresses et des duretés qu'y essuyait
M\textsuperscript{me} la duchesse de Bourgogne, la persuadèrent du beau
dessein qu'elle se mit dans l'esprit, et d'y travailler sans perdre un
moment.

À ce qui vient d'être dit, on peut juger qu'elle n'était ni douce ni
docile aux premiers avis que M\textsuperscript{me} la duchesse d'Orléans
lui voulut donner\,; elle se rebéqua avec aigreur\,; et, sûre de faire
de M. le duc d'Orléans tout ce qu'elle voudrait, elle ne balança pas de
faire l'étrangère et la fille de France avec M\textsuperscript{me} sa
mère. La brouillerie ne tarda pas, et ne fit qu'augmenter sans cesse.
Elle en usa d'une autre façon, mais pour le fond de même, avec
M\textsuperscript{me} la duchesse de Bourgogne, qui avait compté la
conduire et en faire comme de sa fille, et qui sagement retira
promptement ses troupes et ne voulut plus s'en mêler pour éviter noise
et qu'elle ne lui fît des affaires avec M. le duc de Berry qu'elle avait
toujours aimé et traité comme son frère, lequel y avait répondu par
toute la conflance la plus entière et le respect le plus véritable.
Cette crainte ne fut que trop bien fondée, quoique toute occasion en fût
évitée.

Le projet de M\textsuperscript{me} la duchesse de Berry demandait la
discorde entre les deux frères. Pour y parvenir il fallait commencer par
la mettre entre le beau-frère et la belle-sœur. Cela fut extrêmement
difficile. Tout s'y opposait en M. le duc de Berry\,: raison, amitié,
complaisance, habitude, amusements, plaisirs, conseils et appui auprès
du roi et de M\textsuperscript{me} de Maintenon, intimité avec Mgr le
duc de Bourgogne. Mais M. le duc de Berry avait de la droiture, de la
bonté, de la vérité\,; il ne se doutait seulement pas ni de fausseté ni
d'artifice\,; il avait peu d'esprit, et, au milieu de tout, peu d'usage
du monde\,; enfin il était amoureux fou de M\textsuperscript{me} la
duchesse de Berry, et en admiration perpétuelle de son esprit et de son
bien-dire. Elle réussit donc peu à peu à l'éloigner de
M\textsuperscript{me} la duchesse de Bourgogne, et cela mit le comble
entre elles. C'étaient là des sacrifices bien agréables à la cabale à
qui elle voulait plaire, et à qui elle se dévoua. C'est où elle en était
lorsque Monseigneur mourut\,; et c'est ce qui la jeta dans cette rage de
douleur que personne de ce qui n'était pas instruit ne pouvait
comprendre. Tout à coup elle vit ses projets en fumée, elle réduite sous
une princesse qu'elle avait payée de l'ingratitude la plus noire, la
plus suivie, la plus gratuite, qui faisait les délices du roi et de
M\textsuperscript{me} de Maintenon, et qui sans contre-poids allait
régner d'avance en attendant l'effet. Elle ne voyait plus d'égalité
entre les frères par la disproportion du rang de Dauphin. Cette cabale à
qui elle avait sacrifié son âme était perdue pour l'avenir, et pour le
présent lui devenait plus qu'inutile\,; sans secours de la part d'une
mère offensée, ni du côté d'un père faible et léger, mal raffermi auprès
du roi, et foncièrement mal avec M\textsuperscript{me} de Maintenon,
réduite à dépendre du Dauphin et de la Dauphine, et pour le grand, et
pour l'agréable, et pour l'utile, et pour le futile, et à n'avoir de
considération et de consistance qu'autant qu'ils lui en voudraient bien
communiquer\,; et nulle ressource auprès d'eux que M. le duc de Berry
qu'elle avait comme brouillé avec celle qui influait d'une manière si
principale sur le roi, sur M\textsuperscript{me} de Maintenon, et sur
Mgr le duc de Bourgogne, dans tout ce qui n'était point affaires. Elle
sentait encore que M. le duc de Berry serait très-aisément distingué
d'elle, et de plus elle se pouvait dire bien des choses qui la mettaient
en de grands dangers à son égard, pour peu qu'on fût tenté de lui rendre
quelque change, ce qui était très-possible et très-impunément\,; voilà
aussi pourquoi elle lui marqua tant de soins, et tant de tendresse, et
qu'au milieu de son désespoir elle sut mettre à profit à son égard leur
commune douleur. Celle de M. le duc de Berry fut toute d'amitié, de
tendresse, de reconnaissance de celle qu'il avait toujours éprouvée de
Monseigneur, peut-être de sa situation présente avec
M\textsuperscript{me} la duchesse de Bourgogne, et d'avoir assez pris de
M\textsuperscript{me} la duchesse de Berry pour sentir toute la
différence de fils à frère de Dauphin et de roi, et dans la suite le
vide de Meudon et des parties avec Monseigneur aux plaisirs et à
l'amusement de sa vie.

Le roi d'Espagne subsistait dans le cœur de Monseigneur par le sentiment
ordinaire d'aimer davantage ceux pour qui on a grandement fait, et dont
on n'est pas à portée d'éprouver l'ingratitude ou la reconnaissance. La
cabale qui n'avait rien à craindre de si loin, et de plus liée, comme on
l'a vu, avec la princesse des Ursins au point où elle l'était,
entretenait avec soin l'amitié de Monseigneur pour ce prince, et lui
ôtait tout soupçon, en la fomentant pour deux de ses fils, d'aucun
mauvais dessein par leur conduite à l'égard de l'aîné, dont Monseigneur
ne voyait que ce qui se passait auprès de lui là-dessus.

De ce long et curieux détail il résulte que Monseigneur était sans vice
ni vertu, sans lumières ni connaissances quelconques, radicalement
incapable d'en acquérir, très-paresseux, sans imagination ni production,
sans goût, sans choix, sans discernement, né pour l'ennui qu'il
communiquait aux autres, et pour être une boule roulante au hasard par
l'impulsion d'autrui, opiniâtre et petit en tout à l'excès, de
l'incroyable facilité à se prévenir et à tout croire qu'on a vue\,;
livré aux plus pernicieuses mains, incapable d'en sortir ni de s'en
apercevoir, absorbé dans sa graisse et dans ses ténèbres, et que, sans
avoir aucune volonté de mal faire, il eût été un roi pernicieux.

\hypertarget{chapitre-viii.}{%
\chapter{CHAPITRE VIII.}\label{chapitre-viii.}}

1711

~

{\textsc{Obsèques {[}de Monseigneur{]}.}} {\textsc{-
M\textsuperscript{me} de Maintenon à l'égard de Monseigneur et de Mgr
{[}le duc{]} et de M\textsuperscript{me} la duchesse de Bourgogne.}}
{\textsc{- Genre de la douleur du roi.}} {\textsc{- Ses ordres sur les
suites de la mort de Monseigneur.}} {\textsc{- Ses occupations des
premiers jours.}} {\textsc{- Douze mille livres de pension à
M\textsuperscript{lle} Choin, bien traitée du nouveau Dauphin et de la
Dauphine.}} {\textsc{- Gêne de sa vie.}} {\textsc{- Sagesse de sa
conduite après la mort de Monseigneur\,; n'est point abandonnée.}}
{\textsc{- Princesse de Conti veut inutilement se raccommoder avec
M\textsuperscript{lle} Choin.}} {\textsc{- Du Mont justement bien traité
et Casau.}} {\textsc{- Princesse d'Angleterre cède à
M\textsuperscript{me} la Dauphine en lieu tierce.}} {\textsc{- Deuil
drapé de Monseigneur.}} {\textsc{- Situation de M. {[}le duc{]} et de
M\textsuperscript{me} la duchesse de Berry.}} {\textsc{- Les deux
battants des portes, chez les fils et filles de France, ne s'ouvrent que
pour les fils et les filles de France.}} {\textsc{- Colère de
M\textsuperscript{me} la duchesse de Berry.}} {\textsc{- Orage tombé sur
M\textsuperscript{me} la duchesse de Berry.}} {\textsc{- Elle avoue à
M\textsuperscript{me} de Saint-Simon ses étranges projets, avortés par
la mort de Monseigneur, laquelle l'exhorte à n'oublier rien pour se
raccommoder avec M\textsuperscript{me} la Dauphine.}} {\textsc{-
M\textsuperscript{me} la duchesse de Berry se raccommode avec
M\textsuperscript{me} la Dauphine.}} {\textsc{- Service de M. {[}le
duc{]} et M\textsuperscript{me} la duchesse de Berry à Mgr le Dauphin et
à M\textsuperscript{me} la Dauphine.}} {\textsc{- Singulier avis de
M\textsuperscript{me} de Maintenon à M\textsuperscript{me} la
Dauphine.}} {\textsc{- Duc de La Rochefoucauld prétend la garde-robe du
nouveau Dauphin, et la perd contre le duc de Beauvilliers.}} {\textsc{-
Soumission et modération de Mgr le Dauphin\,; veut être nommé et appelé
Monsieur, et non Monseigneur.}} {\textsc{- Marly repeuplé.}} {\textsc{-
Châtillon et Beauvau obtiennent de draper.}} {\textsc{- Deuil singulier
pour Monseigneur.}} {\textsc{- Bâtards obtiennent d'être visités en fils
de France sur la mort de Monseigneur.}} {\textsc{- Manteaux et mantes à
Marly.}} {\textsc{- Indécence et confusion parfaite.}} {\textsc{-
Burlesque ruse de M\textsuperscript{me} la Princesse.}} {\textsc{- Mgr
{[}le Dauphin{]} et M\textsuperscript{me} la Dauphine en mante et en
manteau à Saint-Germain.}} {\textsc{- Ministres étrangers à Versailles,
où les compagnies haranguent Mgr le Dauphin, traité par le parlement de
Monseigneur par ordre du roi.}}

~

Le pourpre, mêlé à la petite vérole dont Monseigneur mourut, et la
prompte infection qui en fut la suite, firent juger également inutile et
dangereuse l'ouverture de son corps. Il fut enseveli, les uns ont dit
par des sœurs grises, les autres par des frotteurs du château, d'autres
par les plombiers mêmes qui apportèrent le cercueil. On jeta dessus un
vieux poêle de la paroisse\,; et, sans aucun accompagnement que des
mêmes qui y étaient restés, c'est-à-dire du seul La Vallière, de
quelques subalternes et des capucins de Meudon qui se relevèrent à prier
Dieu auprès du corps, sans aucune tenture, ni luminaire que quelques
cierges.

Il était mort vers minuit du mardi au mercredi\,; le jeudi il fut porté
à Saint-Denis dans un carrosse du roi, qui n'avait rien de deuil, et
dont on ôta la glace de devant pour laisser passer le bout du cercueil.
Le curé de Meudon et le chapelain en quartier chez Monseigneur y
montèrent. Un autre carrosse du roi suivit aussi sans aucun deuil, au
derrière duquel montèrent le duc de La Trémoille, premier gentilhomme de
la chambre, point en année, et M. de Metz, premier aumônier\,; sur le
devant, Dreux, grand maître des cérémonies, et l'abbé de Brancas,
aumônier de quartier chez Monseigneur, depuis évêque de Lisieux, et
frère du maréchal de Brancas, des gardes du corps, des valets de pied et
vingt-quatre pages du roi portant des flambeaux. Ce très-simple convoi
partit de Meudon sur les six ou sept heures du soir, passa sur le pont
de Sèvres, traversa le bois de Boulogne, et par la plaine de Saint-Ouen
gagna Saint-Denis, où tout de suite le corps fut descendu dans le caveau
royal, sans aucune sorte de cérémonie.

Telle fut la fin d'un prince qui passa près de cinquante ans à faire
faire des plans aux autres, tandis que sur le bord du trône il mena
toujours une vie privée, pour ne pas dire obscure, jusque-là qu'il ne
s'y trouve rien de marqué que la propriété de Meudon, et ce qu'il y a
fait d'embellissement. Chasseur sans plaisir, presque voluptueux mais
sans goût, gros joueur autrefois pour gagner, mais depuis qu'il
bâtissait sifflant dans un coin du salon de Marly, et frappant des
doigts sur sa tabatière, ouvrant de grands yeux sur les uns et les
autres sans presque regarder, sans conversation, sans amusement, je
dirai volontiers sans sentiment et sans pensée, et toutefois, par la
grandeur de son être, le point aboutissant, l'âme, la vie de la cabale
la plus étrange, la plus terrible, la plus profonde, la plus unie,
nonobstant ses subdivisions, qui ait existé depuis la paix des Pyrénées
qui a scellé la dernière fin des troubles nés de la minorité du roi. Je
me suis un peu longuement arrêté sur ce prince presque indéfinissable,
parce qu'on ne le peut faire connaître que par des détails. On serait
infini à les rapporter tous. Cette matière d'ailleurs est assez curieuse
pour permettre de s'étendre sur un Dauphin si peu connu, qui n'a jamais
été rien ni de rien en une si longue et si vaine attente de la couronne,
et sur qui enfin la corde a cassé de tant d'espérances, de craintes et
de projets. Après ce qui a été éparsement expliqué sur Monseigneur, on a
vu par avance quelle sorte de sensation fit sur les personnes royales et
les personnages, sur la cour et sur le public, la perte d'un prince dont
tout le mérite était dans sa naissance, et tout le poids dans son corps.
Je n'ai jamais su qui lui avait captivé les halles et le bas peuple de
Paris, si ce n'est cette gratuite réputation de bonté que j'ai touchée.
Si M\textsuperscript{me} de Maintenon se sentit délivrée par la mort de
Monsieur, elle se la trouva bien plus par celle de Monseigneur, dont
toute la cour intérieure lui fut toujours très-suspecte. Jamais ils
n'eurent l'un pour l'autre que beaucoup d'éloignement réciproque, lui en
presse avec elle, elle en mesure avec lui, et en attention continuelle à
l'observer et à s'instruire de ses plus secrètes pensées, ou pour mieux
dire de celles qui lui étaient inspirées, en quoi M\textsuperscript{me}
d'Espinoy lui servait d'espion, comme il parut dans la suite et comme
j'en ai touché ailleurs un étrange trait d'original, et peut-être
d'espion double à tous les deux. Fort rapprochée de Mgr le duc de
Bourgogne personnellement, depuis la campagne de Lille, et devenue en
effet à l'égard de M\textsuperscript{me} la duchesse de Bourgogne, et
elle au sien, comme une bonne et tendre mère, et la meilleure et la plus
reconnaissante fille et la plus attachée, elle regardait leur
rehaussement comme la sûreté de sa grandeur, et comme le calme et le
rempart de sa vie et de sa fortune, quelque événement qui pût arriver.
Pour le roi, jamais homme si tendre aux larmes, si difficile à
s'affliger, ni si promptement rétabli en sa situation parfaitement
naturelle. Il devait être bien touché de la perte d'un fils qui, à
cinquante ans, n'en avait jamais eu six à son égard. Fatigué d'une si
triste nuit, il demeura fort tard au lit. M\textsuperscript{me} la
duchesse de Bourgogne, arrivée de Versailles, attendait son réveil chez
M\textsuperscript{me} de Maintenon, et toutes deux l'allèrent voir dans
son lit dès qu'il fut éveillé. Il se leva ensuite à son ordinaire. Dès
qu'il fut dans son cabinet, il prit le duc de Beauvilliers et le
chancelier dans une fenêtre, y versa encore quelques larmes, et convint
avec eux que le nom, le rang, et les honneurs de Dauphin devaient dès ce
moment passer à Mgr {[}le duc{]} et à M\textsuperscript{me} la duchesse
de Bourgogne, que désormais je ne nommerai plus autrement. Il décida
ensuite ce qui regardait le corps de Monseigneur, en la manière qui a
été racontée, reçut sa cassette et ses clefs que du Mont lui apporta,
régla ce qui concernait le petit nombre des domestiques personnels du
feu prince, commit le chancelier au partage de la légère succession
entre les trois princes ses petits-fils, et descendit après jusqu'à la
réduction de l'équipage du loup au pied de son premier établissement. Il
remit au dimanche suivant l'admission dans Marly de ce qui avait
accoutumé de l'y suivre, et des autres qu'il chaisirait sur la liste des
demandeurs. Il ne voulut jusque-là que qui que ce soit y entrât, excepté
ceux qui y étaient arrivés avec lui\,; M\textsuperscript{me} la Dauphine
eut seule la permission de l'y venir voir très-peu accompagnée, et sans
y manger ni coucher, pour laisser aérer ce qu'il avait amené, et changer
d'habits à ce même monde. En même temps il envoya le duc de Bouillon,
grand chambellan, à Saint-Germain, donner part au roi, à la reine et à
la princesse d'Angleterre de la perte qu'il venait de faire. Il se
promena dans ses jardins, et M\textsuperscript{me} la Dauphine revint
passer une partie du soir avec lui chez M\textsuperscript{me} de
Maintenon. Cette princesse s'y trouva tous les soirs les jours suivants,
et même à sa promenade. Le jeudi il s'amusa aux listes pour Marly. Il
attacha au Dauphin les mêmes menins qu'avait Monseigneur, et permit à
d'Antin d'en donner à son fils la place qu'il avait.

Il le chargea d'aller assurer de sa part M\textsuperscript{lle} Choin de
sa protection, et de lui porter une pension de douze mille livres. Elle
n'avait ni demandé ni fait nommer son nom. Mgr et M\textsuperscript{me}
la Dauphine lui envoyèrent faire toutes sortes d'amitiés, et toutes deux
lui firent l'honneur de lui écrire. Sa douleur fut de beaucoup moins
longue et moins vive qu'on aurait cru. Cela surprit fort, et persuada
qu'elle entrait en bien moins de choses qu'on ne pensait. Sa vie était
infiniment gênée. Il lui fallait compter de presque tous les gens
qu'elle voyait\,; jamais elle n'eut d'équigage, cinq ou six domestiques
composaient tout son train\,; elle ne paraissait en aucun lieu public,
et si elle allait quelque part, c'était en cinq ou six maisons au plus
de gens de sa liaison, où elle était sûre de n'en point trouver
d'autres\,; toujours le pied à l'étrier, non-seulement pour tous les
voyages de Meudon, mais pour tous les dîners sans coucher que
Monseigneur y allait faire. Elle allait toujours la veille seule avec
une femme de chambre dans un carrosse de louage, le premier venu, tout
au soir, pour arriver de nuit la veille que Monseigneur venait, et s'en
retournait de même à la nuit, après qu'il était parti. Dans Meudon, elle
logeait d'abord dans les entre-sols de Monseigneur, après dans le grand
appartement d'en haut, qu'occupait M\textsuperscript{me} la duchesse de
Bourgogne quand le roi faisait des voyages à Meudon. Mais où qu'elle
logeât, elle ne sortait jamais de son appartement que le matin de bonne
heure pour entendre la messe à la chapelle, et quelquefois sur le minuit
l'été, pour prendre l'air. Dans les premiers temps, elle n'y voyait que
trois ou quatre personnes du secret. Cela s'étendit peu à peu assez
loin\,; mais, quoique cela fût devenu le secret de la comédie, la même
enfermerie, la même cacherie, la même séparation furent toujours de
même. À cette gêne extérieure était jointe celle de l'esprit, et de la
conduite par rapport à la famille royale à cette cour intérieure de
Monseigneur, dont il a été tant parlé, et à Monseigneur lui-même, qui
n'était ni sans épines ni sans ennui. J'en ai ouï parler à de ses amis
comme d'une personne d'esprit, sans ambition ni intérêt quelconque, ni
désir d'être ni de se mêler, fort décente, mais gaie, naturellement
libre, et qui aimait la table et à causer. Une telle contrainte, et de
toute la vie, est bien pesante à qui est de ce caractère, et qui ne s'en
propose rien\,; et la rupture de la chaîne apporte assez tôt
consolation.

Elle était amie intime, de tout temps, de La Croix, riche receveur
général de Paris et fort honnête homme, et modeste pour un publicain qui
a de tels accès. Elle logeait, comme avec lui, dans une portion de
maison attenant le Petit-Saint-Antoine. Elle continua d'y demeurer le
reste de sa vie, avec le même domestique qu'elle avait, sans se répandre
davantage dans le monde. Il ne tint pas à M\textsuperscript{me} la
Dauphine que sa pension ne fût de vingt mille livres.
M\textsuperscript{me} la Duchesse, M\textsuperscript{lle} de Lislebonne,
M\textsuperscript{me} d'Espinoy, les intrinsèques de l'entre-sol de
Meudon, les Noailles et quelques autres amis se sont constamment piqués
de la voir souvent depuis la mort de Monseigneur jusqu'à la sienne, qui
n'arriva que dix ou douze ans après, et qu'elle mena toujours
extrêmement unie et fort réservée sur tout le passé. Malgré tout ce
qu'elle avait fait essuyer à M\textsuperscript{me} la princesse de
Conti, qu'on a vu en son lieu, cette princesse avait fait tout ce
qu'elle avait pu quelques années après pour se raccommoder avec elle et
pour la voir, sans que jamais la Choin y eût voulu entendre, tant
l'extrême faveur, et les idées qu'en tous états on s'en forme, enfantent
d'étranges effets.

Le gouvernement de Meudon fut en même temps confirmé à du Mont avec une
pension qui, avec celles qu'il avait déjà et ses appointements, allait à
plus de trente mille livres de rente, tristes débris de tant et de si
plausibles espérances. Casau eut pour rien la charge de premier maréchal
des logis de M. le duc de Berry, qui par bonheur pour lui n'était pas
encore vendue. Du Mont, en honnête homme qu'il était, souffrait
impatiemment les glaces de Monseigneur pour Mgr le duc de Bourgogne, et
s'était hasardé plus d'une fois de les rapprocher\,; ce prince ne
l'avait pas oublié. Il ne dédaigna pas de l'en remercier avec les
paroles les plus obligeantes, à quoi le duc de Beauvilliers le porta
fort, et y ajouta le présent d'une bague de deux mille pistoles que
Monseigneur portait ordinairement. Il en donna une autre fort belle à La
Croix, en attendant qu'il fût payé d'avances considérables qu'il avait
faites à Monseigneur, dont le Dauphin voulut être le solliciteur.

Ce même jeudi, jour de l'enterrement de Monseigneur, le roi reçut sans
cérémonie la visite de la reine d'Angleterre. Elle vint de Versailles,
où elle avait été de même voir les enfants de Monseigneur, avec la
princesse d'Angleterre, qu'elle fit mettre au salut, qu'elle entendit
avec eux, au-dessous de la Dauphine, parce qu'elle n'était héritière que
possible et non présomptive comme le Dauphin. Elle demeura dans le
carrosse de la reine à Marly, à cause du mauvais air, qui fit rester le
roi d'Angleterre à Saint-Germain.

Le vendredi le roi fut tirer dans son parc. Le samedi il tint le conseil
de finance, et fit sur les hauteurs de Marly la revue des gens d'armes
et des chevau-légers. Il travailla le soir avec Voysin chez
M\textsuperscript{me} de Maintenon. Le même jour il fit une décision
singulière. Il régla que, encore qu'il ne prît point le deuil, il serait
d'un an\,; et que les princes du sang, les ducs, les princes étrangers,
les officiers de la couronne, et les grands officiers de sa maison
draperaient comme ils font lorsqu'il drape lui-même, et qui, parce qu'il
ne prit point le deuil de M\textsuperscript{me} la Dauphine de Bavière,
ne drapèrent point. J'ai conduit le roi dans sa solitude jusqu'au
dimanche que Marly se repeupla à l'ordinaire. Il ne sera pas moins
curieux de voir Versailles pendant ces mêmes jours.

On peut juger qu'on n'y dormit guère cette première nuit. M. {[}le
Dauphin{]} et M\textsuperscript{me} la Dauphine ouïrent la messe
ensemble de fort bonne heure. J'y arrivai sur la fin, et les suivis chez
eux. Leur cour était fort courte, parce qu'on ne s'était pas attendu à
cette diligence. La princesse voulait être à Marly au réveil du roi.
Leurs yeux étaient secs à merveilles, mais très-compassés, et leur
maintien les montrait moins occupés de la mort de Monseigneur que de
leur nouvelle situation. Un sourire, qui leur échappa en se parlant bas
et de fort près, acheva de me le déclarer. En gardant scrupuleusement,
comme ils firent, toutes sortes de bienséances, il n'était pas possible
de le trouver mauvais, ni que cela fût autrement, à tout ce qu'on a vu.
Leur premier soin fut de resserrer de plus en plus l'union avec M. le
duc de Berry, de le ramener sur l'ancienne confiance et intimité avec
M\textsuperscript{me} la Dauphine, et d'essayer, par tout ce qui se peut
d'engageant, de faire oublier à M\textsuperscript{me} la duchesse de
Berry ses fautes à leur égard, et lui adoucir l'inégalité nouvelle que
la mort de Monseigneur mettait entre ses enfants. Dans cet aimable
esprit rien ne coûta à M. {[}le Dauphin{]} et à M\textsuperscript{me} la
Dauphine, et dès ce même jour ils allèrent voir M. le duc et
M\textsuperscript{me} la duchesse de Berry dans leur lit, dès qu'ils les
surent éveillés, ce qui fut de très-bonne heure, et l'après-dînée
M\textsuperscript{me} la Dauphine y retourna encore. M. le duc de Berry,
qui n'avait pu être ébranlé sur l'attachement à Mgr son frère, fut au
milieu de sa douleur extrêmement sensible à ces prévenances d'amitié si
promptement marquées et si éloignées de la différence qui allait être
entre eux, et il fut surtout comblé des procédés de
M\textsuperscript{me} la Dauphine, qu'il sentait avec bon sens, et
meilleur cœur encore, qu'il avait depuis un temps cessé de les mériter
aussi parfaits.

M\textsuperscript{me} la duchesse de Berry paya d'esprit, de larmes et
de langage. Son cœur de princesse même, si elle en avait un, navré de
tout ce qui ne sera point répété ici, et qu'on a développé plus haut,
frémissait au fond de lui-même de recevoir des avances de pure
générosité. Un courage déplacé qui allait à la violence et que la
religion ne retenait pas, ne lui laissait de sentiments que pour la
rage. Bercée, pour la contenir, qu'il se fallait contraindre surtout
pour arriver à un aussi grand mariage, après lequel elle serait
affranchie et maîtresse de faire tout ce qui lui plairait, elle avait
pris ces documents au pied de la lettre. Entièrement maîtresse de M. le
duc d'Orléans et d'un mari dans la première ivresse de sa passion, elle
n'eut pas peine à secouer une mère trop sage pour s'exposer à ce qui ne
lui était que trop connu. Madame était nulle de tout temps à la cour et
dans sa famille\,: excepté les devoirs extérieurs, point de belle-mère,
et un beau-père, tant qu'il vécut, nul ou favorable. Une dame d'honneur
très-affligée de l'être, qui, pour avoir été forcée d'en accepter
l'emploi, n'en faisait que ce qu'elle en voulait bien faire, au
cérémonial près, et qui avait déclaré bien formellement qu'elle n'en
serait pas la gouvernante. L'emploi en roula donc en entier sur
M\textsuperscript{me} la duchesse de Bourgogne, par son amitié pour
M\textsuperscript{me} la duchesse d'Orléans, et son intimité avec
M\textsuperscript{me} de Maintenon, ravie à son âge de se trouver le
chaperon d'une autre\,; elle compta d'autant mieux d'en faire sa poupée,
qu'elle l'avait mise dans la grandeur où elle était.

Elle s'y mécompta bientôt. Mille détails là-dessus, quoique curieux dans
leur temps, perdent leur mérite dans d'autres qui s'éloignent, et
gâteraient le sérieux de ce qui s'expose ici. Il suffit de dire que
l'une, quoique douce et bonne, fut peut-être trop enfant pour tenir une
lisière, et que l'autre, rien moins que tout cela, ne put souffrir d'en
avoir une, quelque lâche et légère qu'elle fût. Le dépit de ne se
trouver que de la cour d'une autre, l'impatience des déférences, la
contrainte des heures, le poids des obligations, des difficultés,
surtout de la reconnaissance, s'accordaient mal avec l'impression de la
pleine liberté de son éducation, de ses goûts irréguliers, de ses
humeurs dans un naturel tel qu'il a été crayonné et gâté encore par de
pernicieuses lectures. L'idée de n'avoir rien à perdre et celle de
figurer aux dépens de Mgr {[}le duc{]} et de M\textsuperscript{me} la
duchesse de Bourgogne, en se livrant aux personnages de Meudon,
achevèrent de tout perdre et brouillèrent les deux belles-sœurs, jusqu'à
ne pouvoir plus se souffrir, à force d'échappées de l'humeur et des
traits les plus méchants de M\textsuperscript{me} la duchesse de
Berry\,; ainsi toutes deux regardèrent comme une délivrance de n'avoir
plus à dîner ensemble, par la formation qui se fit des deux maisons, et
les domestiques du roi {[}comme{]} un grand soulagement de n'avoir plus
à servir la nouvelle mariée.

Un trait entre mille en donnera un échantillon. Un nouvel huissier de la
chambre du roi servait chez elle un matin que M\textsuperscript{me} la
duchesse d'Orléans arriva à la fin de sa toilette pour quelque
ajustement. L'huissier, étourdi et neuf, ouvrit les deux battants de la
porte. M\textsuperscript{me} la duchesse de Berry devint cramoisie et
tremblante de colère\,: elle reçut M\textsuperscript{me} sa mère fort
médiocrement. Quand elle fut sortie, elle appela M\textsuperscript{me}
de Saint-Simon, lui demanda si elle avait remarqué l'impertinence de
l'huissier, et lui dit qu'elle voulait qu'elle l'interdît sur-le-champ.
M\textsuperscript{me} de Saint-Simon convint de la faute, assura qu'elle
y donnerait ordre de façon qu'on ne s'y méprendrait plus et que les deux
battants ne seraient ouverts que pour les fils et les filles de France,
comme c'était la règle, et comme nuls autres ne prétendaient à cet
honneur qu'ils n'avaient pas en effet, mais que d'interdire un huissier
du roi qui n'était point à elle et qui ne la servait que par prêt, et
encore pour avoir fait un trop grand honneur à M\textsuperscript{me} sa
mère et pour l'unique fois que cela était arrivé, elle trouverait bon de
se contenter de la réprimande qu'elle allait lui en faire.
M\textsuperscript{me} la duchesse de Berry insista, pleura, ragea\,;
M\textsuperscript{me} de Saint-Simon la laissa dire, gronda doucement
l'huissier, et lui apprit son cérémonial.

Les maisons faites, la cour, qui trouvait en M\textsuperscript{me} la
duchesse de Bourgogne les jeux, les ris, les distinctions, les
espérances, ne se partagea point, et laissa fort solitaire
M\textsuperscript{me} la duchesse de Berry, où rien de tout cela ne
s'offrait, qui s'en prit à M\textsuperscript{me} la duchesse de
Bourgogne, et fit si bien qu'elle mit M. le duc de Berry de son côté, et
le brouilla avec elle. De l'aveu de M\textsuperscript{me} la duchesse de
Bourgogne, rien de si sensible ne lui est jamais arrivé que cet
éloignement et cette aigreur sans cause ni raison d'un prince avec qui
elle avait toujours vécu dans l'intelligence la plus intime et la plus
entière. Quelques contre-temps forts et trop publics, arrivés à
M\textsuperscript{me} la duchesse de Berry, dont M\textsuperscript{me}
la duchesse de Bourgogne avait doucement abandonné toute conduite dès
avant ce dernier trait, allèrent jusqu'au roi et à M\textsuperscript{me}
de Maintenon, qui leur ouvrirent les yeux. Celle-ci, outrée de s'être si
lourdement trompée, ne put se taire, et M\textsuperscript{me} la
duchesse de Bourgogne, poussée à bout d'être brouillée avec M. le duc de
Berry par la seule malignité de M\textsuperscript{me} la duchesse de
Berry, après tout ce qu'elle avait d'ailleurs essuyé d'elle, rompit
enfin le silence qu'elle avait gardé jusqu'alors. Les choses tendaient à
un éclat\,; mais le roi, qui voulait vivre doucement dans sa famille et
s'y faire aimer, espéra que la frayeur corrigerait M\textsuperscript{me}
la duchesse de Berry, et voulut se contenter qu'elle sût qu'il
n'ignorait rien, et que, pour cette fois, il voulait bien n'en rien
témoigner. Ce ménagement persuada M\textsuperscript{me} la duchesse de
Berry, ou qu'on n'osait lui imposer, ou qu'on ne savait comment s'y
prendre. Au lieu de s'arrêter, elle continua avec plus de licence, et se
mit au point que les matières combustibles qu'elle s'était préparées
s'embrasèrent tout à coup et firent un grand éclat à Marly.

J'étais allé faire seul un tour à la Ferté. M\textsuperscript{me} de
Saint-Simon, avertie de l'orage prêt à crever, craignit d'y être
enveloppée pour s'être tenue dans le silence. Monseigneur était alors
plein de vie et de santé. Elle s'adressa à M\textsuperscript{me} la
duchesse de Bourgogne, et, par son avis, elle eut un entretien avec
M\textsuperscript{me} de Maintenon, où elle apprit avec surprise qu'elle
ignorait peu de choses, et d'avec qui elle sortit fort contente. Elle
crut ensuite devoir dire un mot à M\textsuperscript{me} la duchesse de
Berry. La princesse, d'autant plus outrée qu'elle ne voyait pas moyen
d'échapper, s'en prit à ce qu'elle put, et dans la pensée que
M\textsuperscript{me} de Saint-Simon y avait part, elle voulut lui
répondre sèchement. Je dis exprès qu'elle voulut, parce que
M\textsuperscript{me} de Saint-Simon ne lui en laissa pas le temps. Elle
l'interrompit, l'assura d'abord qu'elle n'avait part ni était entrée en
rien, qu'elle n'avait même rien appris que du monde, mais qu'en peine
d'elle-même pour s'être toujours tenue dans le silence, elle avait parlé
à M\textsuperscript{me} la duchesse de Bourgogne et à
M\textsuperscript{me} de Maintenon, puis ajouta qu'elle ignorait
peut-être la manière dont elle avait été mise auprès d'elle, combien
cela convenait peu à notre naissance, à notre dignité, à nos biens, à
notre union\,; qu'il était bon qu'elle l'apprit une fois pour toutes\,;
que, pour peu qu'elle le désirât, elle se retirerait d'auprès d'elle
avec autant de satisfaction qu'elle y était entrée avec répugnance après
un grand nombre de refus, dont elle lui cita M\textsuperscript{me} la
duchesse de Bourgogne et M. {[}le duc{]} et M\textsuperscript{me} la
duchesse d'Orléans pour témoins. Elle lui dit encore, comme il était
vrai, que, sa conduite n'étant pas telle qu'elle l'avait espérée, elle
avait pris l'occasion d'un éclat fait sans sa participation pour tenter
de se retirer\,; que M\textsuperscript{me} la duchesse de Bourgogne et
M\textsuperscript{me} de Maintenon l'avaient conjurée de n'y pas
penser\,; et que, cela s'étant passé depuis vingt-quatre heures, le
souvenir leur en était assez présent pour qu'elle pût leur en demander
la vérité. M. le duc d'Orléans, qui survint, apaisa la chose le mieux
qu'il put.

M\textsuperscript{me} la duchesse de Berry n'avait point interrompu
M\textsuperscript{me} de Saint-Simon, mais elle crevait de dépit de se
voir sur le point d'une sévère réprimande, et son orgueil souffrait
impatiemment ce qu'elle entendait. Elle répondit néanmoins, avec une
honnêteté forcée, qu'elle voulait demeurer persuadée que
M\textsuperscript{me} de Saint-Simon n'était entrée en rien puisqu'elle
le disait. M\textsuperscript{me} de Saint-Simon la laissa là-dessus avec
M. le duc d'Orléans, outrée de mon absence, dans l'ardeur de quitter
malgré eux tous, quelque dignement et flatteusement qu'elle en fût
traitée. Elle parla aussi à Madame, avec qui en tout temps elle avait
toujours été très-bien, et à M\textsuperscript{me} la duchesse d'Orléans
qu'elle voyait sans cesse, après quoi elle attendit ce que deviendrait
l'orage.

Il fondit le lendemain. Le roi, avant dîner, manda M\textsuperscript{me}
la duchesse de\,:Berry dans son cabinet. La romancine fut longue, et de
l'espèce de celles qu'on ne veut pas avoir la peine de recommencer.
L'après-dînée il fallut aller chez M\textsuperscript{me} de Maintenon,
qui, sans parler si haut, ne parla pas moins ferme. Il est aisé de
concevoir quelle impression cela acheva de faire en
M\textsuperscript{me} la duchesse de Berry à l'égard de
M\textsuperscript{me} la duchesse de Bourgogne, sur qui tout le
ressentiment en tomba. Elle ne tarda guère à voir que
M\textsuperscript{me} de Saint-Simon n'y avait eu aucune part, et à lui
en parler en personne qui le veut et le sait témoigner en réparation du
soupçon.

Cet éclat fit une nouvelle publique, qui mit de plus en plus au
désespoir la princesse qui l'éprouvait. La solitude augmenta chez elle,
les dégoûts lui furent peu ménagés. Elle faisait quelquefois des efforts
pour regagner quelque terrain\,; mais la répugnance qui les accompagnait
leur donnait si mauvaise grâce, et ils étaient d'ailleurs si froidement
reçus, qu'ils en devenaient de tous les côtés de nouveaux sujets
d'éloignement.

Telle était la situation de M\textsuperscript{me} la duchesse de Berry
lorsque Monseigneur mourut, et telles les causes du désespoir extrême où
cette perte la plongea. Dans l'excès de sa douleur elle eut la légèreté,
pour en parler sobrement, d'avouer à M\textsuperscript{me} de
Saint-Simon les desseins qu'elle avait imaginés et sur lesquels elle
cheminait, et que j'ai ci-devant expliqués, avec la terrible cabale qui
gouvernait Monseigneur. Dans l'étonnement d'entendre de si étranges
projets, M\textsuperscript{me} de Saint-Simon tâcha de lui en faire
comprendre le peu de fondement, pour ne pas dire l'absurdité, l'horreur
et la folie, et de la porter à saisir une conjoncture touchante pour se
rapprocher d'une belle-sœur, bonne, douce, commode à vivre, qui l'avait
mariée, et qui, nonobstant tout ce qui s'était passé depuis, était faite
de manière, par sa facilité, à revenir si on savait s'y prendre\,; mais
c'était la nécessité même de le faire, et de le bien faire, qui
aigrissait le courage de celle qui se sentait également chargée de torts
à son égard, et de besoins pour le solide et l'agrément de la vie. Cette
force de nécessité révoltait ce courage altier et l'extrême répugnance à
ployer même en apparence. Accoutumée à un rang égal, ce nom et ce rang
de Dauphine, qui allait mettre tant de différence entre elles, comblait
son désespoir et son éloignement, pour user d'un terme trop doux.
Incapable de regarder derrière elle, et d'où elle était partie pour
monter où elle se voyait\,; aussi peu de se faire une raison que ce qui
venait d'arriver devait arriver tôt ou tard, beaucoup moins encore que
cette supériorité qui la désolait n'était qu'un degré pour monter sur le
trône et la voir reine, de qui même elle n'aurait pas l'honneur d'être
la première sujette, elle ne pouvait supporter l'état nouveau où elle se
trouvait. Après bien des plaintes, des larmes et des élans, pressée par
les raisons sans nombre et sans réplique, plus encore par ses besoins
qu'elle sentait malgré elle dans toute leur étendue, elle promit à
M\textsuperscript{me} de Saint-Simon d'aller le lendemain jeudi chez la
nouvelle Dauphine, de lui demander une audience dans son cabinet, et d'y
faire tout son possible pour se raccommoder avec elle.

Ce jeudi était le jour que Monseigneur fut porté à Saint-Denis, et avec
lui tous les beaux projets de M\textsuperscript{me} la duchesse de
Berry. Elle tint parole et l'exécuta en effet très-bien. Son aimable
belle-sœur lui en aplanit tout le chemin, et entra en propos la
première. Par ce que toutes deux ont redit séparément de ce tête-à-tête,
M\textsuperscript{me} la Dauphine agit et parla comme si elle-même eût
offensé M\textsuperscript{me} la duchesse de Berry, comme si elle lui
eût tout dû, comme si elle eût tout attendu d'elle\,; et
M\textsuperscript{me} la duchesse de Berry aussi se surpassa.
L'entretien dura plus d'une heure. Elles sortirent du cabinet avec un
air naturel de satisfaction réciproque qui réjouit autant les honnêtes
gens qu'il déplut à ceux qui n'espèrent qu'en la division et au
désordre. M. {[}le duc{]} et M\textsuperscript{me} la duchesse d'Orléans
eurent une joie extrême de cette réconciliation, et M. le duc de Berry
en fut si content que sa douleur en fut fort adoucie. Il aimait
tendrement Mgr le Dauphin, il aimait encore beaucoup
M\textsuperscript{me} la Dauphine\,; ce lui était une contrainte
mortelle de se conduire avec elle comme M\textsuperscript{me} la
duchesse de Berry l'exigeait. Il embrassa cette occasion de tout son
cœur et en vrai bon homme\,; et M\textsuperscript{me} la Dauphine les
étant venue voir l'après-dînée du même jour que cette réconciliation
s'était faite le matin, elle prit M. le duc de Berry en particulier et
ils pleurèrent ensemble de tendresse. Ce qui s'était passé le matin y
fut confirmé de sa part avec toutes les grâces qui lui étaient si
naturelles\,; mais de celle de M\textsuperscript{me} la duchesse de
Berry il se trouva bientôt une pierre d'achoppement\,: ce fut de
présenter le service à Mgr et à M\textsuperscript{me} la Dauphine.

On s'attendait chez eux que ce devoir ne serait pas différé. La bonne
grâce y était même, à la suite d'une réconciliation si prompte, et des
visites si peu ménagées et si redoublées de l'aîné au cadet. Néanmoins,
lorsque M\textsuperscript{me} de Saint-Simon leur voulut insinuer, ce
même jeudi, après que M\textsuperscript{me} la Dauphine fut sortie de
chez eux, d'aller le lendemain donner la chemise, l'un à Mgr le Dauphin,
l'autre à M\textsuperscript{me} la Dauphine, M\textsuperscript{me} la
duchesse de Berry s'éleva avec fureur, et prétendit qu'entre frères ce
service n'était point dû, que l'exemple de Monsieur, oncle de feu
Monseigneur, n'en était pas un pour eux, et s'emporta fort contre ce
devoir, qu'elle appelait un valetage. M. le duc de Berry, qui savait que
cela se devait, et que son cœur portait en tout vers Mgr et
M\textsuperscript{me} la Dauphine, fit tout ce qu'il put pour la ramener
par raisons et par caresses. Elle se fâcha contre lui, le maltraita, lui
dit qu'elle aurait le dernier mépris pour lui s'il se soumettait à une
chose si servile, et de là aux pleurs, aux sanglots, aux hauts cris, de
façon que M. le duc de Berry, qui avait compté d'aller le lendemain au
lever de Mgr le Dauphin, ne l'osa de peur de se brouiller avec elle.

Le bruit avec lequel cette dispute s'était passée éveilla la curiosité,
qui eut bientôt éventé le fait, parce que M\textsuperscript{me} la
duchesse de Berry en était si pleine qu'elle se répandit. Tout aussitôt
voilà les dames de M\textsuperscript{me} la Dauphine en l'air comme sur
chose qui allait presque à leur déshonneur, et cette affaire devint
publique.

M. le duc d'Orléans accourut au secours de M. le duc de Berry, qui
n'osait presque rien dire dans cette impétuosité. Tous deux ne mettaient
pas le devoir et la règle en doute\,; tous deux, si aises du
raccommodement, sentaient le danger d'une rechute, l'affront certain
auquel la princesse s'exposait d'en recevoir du roi l'ordre et la
réprimande, et l'effet intérieur et au dehors que produirait un
entêtement si mal fondé, et dans des circonstances pareilles. Tout le
lendemain vendredi fut employé à la persuader. Enfin, la peur de
l'ordre, de la romancine et de l'affront, arracha d'elle la permission à
M. le duc de Berry de dire qu'ils donneraient la chemise et le service,
mais à condition de délai pour se résoudre à l'exécution.

Elle le voulait aussi pour M. le duc de Berry, mais ce prince fut si
aise d'être affranchi là-dessus qu'il voulut servir M. le Dauphin le
samedi matin. M. le Dauphin et M\textsuperscript{me} la Dauphine
n'avaient pas ouvert la bouche là-dessus. Mais ce prince, pour faire une
honnêteté à M. son frère, refusa d'en être servi jusqu'à ce qu'ils
eussent vu le roi. Ils le virent le dimanche suivant, et le lendemain
lundi M. le duc de Berry alla exprès au coucher de Mgr le Dauphin et lui
donna sa chemise, qui, dans le moment qu'il l'eut reçue, embrassa
tendrement M. son frère.

Il fallut encore quelques jours à M\textsuperscript{me} la duchesse de
Berry pour se résoudre. À la fin il fallut bien finir. Elle fut à la
toilette de M\textsuperscript{me} la Dauphine, à qui elle donna la
chemise, et à la fin de la toilette lui présenta la sale\footnote{La
  sale était une soucoupe de Vermeil, sur laquelle on présentait à la
  reine et aux princesses les boîtes, étuis, montres, éventail, etc.,
  couverts d'un taffetas brodé, qu'on levait en leur offrant ces objets.}.
M\textsuperscript{me} la Dauphine, qui n'avait jamais fait semblant de
se douter de rien de ce qui s'était passé là-dessus, ni de prendre garde
à un délai si déplacé, reçut ces services avec toutes les grâces
imaginables, et toutes les marques d'amitié les plus naturelles. Le
désir extrême de la douceur de l'union fit passer M\textsuperscript{me}
la Dauphine généreusement sur cette nouvelle frasque, comme si, au lieu
de M\textsuperscript{me} la duchesse de Berry, c'eût été elle qui eût eu
tout à y gagner ou à y perdre.

J'ai remarqué que M\textsuperscript{me} la Dauphine allait voir le roi
tous les jours à Marly. Elle y reçut un avis de M\textsuperscript{me} de
Maintenon qui mérita sans doute quelque surprise, d'autant plus que ce
fut dès sa seconde visite, c'est-à-dire dès le lendemain de la mort de
Monseigneur qu'elle fut voir le roi à son réveil, et le soir encore chez
M\textsuperscript{me} de Maintenon\,: ce fut de se parer avec quelque
soin, parce que la négligence de son ajustement déplaisait au roi. La
princesse ne croyait pas devoir songer à des ajustements alors\,; et
quand elle en aurait eu la pensée, elle aurait cru avec grande raison
commettre une grande faute contre la bienséance, et qui lui aurait été
d'autant moins pardonnée qu'elle gagnait trop en toutes façons à ce qui
venait d'arriver pour n'être pas en garde là-dessus contre elle-même. Le
lendemain donc elle prit plus de soin d'elle\,; mais cela n'ayant pas
encore suffi, elle porta le jour suivant de quoi s'ajuster en cachette
chez M\textsuperscript{me} de Maintenon, où elle le quitta de même avant
d'en revenir à Versailles, pour, sans choquer le goût du roi, ne pas
blesser le goût du monde, qui aurait été difficilement persuadé qu'il
n'entrait que de la complaisance dans une recherche de soi-même si à
contre-temps. La comtesse de Mailly, qui trouva cette invention de
porter la parure pour la prendre et la quitter chez
M\textsuperscript{me} de Maintenon, et M\textsuperscript{me} de Nogaret,
qui toutes deux aimaient Monseigneur, me le contèrent et en étaient
piquées. On peut juger de là, et par les occupations et les amusements
ordinaires qui reprirent tout aussitôt, comme on l'a vu, leurs places
dans les journées du roi, sans qu'il parût en lui aucune contrainte, que
si sa douleur avait été amère, elle avait aussi le sort de celles dont
la violence fait augurer qu'elles ne seront pas de durée.

Il y eut une assez ridicule dispute élevée tout aussitôt sur la
garde-robe du nouveau Dauphin, dont M. de la Rochefoucauld prétendit
disposer, comme il faisait de celle du roi, par sa charge de
grand-maître de la garde-robe. Il aimait encore, tout vieux et aveugle
qu'il était, à tenir et à conserver, et il alléguait qu'il ne demandait,
à l'égard du nouveau Dauphin, que ce qu'il avait eu, et sans difficulté
exercé, pendant la vie de Monseigneur. Il avait oublié sans doute qu'il
ne se mêla de la garde-robe de ce prince qu'après la mort de M. de
Montausier qui s'en faisait soulager par la duchesse d'Uzès sa fille, et
de la colère où, sur les fins de la vie du duc de Montausier, le roi se
mit contre elle, fort au delà de ce que la chose valait, pour un habit
de Monseigneur, dans le temps que le roi avait entrepris de bannir les
draps étrangers, et de donner vogue à une manufacture de France dont les
draps étaient rayés partout. Je me souviens d'en avoir porté comme tout
le monde, et que cela était fort vilain. Les raies de l'habit de
Monseigneur ne parurent pas tout à fait comme les autres, et le roi
avait le coup d'oeil fort juste\,; vérification faite, il se trouva que
le drap était étranger et contrefait, et que M\textsuperscript{me}
d'Uzès y avait été attrapée. Le duc de Beauvilliers allégua sa charge,
et ses provisions de premier gentilhomme de la chambre, et de maître de
la garde-robe du prince dont il avait été gouverneur, et l'exemple
dernier du duc de Montausier. Il n'en fallut pas davantage, et le duc de
La Rochefoucauld fut tondu.

Le roi, dès les premiers jours de sa solitude, se laissa entendre au duc
de Beauvilliers, qui allait tous les jours à Marly, qu'il ne verrait pas
volontiers le nouveau Dauphin faire des voyages à Meudon. C'en fut assez
pour que ce prince déclarât qu'il n'y mettrait pas le pied, et qu'il ne
sortirait point des lieux où le roi se trouverait\,; et, en effet, il
n'y fit jamais depuis une seule promenade. Le roi lui voulut donner
cinquante mille livres par mois comme Monseigneur les avait\,; M. le
Dauphin en remercia. Il n'avait que six mille livres par mois, il se
contenta de les doubler et n'en voulut pas davantage. C'était le
chancelier qui étant contrôleur général avait fait pousser le traitement
de Monseigneur jusqu'à cette somme. Ce désintéressement plut fort au
public. M. le Dauphin ne voulut quoi que ce soit de particulier pour
lui, et persista à demeurer à cet égard comme il était pendant la vie de
Monseigneur. Ces augures d'un règne sage et mesuré firent concevoir de
grandes espérances.

J'ai expliqué ailleurs la très-moderne et fine introduction de l'art des
princes du sang, et de leurs valets principaux, de les appeler
\emph{Monseigneur}, qui, comme tous leurs autres honneurs, rangs, et
distinctions, devinrent bientôt communs avec les bâtards. Rien n'avait
tant choqué Mgr le duc de Bourgogne, qui jusque-là n'avait jamais été
appelé que \emph{Monsieur}, et qui ne le fut \emph{Monseigneur} que par
la manie de les y appeler tous\ldots{} Aussi, dès qu'il fut Dauphin, il
en fit parler au roi par M\textsuperscript{me} la Dauphine\,; puis,
avant d'aller à Marly, déclara qu'il ne voulait point être ni nommé
Monseigneur, comme Monseigneur son père, mais M. le Dauphin, ni, quand
on lui parlerait, autrement que Monsieur. Il y fut même attentif et
reprenait ceux qui dans les commencements n'y étaient pas accoutumés.
Cela embarrassa un peu les princes du sang\,; mais, à l'abri de M. le
duc de Berry et de M. le duc d'Orléans, ils retinrent le
\emph{Monseigneur} que Mgr le Dauphin ne leur aurait pas laissé s'il fût
devenu le maître.

Le dimanche 18 avril finit la clôture du roi à Marly. La famille royale
et les personnes élues parmi les demandeurs, repeuplèrent ce lieu qui
avait été quatre jours entiers si solitaire. Les deux fils de France et
leurs épouses y arrivèrent ensemble après le salut ouï à Versailles\,;
ils entrèrent tous quatre chez M\textsuperscript{me} de Maintenon où le
roi était, qui les embrassa. L'entrevue ne dura qu'un moment\,; les
princes allèrent prendre l'air dans les jardins\,; le roi soupa avec les
dames, et la vie ordinaire recommença à l'exception du jeu. La cour prit
le deuil ce même jour, qui fut régie pour un an comme de père.

Les différences de rang à porter les deuils sur sa personne s'étaient
peu à peu réduites à rien depuis dix ou douze ans. Je les avais vues
auparavant observées\,; tout s'était réduit à celle de draper, qui
jusqu'à ce deuil s'était maintenue dans les règles. Plusieurs petits
officiers de la maison du roi, comme capitaines des chasses et autres,
l'usurpèrent en celui-ci\,; et, comme on aimait la confusion pour
anéantir les distinctions, on les laissa faire. Le comte de Châtillon en
profita pour s'en forger une toute nouvelle à laquelle ses pères étaient
bien loin de penser. Voysin, son beau-père, étala au roi la grandeur de
la maison de Châtillon, le duché de Bretagne qu'elle avait prétendu et
possédé quelques années, ses douze ou treize alliances directes avec la
maison royale, même avec des fils et des filles de France\,; le nombre
des plus grands offices de la couronne qu'elle avait eus, et les
prodigieux fiefs qu'elle avait possédés\,: il se garda bien d'ajouter
que de toute cette splendeur il n'en rejaillissait rien ou comme rien
sur son gendre, dont la mère et la grand'mère paternelle étaient de la
lie du peuple\,; que toutes les branches illustres de Châtillon étaient
éteintes depuis longtemps, que celle de son gendre n'avait participé à
aucune des grandeurs des autres, et que, s'il sortait de deux filles de
la branche de Dreux, dont même la seconde était fille du chef de la
branche de Beu, et par l'injustice des temps n'était pas sur le pied des
autres du sang royal, c'était avant la séparation de sa branche\,; qu'il
en était de même des deux charges de souverain maître d'hôtel et de
grand maître des eaux et forêts\,; il se garda encore mieux de faire
mention du sieur de Boisrogues, père du père de son gendre, qui était
gentilhomme servant de M. Gaston avec du Rivau qui fut depuis dans ses
Suisses, et que le crédit de M\textsuperscript{lle} de Saujon sur Gaston
en fit enfin capitaine, par le mariage de sa nièce, mais qui laissa
Boisrogues gentilhomme servant. Voysin sans doute ne parla pas de la
dispute sur la légitimité ou la bâtardise que M. le duc d'Orléans m'a
plus d'une fois assurée, et que les Châtillon étaient éteints depuis
longtemps. Voysin était ministre et favori, il l'était aussi de
M\textsuperscript{me} de Maintenon\,: il parlait tête à tête, elle en
tiers, il demanda que son gendre drapât comme ayant l'honneur
d'appartenir au roi, et il ne lui appartenait en aucun degré, mais il
n'avait point de contradicteur, et son gendre drapa.

Cette nouveauté réveilla La Vallière et M\textsuperscript{me} la
princesse de Conti, pour les Beauvau, dont avec trop de raison ils
s'honoraient fort de l'alliance. La grand'mère de M\textsuperscript{me}
de La Vallière, mère de M\textsuperscript{me} la princesse de Conti, et
sœur du père de La Vallière était Beauvau par un cas fort étrange.

La sixième aïeule paternelle du roi était Beauvau, et il était au
huitième degré de tous les Beauvau. La parenté était bien éloignée, mais
au moins était-elle, et à cela il n'y avait point de parité avec M. de
Châtillon qui n'en eut jamais l'apparence, et à qui il fut permis de
draper. Sur cet exemple et cette sixième grand'mère,
M\textsuperscript{me} la princesse de Conti obtint aussi de faire draper
les Beauvau, qui non plus que les Châtillon n'y avaient jamais songé
jusqu'alors.

Le roi avait déclaré que de trois mois il ne quitterait Marly à cause du
mauvais air répandu à Versailles, et qu'il recevrait à Marly, le lundi
20 avril, les compliments muets de tout le monde, en manteaux et en
mantes, soit des gens qui étaient à Marly, soit de ceux qui étaient à
Paris. M. du Maine qui, comme on a vu, n'avait pas perdu de temps à
mettre à profit pour le rang de prince du sang de ses enfants la mort
des seuls princes du sang en âge et en état de l'empêcher se trouva bien
autrement à son aise de la mort de Monseigneur, qui avait si mal reçu ce
rang nouveau de ses enfants, après avoir été si peu content du sien
même. Il avait plus que raison d'appréhender d'en tomber sous son règne,
et on a vu que Monseigneur ne se contraignit pas là-dessus avec lui, et
quel fut son silence, et celui de Mgr le duc de Bourgogne, lorsque le
roi s'humilia, pour ainsi dire, devant eux pour leur faire agréer et en
obtenir quelque parole si constamment refusée, en leur présentant M. du
Maine pour les toucher. Monseigneur mort, le duc du Maine n'eut plus
affaire qu'à Mgr le duc de Bourgogne. C'était beaucoup trop. Mais
pourquoi ne pas espérer d'en voir la fin comme il voyait celle du père
et en attendant pousser son bidet\,? II connaissoit la faiblesse et
l'incurie de M. le duc d'Orléans, dont le fils était enfant, il voyait
quel était M. le duc de Berry. Il sentit qu'avec M\textsuperscript{me}
de Maintenon il n'avait plus rien à craindre pour s'élever aussi haut
qu'il pourrait dans le présent, et remit le futur à son industrie et à
sa bonne fortune.

Le duc de Tresmes était en année, c'en était déjà une, et il en sut
profiter. Avec beaucoup d'honneur et de probité, Tresmes était sans le
moindre rayon d'esprit que l'usage de la cour et du grand monde, et de
l'ignorance la plus universelle. Avec cela plus valet que nul valet
d'extraction, et plus avide de faire sa cour et de plaire que le plus
plat provincial. Avec ces qualités ce fut l'homme de M. du Maine.

C'était à lui à recevoir et à donner les ordres pour ces révérences de
deuil. Il mit au roi en question si on irait les faire à ses enfants
naturels, comme étant frères et sœurs de Monseigneur. Le roi, toujours
éloigné de ces gradations par lesquelles il a été peu à peu mené à tout
pour eux contre son sens, comme on l'a vu sans cesse, trouva d'abord la
proposition du duc de Tresmes ridicule. Il ne répondit pourtant pas une
négative absolue, mais il marqua seulement que cela ne lui plaisait pas.
M. du Maine, qui s'y était attendu par toutes ses expériences pareilles,
n'avait lâché le duc de Tresmes que le dimanche, pour ne laisser pas de
temps, mais pour donner lieu au roi d'en parler le soir à
M\textsuperscript{me} de Maintenon. Nonobstant cette ruse, il n'y fut
rien décidé, mais c'était beaucoup que ce ne fût pas une négative, et
que M\textsuperscript{me} de Maintenon en eût assez fait pour le laisser
dans la balance. Il y était encore le lundi matin, jour de ces
révérences. Mais entre le conseil et le petit couvert, M. du Maine
secondé de son fidèle second l'emporta, et le duc de Tresmes, en ayant
pris l'ordre du roi, le publia aussitôt. La surprise en fut si grande
que presque chacun se le fit répéter.

Le moment de la déclaration fut pris avec justesse. Le roi se mettait à
table, tout le monde y était déjà ou s'y allait mettre, et la cérémonie
commençait à deux heures, c'est-à-dire tout au sortir de dîner\,; ainsi
point de temps à raisonner, encore moins à faire\,; et on obéit, avec la
soumission aveugle et douloureuse à laquelle on était si fort accoutumé.

Par cette adresse les bâtards furent pleinement égalés aux fils et aux
filles de France, et mis en plein parallèle avec eux\,: pierre d'attente
pour laquelle le roi n'a pas tout à fait assez vécu.

Ce même jour lundi, 20 avril, le roi fit ouvrir les portes de ses
cabinets devant et derrière à deux heures et demie. On entrait par sa
chambre. Il était en habit ordinaire, mais avec son chapeau sous le
bras, debout et appuyé de la main droite sur la table de son cabinet la
plus proche de la porte de sa chambre. M. {[}le Dauphin{]} et
M\textsuperscript{me} la Dauphine, M. {[}le duc{]} et
M\textsuperscript{me} la duchesse de Berry, Madame, M. {[}le duc{]} et
M\textsuperscript{me} la duchesse d'Orléans, M\textsuperscript{me} la
Grande-Duchesse, M\textsuperscript{me} la Princesse,
M\textsuperscript{me} la Duchesse, ses deux fils et ses deux filles, M.
du Maine et le comte de Toulouse se rangèrent en grand demi-cercle
au-dessous du roi à mesure qu'ils entrèrent, tous en grands manteaux et
en mantes, hors les veuves qui n'en portent point et n'ont que le petit
voile. M\textsuperscript{me} la princesse de Conti douairière était
malade dans son lit, l'autre princesse de Conti avec ses enfants restée
à Paris à cause de l'air de la petite vérole, et M\textsuperscript{me}
du Maine avec les siens à Sceaux pour la même raison. Tout Paris, vêtu
d'enterrement ainsi que tout Marly, remplissait les salons et la chambre
du roi. Douze ou quinze duchesses entrèrent à la file les premières,
puis dames titrées et non titrées comme elles se trouvèrent, et les
princesses étrangères, arrivées tard contre leur vigilance ordinaire, y
furent mêlées\,; après les dames, l'archevêque de Reims, suivi d'une
quinzaine de ducs, et ces deux têtes en rang d'ancienneté, entrèrent\,;
puis tous les hommes titrés et non titrés, princes étrangers, prélats,
mêlés au hasard. Quatre ou cinq pères ou fils de la maison de Rohan se
mirent ensemble à la file en rang d'aînesse vers le milieu de la
marche\,; quelques gens de qualité qui s'aperçurent de cette affectation
les coupèrent, en sorte qu'ils furent tous mêlés, et entrèrent ainsi
dans le cabinet. On allait droit au roi l'un après l'autre\,; et, à
distance de lui, on lui faisait une profonde révérence qu'il rendait
fort marquée à chaque personne titrée, homme et femme, et point du tout
aux autres. Cette révérence unique faite, on allait lentement à l'autre
cabinet, d'où on sortait par le petit salon de la chapelle. La mante et
le grand manteau était une distinction réservée aux gens d'une certaine
qualité, mais elle avait disparu avec tant d'autres, jusque-là qu'il en
passa devant le roi que ni lui ni pas un du demi-cercle ne connut, et
personne même de la cour qui pût dire qui c'était, et il y en eut
plusieurs de la sorte. Il s'y mêla aussi des gens de robe, ce qui parut
tout aussi singulier.

Il est difficile que la variété des visages, et la bigarrure de
l'accoutrement de bien des gens peu faits pour le porter, ne fournissent
quelque objet ridicule qui ne démonte la gravité la plus concertée. Cela
arriva en cette occasion, où le roi eut quelquefois peine à se retenir,
et où même il succomba une fois avec toute l'assistance au passage de je
ne sais plus quel pied plat à demi abandonné de son équipage.

Quand tout fut fini chez le roi, et cela fut long, tout ce qui devait
être visité se sépara, pour aller chacun chez soi recevoir les visites.
Les visités ne furent autres que les fils et filles de France, et les
bâtards et bâtardes, et M. le duc d'Orléans comme mari de
M\textsuperscript{me} la duchesse d'Orléans, et celui-là parut comique.
Les moindres d'aînesse ou de rang allèrent chez leurs plus grands, qui
ne leur rendirent point la visite, excepté Madame, qui, comme veuve du
grand-père de M\textsuperscript{me} la Dauphine et grand'mère de
M\textsuperscript{me} la duchesse de Berry, fut visitée des fils et
filles de France, mais non M. {[}le duc{]} et M\textsuperscript{me} la
duchesse d'Orléans. On alla donc comme on put faire cette tournée. On
entrait et sortait pêle-mêle, et on ne faisait que passer entrant par
une porte et sortant par une autre, où il y avait des dégagements. C'est
ce qui se rencontra chez M\textsuperscript{me} la Duchesse, et à la
faveur de cette commodité, une subtilité de M\textsuperscript{me} la
Princesse, fort prompte à saisir ses avantages tout dévotement. Sortant
de chez M\textsuperscript{me} la. Duchesse par le dégagement de son
cabinet, on y trouva M\textsuperscript{me} la Princesse qui se
présentait à la compagnie pour recevoir les révérences, qui ne lui
étaient ni dues ni ordonnées. On en fut si surpris que beaucoup de gens
passèrent sans la voir, beaucoup plus sans faire semblant de
s'apercevoir d'elle. Les deux petits princes du sang ne s'y présentèrent
point.

Le duc du Maine et le comte de Toulouse reçurent les visites ensemble
dans la chambre de M. du Maine, où on entrait de plain-pied et
directement du jardin. Ils avaient leur compte, et voulurent faire les
modestes et les attentifs pour ne pas donner la peine d'aller séparément
chez tous les deux. M. du Maine se dépeça en excuses embarrassées de la
peine qu'on prenait, et se tuait à conduire les gens titrés, et à en
manquer tout le moins qu'il pouvait. M. le comte de Toulouse conduisait
aussi avec soin, mais sans affectation.

J'oubliais M\textsuperscript{me} de Vendôme, qui parut aussi chez le roi
en rang d'oignon, mais qui ne fut point visitée, parce que la bâtardise
de son mari venait de plus loin. Elle ne s'embusqua point avec
M\textsuperscript{me} sa mère pour enlever les révérences aux passants.

Ni le roi, ni princes, ni princesses visités ne s'assirent ni n'eurent
de siége derrière eux. Si on se fût assis chez ceux où on le doit être,
cela n'eût point fini de la journée chez chacun\,; et des siéges sans
s'asseoir auraient culbuté le monde dans l'excès de la foule et des
petits lieux.

Le lendemain, mardi 21 avril, M. {[}le Dauphin{]} et
M\textsuperscript{me} la Dauphine, M. {[}le duc{]} et
M\textsuperscript{me} la duchesse de Berry, Madame, M. {[}le duc{]} et
M\textsuperscript{me} la duchesse d'Orléans allèrent, l'après-dînée, en
même carrosse, à Saint-Germain, tous en mante et en grand manteau. Ils
allèrent droit chez le roi d'Angleterre, où ils ne s'assirent point,
ensuite chez la reine, où ils s'assirent dans six fauteuils\,; M. {[}le
duc{]} et M\textsuperscript{me} la duchesse d'Orléans et M. du Maine sur
un ployant chacun. Il était allé les y attendre pour jouir de cet
honneur, et s'y égaler à un petit-fils de France. La reine fit des
excuses de n'être pas en mante pour les recevoir, c'est-à-dire en petit
voile, parce que, au moins en France, les veuves ne portent de mante en
nulle occasion\,; elle ajouta que le roi le lui avait défendu. Cette
excuse fut le comble de la politesse. Le roi, très-attentif à ne faire
sentir à la reine d'Angleterre rien de sa triste situation, n'avait
garde de souffrir qu'elle prît une mante, ni le roi d'Angleterre un
grand manteau, pour recevoir le grand deuil de cérémonie d'un Dauphin et
qui n'était pas roi. En se levant ils voulurent aller chez la princesse
d'Angleterre\,; mais la reine les arrêta et l'envoya chercher. Elle se
contenta que la visite fût marquée. On ne se rassit point. La princesse,
qui à cause de la reine était sans mante, ne pouvait avoir de fauteuil
devant elle, ni les fils et filles de France {[}être{]} sans fauteuil
devant la reine dans le sien, ni garder le leur en présence de la
princesse d'Angleterre sur un ployant. La visite finit de la sorte. De
toute la cour de Saint-Germain aucune dame ne parut en mante, ni aucun
homme en manteau long que le seul duc de Berwick, à cause de ses
dignités françaises.

Le lundi suivant, 29 avril, le roi s'en alla, sur les onze heures du
matin, à Versailles, où il reçut les compliments de tous les ministres
étrangers\,; après eux de beaucoup d'ordres religieux\,; et après son
dîner au petit couvert, les harangues du parlement, de la chambre des
comptes, de la cour des aides, de celle des monnaies, et de la ville de
Paris. La compétence du grand conseil et du parlement mit une heure
d'intervalle, après laquelle il vint aussi faire sa harangue, suivi de
l'Université et de l'Académie française, pour laquelle Saint-Aulaire
porta fort bien la parole. Le parlement alla aussi haranguer Mgr le
Dauphin\,; le premier président ne voulut pas lui laisser ignorer que
c'était par ordre du roi qu'il le haranguait et qu'il le traitait de
Monseigneur. Cette insolente bagatelle mériterait des réflexions.

Tout ce qui avait complimenté ou harangué le roi rendit aussi les mêmes
devoirs à Mgr et à M\textsuperscript{me} la Dauphine. Le roi revint sur
le soir à Marly.

\hypertarget{chapitre-ix.}{%
\chapter{CHAPITRE IX.}\label{chapitre-ix.}}

1711

~

{\textsc{Mort et caractère de la duchesse de Villeroy.}} {\textsc{- Mort
de l'empereur Joseph.}} {\textsc{- Prince Eugène mal avec son
successeur.}} {\textsc{- Mort de M\textsuperscript{me}s de Vaubourg et
Turgot.}} {\textsc{- Mort de Caravas.}} {\textsc{- Mariage des deux
filles de Beauvau avec Beauvau et Choiseul.}} {\textsc{- Reprise de
l'affaire d'Épernon.}} {\textsc{- Force prétentions semblables prêtes à
éclore.}} {\textsc{- Leur impression sur les parties du procès
d'Épernon.}} {\textsc{- Ancien projet de règlement sur les
duchés-pairies en 1694\,; son sort alors.}} {\textsc{- Perversité du
premier président d'Harlay, qui le dressa.}} {\textsc{- Duc de
Chevreuse, de concert avec d'Antin, gagne le chancelier pour un
règlement sur ce modèle.}} {\textsc{- Le chancelier m'en confie l'idée
et l'ancien projet.}} {\textsc{- Raisons qui m'y font entrer sans en
prévoir le funeste, et j'y travaille seul avec le chancelier.}}
{\textsc{- Ancien projet et mes notes dessus.}} {\textsc{-Grâce de
substitution accordée au duc d'Harcourt enfourne ce règlement.}}
{\textsc{- Sagesse et franchise d'Harcourt avec moi sur les bâtards.}}
{\textsc{- Je joins le maréchal de Boufflers au secret, qui est
restreint d'une part entre nous deux et Harcourt en général, de l'autre
entre Chevreuse et d'Antin en général, et sans nous rien communiquer.}}
{\textsc{- Harcourt parle au roi, et la chose s'enfourne.}} {\textsc{-
Chimères de Chevreuse et de Chaulnes.}} {\textsc{- Duc de Beauvilliers
n'approuve pas les chimères\,; ne peut pourtant être admis au secret du
règlement par moi.}} {\textsc{- Secret de tout ce qui se fit sur le
règlement uniquement entre le chancelier et moi.}} {\textsc{- Trait
hardi et raffiné du plus délié courtisan de d'Antin, qui parle au roi.}}
{\textsc{- Le roi suspend la plaidoirie sur le point de commencer sur la
prétention d'Épernon.}}

~

Je perdis en même temps une amie que je regrettai fort\,; ce fut la
duchesse de Villeroy, dont j'ai parlé plus d'une fois. C'était une
personne droite, naturelle, franche, sûre, secrète, qui sans esprit
était parvenue à faire une figure à la cour, et à maîtriser mari et
beau-père. Elle était haute en tous points, surtout pour la dignité, en
même temps qu'elle se faisait une justice si exacte et si publique sur
sa naissance, même sur celle de son mari, qu'elle en embarrassait
souvent. Elle était fort inégale, sans que, pour ce qui me regarde, je
m'en sois jamais aperçu. Elle avait de l'humeur, son commerce était rude
et dur. Elle tenait fort là-dessus de sa famille. Elle était depuis
longtemps dans la plus grande intimité de M\textsuperscript{me} la
duchesse d'Orléans, et dans une grande confidence de
M\textsuperscript{me} la Dauphine, qui toutes deux l'aimaient et la
craignaient aussi. Elle avait des amis et des amies\,; elle en méritait.
Elle était bonne, vive et sûre amie, et les glaces ne coûtaient rien à
rompre. Elle devenait personnage, et on commençait à compter avec elle.
Son visage très-singulier était vilain d'en bas, surtout pour le rire,
était charmant de tout le haut. Sérieuse et parée, grande comme elle
était, quoique avec les hanches et les épaules trop hautes, personne
n'avait si grand air et ne parait tant les fêtes et les bals, où il
n'était aucune beauté et bien plus qu'elle qu'elle n'effaçât. Quelques
mois avant sa mort et toujours dans une santé parfaite, elle disait à
M\textsuperscript{me} de Saint-Simon qu'elle était trop heureuse\,; que,
de quelque côté qu'elle se tournât, son bonheur était parfait\,; que
cela lui faisait une peur extrême, et que sûrement un état si fort à
souhait ne pouvait durer\,; qu'il lui arriverait quelque catastrophe
impossible à prévoir, ou qu'elle mourrait bientôt. Le dernier arriva.
Son mari servait de capitaine des gardes pour le maréchal de Boufflers,
demeuré à Paris pour la mort de son fils. Elle craignait extrêmement la
petite vérole, qu'elle n'avait point eue. Malgré cela, elle voulut que
M\textsuperscript{me} la Dauphine la menât à Marly dans ces premiers
jours de la solitude du roi, sous prétexte d'aller voir son mari. Rien
de tout ce qu'on put lui dire ne put l'en détourner, tant les petites
distinctions de cour tournent les têtes. Elle y eut une frayeur
mortelle, tomba incontinent après malade de la petite vérole, et en
mourut à Versailles. L'abbé de Louvois et le duc de Villeroy
s'enfermèrent avec elle. Le premier en fut inconsolable, l'autre ne le
fut pas longtemps, et bientôt jouit du plaisir de se croire hors de
page. Il n'était pas né pour y être\,; son père trop tôt après le remit
sous son joug.

L'empereur mourut en même temps à Vienne de la même maladie, et laissa
peu de regrets. C'était un prince emporté, violent, d'esprit et de
talents au-dessous du médiocre, qui vivait avec fort peu d'égards pour
l'impératrice sa mère, qu'il fit pourtant régente, peu de tendresse pour
l'impératrice sa femme, et peu d'amitié et de considération pour
l'archiduc son frère. Sa cour était orageuse, et les plus grands y
étaient mal assurés de leur état. Le prince Eugène fut peut-être le seul
qui y perdit. Il avait toute sa confiance, et il était fort mal avec
l'archiduc, qui se prenait à lui du peu de secours qu'il recevait de
Vienne, et qui ne lui pardonnait pas d'avoir refusé d'aller en Espagne.
Ce mécontentement ne fut que replâtré par le besoin et les
conjonctures\,; mais jamais le prince Eugène ne se remit bien avec lui.
Il n'y eut que du dehors sans amitié et sans confiance, et, quant à la
considération et au crédit, ce qui seulement ne s'en pouvait refuser,
quoi que le prince Eugène pût faire, sans se lasser de ramer inutilement
là-dessus jusqu'à la mort. Celle de l'empereur fut un grand coup, et de
ces fortunes inespérables, pour conduire à la paix et conserver la
monarchie d'Espagne. Je ne m'arrêterai pas à ces grandes suites, parce
qu'elles font partie de ce qui se passa en Angleterre, pour préparer au
traité de paix signé à Utrecht, et ensuite avec l'empereur
nouveau\footnote{Charles VI avait été couronné empereur à Francfort, le
  22 décembre 1711. On a déjà parlé des pièces auxquelles renvoie ici
  Saint-Simon et qui se trouvent dans les Mémoires de Torcy.}, et que
ces choses se trouveront mieux dans les Pièces que je ne pourrais les
raconter, comme y étant de main de maître\,; je dirai seulement ici que
Torcy alla, incontinent après, trouver l'électeur de Bavière à
Compiègne, où il demeura un jour avec lui.

Voysin perdit M\textsuperscript{me} de Vaubourg, sa sœur, femme de
mérite, dont le mari, conseiller d'État, capable et d'une grande vertu,
était frère de Desmarets. Ce lien les entretenait ensemble, et sa
rupture eut des suites entre eux. Pelletier de Sousy perdit aussi
M\textsuperscript{me} Turgot, sa fille, qu'il aimait avec passion, et
avec grande raison. Son gendre était un butor qu'il ne put jamais sentir
dans les intendances, ni faire conseiller d'État. Le fils de celui-là
l'est devenu avec beaucoup de réputation, après s'en être acquis une
grande d'intégrité et de capacité dans la place de prévôt des marchands,
et dans des temps fort difficiles.

Le vieux Caravas mourut aussi, qui allait mentir partout à gorge
déployée. Il était Gouffier, et avait, par je ne sais quelle aventure,
épousé autrefois en Hollande la tante paternelle de ce Riperda, dont la
subite élévation au premier ministère d'Espagne, la rapide chute et la
fin, ont tant fait de bruit dans le monde.

Beauvau, qui avait été capitaine des gardes de Monsieur, et qui s'était
retiré de la cour, et presque du monde, depuis longtemps d'une manière
fort obscure, n'avait que deux filles fort riches. Il les maria toutes
deux en ce temps-ci\,: l'une au comte de Beauvau, mort bien longtemps
depuis lieutenant général, gouverneur de Douai, et chevalier de l'ordre
de 1724\,; l'autre au marquis de Choiseul, le seul de cette grande
maison qui fût à son aise.

Ce serait ici le lieu de présenter un nouveau tableau de la cour, après
un changement de théâtre qui dérangea si parfaitement toute la scène\,;
mais cette nouvelle qui succéda a tant de liaison avec toutes les suites
qu'il est à propos de la rejeter après le récit d'une affaire trop
importante pour être omise, quelque longue et ennuyeuse qu'elle puisse
être, et qui eut tant de trait à d'autres temps, d'autant plus que,
commencée avant la mort de Monseigneur, elle a été différée jusqu'au
temps de sa conclusion pour ne la pas interrompre. Il faut donc
retourner sur nos pas. Outre l'importance, il ne laissera pas de s'y
trouver quelques traits curieux.

C'est l'affaire de d'Antin, qu'il s'agit de reprendre jusqu'à sa
conclusion. Ce n'était pas la seule dont il pût être question. Une
quinzaine de chimères, plus absurdes les unes que les autres, étaient
prêtes à éclore. Les visions attendaient l'événement de celle de
d'Antin, pour différer à un autre temps, ou pour entrer en lice si la
sienne réussissait, avec la confiance que le roi et les juges les
protégeaient volontiers, pour montrer que, sans être favori, on gagnait
des causes contre toutes sortes de règles. Les procès existants étaient
celui de M. de Luxembourg, qu'il venait de remettre en train judiciaire,
en même temps qu'il s'était joint aux opposants à la prétention de
d'Antin\,; et j'agissais déjà pour tâcher d'annuler l'arrêt sans force
et sans mesure qu'il avait obtenu, et le réduire à l'ancien détroit
d'option entre son érection nouvelle ou n'être point pair. Je passe
légèrement sur cette affaire si bien expliquée au commencement de ces
Mémoires, et par les factums imprimés de part et d'autre qui sont entre
les mains de tout le monde, et celui d'entre M. de La Rochefoucauld et
moi. Ceux qui n'étaient pas encore formés, mais tout prêts à l'être,
celui d'Aiguillon et celui d'Estouteville.

Les chimères encore recluses, mais qui n'attendaient pas moins
impatiemment la conjoncture de paraître en prétentions, étaient celle de
l'ancienneté de Chevreuse, de l'érection en faveur des Lorrains, et
celle de Chaulnes, toutes deux dans la tête et dans la volonté du duc de
Chevreuse\,; celle de l'ancienneté de Rohan, du grand-père maternel du
duc de Rohan-Chabot\,; celle des premières érections d'Albret et de
Château-Thierry, dont M. de Bouillon ne pouvait se départir, et dont on
a vu ailleurs que le premier présidant Harlay s'était moqué si
cruellement en parlant à sa personne. Il n'y avait pas jusqu'aux Bissy à
qui l'ivresse de la faveur de leur évêque de Meaux ne tournât la tête,
jusqu'à prétendre la dignité de Pont-de-Vaux, et cinq ou six autres de
même espèce {[}qui{]}, par les tortures prétendues applicables aux
duchés femelles, eussent eu lieu, et tombées dans la boue par des
alliances et des arrière-alliances déjà contractées.

C'est ce qui nous faisait peur pour le renversement entier de tout ordre
et de toute règle parmi nous, par l'achèvement de toute ignominie dans
la transmission de ces dignités sans mesure\,; et même en réussissant
contre elles, par une vie misérable de chicanes, de procès et de
procédés, chacun ne manquant point de chicanes et de subterfuges pour
détourner de dessus soi la condamnation de son voisin et même de son
semblable, et se présenter hardiment sous des apparences d'espèces
différentes. C'était néanmoins ce qui nous pouvait arriver de mieux que
de gagner en luttant, et de nous consumer en luttes.

Nous ne cessions de nous plaindre de ces amas de prétentions et de
procès, que nous nous voyions pendre sur la tête par le fait de d'Antin,
que son exemple avait ranimés\,; et nous nous servions de ce débordement
pour aggraver l'importance de laisser les choses dans les règles de tout
temps suivies et reconnues. D'Antin, qui s'en aperçut, et que ce que
nous alléguions là-dessus ne nous était pas inutile, sut tourner court
et prendre au bond cette balle avec finesse pour s'en servir lui-même
avec avantage. Outre tout le mauvais de sa cause en soi, dont il fut
toujours très-persuadé comme il nous l'a avoué depuis, il sentait
l'extrême embarras où il allait tomber par nos fins de recevoir qu'il ne
pouvait assez s'étonner que nous eussions découvertes, ce qui était
l'ouvrage de Vesins, l'un de nos meilleurs avocats. La cause dirimante
par la mésalliance de Zamet, de laquelle seule il tirait son prétendu
droit, était sans réponse\,; et il n'avait garde d'être tranquille sur
son acquisition d'Épernon, autre fait dirimant. Monseigneur qui y était
mêlé eût pu le lui reprocher durement, et donner lieu à ses ennemis de
Meudon, qui commençaient à prévaloir, de lui faire un crime auprès de ce
prince d'avoir abusé de sa faveur pour une acquisition dont il ne lui
avait pas montré l'objet, et lui faire faire ainsi bien du chemin dans
la descente. Il s'y joignait un malaise du roi importuné de ses
absences, qui pouvait aisément se tourner en dégoût, ou en habitude de
se passer de lui pour les bagatelles dont il savait faire un si habile
usage.

Un contraste assez ferme qu'il eut à la porte de Dongois, greffier du
parlement, avec les ducs de Charost et de Berwick sur des procédés, et
qui furent poussés assez loin de la part des nôtres sur quelques
longueurs dont il voulut se plaindre, tandis qu'il nous y avait forcés
par un piége, et la hauteur dont la chose fut prise de notre part à tous
enfin, le changement de l'air du monde et même de celui de la cour, le
bruit sourd du palais qui ne lui était pas favorable, toutes ces choses
ensemble l'avaient effrayé dès le carême, jusqu'à le désespérer
intérieurement du succès, et lui faire craindre de perdre encore autre
chose que son procès.

Ces mêmes choses firent une impression pareille au duc de Chevreuse pour
ce qui le regardait, qui, né timide et chancelant, crut voir sa
condamnation écrite par les épines que le favori éprouvait. Ennemis de
cabale, et sur toute autre chose, mais liés tous deux sur ces matières,
tant l'intérêt a de pouvoir jusque sur les plus honnêtes gens tels que
l'était Chevreuse, il tourna ses pensées au souvenir d'un règlement
général projeté lors du procès de feu M. de Luxembourg, et il espéra du
crédit de d'Antin de remettre ce règlement sus, et de faire passer son
second fils duc de Chaulnes avec lui, en abandonnant leurs prétentions
de l'ancienneté d'Épernon et de celle de Chevreuse. Ce point si
funestement capital mérite d'être un peu plus expliqué dès son origine.

Lors du plus grand mouvement, en 1694, du procès entrepris par M. de
Luxembourg contre ses anciens, il fut fait un projet, que j'ignorai
longtemps depuis, qui réglait en forme de déclaration du roi les
transmissions contestées de la dignité de duc et pair, laquelle excluait
presque entièrement les femelles, mais qui, avec cet appât aux ducs, les
assommait par l'établissement du grand rang des enfants naturels du roi.
Harlay, premier président, qui papegeait\footnote{\emph{Papegeait pour
  la place} signifie \emph{visait à la place}. Le verbe qu'emploie
  Saint-Simon vient du mot \emph{papegai}, qui désignait un oiseau de
  carton ou de bois peint que l'on plaçait au bout d'une perche ou d'un
  poteau pour exercer à tirer de l'arc, de l'arbalète ou de l'arquebuse.}
pour la place de chancelier que le cadavre de Boucherat remplissait
encore\,; qui, procureur général, avait ouvert la voie en faisant
légitimer le chevalier de Longueville, tué depuis, sans nommer la
mère\,; qui avait eu pour cet exécrable service, parole réitérée des
sceaux, voulut, vil et détestable esclave du crime et de la faveur,
cueillir les fruits de son ouvrage par ce couronnement inouï de ces
enfants, qui, sans lui et son invention cauteleuse et hardie, eussent
forcément été ceux de M. de Montespan, peut-être des enfants trouvés
dans l'impuissance d'énoncer père ni mère. C'était donc bien moins en
faveur de la paix que cette déclaration avait été conçue, et pour mettre
des bornes fixes et précises aux transmissions des duchés femelles que
pour la grandeur des bâtards. Harlay y avait fait consentir M. de
Luxembourg et son fils. Mais ce projet fut tant tourné, rebattu,
rajusté, que le roi, du goût duquel ces choses ne furent jamais,
l'abandonna, sitôt que par une voie plus militaire, et telle qu'elle a
été racontée, il eut trouvé plus court de donner à ses fils naturels, et
bientôt après à leur postérité, en la personne du duc de Vendôme, une
préséance énorme, qui, lui ayant paru alors le comble de leur grandeur
et de sa toute-puissance, ne devint pourtant que le piédestal des
horribles prodiges qu'on a vus depuis en ce genre.

Le duc de Chevreuse d'accord avec d'Antin parla au chancelier. Il lui
donna envie de la gloire d'un ouvrage qui finirait toutes ces fâcheuses
contestations\,; et toucha peut-être en lui la partie faible du
courtisan, désireux d'aplanir à son maître la voie d'élever de plus en
plus ses enfants naturels, et d'achever la fortune de son favori, en se
conciliant ces grands personnages du temps présent. Le chancelier gagné
m'en parla d'abord avec une entière ouverture, mais une imposition
étroite du secret. Nous agitâmes la matière, et j'avouerai à ma honte,
ou à celle d'autrui que, n'imaginant pas qu'il fût dans la possibilité
de trouver pour les bâtards rien au delà de ce qu'ils avaient, il ne
m'entra pas dans l'esprit qu'ils profitassent du règlement qui se
pouvait mettre sur le tapis, autrement que par une confirmation de tout
ce dont ils étaient en possession, qui n'ajoutait rien à leur droit ni à
leur jouissance. Ce fut par où nous commençâmes.

Le chancelier me fit bien entendre, et sans peine, que le chausse-pied
de la déclaration (ce fut son terme) serait inévitablement l'intérêt des
bâtards, \emph{causa sine qua non} du roi en toutes ces matières\,; mais
avec ma sotte présupposition qu'il appuya, et je crois de bonne foi
alors, je conclus qu'il valait mieux à ce prix sortir tout d'un coup,
par une bonne déclaration, de tant d'affaires que de nous y laisser
consumer. Je pensais que couper à jamais toutes racines de questions de
préséance entre nous nous mettrait à couvert des schismes qui se
mettaient si souvent parmi nous, et que nous délivrer une bonne fois des
ambitions femelles nous délivrerait des désordres et des successions
indignes qui achevaient la confusion. Je considérais une barrière aux
favoris présents et futurs d'autant plus à désirer que l'âge du roi en
faisait craindre de capables de s'en prévaloir avec hardiesse\,; et il
est vrai encore que mon repos particulier acheva de me déterminer, parce
que le poids de toutes ces sortes d'affaires tombait toujours sur moi,
en tout ou en la plus grande partie, pour le travail dont je ne me
pouvais défendre, et pour la haine qui en résultait, avec peu ou point
de secours ni d'appui.

Ce parti bien pris en moi-même, et justement fondé sur nos misères
intérieures dont je n'avais qu'une trop continuelle expérience, il fut
question d'y travailler. Pour le faire utilement, le chancelier me
montra le projet du premier président d'Harlay. Nous l'examinâmes
ensemble\,; et pour mieux faire, il me le confia pour en tirer une
copie, et pour, sur cette copie, faire mes notes, afin de les discuter
après avec lui, et arrêter ensemble un nouveau projet sur cet ancien,
qui nous fît trouver notre compte par des lois sages et justes, et par
des avantages qui, autant que le temps le pouvait comporter, nous
dédommageassent de la confirmation de la grandeur des bâtards, qu'il
fallait bien s'attendre devoir être énoncée dans ce règlement.

Pour mieux entendre ce qu'il en arriva, il ne sera pas peu à propos ni
peu curieux d'insérer ici, plutôt que le renvoyer aux Pièces, cet ancien
projet du premier président d'Harlay, avec les notes que je mis à chaque
article de ce que je crus qui y devait être changé, retranché ou
ajouté\,; l'ancien projet d'un côté à mi-marge, mes notes de l'autre,
vis-à-vis chaque article, tel que je le donnai au chancelier. Cet ancien
projet avait été concerté entre le chancelier, lors contrôleur général
et secrétaire d'État de la maison du roi et ministre, le premier
président d'Harlay, et d'Aguesseau, lors avocat général, aujourd'hui
chancelier, communiqué par ordre du roi, et revu par le duc de
Chevreuse, qui en-avait, disait-il, perdu la copie qu'il en avait eue,
et convenu pour lui-même, et par MM. de Luxembourg père et fils pour
eux, et resté en 1696 fixé entre eux tel qu'il suit\,:

ANCIEN PROJET.

NOTES.

\begin{enumerate}
\def\labelenumi{\Roman{enumi}.}
\tightlist
\item
\end{enumerate}

Les princes du sang seront honorés en tous lieux, suivant le respect qui
est dû à leur naissance\,; et, en conséquence, auront droit d'entrée,
séance et voix délibérative au parlement de Paris à l'âge de\ldots, tant
aux audiences qu'au conseil, sans aucune formalité.

Ce premier article pourrait être omis comme tout à fait inutile.

\begin{enumerate}
\def\labelenumi{\Roman{enumi}.}
\setcounter{enumi}{1}
\tightlist
\item
\end{enumerate}

Les enfants naturels des rois qui auront été légitimés, et leurs enfants
et descendants mâles qui posséderont des duchés-pairies, auront droit
d'entrée, séance et voix délibérative en ladite cour, à l'âge de\ldots{}
ans, en prêtant le serment ordinaire des pairs, avec séance
immédiatement après et au-dessous des princes du sang, et y précéderont,
ainsi qu'en tous autres lieux, tous les ducs et pairs, quand leurs
duchés-pairies seraient moins anciennes que celles des ducs et pairs.

Ce second article pourrait être omis comme tout à fait inutile. Il y en
a une déclaration expresse, qui n'était pas lors, et qui est enregistrée
et confirmée par un usage constant depuis.

\begin{enumerate}
\def\labelenumi{\Roman{enumi}.}
\setcounter{enumi}{2}
\tightlist
\item
\end{enumerate}

Les ducs et pairs auront rang et séance entre eux du jour de l'arrêt de
l'enregistrement, qui sera fait au parlement de Paris, des lettres
portant érection du duché-pairie qu'ils possèdent, et seront reçus audit
parlement à l'âge de vingt-cinq ans, en la manière accoutumée.

Le duché de Brancas n'est point vérifié au parlement de Paris, et c'est
le seul existant. Il est du feu roi, et perdrait beaucoup à prendre rang
de l'enregistrement qu'il en faudrait faire présentement au parlement de
Paris, aux termes de ce troisième article. On n'oserait proposer d'y
ajouter la pairie pour dédommagement, en prenant la queue de tout par un
enregistrement de duché-pairie au parlement de Paris, laissant caduc
celui du parlement d'Aix. Il y a de grandes raisons pour fixer le rang
des pairs au jour de la réception de l'impétrant au parlement, celui de
l'enregistrement fixerait le rang des ducs vérifiés qui ne sont pas
pairs.

Quant à l'âge, on ne peut contester l'indécence et l'inconvénient d'un
trop jeune âge, mais on ne peut contester aussi qu'il n'y en a non plus
de réglé pour les pairs que pour les princes du sang, témoin le feu duc
de Luynes, reçu à quinze ans, et bien d'autres. Puis donc qu'un âge ne
peut être fixé sans faire une nouveauté intéressante, et que les pairs
les plus avancés en âge ne savent pas plus de jurisprudence que les plus
jeunes, dont l'étude est la raison principale qui a fixé l'âge pour la
magistrature, à laquelle étude les pairs ne sont en rien assujettis, il
paraît qu'un tempérament convenable serait de fixer l'âge de la
réception des pairs à vingt ans, pour différence d'avec les magistrats.

Si on omet les deux premiers articles, il serait utile d'ajouter en
celui-ci que les pairs auront entrée, séance et voix délibérative, tant
aux audiences qu'au conseil, pour éviter équivoque par une expression
différente ou tacite.

Il serait nécessaire, pour couper court à mille nouvelles et
insoutenables difficultés, d'ajouter que les pairs garderont, dans tous
les parlements du royaume, la même forme d'entrer dans le lieu de la
séance et d'en sortir qu'ils ont accoutumé de garder en celui de Paris,
cour ordinaire des pairs et le premier de tous les parlements, dont
l'exemple ne peut et ne doit être refusé d'aucun autre.

\begin{enumerate}
\def\labelenumi{\Roman{enumi}.}
\setcounter{enumi}{3}
\tightlist
\item
\end{enumerate}

Les termes \emph{d'ayant cause} n'auront aucun effet dans les lettres
d'érection des duchés-pairies qui auront été accordées jusqu'à cette
heure où ils auraient été mis, et ne seront plus insérés dans aucunes
lettres à l'avenir.

Il ne faut point supprimer un terme consacré par un long usage, et qui,
en effet, est essentiel, mais lui donner seulement une interprétation
générale pour toutes les lettres, tant expédiées qu'à expédier, qui soit
fixe et certaine. Il faut donc exprimer que, par \emph{ayant cause}, le
concesseur entend les mâles issus de l'impétrant, étant de son nom et
maison, en quelque degré et ligne collatérale que ce puisse être, en
gardant entre eux l'ordre et le rang de branche et d'aînesse, afin que
la dignité se conserve et perpétue dans les issus mâles de l'impétrant
de son nom et maison, tant et si longtemps qu'il restera un seul mâle
issu de l'impétrant de son nom et maison.

Les clauses générales insérées ci-devant en quelques lettres d'érection
de duchés-pairies en faveur des femelles, n'auront aucun effet qu'à
l'égard de celles qui descendront et seront du nom et maison de
l'impétrant\footnote{On peut ajouter\,: si ce n'est qu'il plaise au roi
  d'étendre sa grâce aux filles des filles par une clause expresse.
  \emph{(Note de l'auteur du projet.)}}, et à la charge qu'elles
épouseront des personnes que le roi jugera dignes de posséder cet
honneur, et dont Sa Majesté aura agréé le mariage par des lettres
patentes qui seront adressées au parlement.

Ajouter à cet article, où aucun mot n'est à changer, que du mariage
d'une fille, qui, aux termes dudit article, fera son mari duc et pair,
sortira une race ducale masculine, c'est-à-dire qu'en la personne du
fils de cette fille la duché-pairie femelle deviendra masculine, dont la
succession à la dignité sera semblable en tout à la succession de tout
autre dignité de duc et pair qui n'a jamais été femelle, et qui n'a été
érigée qu'en faveur des seuls mâles.

Exprimer si le gendre aura le même rang que le beau-père, ou de la date
des lettres patentes adressées au parlement pour son mariage, et alors
conséquemment de sa réception s'il est pair, ce qui fixe le rang de ce
duché, devenu alors masculin. Il semble que, avec cette restriction
apportée aux duchés femelles, on pourrait laisser au gendre le rang de
son beau-père\,; bien entendu que cet édit ait un effet rétroactif en
tous ses points et articles. Pour ce qui est des filles des filles,
c'est une chose à bannir et à proscrire à jamais, comme une porte
funestement ouverte aux inconvénients contre lesquels cet édit est
principalement salutaire.

\begin{enumerate}
\def\labelenumi{\Roman{enumi}.}
\setcounter{enumi}{5}
\tightlist
\item
\end{enumerate}

Permettre à ceux qui ont des duchés d'en substituer à perpétuité, ou
pour un certain nombre de personnes plus grand que celui de deux, outre
l'institué, prescrit par l'ordonnance de Moulins, art. 59, le chef-lieu
avec une certaine partie de leur revenu, montant jusqu'à\ldots{} de
rente, auquel le titre et dignité des duchés-pairies demeurera annexé,
sans pouvoir être sujet à aucunes dettes ni détractions, de quelque
nature qu'elles puissent être, après qu'on aura observé les formalités
prescrites par les ordonnances pour la publication des substitutions.

Il serait beaucoup plus à propos qu'à l'exemple des majorasques
d'Espagne, cet édit marquât que toute érection de duché porte
substitution perpétuelle de la terre érigée, c'est-à-dire du chef-lieu
et d'un certain nombre de paroisses aux environs, faisant un revenu de
quinze mille livres de rente, avec privilége, outre ceux contenus en ce
sixième article\,; que ce revenu ne pourra être saisi pour aucune cause
que ce puisse être\,; que s'il y a des duchés entiers qui ne les valent
pas, tant pis pour leurs titulaires possesseurs, qui néanmoins les
pourront accroître par des acquisitions\,; que s'il se trouve des ducs
trop obérés pour que cette concession ne préjudiciât pas à leurs
créanciers, donner pouvoir aux petits commissaires de la grand'chambre
du parlement de Paris de changer l'hypothèque des créanciers sur les
biens libres de la femme du duc, et de faire en sorte de rendre le duché
capable de jouir du bénéfice de cette disposition, qui, une fois connue,
ne peut plus préjudicier à l'avenir, et assure une subsistance modique
aux plus grands dissipateurs pour soutenir leur dignité, et délivre les
maisons de la négligence de plusieurs ducs à se servir de cette grâce,
si elle n'était qu'offerte et ouverte à volonté, comme elle l'est dans
cet article sixième. On sait que les fiefs de dignité sont à peu près
revêtus de tous ces avantages par toute l'Allemagne\,; que ceux d'Italie
ne se peuvent, à proprement parler, réputer tels, hors les vraies
souverainetés, et que ceux d'Angleterre ne sont que des noms et des
titres vains, jamais possédés par ceux qui les portent.

\begin{enumerate}
\def\labelenumi{\Roman{enumi}.}
\setcounter{enumi}{6}
\tightlist
\item
\end{enumerate}

Permettre aux mâles descendants en ligne directe de l'impétrant de
retirer le duché-pairie des filles qui se trouveront en être
propriétaires, en leur en remboursant le prix dans\ldots, sur le pied du
denier\ldots{} du revenu actuel.

Le remboursement du prix doit être reçu forcément par les femelles, et
réduit à un denier fort au-dessous du revenu de la terre, payable par un
contrat de constitution\footnote{Contrat de constitution de rente.}. La
pratique très-embarrassante de cet article serait supprimée par la
substitution de droit perpétuelle, proposée sur l'article précédent.

\begin{enumerate}
\def\labelenumi{\Roman{enumi}.}
\setcounter{enumi}{7}
\tightlist
\item
\end{enumerate}

Ordonner que ceux qui voudront former quelque contestation sur le sujet
des duchés-pairies, et des rangs, honneurs et préséances accordés par le
roi aux ducs et pairs, princes et seigneurs de son royaume, seront tenus
de représenter, chacun en particulier, à Sa Majesté l'intérêt qu'ils
prétendent y avoir, afin d'en obtenir la permission de le poursuivre, et
qu'elle puisse y prononcer elle-même, si elle le trouve à propos, ou
renvoyer par un arrêt de son conseil d'État les parties pour procéder et
être jugées en son parlement\,; et en cas qu'après y avoir renvoyé une
demande, les parties veulent en former d'autres incidemment qui soient
différentes de la première, elles soient tenues d'en obtenir de
nouvelles permissions de Sa Majesté.

Bon. Pourvu qu'il n'émane aucun arrêt qui dès là que ce serait un arrêt,
attaquerait le droit et la dignité de la cour des pairs, mais bien un
ordre verbal du roi, ou une lettre de cachet au parlement, ou du
secrétaire d'État de la maison du roi au premier président, au procureur
général, et au premier avocat général du parlement de Paris, marquant la
volonté du roi par son ordre.

Il parait équitable de donner aux ducs vérifiés non pairs, et aux duchés
vérifiés sans pairie, les mêmes avantages qu'aux ducs et pairs et aux
duchés-pairies, en les comprenant en cet édit, si ce n'est que le revenu
perpétuellement substitué des duchés vérifiés non pairies pourrait être
modéré à dix mille livres de rente.

\begin{enumerate}
\def\labelenumi{\Roman{enumi}.}
\setcounter{enumi}{8}
\tightlist
\item
\end{enumerate}

Ordonner enfin que M. de Luxembourg aura son rang de 1662.

À la bonne heure, mais en disant\,: \emph{et voulant traiter
favorablement, etc}., parce que ce rang même aujourd'hui n'est pas
invulnérable, et qu'il ne faut pas révoquer en doute ce qui le peut et
doit attaquer, chose en soi très-indifférente à M. de Luxembourg par
quels termes qu'il conserve ce rang, dès là qu'il le conserve, et que
c'est par des termes honnêtes pour lui.

Tel était l'ancien projet et telles les notes que j'y mis\,; ce qui fut
bientôt fait de ma part, mais non pas sitôt convenu entre le chancelier
et moi. Avant de rapporter cette dispute, qu'interrompit mon voyage de
Pâques à la Ferté, et la mort de Monseigneur ensuite, il est à propos
d'expliquer comment la chose s'enfourna parmi nous.

Le duc d'Harcourt, toujours attentif à ses affaires, demandait en ce
temps-là une grâce qui donna le branle à tout. C'était une déclaration
du roi qui donnât une préférence à tous ses issus mâles, exclusive de
tout issu par femelles, à la succession de son duché-pairie, pour éviter
l'inconvénient des héritières des branches aînées qui, emportant la
terre à titre de plus proches, mettaient, par là, ou par un prix trop
fort, les cadets mâles hors d'état de recueillir une glèbe, sans la
possession de laquelle ils ne peuvent recueillir la dignité, qui
s'éteint ainsi sur eux forcément, comme il avait pensé arriver tout
récemment aux ducs de Brissac et de Duras. Le roi y consentit\,; mais la
forme n'était pas aisée, parce que Harcourt, qui voulait travailler
solidement, cherchait à la rendre telle que la coutume de Normandie, où
son duché était situé, ne pût en d'autres temps donner atteinte à son
ouvrage.

Quand donc j'eus consenti, le chancelier me permit d'en parler à
Harcourt qui, pour une saignée au pied qui avait peine à se fermer,
gardait la chambre dans l'appartement des capitaines des gardes en
quartier, qu'il servait pour le maréchal de Boufflers navré de douleur
de la mort de son fils, et que le duc de Villeroy servit bientôt après,
pour laisser Harcourt se préparer à son départ pour Bourbonne et pour le
Rhin.

Harcourt trouvait doublement son compte dans la proposition que je lui
fis, puisque la grâce qu'il demandait devenait bien plus sûre par un
article exprès d'un édit général, et par se voir délivré d'être la
partie du favori. Mais ma surprise fut extrême lorsque j'entendis ce
courtisan intime de M\textsuperscript{me} de Maintenon, et de M. du
Maine, auquel je savais qu'il s'était prostitué par des traits de la
dernière bassesse, me dire sans détour que, dès qu'on ne pouvait espérer
de déclaration du roi qu'en y confirmant les avantages des bâtards (car
ce fut son propre terme, et avec un ton de dépit), rien n'en pouvait
être bon. Je répondis que cette confirmation n'ajoutait rien à ce qu'ils
avaient, et partant ne nous nuirait pas davantage\,: «\,Voyez-vous,
monsieur, me répliqua-t-il avec feu, je vis très-bien avec eux et suis
leur serviteur\,; mais je vous avoue que leur rang m'est insupportable.
Il n'y a de parti présent que de se taire, mais dans d'autres temps il
faut culbuter tout cela, comme on renverse toujours les choses violentes
et odieuses, comme le rang de Joyeuse et d'Épernon a fini avec Henri
III, et comme dans eux-mêmes le rang du bonhomme Vendôme finit avec
Henri IV. C'est ce que nous devons toujours avoir devant les yeux comme
ce qu'il y a de plus important, car c'est là ce qui nous blesse le plus
essentiellement. Ainsi, avec ce dessein-là, que nous ne devons jamais
perdre de vue, je ne puis être d'avis de passer une déclaration qui
fortifie ce qui ne l'est déjà que trop, et ce que nous devons détruire.
Je vous parle à cœur ouvert, ajouta-t-il avec un air plus serein,
sentant peut-être ma surprise\,; je sais qu'on peut vous parler ainsi,
tous ceux qui ont un reste de sentiment ne peuvent penser autrement.\,»

Quelque étourdi que je fusse d'une franchise si peu attendue, je lui
avouai que je sentais la même peine que lui sur les bâtards, ravi de le
trouver sur ce chapitre tout autre que j'avais lieu de le croire. Nous
nous y étendîmes un peu avec ouverture et une secrète admiration en
moi-même de tout ce que cachent les replis du cœur d'un véritable
courtisan. Ensuite je lui dis qu'étant entièrement de son avis sur le
futur, je croyais pouvoir n'en être pas sur le présent, parce que, ce
qui était fait ne subsistant pas, il ne fallait pas compter qu'une
confirmation de plus ou de moins fût le salut ou la ruine de rangs de
cette nature\,; que si dans la suite ils se pouvaient renverser,
l'article de l'édit dont je lui parlais ne serait pas plus considérable
que les déclarations enregistrées qui les regardaient expressément, ni
que leur possession\,; que cet article, regardé alors du même œil, et
d'un œil sain, serait détaché de l'édit sans en altérer le corps, dont
la disposition en soi juste conserverait toute sa force et ne blessait
personne\,; et que nous pouvions aisément compter sur ce crédit\,; si
nous en avions assez pour réussir dans une chose aussi considérable que
de remettre les bâtards à raison, et au rang de leur ancienneté parmi
nous\,; que si, au contraire, ils demeuraient ce qu'ils ont été faits,
ce serait un assez grand malheur pour nous, pour ne pas y vouloir
joindre celui de nous priver d'un édit aussi avantageux pour tout le
reste, dont je lui fis sentir toute l'importance. Ce raisonnement
l'ébranla, et il s'y rendit le lendemain.

Je ne voulus point passer outre sans obtenir du chancelier la liberté de
m'ouvrir au maréchal de Boufflers, que je regardais avec une tendresse
et un respect de fils à père, et qui vivait avec moi, depuis bien des
années, dans la plus entière confiance. Le chancelier y consentit, et je
persuadai ce maréchal par le même raisonnement qui avait emporté
l'autre. Après cela, il fut question d'entamer l'affaire. Le comment fut
résolu d'un côté entre Boufflers, Harcourt et moi, qui seuls des
opposants à d'Antin en avions le secret\,; de l'autre, entre Chevreuse
et d'Antin, et le chancelier au milieu de nous, qui nous servait
là-dessus de lien, sans nous rien communiquer d'un côté à l'autre. Ce
comment fut\,: qu'il fallait s'y prendre par la demande qu'Harcourt
avait faite pour son duché, et à ce propos remettre l'ancien projet sus.
Harcourt guéri vit le chancelier, et parla au roi comme pour fortifier
sa demande de cet ancien projet dont il avait ouï parler confusément. Le
roi lui dit qu'en effet il y en avait eu un, et d'en parler au
chancelier et au duc de Chevreuse qui tous deux s'en devaient souvenir.
Le roi, aussitôt après, parla au chancelier de cet ancien projet, avec
surprise et chagrin de ce que quelques ducs en avaient eu connaissance,
puisque Harcourt lui en avait parlé. Le chancelier le fit souvenir que
par son ordre le duc de Chevreuse et feu M. de Luxembourg en avaient eu
part, d'où cela avait pu se répandre à quelques autres. Le roi, contenté
là-dessus, demanda au chancelier s'il en avait encore quelque chose\,;
et sur ce qu'il lui dit en avoir conservé soigneusement tous les
papiers, il en reçut ordre de les revoir pour lui en pouvoir rendre
compte. On en était là lorsque la semaine sainte sépara la compagnie,
qui fut suivie de celle de Pâques, et tout de suite de la maladie et de
la mort de Monseigneur, sur laquelle il nous parut indécent de commencer
nos plaidoiries, que nous remîmes à un peu d'éloignement, de concert
avec d'Antin et le premier président. Je prendrai cet intervalle pour
exposer courtement l'intérêt du duc de Chevreuse qui prétendait en avoir
deux, l'un et l'autre parfaitement pitoyables.

Sans s'étendre sur la prodigieuse fortune des Luynes ni sur leur
généalogie, tout le monde sait que MM. de Luynes, Brantes et
Cadenet\footnote{Les trois frères dont parle Saint-Simon étaient\,:
  Charles d'Albert, duc de Luynes, Honoré d'Albert, seigneur de Cadenet,
  et Léon d'Albert, seigneur de Brantes au comtat Venaissin et non de
  Brancas, comme on l'a imprimé dans les précédentes éditions.} étaient
frères, que l'aîné fut duc et pair de Luynes et connétable\,; que
Brantes fut duc et pair de Piney-Luxembourg par son mariage, dont il a
été amplement parlé en son lieu sur le procès de préséance prétendue par
le maréchal-duc de Luxembourg\,; et que Cadenet, ayant épousé
l'héritière d'Ailly, fut fait duc et pair de Chaulnes, étant déjà
maréchal de France. Il résulte de là qu'il était oncle du duc de Luynes,
et grand-oncle du duc de Chevreuse. Cette érection est de mars 1621,
huit mois avant la mort du connétable. M. de Chaulnes laissa deux fils.
L'aîné, gendre du premier maréchal de Villeroy, mourut sans enfants. Son
frère cadet devint ainsi duc de Chaulnes. Il fut célèbre par sa capacité
dans ses diverses ambassades, gouverneur de Bretagne, puis de Guyenne,
et il a été souvent fait mention de lui ici en divers endroits. Il était
donc cousin germain du duc de Luynes, père du duc de Chevreuse. Lorsque
ce dernier épousa la fille aînée de M. Colbert, au commencement de 1667,
M. de Chaulnes fit donation de tous ses biens au second mâle qui
naîtrait de ce mariage, au cas qu'il n'eût point d'enfants. Le cas
arriva en 1698, et le vidame d'Amiens, second fils du duc de Chevreuse,
hérita des biens de M. de Chaulnes fort chargé de dettes, dont il ne
s'était pas soucié de débarrasser son héritier, et le duché de Chaulnes
fut éteint. M. de Chevreuse était petit-fils du connétable, et ne venait
point du premier duc de Chaulnes, le duché de Chaulnes n'était que pour
l'impétrant et les mâles issus de lui, aucun autre n'y était appelé,
rien donc de plus manifeste que son extinction à faute d'hoirs mâles
issus par mâles de l'impétrant. M. de Chevreuse de plus était
personnellement exclu des biens du dernier duc de Chaulnes par son
propre contrat de mariage, qui étaient donnés au second fils qu'il
aurait, tellement que, à toute sorte de titres on ne peut concevoir quel
pouvait être le fondement de M. de Chevreuse de prétendre pour lui-même,
et aussi pour son second fils, la dignité de Chaulnes, dont lui ne
pouvait posséder le duché, et auquel lui et ses enfants n'étaient point
appelés, ni sortis du premier duc de Chaulnes. À force d'esprit et de
désir, d'interprétations sans bornes des termes de \emph{successeurs et
ayants cause} employés dans l'érection de Chaulnes, comme en toutes les
autres\,; par des raisonnements subtils, forcés, faux\,; à force
d'inductions multipliées et de sophismes entortillés, M. de Chevreuse,
dupe de son cœur et de son trop d'esprit et d'habileté, se persuada
premièrement à lui-même qu'il avait droit, et son second fils après lui,
et voulut après en persuader les autres.

Sur Chevreuse, voici le fait\,: cette terre fut érigée en faveur du
dernier fils de M. de Guise, tué aux derniers états de Blois en décembre
1588. Ce dernier fils, si connu sous le nom de duc de Chevreuse, le fut,
comme on dit improprement, à brevet, depuis 1612, que l'érection fut
faite pour lui et ses descendants mâles, jusqu'en 1627, que ce
duché-pairie fut enregistré. Ce duc de Chevreuse épousa Marie de Rohan,
veuve du connétable de Luynes, et mère du duc de Luynes père du duc de
Chevreuse dont il s'agit\,; et c'est cette M\textsuperscript{me} de
Chevreuse qui a fait tant de figure et de bruit, surtout dans les
troubles de la minorité de Louis XIV. Elle n'eut que deux filles du
Lorrain, dont aucune ne fut mariée\,; Elle survécut à ce second mari, et
eut le duché de Chevreuse pour ses reprises, et elle le donna au duc de
Luynes, son fils du premier lit. Le duc de Luynes le donna en mariage à
son fils, qui, par le crédit de Colbert, son beau-père, obtint une
nouvelle érection, en sa faveur, de Chevreuse en duché sans pairie, qui
fut vérifiée tout de suite. De prétendre de là la pairie et l'ancienneté
de M. de Chevreuse-Lorraine, mieux encore l'ancienneté de l'érection en
duché sans pairie enregistrée en 1555 pour le cardinal Charles de
Lorraine, qui fut éteint par sa mort, c'est ce qui est inconcevable.

On ferait un volume des absurdités de ces chimères. Cependant ce furent
ces chimères qui portèrent toujours M. de Chevreuse du côté de toutes
celles qui se présentèrent, et sinon à prendre parti pour elles à
découvert et en jonction, à demeurer au moins neutre en apparence, et
leur fauteur et défenseur en effet.

J'avais vécu avec lui dans la confiance et l'amitié la plus intime et la
plus réciproque. Il n'ignorait donc pas que l'intérêt de la dignité en
général, et celui de mon rang en particulier, ne l'emportassent à cet
égard sur tout autre sentiment et sur toute autre considération\,; ainsi
il voulut essayer de me persuader, et n'oublia rien, en plusieurs
différents temps, pour m'emporter par toute la séduction de l'amitié et
celle du raisonnement joints ensemble.

Il me trouva inébranlable. Sur l'amitié, je lui dis que je serais
très-aise qu'il fît obtenir des lettres nouvelles à son second fils,
mais que je ne pouvais trahir ma dignité en connivant à un abus si
préjudiciable que serait celui d'une si vaste et si large succession de
dignité, telle qu'il la prétendait. Sur le raisonnement, je démêlai ses
sophismes, que je ne rendrai point ici, pour n'allonger point ce récit
d'absurdités si arides et si subtilisées, et inutiles puisque la
prétention n'osa se présenter en forme. Je dirai seulement, pour en
donner une idée, que je le poussai un jour entre autres d'absurdités en
absurdités, auxquelles son raisonnement le jetait nécessairement,
jusqu'au point de me soutenir qu'un duc et pair dont le duché serait
situé dans la même coutume où Chaulnes est situé, et qui aurait deux
fils, pourrait, de droit et sans aucune difficulté, ajuster les deux
partages, en sorte que l'aîné ayant pour la quantité de biens tous les
avantages de l'aînesse, le cadet serait néanmoins duc et pair à son
préjudice, en faisant tomber le duché-pairie dans son lot, sans que
l'aîné eût démérité ni qu'il pût l'empêcher. Quelquefois des
conséquences si grossières, dont il ne se pouvait tirer, lui donnaient
quelque sorte de honte\,; mais sa manière de raisonner, subtile au
dernier point, le réconfortait à son propre égard, l'empêchait de se
laisser aller à la droite et vraie raison, et le laissait en liberté de
poursuivre avec candeur la plus déplorable de toutes les thèses. Je
finis avec lui par lui dire qu'il était inutile de disputer davantage
là-dessus\,; que, s'il entreprenait ce procès, il devait compter de me
trouver contre lui de toutes mes forces, sans pour cela l'aimer moins\,;
et que la plus grande preuve que je lui en pusse donner était mon
souhait sincère qu'il réussît pour son second fils par des lettres
nouvelles. Cette marque d'amitié était en effet grande pour moi\,; et il
en sentit le prix, parce qu'il connaissoit parfaitement mon éloignement
extrême de notre multiplication, et l'extrême raison de cet éloignement.

Nous demeurâmes donc de la sorte muets sur Chaulnes, qu'il avait bien
plus à cœur que son ancienneté de Chevreuse qu'il ne regardait qu'en
éloignement, moi en garde avec lui sur Épernon, et lui refusant
quelquefois nettement toute réponse à ses questions là-dessus, mais, du
reste, aussi étroitement unis, et en confiance aussi entière, sur tout
ce qui ne touchait pas ces matières, que nous étions auparavant.

Quelque uns, car c'est trop peu de dire unis, que fussent en tout M. de
Chevreuse et M. de Beauvilliers, ce dernier était bien éloigné
d'approuver les chimères de son beau-frère\,; on l'a vu par le conseil
qu'il me donna, sans que je le lui demandasse, de m'opposer sagement,
mais fermement à la prétention d'Épernon, et par le même qu'il me dit
avoir donné à son frère, qui fut fidèlement des nôtres. Mais, par son
unité d'ailleurs avec M. de Chevreuse, il ne voulait pas le blâmer, et
se tenait là-dessus tellement à l'écart que, avec le plus qu'éloignement
qui était entre lui et le chancelier, il ne put être question que,
quoique sans aucun secret mien pour lui, je pusse lui parler du
règlement de ce dont il s'agissait. C'est où nous en étions lorsque,
après la mort de Monseigneur, il fut enfin temps de commencer nos
plaidoiries sur la prétention d'Épernon, ou de finir tout par le
règlement en forme de déclaration ou d'édit dont j'ai parlé.

Le duc de Chevreuse et M. d'Antin le désiraient passionnément par les
raisons que j'ai racontées, et je ne le désirais pas moins par celles
que j'ai rapportées. Ce secret, comme je l'ai dit, était renfermé entre
eux deux d'une part, les maréchaux de Boufflers et d'Harcourt et moi
d'autre part, et le chancelier\,; point milieu des deux côtés qui ne se
communiquaient que par lui\,; et à la fin se renferma uniquement entre
le chancelier et moi seul pour tout ce qu'il s'y fit. Le maréchal de
Boufflers s'en alla malade à Paris, dès que la revue des gardes du corps
fut faite\,; Harcourt partit assez tard pour Bourbonne, et de là pour le
Rhin, et on verra pourquoi je ne fus pas pressé de lui parler\,; d'Antin
et moi n'étions pas en mesure de nous entretenir d'affaires\,; le duc de
Chevreuse demeura le seul à qui je pusse parler, mais tellement en
général que je n'eus pas la liberté de lui avouer que j'eusse
connaissance du projet du premier président d'Harlay, moins encore de
tout ce qui se passait sur cette base. Tel était le secret que le
chancelier m'avait imposé, ne me laissant que la simple liberté de
parler en général à M. de Chevreuse, comme sachant bien qu'on pensait à
un règlement, comme le désirant, mais rien du tout au delà.

Nous étions à Marly. Ce séjour rendait tout lent et incommode, et me
faisait un contre-temps continuel. Le chancelier, passionné pour sa
maison de Pontchartrain, n'allait presque plus à Marly, et n'y venait
que pour les conseils. Du mercredi au samedi, il était à sa chère
campagne, l'autre partie à Versailles, pour être les matins au conseil à
Marly et s'en retourner dîner à Versailles. Le lundi, qui lui était
libre, il tenait le matin conseil des parties, et le sceau\footnote{Il a
  été question du conseil des parties, t. I\^{}er, p. 445. Le chancelier
  tenait le sceau avec des conseillers d'État et des maîtres des
  requêtes, et scellait après leur rapport les édits et déclarations,
  lettres d'anoblissement, de légitimation, etc. Il pouvait rejeter les
  ordonnances présentées au sceau, si elles étaient déclarées contraires
  aux lois du royaume. Voy. notes à la fin du volume.} l'après-dînée, de
sorte qu'il n'y avait presque que l'après-dînée du mardi d'accessible
chez lui à Versailles. Nous avions, lui et moi beaucoup à conférer,
ainsi tout était coupé et retardé, et nous jetait sans cesse dans les
lettres de l'un à l'autre. Les ducs de Charost et d'Humières étaient à
Paris\,; cela me sauvait du juste embarras d'avoir la bouche fermée pour
des amis intimes, dans un intérêt commun, et qui avaient le timon de
l'affaire d'Épernon, auxquels néanmoins il fallut bien tenir rigueur
jusqu'au bout.

D'Antin à la fin, informé par le chancelier de l'ordre qu'il avait reçu
du roi sur le projet ancien, après qu'Harcourt en eut parlé au roi,
seconda la chose par un trait hardi de raffiné courtisan. Il avait
embarqué son affaire par des protestations au roi qu'il ne lui demandait
pour toute grâce que la permission, qu'il ne refusait à personne, de
pousser son procès. Cela ne l'embarrassa point quand il lui convint de
changer de langage. Il dit au roi que son procès était indubitable, mais
cependant qu'il croyait que son crédit soutiendrait difficilement le
nôtre\,; que deux autres choses lui faisaient aussi beaucoup de peine\,:
la longueur qui le priverait d'une assiduité auprès de sa personne, qui
faisait tout son devoir et tout son bonheur\,; et une aigreur qui lui
attirerait tous les ducs, lui qui ne cherchait qu'à être bien avec tout
le monde\,; que, quelque bonne que fût son affaire, il avouait qu'il
aurait toujours à contre-cœur de devoir son élévation à la justice de sa
cause, au lieu de la recevoir de sa grâce et de sa libéralité, qui
serait la seule chose qui lui ferait plaisir\,; que ce plaisir même le
toucherait de telle sorte qu'il lui sacrifierait de tout son cœur toute
l'ancienneté qu'il avait lieu d'attendre, et qu'il se verrait avec cent
fois plus de joie le dernier pair par la bonté du roi, avec les bonnes
grâces des autres, que le second par l'heureuse issue de son procès\,;
que ce n'était pas, encore une fois, qu'il ne le crût indubitable\,;
qu'il arrivait encore de Paris, où il avait vu les meilleures têtes du
parlement, qui l'en avaient assuré (il mentait bien à son escient, comme
il l'a avoué depuis)\,; mais qu'il se déplaisait tellement en cette vie
de courses et d'éloignement d'auprès de lui\,; qu'il était si accoutumé
à ne rien tenir que de lui, {[}qu'{]} il osait le conjurer d'abréger
toutes ses peines, en lui donnant comme une grâce la dernière place
parmi les ducs et pairs, où il était persuadé que la seconde lui était
due. Cela dit en distance de plusieurs mois qu'il avait dit tout le
contraire pour enfourner son affaire, et dit dans un moment
d'ébranlement sur l'ancien projet de règlement, mit le roi au large de
contenter tout le monde, et en chemin d'être conduit où on voulait. Il
ne répondit rien de précis à d'Antin\,; mais il ne le fit point souvenir
non plus qu'il l'avait assuré d'abord qu'il ne lui demanderait point de
grâce\,; ensuite il lui parla de lui-même de cet ancien projet, à quoi
d'Antin, tout préparé, prit, de façon qu'il se fit ordonner de voir
là-dessus le duc de Chevreuse et le chancelier.

L'amorce prise, le chancelier représenta au roi qu'il était à propos de
suspendre les plaidoiries qui allaient commencer sur la prétention
d'Épernon, en cas qu'il voulût reprendre les anciens errements du
règlement\,; et, quoique le roi n'y fût pas encore résolu, il consentit
à la suspension. Le chancelier la fit aussitôt savoir au premier
président, aux gens du roi et aux parties. La surprise en fut grande
parmi les opposants à d'Antin et parmi leurs avocats. Ils ne savaient à
quoi attribuer ce coup d'autorité\,; ils ne doutèrent même pas que ce ne
fût un trait de favori inquiet de la face que son affaire avait prise.
Tout ce que je pus faire pour les rassurer, fut de dire aux ducs de
Charost et d'Humières de ne s'inquiéter point, et à nos avocats d'avoir
bon courage.

\hypertarget{chapitre-x.}{%
\chapter{CHAPITRE X.}\label{chapitre-x.}}

1711

~

{\textsc{Discussion du projet de règlement entre le chancelier et moi.}}
{\textsc{- Friponnerie insigne et ambitieuse du premier président
d'Harlay.}} {\textsc{- Apophtegme du premier maréchal de Villeroy.}}
{\textsc{- Je fais comprendre les ducs vérifiés en l'édit.}} {\textsc{-
L'amitié m'intéresse aux lettres nouvelles de Chaulnes, et le chancelier
s'y porte de bonne grâce.}} {\textsc{- Je l'y soutiens avec peine,
dépité qu'il devient des sophismes du duc de Chevreuse.}} {\textsc{- Le
chancelier travaille seul avec le roi sur le règlement.}} {\textsc{- Son
aversion des ducs et sa cause.}} {\textsc{- Scélératesse du premier
président d'Harlay sur le sacre et la propagation des bâtards.}}
{\textsc{- Je propose le très-faible dédommagement de la double séance
de pairs démis.}} {\textsc{- Le roi, uniquement pour son autorité,
favorable à M. de La Rochefoucauld contre moi.}} {\textsc{- Chaulnes
enfourné.}} {\textsc{- Mémoire uniquement portant sur l'autorité du roi,
qui me vaut la préséance sur M. de La Rochefoucauld.}} {\textsc{- Défaut
de foi et hommage\,; explication et nécessité de cet acte.}} {\textsc{-
Alternative ordonnée en attendant jugement, et commencée par la tirer au
sort.}} {\textsc{- Préjugés célèbres du roi en faveur de M. de
Saint-Simon.}} {\textsc{- Singulier procédé entre les ducs de
Saint-Simon et de La Rochefoucauld lors et à la suite de la réception au
parlement du premier.}} {\textsc{- Autre préjugé du roi tout récent en
faveur de M. de Saint-Simon.}} {\textsc{- L'autorité du roi favorable à
M. de Saint-Simon.}} {\textsc{- Enregistrement sauvage des lettres
d'érection de La Rochefoucauld.}} {\textsc{- Lettres de M. le duc de
Saint-Simon à M. le chancelier\,; de M. le chancelier à M. le duc de
Saint-Simon\,; de M. le duc de Saint-Simon à M. le chancelier.}}
{\textsc{- Éclaircissement de quelques endroits de mes lettres.}}
{\textsc{- Anecdote curieuse de l'enregistrement de La Rochefoucauld.}}

~

Alors il fut question, entre le chancelier et moi, d'en venir à un
sérieux examen de cet ancien projet du premier président d'Harlay, que
j'avais copié et noté, qui devait servir de base au règlement qu'on
voulait faire. Le premier article devint la première matière de
contestation\,: c'était celui des princes du sang, qui était vague, hors
d'œuvre, et qui ne disait rien. Par cela même, j'en craignais une
approbation implicite des usurpations à notre égard, dont M. le prince
de Conti convenait de si bonne foi du nombre et de l'injustice\,; et
sans m'expliquer là-dessus avec le chancelier, j'insistai sur
l'inutilité, et dès là sur l'indécence d'un article qui ne réglait rien,
parce qu'il n'y avait rien alors à décider à cet égard. Le chancelier me
répondit qu'ayant nécessairement à parler des légitimés, on ne pouvait
passer sous silence les légitimes. Je ne voyais point cette nécessité.
Il ne s'agissait de rien sur les princes du sang\,: il n'y avait point
de concessions à confirmer pour eux comme pour les bâtards, puisqu'on
voulait prendre cette occasion de le faire\,; mais cette bienséance de
ne pas parler de ceux-ci sans avoir d'abord fait mention de ceux-là
parut au chancelier une raison péremptoire. Comme, dans le fait, ce
premier article n'énonçait rien, je ne m'opiniâtrai pas trop\,; mais
j'essayai de faire supprimer le second, qui portait la confirmation dont
je viens de parler, et avec lequel le premier tombait de soi-même. Mais
le chancelier, ferme sur son principe que cet article seul serait le
chausse-pied du règlement, m'ôta toute espérance qu'il pût être
supprimé, et je me tournai à le faire dresser, en sorte qu'il ne donnât
pas au moins une force nouvelle à ce qui avait été fait pour les bâtards
et que la confirmation, puisqu'il en fallait passer par là, fût la plus
simple et la plus exténuée qu'il serait possible. Le troisième article
fut une ample matière. Harlay, par ce projet, ne songeait qu'à son
ambition. Il avait parole réitérée d'être chancelier pour ses bons
services aux bâtards. Le brillant de M. de Luxembourg, soutenu de la
faveur pleine de M. de Chevreuse, l'avait ébloui jusqu'à lui faire tenir
la partiale conduite qui le fit récuser dans cette affaire de préséance,
et qui nous fit rompre tous ouvertement avec lui. Il était lors au fort
de cette brouillerie, dans laquelle le duc de La Rochefoucauld se montra
des plus animés. Harlay le redouta pour les sceaux, et le voulut ramener
à soi par la même voie qui l'en avait aliéné. Il était bien au fait de
la question de préséance qui était entre lui et moi, et, sans faire
semblant d'y penser, il dressa ce troisième article pour m'étrangler,
sans que je m'en défiasse, et pour se raccommoder par là avec M. de La
Rochefoucauld. Comme cet article fut la matière de divers mouvements
auxquels il faudra revenir plus d'une fois, je passerai aux autres sans
m'arrêter maintenant à celui-ci, sinon sur ce qui ne me regarde pas en
particulier.

Je trouvais juste que les duchés ne fussent vérifiés qu'à Paris, cour
des pairs et le premier de tous les parlements\,; ce fut pour cela que,
sans la plus légère liaison avec les Brancas, je proposai ce qui se voit
dans la note sur cet article. Mais comme les choses se réglaient avec le
roi bien plus par goût que par principes, cela fut laissé à côté dès
qu'il ne fut plus question d'enregistrement, comme on verra dans la
suite. L'âge compris dans cet article forma une grande dispute entre le
chancelier et moi. La réception des pairs n'y avait jamais été
assujettie\,; je ne pouvais souffrir qu'elle la fût, et uniquement pour
servir de degré à la distinction sur eux des bâtards et des princes du
sang, qui tous ne peuvent nier, malgré toutes leurs usurpations, qu'ils
n'entrent au parlement que comme pairs, et, malgré toutes leurs
distinctions, comme pairs tels que tous les autres. La raison de l'âge
pour les gens de loi, et qui n'a rien de commun avec les pairs, fut par
moi déployée dans toute sa force.

Le malheur était que celui contre qui je disputais était juge et partie.
L'homme de loi, le magistrat blessé en lui de cette différence, se
sentit en situation de l'anéantir\,; il se garda bien d'en manquer
l'occasion si favorable, et, à faute de mieux, de ne pas mettre pour
l'âge les pairs à l'unisson des magistrats.

Le vieux maréchal de Villeroy disait avec un admirable sens qu'il
aimerait mieux pour soi un premier ministre son ennemi, mais homme de
qualité, qu'un bourgeois son ami. Je me trouvai ici dans le cas.

Le chancelier, qui m'en voulait détourner l'esprit, s'appuya tant qu'il
put de l'indécence et de l'inconvénient même quelquefois du pouvoir
d'opiner dans les plus grandes affaires, avant l'âge sagement prescrit
pour pouvoir disposer des siennes particulières. J'opposai l'extrême
rareté de ces occasions de juger pour les pairs, et le continuel usage
des dispenses d'âge des magistrats qui jugent tous les jours de leur
vie. J'eus beau me récrier sur l'iniquité de la disparité d'avec les
princes du sang et les bâtards, et la parité entière avec les
magistrats, jusqu'alors inouïe\,; je parlais à un sourd enveloppé de sa
robe, qui lui était plus chère que justice, raison ni amitié, et il
fallut passer aux autres articles.

J'eus bon marché du quatrième et cinquième, qui regardaient les ayants
cause et les duchés femelles. Ce dédommagement était bien mince des
trois premiers, mais le contraire aurait été fort nuisible dans un temps
si malheureux\,; et si nous n'y gagnâmes rien, au moins fûmes-nous à
l'abri d'y perdre. Il n'y avait que les audiences du parlement de Paris
d'exprimées\,; je craignis les suites d'une omission de cette nature,
sur l'exemple de celle qui, par la faute des pairs de ces temps-là, nous
a par la suite exclus du conseil des parties. Je fis donc ajouter, et
sans peine, le conseil, c'est-à-dire les procès par écrit, et les autres
parlements à celui de Paris.

J'essayai après d'y faire cesser les ineptes difficultés que font
quelques autres parlements sur la manière d'entrer et de sortir de
séance, et de faire ajouter un mot qui les fixât tous à celles dont les
pairs entrent et sortent de séance au parlement de Paris, le plus ancien
et le modèle de tous les autres. Mais le magistrat se trouva encore ici
avec sa précieuse robe, qui me répondit que c'était des choses
étrangères à la matière dont il s'agissait dans ce règlement, et que le
roi ne pouvait entrer dans ces vétilles, terme très-familier à ceux qui
n'ont rien de fâcheux à essuyer. Ainsi, en choses de parlement, un homme
de robe, en celles qui regardaient les princes du sang ou les bâtards,
un courtisan, était ce que j'avais en tête, et avec qui lutter trop
inégalement. Ces deux articles et les deux suivants n'avaient rien qui
touchât aux princes du sang, aux bâtards, ni à la robe. C'étaient
néanmoins les importants pour finir tous les procès de préséance, et
nous garantir des pluies de la faveur et des prétentions de toute espèce
qui renversent tout droit et tout ordre dans la dignité\,; aussi le
chancelier m'en fit-il bon marché. Nous les tournâmes tout aussi
avantageusement que je voulus, et mieux encore, non-seulement sur
l'ayant cause, mais sur les femelles, où le gendre fut exclu de
l'ancienneté du beau-père. Ce furent deux grands points. Le sixième fut
extrêmement discuté, non par la fantaisie du chancelier, mais par la
difficulté de sa nature. Ma pensée était que la faculté de substituer
était insuffisante des ducs indifférents, mal entendus ou mal dans leurs
affaires, et mon dessein était de conserver la dignité et sa glèbe
perpétuellement à tous les appelés, de les dérober à l'incurie de leurs
auteurs jusqu'à extinction de race, et tout à la fois de procurer aux
ducs de quoi vivre au moins dans la plus grande décadence de leurs
affaires, avec un lustre à leur dignité, de la solidité duquel ils
tireraient leur subsistance. Il faut dire, à l'honneur du chancelier,
qu'il entra parfaitement dans ces vues, et qu'il n'y eut que les
obstacles insurmontables de l'exécution, par les difficultés de la chose
en elle-même, et qui ne se purent résoudre, et qui empêchèrent la
substitution de droit par l'érection, et qui la réduisirent à la simple
faculté aux ducs de la faire, à laquelle nous donnâmes toute l'étendue
possible, pour remplir toutes les vues que je viens d'expliquer. Le
septième article fut encore extrêmement discuté. Je voulais un denier
plus faible\,; l'équité en exigea un plus fort, et je m'y rendis. Le
chancelier alla plus loin que moi, il ne faut pas lui en dérober
l'honneur. Je ne pensais qu'au premier mâle en ordre de succéder, le
chancelier étendit de lui-même la faculté du remboursement forcé de la
femelle à tout mâle appelé à la dignité, chacun en son ordre, au refus
par incurie ou par impossibilité des mâles avant appelés, ce qui fut une
extension très-avantageuse pour la conservation des dignités dans la
descendance de l'impétrant. Le huitième article passa sans difficulté
entre nous deux, sinon que je m'opposai tellement à la forme d'un arrêt
du conseil pour le renvoi des causes de prétentions ducales au
parlement, que j'obtins que cette forme d'arrêt du conseil serait omise.
Ma raison fut que les magistrats du conseil ne sont pas juges compétents
de ces matières. L'article neuvième allait tout seul. La prétention de
l'ancienne érection de Piney était éteinte par les articles précédents.
Le rang de sa réérection de 1662, faite pour le feu maréchal de
Luxembourg, fut établi par celui-ci\,; et en même temps l'érection
nouvelle et le rang nouveau de d'Antin y fut compris. Le premier avait
été le motif de l'ancien projet, le second de le remettre sur le tapis.
Il finissait ces deux affaires, et il était devenu épineux de faire
juridiquement déclarer Piney éteint de la première et de la seconde
érection, depuis le monstrueux arrêt de l'inique Maisons, qui a été
expliqué en son temps, chose néanmoins à laquelle nous allions donner
tous nos soins, si ceci ne nous en eût ôté la peine.

Jusqu'ici il ne s'agissait du tout que des pairs, et l'ancien projet ne
faisait aucune mention des ducs simplement vérifiés ou héréditaires,
comme on les appelle mal à propos, puisque les pairs le sont aussi.
L'équité, aiguisée de l'intérêt de la maison de M\textsuperscript{me} de
Saint-Simon, me fit penser à eux, par celui de l'aîné de sa maison et
son cousin germain, de son frère et de son beau-frère, tous trois ducs
vérifiés. Je proposai donc au chancelier d'ajouter à la fin de l'édit un
article qui y comprît les ducs simplement vérifiés, autant qu'ils en
étaient susceptibles. Il ne m'en fît aucune difficulté.

Tout cela convenu entre lui et moi, je vins à mon fait particulier de
l'ancienneté à régler par la date de l'enregistrement des lettres, comme
M. de La Rochefoucauld le prétendait contre moi, et comme le portait
l'ancien projet du premier président d'Harlay, pour lui complaire et se
le rapprocher, ou, comme je le prétendais, par la date de la réception
de l'impétrant au parlement. Je diffère à expliquer plus bas les raisons
de part et d'autre, pour ne pas interrompre la suite du récit du
règlement. Il suffit ici de dire que je convainquis le chancelier de mon
droit. Je mis ensuite sur le tapis ce qui regardait M. de Chevreuse.

C'était un des grands épisodes. De l'ancienneté de Chevreuse-Lorraine,
ce n'était pas le plus pressé\,; Luynes était plus ancien. Le point
pressant était Chaulnes. Il n'existait plus depuis 1698, que le dernier
duc de Chaulnes était mort\,; et le vidame d'Amiens, second fils de M.
de Chevreuse, se morfondait cependant, et, suivant M. son père,
souffrait, et lui aussi, une grande injustice, sans toutefois que ni
l'un ni l'autre eussent osé encore se présenter juridiquement à
recueillir cette dignité. Le chancelier et moi convînmes bientôt que
cette prétention ne pouvait se soutenir. Alors je lui dis que c'était là
une occasion essentielle de se souvenir de l'amitié personnelle qui
avait toujours été entre M. de Chevreuse et lui, et je l'exhortai à le
servir en cette occasion si importante, pour obtenir à son second fils
des lettres nouvelles avec un nouveau rang. Le chancelier ne se fit
point prier, et me répondit d'un air ouvert qu'il était ravi de me voir
dans ce sentiment, et que cela même le mettait là-dessus à son aise.
Nous discourûmes de la manière de s'y prendre\,; nous convînmes que
l'unique était de ne pas faire au roi la prétention si mauvaise, afin
d'y laisser une queue d'équité, de la terminer par une nouvelle
érection, à quoi le chancelier me promit de faire tout son possible.

M\textsuperscript{me} de Saint-Simon avait quitté Marly avec la
fièvre\,; elle était demeurée depuis à Paris assez incommodée, et je l'y
allais voir le plus souvent que je pouvais. Le duc de Chevreuse y était
aussi, qui, fort mal à propos pour ses vues de Chaulnes, avait esquivé
ce Marly, dont le roi n'était pas trop content\,; car à lui qui était
réellement ministre, bien qu'incognito, il lui fallait des permissions
pour ces absences, que le roi ne lui donnait pas volontiers.
L'inquiétude le prit\,; il me vint trouver à Paris\,: il se mit à me
haranguer avec ses longueurs ordinaires\,; moi à lui couper court que sa
prétention de Chaulnes était insoutenable, et n'aurait pas un plus
ardent adversaire que moi, s'il se mettait à la plaider. J'ajoutai tout
de suite que, pour lui montrer la vérité de mon amitié, je lui
promettais tous bons offices s'il en avait besoin pour des lettres
nouvelles\,; et je lui dis ce qui s'était passé là-dessus entre le
chancelier et moi, mais sans un seul mot qui approchât du règlement.
Cette franchise le charma\,; il me fit mille remercîments, et me pria de
soutenir le chancelier dans ce bon dessein. Dès qu'il m'eut quitté, il
se mit à travailler à un mémoire, qui ne valut rien, parce que sa
prétention était sans aucune sorte de fondement. Il l'envoya au
chancelier. Les raisonnements en étaient tellement tirés à l'alambic
qu'ils l'impatientèrent, et plus encore une conversation qu'il eut avec
lui à Versailles, où il l'alla trouver, tellement qu'il fut grand besoin
que je remisse le chancelier de cette mauvaise humeur qu'il avait prise.
Je n'en voulus pas donner l'inquiétude à M. de Chevreuse, quoiqu'il s'en
fût un peu aperçu.

Le chancelier cependant travailla avec le roi. Ce tête-à-tête non
accoutumé réveilla tout le monde, qui, joignant à cette singularité la
surséance arrivée à notre affaire de d'Antin, ne douta pas qu'il n'y en
fût question. Le chancelier proposa au roi de communiquer le projet de
règlement à quelques ducs, et de travailler là-dessus, avec eux,
puisqu'il s'agissait de faire une loi à eux si importante. Le roi,
hérissé de la proposition, répondit avec un mépris assez juste sur leur
capacité en affaires, et la difficulté d'en trouver quelques-uns qui
entendissent celles-là assez bien. Le chancelier lui en nomma
quelques-uns, moi entre autres, et en prit occasion de faire valoir son
amitié sans la montrer trop. Il insista même assez ferme\,; mais le roi
demeura inébranlable en ses usages, ses préjugés, et ses ombrages
\emph{mazarins} d'autorité qui l'animaient contre les ducs, dont la
dignité lui était odieuse par sa grandeur intrinsèque, indépendante par
sa nature des accidents étrangers. Elle lui faisait toujours peur et
peine par les impressions que ce premier ministre italien lui en avait
données pour son intérêt particulier, et lui avait sans cesse fait
inspirer par la reine mère, ce qui le rendit si constamment contraire,
jusqu'à franchir les injustices les plus senties, et même avouées en
bien des occasions.

Le projet, tel que le chancelier et moi {[}en{]} étions convenus fut par
lui communiqué au premier président et au procureur général. Pelletier,
qui n'était pas grand clerc, ne fit que le voir à sa campagne où il
était allé, et le renvoya aussitôt. D'Aguesseau écrivit un long verbiage
qui, pour en dire le vrai, ne signifiait rien. Le chancelier, content de
sa communication de bienséance, poussa sa pointe.

M. de Chevreuse, en éveil sur ce travail du roi avec le chancelier seul,
redoubla d'un mémoire à celui-ci. Ce mémoire n'était point correct dans
ses principes, peu droit dans ses raisonnements qui tous conduisaient à
ses fins, comme le chancelier me le manda avec dégoût et même avec
amertume. Il ajouta qu'en le lui donnant M. de Chevreuse lui avait dit,
pour le faire valoir, qu'il m'avait fait presque convenir de tout. Il
n'en était rien, et je le sus bien dire à l'un et à l'autre. Quelque
étrange qu'un semblable allégué doive paraître à qui n'a pas connu le
duc de Chevreuse, je suis convaincu qu'il se trompait soi-même, et qu'à
force de désirer, de se figurer, de se persuader, il croyait tout ce
qu'il souhaitait et tout ce dont il se persuadait de la chose, de
lui-même et des autres. Toutefois je ne pus m'empêcher de lui en parler
avec force, mais en même temps je soutins le chancelier dépité, et avec
travail, qui voulait laisser faire M. de Chevreuse, l'abandonner à ses
sophismes et à tout ce qu'il en pourrait tirer sans autre secours pour
son affaire.

Ce qui le gâtait encore avec le chancelier, c'est que, se doutant bien,
qu'il était question d'un règlement, puisqu'il en avait parlé lui-même,
il le tracassait pour pénétrer ses sentiments, et encore pour avoir
communication de l'ancien projet qu'il avait vu dans le temps que le
premier président d'Harlay le fit, qu'il jugeait bien devoir servir de
base à ce qu'on allait faire, mais dont il ne lui restait rien qu'en
gros et imparfaitement dans la mémoire. Or le chancelier s'en trouvait
d'autant plus importuné qu'il ne voulut ni lui communiquer l'ancien
projet, ni moins encore lui laisser rien entrevoir de ce qui entrerait,
ni de ce qu'il pensait devoir entrer dans ce qu'on voulait faire.

Je n'étais pas moi-même moins circonvenu toutes les fois que je venais à
Paris, et je n'avais pas peu à me défendre d'un ami si intime, si
supérieur en âge et en situation, et si adroit à pomper, dans la pensée
que le chancelier me communiquait tout, et ne me cachait rien. Il eut
beau faire, jamais il ne put rien tirer de moi que des avis sur son
fait, et des services très-empressés et très-constants auprès du
chancelier, qui ne furent pas inutiles.

Le chancelier avait travaillé avec le roi trois fois tête à tête.
J'appris de lui, après ce troisième travail, que le roi s'était souvenu
de deux articles de l'ancien projet du premier président d'Harlay, que
je n'avais point vus dans la copie que le chancelier m'avait
communiquée\,: c'étaient les deux derniers coups de foudre. Le premier
était la représentation de six anciens pairs au sacre, attribuée,
exclusivement aux pairs\footnote{À l'exclusion des pairs.}, à tous les
princes du sang, à leur défaut aux légitimés pairs, sans que les autres
pairs y pussent être admis qu'à faute de nombre des uns et des autres.
L'autre était l'attribution, aux légitimés qui auraient plusieurs
duchés-pairies, de les partager entre leurs enfants mâles qui
deviendraient ainsi ducs et pairs et feraient autant de souches de ducs
et pairs, avec les rangs, honneurs et priviléges maintenant accordés aux
légitimés, au-dessus de tous autres pairs plus anciens qu'eux.

Ce que je sentis à deux nouveautés tout à la fois si inimaginables et si
destructives serait difficile à rendre. Je disputai contre le chancelier
qui me montra l'article du sacre dans la minute de cet exécrable Harlay,
qu'il n'avait, disait-il, recouvrée que depuis peu. Je lui remontrai
l'antiquité de la fonction des pairs égale à celle du sacre même, et non
interrompue jusqu'à présent\,; qu'il n'y en avait jamais eu où les
pairs, quand il s'en trouvait, n'eussent servi, lors même qu'il y avait
plus de princes du sang qu'il n'en fallait pour cet auguste service. Je
le fis souvenir de la préférence des pairs par ancienneté sur les
princes du sang, aux sacres d'Henri II et de ses fils. Je lui démontrai
que cette loi si juste par laquelle Henri III fait tous les princes du
sang pairs à titre de naissance, et leur donne la préséance sur tous les
autres pairs, n'avait fait aucune altération à leurs fonctions du sacre.
Je lui expliquai le fond, la raison, l'esprit de cette grande cérémonie,
par l'histoire, et tout ce qu'elle a de figuratif, dont il n'est pas
possible de convenir\footnote{Le manuscrit porte \emph{convenir} et non
  \emph{disconvenir}, comme on l'a imprimé dans les précédentes
  éditions.}.

Je lui rendis évident le peu de solidité d'un couronnement fait par tous
les parents masculins d'un roi héréditaire, et d'une monarchie qui est
l'unique soumise à la loi salique. Je lui fis honte de l'infamie d'une
représentation si éminente par des bâtards, et à titre de bâtards. Enfin
je n'oubliai rien de ce que la douleur la plus pathétique et
l'instruction la plus puissamment réveillée me purent suggérer.

Mais ce fut là où je trouvai tout à la fois le magistrat et le
courtisan, contre lequel j'eus enfin peine à me retenir. Il me protesta
que ce souvenir était venu du roi tout seul, et qu'il n'avait pu le
détourner de cet article non plus que de l'autre, à quoi je pense bien
qu'il n'épuisa pas ses efforts. J'essayai de le frapper par le nombre et
le poids de nos pertes. Voyant enfin que je ne gagnais rien, je me
tournai à le prier de faire arrêter le projet de règlement. Ce fut là
que les grands coups se ruèrent de part et d'autre. Il ne put souffrir
cette proposition, ni moi de m'en désister. Je lui soutins que cette
plaie portait droit au cœur, et qu'en attaquant jusqu'à cet excès tout
ce que la dignité avait de plus ancien, de plus auguste, de plus
inhérent, rien ne pouvait être bon. Il étala les avantages de tous les
procès retranchés par les articles des ayant cause et des femelles, et
de ceux des substitutions et du rachat forcé des héritières femelles. Je
convins de l'avantage de ces articles\,; mais j'ajoutai que
non-seulement ceux-là, mais qu'un règlement composé par moi-même en
pleine liberté, et tout à mon gré, mais à condition de cet article du
sacre, ne nous pourrait être que parfaitement odieux. Je le pressai de
reparler au roi là-dessus, qui avait souvent dit lui-même que, outre des
princes du sang, il fallait des pairs pour représenter les anciens au
sacre, qui pouvait être ramené sur une chose qu'il ne pouvait jamais
voir. Le chancelier fut ébranlé\,; il me promit même toute assistance\,;
mais j'eus lieu de croire, par une réponse que j'en reçus le lendemain à
une lettre dont j'avais redoublé mon instance, que l'homme de robe, bien
tranquille sur une énormité qui ne la touchait pas, avait laissé faire
le roi en courtisan qui veut plaire, et qui sent bien que ce n'est pas à
ses dépens.

Cet article, plutôt contraint par l'heure qu'épuisé, nous vînmes au
second. Il est si étrange, si monstrueux et si surprenant, qu'il est
inutile de s'y étendre après l'avoir expliqué. Il avait été suggéré par
le duc du Maine, à qui le roi parla d'abord de ce dont il était
question, et qui ne s'épargna pas à en profiter. Je m'étendis avec le
chancelier sur un pouvoir donné à des bâtards comme tels, à exercer
indépendamment du roi sur un privilége, à raison de dignité multipliée
dont ils sauraient bien ne pas manquer, qui revenait pour l'effet au
même que l'édit d'Henri III qui avait fait les princes du sang pairs
nés, en un mot sur un rang monstrueux qui en nombre comme en choses
n'aurait plus de bornes\,; Finalement je me tus, voyant bien que ce qui
était imaginé, demandé et accordé pour le duc du Maine, en faveur de sa
bâtardise, ne pouvait plus être abandonné par le roi, qui en faisait son
idole d'amour et d'orgueil. Je me rabattis donc à quelque sorte de
dédommagement. Tous étaient bien-difficiles à tirer du roi si jaloux
d'une dignité qu'il avait continuellement mutilée, et qui
s'effaroucherait de toute restitution, surtout si elle touchait autrui.
Cette considération me porta à en proposer un très-médiocre, et qui ne
portait sur personne\,: ce fut la double séance au parlement des pairs
démis, avec leurs fils pairs par leur démission.

Je fis remarquer au chancelier que cette nouveauté n'était aux dépens de
personne, que les pairs démis ne se privaient par leur démission que de
la séance au parlement\,; que cela ne changeait donc rien pour eux, ni
pour leur rang, ancienneté, préséance et honneurs en pas un autre lieu,
puisque leur démission ne les excluait d'aucune cérémonie, ni de la
jouissance partout de ce qu'ils avaient avant leur démission\,; que les
ducs vérifiés ne perdaient rien à la leur, parce qu'il n'y avait à y
perdre que l'entrée au parlement, qu'ils n'ont pas\,; que ce ne serait
même rien de nouveau en soi dans le parlement, puisque les présidents à
mortier qui cèdent leurs charges à leurs fils n'y sont privés de rien,
sinon de pouvoir présider en chef, mais jouissent d'ailleurs de leur
séance et de leur ancienneté, et de leur voix délibérative\,; que la
même chose se pouvait faire en faveur des pairs si on voulait conserver
un air d'apparence, sinon de justice, lorsqu'on s'en éloignait à leur
égard d'une manière si violente et si inouïe. Le chancelier contesta peu
là-dessus. Il ne laissa pas d'alléguer que le père et le fils ne
pouvaient siéger ensemble. Je lui demandai pourquoi cette exclusion,
tandis qu'elle n'était pas pour la robe\,; qu'en cela seulement il était
juste qu'il en fût des pairs père et fils comme des magistrats père et
fils\,; qu'étant de même avis, leurs voix ne seraient comptées que pour
une\,; et que d'avis différent, elle serait caduque. J'ajoutai que ce
n'était qu'une extension à tous d'un droit qui appartenait à
quelques-uns\,; que MM. de Richelieu, Bouillon et Mazarin avaient chacun
deux duchés-pairies\,; que les deux derniers s'étaient démis de l'une
des deux\,; que par conséquent c'étaient deux pères et deux fils
siégeant ensemble au parlement, toutes fois et quantes bon leur semblait
et semblerait, sans moyen aucun de l'empêcher, et sans qu'on se fut
avisé jusqu'à cette heure d'y trouver le moindre inconvénient. Le
chancelier n'eut point de réplique à me faire\,; il avoua la proposition
très-raisonnable, et me promit de faire tout de son mieux pour la faire
passer.

Ce point achevé, il me dit que le roi n'avait pu goûter mes raisons
contre M. de La Rochefoucauld, quoi qu'il eût pu lui dire\,; que la
réplique du roi avait été que son autorité y serait intéressée et qu'il
était demeuré fermé là-dessus.

Un homme moins sensible que je ne l'étais en aurait eu sa suffisance de
ces trois points dans une même conversation. Ce dernier néanmoins, qui
étant seul m'eût extrêmement touché, ne me fit pas grande impression
tant celle des deux autres me fut douloureuse. Elles attaquaient tant,
et mon affaire ne touchait presque pas la dignité. Je ne laissai pas de
disputer ma cause avec le chancelier, qui pour toute réponse convint et
haussa les épaules, m'avoua qu'il était pour moi, qu'il avait combattu
le roi tant qu'il lui avait été possible, que les réponses du roi sur le
fond et sur le droit avaient été nulles, et qu'il n'avait répliqué que
par le seul intérêt de son autorité. Je priai le chancelier de ne me pas
tenir pour battu, ni lui non plus, en portant ma cause\,; je lui dis
que, dès qu'il la trouvait bonne par le mérite du fond, du droit, des
règles et de la justice, qui ne touchaient point celles du roi,
affranchi d'avoir à le persuader lui, puisque de son aveu il l'était,
j'allais me tourner à persuader le roi sur son autorité comme je
pourrais, par un autre mémoire que je prévoyais bien qu'il ne trouverait
pas bon, mais qu'il se souvînt du premier qu'il avait trouvé tel, et
qu'il se servît de celui que j'allais faire en faveur de l'autre,
puisque ce n'était que par là que je pouvais réussir.

Nous finîmes par l'article de Chaulnes qu'il me dit avoir enfourné assez
heureusement. Après cet entretien dans son cabinet à Versailles, qui
dura plus de trois heures, je m'en allai dans la situation de cœur et
d'esprit qu'il est aisé d'imaginer. En arrivant chez moi, je me mis à
travailler au mémoire dont il vient d'être parlé. J'étais fâché\,; je le
brusquai en deux heures pour l'envoyer au chancelier aussitôt, qui
devait travailler incessamment avec le roi, et essayer avec ce nouveau
secours de remettre ma prétention à flot. L'adresse réussit\,; elle est
telle que je l'insère ici plutôt que dans les Pièces. C'est un mémoire
curieux pour bien connaître Louis XIV qui, uniquement sur cette pièce,
me donna partout la préséance sur M. de La Rochefoucauld. La voici.

«\,On n'a pas dessein d'entrer dans le fond de la question par ce
mémoire. On s'y propose seulement de faire très-succinctement l'histoire
de ce qui s'est passé entre les titulaires de ces deux duchés-pairies,
depuis leur érection jusqu'à présent, et d'y ajouter dans les endroits
nécessaires de courtes réflexions, d'où on espère qu'il résultera avec
évidence que cette question n'en fut jamais une, et que, si la
considération de M. de La Rochefoucauld l'a tenue jusqu'à présent sans
être jugée, tous les préjugés même du roi lui ont été manifestement et
uniformément contraires. Il est seulement bon de représenter en un mot
que, s'il arrivait qu'il fût besoin d'une plus ample instruction, et
d'entrer dans le fond de l'affaire, on est prêt d'y satisfaire par un
mémoire tout fait il y a sept ou huit ans, et de suppléer encore à ce
mémoire s'il n'était pas trouvé suffisant sans demander une heure de
délai.

«\,L'érection de La Rochefoucauld est de 1622. L'enregistrement est de
1631. On supprime ici, avec un religieux silence, les causes d'un si
long délai, et la manière dont cet enregistrement fut fait. Ni l'un ni
l'autre ne seraient pas favorables à la cause de M. de La
Rochefoucauld\,; et si cette remarque, toute monosyllabe qu'elle est,
n'était indispensable pour faire voir que ce n'est pas se prévaloir de
la négligence de M. de La Rochefoucauld, on n'en aurait fait aucune
mention.

«\,On souhaiterait encore pouvoir taire un autre inconvénient qui a même
jeté M. le duc de Saint-Simon dans un grand embarras, lorsqu'il a été
obligé de faire travailler à cette affaire pour n'en pas tirer un
avantage trop ruineux à M. de La Rochefoucauld. C'est le défaut
d'hommage rendu au roi. Une érection en duché, marquisat ou comté, plus
essentiellement en duché-pairie, est constamment la remise d'un fief que
le vassal possède entre les mains du roi\,; que le roi, après l'avoir
repris, lui rend avec une dignité dont il l'investit par l'érection aux
conditions portées par icelle qui sont respectives, savoir d'honneur et
d'avantage pour le sujet, d'hommage et de service envers le seigneur,
dont la principale, qui donne l'être aux autres, est constamment
l'hommage. Par l'érection le roi investit son sujet, par l'hommage le
sujet accepte et se soumet aux conditions sans lesquelles le roi
n'entend lui rien donner, et le sujet n'entend rien recevoir. Cela n'est
pas douteux. Dans l'hommage du sujet nouvellement investi consiste donc
toute la forme\,; la force et la réalité de l'effet de l'érection et de
l'investiture, sans quoi les choses demeureraient nulles et comme non
avenues, puisque le sujet ne fait point de sa part ce qui est requis
pour recevoir la grâce que son souverain lui fait, qui est de l'accepter
de sa main et de le reconnaître pour son seigneur singulier en ce genre.
Cette action d'hommage ne se peut faire qu'en trois façons, ou au roi
même en personne, ce qui est devenu très-rare, ou, en la place de Sa
Majesté, à son chancelier qui la tient pour ce, ou encore en la chambre
des comptes. Il en demeure un acte solennel au souverain et au nouveau
vassal, qui est le titre du changement de son fief en dignité plus
éminente, et en mouvance plus auguste, puisque alors ce fief érigé ne
relève plus que de la couronne, et c'est l'instrument qui déclare au
public le changement arrivé dans le fief et dans son possesseur, puisque
l'érection sans cela n'est qu'un témoignage de la volonté du roi
demeurée imparfaite, dès là que par l'omission de l'hommage, condition
si essentielle, le sujet n'accepte pas la grâce de son seigneur, et ne
se lie pas à son joug par un nouveau serment, et acte d'obéissance, de
service et de fidélité.

«\,C'est néanmoins ce qui ne se trouvera pas que feu M. le duc de La
Rochefoucauld, ait fait, en aucun temps, au roi, à son chancelier, ni à
la chambre des comptes, chose pourtant si essentielle qu'on ne craint
point d'avancer que la dignité de duc et pair pourrait être justement
contestée à M. de La Rochefoucauld\,; rien ne peut couvrir ce défaut que
la bonté du roi, en lui accordant un rang nouveau, en faisant
présentement son hommage, et c'est cet étrange inconvénient que M. de
Saint-Simon a cherché par tous moyens de pallier, pour n'émouvoir pas
une question si fâcheuse à un seigneur qu'il respecte, et qu'il a
toujours constamment honoré. Pour en venir à bout, M. de Saint-Simon
s'est trouvé réduit à dire que lorsque feu M. de La Rochefaucauld prêta
serment en la manière accoutumée lorsqu'il fut reçu au parlement, ce
serment emporta hommage, qui donc au-moins ne fut rendu qu'en cet
instant\,; et pareillement que la chambre des comptes établie si
spécialement sur les foi et hommage, aveux et dénombrements\footnote{Il
  a été question de l'hommage et des cérémonies qui l'accompagnaient, t.
  II, p.~449. \emph{L'aveu} était encore une espèce d'hommage, par
  lequel on se reconnaissait l'homme du seigneur. Voici une formule
  d'aveu extraite du \emph{Grand coutumier} (t. II, p.~31)\,: «\,Tu me
  jures que d'ici en avant tu me porteras foi et loyauté comme à ton
  seigneur, et que tu te maintiendras comme homme de telle condition
  comme tu es\,; que tu me payeras mes dettes (ce qui m'est dû) et
  devoirs bien et loyaûment, toutefois que payer les devras, ni ne
  pourchasseras choses pourquoi je perds l'obéissance de toi et de tes
  hoirs (héritiers), ni ne te partiras de ma cour, si ce n'est par
  défaut de droit et de mauvais jugement. En tout cas tu \emph{advoues}
  ma cour pour toi et pour tes hoirs.\,» Le \emph{dénombrement} était
  une déclaration que chaque vassal était tenu de faire à son seigneur
  quarante jours après l'hommage. Elle devait contenir rénumération de
  toutes les terres et droits qui dépendaient du seigneur. Ce dernier
  avait aussi quarante jours pour constater l'exactitude du
  dénombrement.} de la couronne, ne le put reconnaître, à faute
d'hommage, qu'alors et deux mois après, lorsque son érection y fut
vérifiée, c'est-à-dire en 1637.

«\,Deux ans auparavant, c'est-à-dire en 1635, le 2 février, l'érection
de Saint-Simon avait été faite et fut enregistrée. Feu M. le duc de
Saint-Simon avait rendu sa foi et hommage\,; il avait été reçu duc et
pair au parlement, et feu M. le duc de La Rochefoucauld n'y avait formé
nulle opposition pour son rang. Il est vrai qu'étant reçu deux ans après
il prétendit la préséance, et il ne l'est pas moins qu'il ne la put
jamais obtenir, chose qui s'accorde si aisément par provision à ceux
dont le droit est jugé le meilleur, en attendant un jugement définitif,
comme il est arrivé en pairie en tant d'occasions, et comme il en
subsiste encore un exemple dans l'affaire de M. de Luxembourg. M. le duc
de Retz se trouvait dans le même cas à l'égard de M. le duc de La
Rochefoucauld, et ils s'accommodèrent ensemble, sans qu'on ait pu en
démêler la raison, à se précéder alternativement. Ces accords se peuvent
pour les cérémonies de la cour quand le roi le trouve bon, mais au
parlement il faut un titre. C'est ce qui fut cause d'un brevet du roi,
du 6 septembre 1645, qui, en attendant le jugement, ordonna cette
alternative dont le commencement solennel fut au lit de justice du
lendemain, et comme il importait aux parties par laquelle la préséance
commencerait, le sort en décida contre M. de La Rochefoucauld. Il ne se
peut une balance plus exacte\,; depuis, l'alternative a toujours
subsisté. Retz s'est éteint\,; Saint-Simon seul est resté dans cet
intérêt, qui quant à présent ne regarde aucun autre duc que MM. de La
Rochefoucauld et Saint-Simon.

«\,Cette question a toujours paru au roi sinon si sûre {[}du moins{]} en
faveur de M. de Saint-Simon, c'est-à-dire de la première réception,
qu'il en est émané de Sa Majesté deux grands préjugés célèbres dans une
de ses plus augustes fonctions. Le roi ayant élevé à la fin de 1663
quatorze seigneurs à la dignité de pairs de France, Sa Majesté tint son
lit de justice, et en sa présence fit enregistrer les érections et
recevoir les nouveaux pairs l'un après l'autre dans le rang qu'elle
avait déterminé de leur donner. M. le duc de Bouillon avait été fait duc
et pair quelques années auparavant avec une clause d'ancienneté première
de Château-Thierry et d'Albret, que le parlement modifia en enregistrant
le contrat d'échange de Sedan, au jour de la date de ce contrat, pour,
en modérant une ancienneté qui l'eût mis à la tête de tous les ducs et
pairs, lui en donner une insolite en manière de dédommagement, et la
fixer avant l'enregistrement de ses lettres, et avant sa première
réception, ce que le roi trouva si juste, attendu le jeune âge de M. de
Bouillon, depuis grand chambellan de France, et sentit en même temps si
bien qu'il perdrait son ancienneté, s'il n'y était autrement pourvu,
qu'il fit prononcer par M. le chancelier un arrêt exprès pour la
conservation de son rang au jour de la date susdite, en ce même lit de
justice. Il y a plus\,: M. le maréchal de La Meilleraye, l'un des
quatorze nouveaux pairs, était lors absent et en Bretagne pour le
service du roi.. Il ne parut pas juste à Sa Majesté que son absence
préjudiciât au rang qu'elle lui avait destiné le quatrième parmi les
autres, et il fut encore rendu un autre arrêt pour la conservation de
son rang. Il faut convenir que rien n'est plus formel en faveur de M. de
Saint-Simon que ces deux arrêts si solennels sur cette même et précise
question, émanés du roi même, séant en son lit de justice, uniquement
tenu pour les pairs.

«\,Lorsqu'en 1702, M. de Saint-Simon d'aujourd'hui songea, avec la
permission du roi, à se faire recevoir au parlement, il supplia M. le
duc de La Rochefoucauld de s'y trouver et de l'y précéder sans
rechercher qui avait eu la dernière alternative, dont l'âge avancé de
feu M. de Saint-Simon et la jeunesse de celui-ci avaient ôté les
occasions depuis longtemps. M. de La Rochefoucauld fut sensible à
l'honnêteté qui certainement était grande, mais embarrassé. On était à
Marly. M. le duc de Saint-Simon fut à Paris voir M. le premier président
d'Harlay, qui lui demanda comme il ferait avec M. le duc de La
Rochefoucauld. M. de Saint-Simon lui dit l'honnêteté qu'il lui avait
faite qui levait tout embarras\,; mais il ne fut pas peu surpris de la
réponse de ce magistrat, qui se piquait de n'ignorer rien. Cette réponse
fut que les rangs des pairs entre eux ne dépendaient pas d'eux au
parlement, et que cela ne levait aucune difficulté. M. de Saint-Simon
était jeune\,: il craignait les exemples des réponses fâcheuses de ce
premier président. Il s'y voulait d'autant moins exposer qu'il savait
par l'expérience de ses affaires que, depuis le procès de M. de
Luxembourg, il était fort mal avec lui, et que d'ailleurs il avait
cherché à se raccommoder par feu M\textsuperscript{me} de La Trémoille
avec M. de La Rochefoucauld, que ce même procès avait brouillé avec lui.
Ainsi M. de Saint-Simon se tut et ne jugea pas à propos de l'irriter en
lui parlant du brevet de 1645, que le parlement avait enregistré, que ce
magistrat ignorait ou voulait ignorer, et se retira sans lui rien
répondre là-dessus. De retour qu'il fut le soir même à Marly, il apprit
par feu M. le duc de La Trémoille que M. de La Rochefoucauld désirait
que le procès se jugeât entre eux. M. de Saint-Simon pria M. de La
Rochefoucauld de s'expliquer franchement avec lui, lequel lui dit que
Retz étant éteint, l'âge et l'état de la famille de feu M. de
Saint-Simon avait toujours fait juger que sa dignité s'éteindrait de
même, que cette considération avait toujours arrêté toute pensée de
jugement, mais que présentement l'état des choses qui avait changé
faisait aussi changer de sentiment, et qu'il désirait que l'affaire fût
jugée. Ils parlèrent ensuite de la manière d'en user réciproquement, et
M. de La Rochefoucauld voulut des arbitres pairs. M. de Saint-Simon lui
représenta que le roi seul ou le parlement étaient les juges uniquement
compétents, et que jamais un autre jugement ne pourrait être solide\,;
mais il n'y eut pas moyen de le persuader, et tous deux convinrent de
sept juges, qui furent MM. de Laon, Sully, Chevreuse, Beauvilliers,
Noailles, Coislin et Charost. M. de Saint Simon insista pour qu'il y eût
au moins un magistrat rapporteur. Cela fut également rejeté par M. de La
Rochefoucauld, tellement qu'il fut convenu que M. de Laon présiderait et
rapporterait en même temps, et que, pour tenir lieu de significations,
les copies des pièces et des mémoires dont on voudrait se servir
seraient remises à M. de Laon par les parties signées d'eux, et
communiquées de l'une à l'autre par M. de Laon, qui aurait pouvoir de
limiter le temps qu'on serait obligé de les lui rendre.

«\,Les choses en cet état agréées par le roi, M. de Saint-Simon demanda
du temps pour revoir une affaire si vieillie, et qu'il comptait laisser
en alternative tant qu'il plairait à M. de La Rochefoucauld, et que cela
lui plairait toujours. Ce fut alors que M. de Saint-Simon fut arrêté et
fort embarrassé de l'omission de foi et hommage par feu M. de La
Rochefoucauld, qu'il suppléa, comme il a été dit ci-dessus, pour ne se
pas donner la douleur de faire perdre à M. de La Rochefoucauld un rang
si ancien, et le réduire à prendre la queue de tous les ducs, en lui
contestant, comme il serait trop bien fondé à le faire, la validité de
sa dignité.

«\,Lorsque M. de Saint-Simon fut prêt, il le déclara à M. de Laon pour
le dire à M. de La Rochefoucauld, lequel fut longtemps à prétendre que
M. de Saint-Simon communiquât ses papiers le premier. M. de Saint-Simon
répondit que c'était à M. La Rochefoucauld à commencer, puisque c'était
lui qui ne voulait plus l'alternative et qui désirait le jugement\,;
que, ne donnât-il que six lignes contenant sa prétention toute nue avec
ses lettres d'érection et ses autres pièces conséquentes, M. de
Saint-Simon s'en contenterait et répondrait. Après un assez long temps,
on ne sait quel en fut le motif, M. de La Rochefoucauld déclara à M. de
Laon, en lui donnant sa prétention toute sèche en douze lignes, qu'il
n'avait pièces ni raisons quelconques à présenter, et qu'il n'en voulait
plus ouïr parler\,; on n'oserait dire qu'il paya d'humeur, mais on ne
peut taire qu'il ne paya d'aucune raison. Il y a sept ou huit ans que
les choses en sont là, sans que M. de La Rochefoucauld se soit présenté
en aucune occasion d'alternative, ne s'étant pas même trouvé à la
réception de M. le duc de Saint-Simon, qui avant tout a songé à se
conserver l'honneur de l'amitié de M. le duc de La Rochefoucauld, et n'a
pas parlé depuis de leur affaire qui est demeurée là.

«\,Deux courtes observations uniront ce mémoire.

«\,La première\,: Qu'on ne peut pas dire qu'il n'y ait pas un procès
certainement existant et très-ancien entre MM. de Saint-Simon et de La
Rochefoucould, repris et laissé en divers temps entre leurs pères, et
depuis par eux-mêmes\,; «\, Que le roi en a eu en tous les temps une
connaissance si effective qu'il est émané de Sa Majesté un brevet pour
l'établissement d'une alternative au parlement, qui exclut toute
provision de préséance, et deux arrêts en plein lit de justice, qui sont
un préjugé formel et le plus précis qui puisse être en faveur de M. de
Saint-Simon\,; «\,Que tout nouvellement, le roi, sur la représentation
de M. le maréchal de Villars de lui accorder un arrêt semblable à ceux
de Bouillon et de La Meilleraye, ou d'empêcher que M. le maréchal
d'Harcourt fût reçu pair au parlement avant que sa blessure lui eût
permis de l'être lui-même, Sa Majesté a pris ce dernier parti, ce qui
n'est pas un moindre préjugé en faveur de M. de Saint-Simon que les deux
autres.

«\,Conséquemment que le roi a dans tous les temps regardé cette question
comme une vraie et très-importante question, et par plusieurs actes
solennels émanés de Sa Majesté jusque tout récemment, comme une question
très-favorable pour M. le duc de Saint-Simon. Voilà pour ce qui est de
la chose en soi.

«\,L'autre observation regarde l'autorité du roi.

«\,Rien ne serait plus contraire au devoir de vassal à son seigneur,
bien pis encore d'un sujet à son souverain, que de jouir de l'effet
d'une grâce, qui est ce que le prince donne, sans rendre foi et hommage,
qui est un lien prescrit par sa grâce même, et un échange pour la grâce
que le sujet en la recevant rend au prince qui l'honore d'un nouveau
titre, en conséquence duquel il lui est par la foi et hommage, pour
raison de ce, plus nouvellement et plus étroitement soumis, attaché et
fidèle. C'est néanmoins ce qui manque à M. de La Rochefoucauld, et ce
qui n'a pu être suppléé que par son serment de pair prêté en 1637, deux
ans après l'hommage de feu M. le duc de Saint-Simon et sa réception au
parlement postérieure à cet hommage.

«\,Rien ne marquerait moins l'autorité du roi que la fixation du rang
des pairs à la date de l'enregistrement de leurs lettres, et rien en
particulier n'y serait plus spécialement opposé que la fixation du rang
de M. de La Rochefoucauld à la date de l'enregistrement des siennes. Sur
le premier point, il est constant que ce serait prendre rang par
l'autorité du parlement qui a toujours prétendu pouvoir admettre,
retarder, avancer ou rejeter les enregistrements des lettres, et qui
souvent l'a osé faire\,; sur le second point, c'est l'espèce présente,
puisque les lettres de La Rochefoucauld furent enregistrées pendant la
disgrâce de feu M. de La Rochefoucauld et contre la volonté du roi
connue, et lors absent de Paris. Ce fait est certain, et M. de La
Rochefoucauld, qui se souvient bien de la manière dont cela se passa,
pour l'avoir ouï souvent raconter chez lui, n'en disconviendra pas.

«\,Reste donc, pour faire chose séante à l'autorité royale, de fixer le
rang à la date des lettres ou à la réception de l'impétrant au
parlement, puisqu'on vient de montrer l'indécence de la fixer à la date
de l'enregistrement des lettres. De le faire à la date de leur
expédition est impossible, puisque des lettres non enregistrées
n'opèrent qu'une volonté du roi non effective ni effectuée, qui ne
produit que ce qu'on appelle improprement duc à brevet, comme l'est
encore M. de Roquelaure, c'est-à-dire un homme que le parlement ne
reconnaît point duc et pair, qui n'a nul rang, qui ne jouit que de
quelques honneurs qui ne peuvent passer à son fils sans grâce nouvelle,
et dont les lettres sont incapables de lui fixer un rang parmi ceux du
nombre desquels il ne peut être tant que ses lettres demeurent sans
vérification.

«\,On ne peut donc fixer le rang d'ancienneté qu'à la réception de
l'impétrant pour deux grandes raisons\,: la première parce qu'alors
seulement la dignité se trouve complète et parachevée sans que rien de
ce qui est d'elle y puisse plus être ajouté, comme on le montrerait
évidemment si on entrait dans le fond. L'autre, c'est qu'alors seulement
la volonté du roi, non suffisante par l'expédition des lettres
d'érection, non toujours suivie par leur enregistrement, et spécialement
en celle de La Rochefoucauld, est la règle unique de cette réception
dont on ne trouvera aucun exemple contre la volonté des rois. C'est donc
alors seulement qu'opère indépendamment de tout le reste la puissance de
cette volonté souveraine, qui vainement a érigé, qui pour
l'enregistrement n'est pas toujours obéie, et qui, quand elle la serait,
ferait donner par le parlement ce qu'elle-même n'a pu donner sans son
concours\,; mais qui seule suspend ou presse à son gré la réception au
parlement de celui qu'elle a fait pair de France, et par cet acte elle
le tient suspendu en ses mains tant que bon lui semble, et tient ainsi
sa fortune en l'air quoique achevée, et ce semble déterminée par là
puissance étrangère de l'enregistrement, et permet seulement que tout
acte de pairie s'achève en effet et s'accomplisse en l'impétrant, quand
elle veut, par cette grâce dernière de sa première réception au
parlement, couronner toutes les autres qui n'y sont qu'accessoires, et
manifeste seulement alors à l'État un assesseur et un conseiller nouveau
qu'elle s'est choisi, aux grands vassaux de la couronne un compagnon
qu'ils ont reçu de sa main toute-puissante, et à tous ses sujets un juge
né qu'elle a élevé sur eux. Alors la dignité complète est seulement
proposée telle, et le rang d'ancienneté fixé pour jamais dans cette
famille par un dernier coup de volonté pleine qui ne dépend que du roi
tout seul, sans concours du parlement, et sans qu'autre que la majesté
royale mette la main à l'ouvrage alors entier et en sa perfection.

«\,C'est ce que plus de loisir et de licence d'entrer dans un fond plus
détaillé de la matière du procès pendant entre MM. de Saint-Simon et de
La Rochefoucauld, et pour le droit en soi, et pour le fait en exemples,
démontrerait encore plus invinciblement. En voilà assez au moins sinon
pour déterminer le roi en faveur de son autorité et de son
incommunicable puissance, des préjugés émanés de Sa Majesté même, en
tous les temps et avec grande solennité, et de la bonté en soi de la
cause de M. de Saint-Simon, pour détourner au moins sa bonté, et on ose
ajouter son équité, de décider rien là-dessus sans lui avoir fait la
grâce de l'entendre, sinon par elle-même, au moins par ceux sur qui elle
s'en voudra décharger, dont M. de Saint-Simon n'aura aucun possible pour
suspect, par sa confiance en la bonté et en la justice de son droit.\,»

Deux lettres que nous nous écrivîmes le chancelier et moi donneront
maintenant toute la lumière dont la suite de cette affaire a besoin. La
première est du lendemain que j'eus appris de lui à Versailles les
articles du sacre et de l'extension des bâtards en autant de pairs
qu'ils auraient de pairies\,; l'autre, aussitôt que j'eus achevé le
mémoire ci-dessus. Ce fut le 3 mai, à Paris où j'étais venu coucher.

«\,Je vous avoue, monsieur, que je revins hier plus affligé que je ne
puis vous le dire, et qu'après avoir pensé à la nouvelle et horrible
plaie générale, je songeai à la mienne particulière. Ce matin, j'ai fait
un mémoire sur mon affaire, le plus court et précis que j'ai pu, et je
viens de vous écrire une lettre ostensible, compassée au mieux que j'ai
pu pour y joindre. D'Antin a dit le fait à M. de Chevreuse\,; puisqu'il
l'a su sans vous, et ce dernier me l'a dit à moi, comme je vous en
rendis hier compte\,; j'espérais que mon mémoire serait assez tôt mis au
net pour pouvoir vous le porter ce soir, mais mon lambin de secrétaire
ne finit point. Il me serait néanmoins très-important d'avoir l'honneur
de vous entretenir, et je vois vos journées si prises, que je ne sais
pas quand. D'aller à Pontchartrain ne me semble pas trop à propos dans
cette conjoncture, et je ne vois que samedi prochain comme hier à
Versailles, ce qui est long et étranglé\,; en attendant je vous enverrai
mon mémoire que j'aurai grand regret de vous laisser lire tout seul.
Cependant commandez à votre serviteur muet comme un poisson, et qui va
être en général et en particulier brisé comme vile argile. Qu'il y
aurait un beau gémissement à faire là-dessus, qui me ferait encore
dérouiller du latin et des passages, mais vous diriez que ce serait les
profaner\,! Permettez-moi du moins, un \emph{heu\,!} profondément
redoublé, en vous assurant d'un attachement et d'une reconnaissance
parfaite.\,»

Le chancelier, qui en magistrat et en courtisan comptait pour rien les
deux nouveaux articles du sacre et des bâtards\,; qui espérait, en
quelque dédommagement du second, faire passer la double séance des pairs
démis, piqué de n'avoir pu emporter ma préséance sur M. de La
Rochefoucauld, de la justice de laquelle il était convaincu, et se
voulant persuader, et plus encore à nous, que nous devions être gorgés
et nous tenir comblés des autres articles, me renvoya sur-le-champ ma
lettre dont il déploya l'autre feuille, sur laquelle il m'écrivit cette
réponse\,: «\,Permettez-moi, monsieur, cette manière de vous répondre
pour une fois seulement et pour abréger, et permettez-moi aussi de vous
gronder en peu de mots, en attendant plus. N'avez-vous point de honte de
n'être jamais content de ce que pensent les autres\,? serez-vous
toujours partial en toute affaire\,? ramperez-vous toujours dans le rang
des parties sans entrer jamais dans l'esprit de législateur\,? La
besogne est bonne, je la soutiens telle, et si bonne que c'est pour
l'être trop qu'elle ne passera peut-être pas\,; et cette bonne besogne,
c'est pour vous une horrible plaie générale et une plaie particulière
qui vous afflige au delà de l'expression. Qu'entendez-vous par cette
lettre ostensible\,? à qui la voudrais-je ou pourrais-je montrer\,? Non,
monsieur, il n'y a que samedi prochain de praticable\,; un siècle entier
de conversation vous paraîtroit un moment étranglé si on ne finissait
pas par être de votre avis. Envoyez-moi toujours votre mémoire,
monsieur\,; cela en facilitera une seconde lecture avec vous et la
rendra plus intelligible. Soyez toujours très-muet, mais exaltez-vous
dans l'esprit de vérité, et ne vous abaissez pas au-dessous de l'argile
pour perdre un cheveu de votre perruque quand vous en gagnez une
entière. Permettez-moi, à mon tour, un \emph{heu\,!} profondément
redoublé sur les torts d'un ami aussi estimable que vous l'êtes pour
moi, et aussi aimable en toute autre chose.\,» Ces deux lettres
caractérisent merveilleusement ceux qui les ont écrites, et pour le
moins aussi bien celui à qui ils avaient affaire\,: les deux suivantes
le feront encore mieux. Voici celle du chancelier du 5 mai.

«\,J'ai lu, monsieur, et relu avec toute l'attention et le plaisir
qu'une telle lecture donne à un homme comme moi, et avec toutes les
pauses et les réflexions réitérées qu'une pareille matière exige, et
votre lettre et votre mémoire, et votre abrégé de mémoire. Je vous
renvoie la lettre. Les raisons de ce renvoi sont dans ma réponse d'hier.
Je garde le reste\,; il est pour moi, s'il vous plaît\,; vous en avez la
source dans votre esprit, les minutes dans vos papiers. Ce que je garde
me tiendra lieu de tout cela, c'est beaucoup pour moi. À l'égard de la
question, je suis pour vous, monsieur\,; je vous l'ai déjà dit, mon
suffrage sera toujours à votre avantage. Ce qui vous surprendra, c'est
que ce ne serait pas par vos raisons. Votre première et grande raison,
que vous tirez des foi et hommage, n'est pas vraie dans le principe des
fiefs, et votre dernière grande raison, que vous tirez de l'intérêt des
rois mêmes, n'est en bonne vérité qu'un jeu d'esprit, et qu'un sophisme
aussi dangereux qu'il est aussi bien tourné qu'il puisse l'être, et
aussi noblement et artistement conçu qu'on puisse l'imaginer. Mais après
mille et mille ans de discussion, où, sans en rien dire davantage,
trouvez-vous, suivant votre terme d'hier, que cette discussion soit
étranglée, puisque je me déclare pour vous, et que je ne me départirai
jamais de cet avis tant que ce sera mon avis qu'on me demandera\,? Mais
quand, après avoir tout représenté, je n'ai plus qu'à écrire ce que l'on
me dicte et qu'à obéir, puis-je faire autrement\,? D'ailleurs, en bonne
foi, quand tout l'ouvrage en lui-même est si bon et si désirable, que
vous consentez vous-même que l'on juge deux procès existants sans
entendre les parties, et que l'on en prévienne douze prêts à éclore sans
y appeler aucune des parties, pouvez-vous en justice, en honneur, en
conscience désirer que l'on fasse renaître le vôtre oublié du parlement
comme du roi même, et que l'on renverse un projet d'édit de cette
importance, bon de votre propre aveu en tout ce qui est de votre goût,
et qui ne regarde point votre petit intérêt à qui vous voulez que tout
cède\,? J'en appelle à la noblesse de votre cœur et à votre droite
raison, monsieur\,; vous êtes citoyen avant d'être duc, vous êtes sujet
avant d'être duc, vous êtes fait par vous-même pour être homme d'État,
et vous n'êtes duc que par d'autres. Pour me confirmer davantage dans
mon avis, donnez-moi, je vous conjure, une copie du brevet de 1645\,;
expliquez-moi bien 1622, 1631 et la réception de 1637. Je vois que par
un excès de charité vous en faites une réticence éloquente dans votre
mémoire. Moi, qui ne suis ni éloquent ni charitable, que j'en sache, je
vous prie, l'anecdote dans tous ses points et dans tous ses détails.
Vous savez comme moi tout ce que je vous suis, monsieur.\,»

Voici ma réponse à cette lettre, de Marly, 6 mai.

«\,J'ai reçu ce matin, monsieur, l'honneur de vos deux dernières
lettres, l'une revenue de Paris, l'autre droit ici\,; j'en respecte la
gronderie, j'en aime l'esprit, permettez-moi la liberté du terme. Je
reçois avec action de grâces le rendez-vous de samedi à Versailles. Je
suis ravi de la peine que vous avez bien voulu prendre de tout lire, et
je ne puis différer de vous remercier très-humblement des
éclaircissements que vous me demandez. J'aurai l'honneur de vous les
porter samedi avec votre lettre même pour que, sans rappeler votre
mémoire, vous voyiez si je satisfais à tout. J'aurais trop à m'étendre
sur ce qu'il vous plaît de me dire de flatteur\,; en m'y arrêtant je
m'enflerais trop. J'aime mieux m'arrêter au blâme, et vous rendre
courtement et sincèrement compte de mes sentiments, comme on rend raison
de sa foi.

Pour mes sentiments, pardonnez-moi si, avec tout respect, je demeure
navré de ce qui regarde le sacre, et si je suis trop partie, ne soyez
vous-même législateur qu'en vous mettant en la place de {[}ceux{]} sur
qui portent les lois. C'est notre fonction la plus propre, la plus
ancienne, la plus auguste, dont rien ne peut consoler et à laquelle
d'ailleurs je ne me flatterais pas personnellement de pouvoir prétendre.
Ainsi ce n'est pas moi que je pleure, mais la plaie de la dignité. Du
reste, tout est si excellemment bon que si on venait à mon avis que tout
le reste passât tel qu'il est maintenant, ou que tout ce reste demeurât
comme non avenu, je le ferais plutôt signer, sceller et enregistrer ce
soir que demain matin, encore que le second article soit fâcheux en
général, et que par un autre article je perde une cause personnelle que
je tiens sans question, de bonne foi, et que vous-même trouvez bonne et
juste. Voyez, monsieur, si c'est là être attaché à ses intérêts
particuliers, et je vous parle en toute vérité.

«\,À l'égard de mon mémoire, oserais-je vous dire que je ne me crois pas
tout à fait battu sur le défaut et la nécessité de l'hommage, et que,
s'il en était question, et que vous me voulussiez traiter comme
Corneille faisait sa grossière servante, je crois que vous ne trouveriez
pas mon opinion si déraisonnable. Je sais que la grande et indisputable
raison est celle des offices et des officiers, mais comme elle n'est pas
entrée lorsqu'elle a été mieux représentée que je ne pourrais faire en
cent ans, je l'ai omise. Pour ce qui est de ce que vous appelez sophisme
sur l'autorité des rois, trouvez bon que je vous suggère un terme plus
fort et plus vrai, c'est une fausse raison\,; non que le raisonnement
n'en soit juste et certain, mais c'est que ce n'est pas par là que la
question doit se décider\,; cependant c'est uniquement par rapport à
l'autorité qu'on se détermine contre moi. Puisque je l'ai pour moi,
n'ai-je pas raison de l'expliquer, et puisque ma cause est bonne et
juste, ne dois-je pas lever la difficulté qui me la fait perdre, et
prendre mon juge par l'endroit dont il est uniquement susceptible, et
appuyer dessus en disant ce qui est, puisque sur cela seul je serai
jugé, sans aucune considération pour nulle autre raison.

«\,De m'opposer qu'il est injuste à moi de prétendre être ouï, tandis
que j'approuve que tant d'autres soient jugés sans être entendus, un mot
vous fera voir, monsieur, que cela ne doit pas m'être objecté.

«\,De tout ce nombre de prétendants prêts à éclore, aucun jamais n'a
intenté de procès, un seul en a eu la permission, et il en est encore à
en faire le premier usage, par quoi il est encore dans la condition des
autres qui ont des prétentions, mais n'ont jamais eu de procès. Ceux-là,
qu'on les juge par un règlement sans les entendre, que peuvent-ils
opposer\,? leurs prétentions sont dans leurs têtes\,; est-on tenu de les
supposer, et de discuter des êtres de raison qui n'ont pas la première
existence, et n'est-ce pas au contraire très-bien fait d'ôter aux
chimères, aux êtres de raison toute possibilité d'exister\,? Mais pour
ceux dont les prétentions sont par l'aveu du roi juridiquement au jour,
expliquées à des juges ou naturels ou pour ce permis, qu'un tribunal est
saisi, que les parties sont en pouvoir de faire juger entre elles, il ne
paraît pas juste de former un article entre elles sans y avoir égard, et
c'est en effet ce qui a été trouvé si peu juste par le roi et par
vous-même, que le consentement de feu M. de Luxembourg fût demandé et
intervînt sur le point qui le regarde dans le règlement projeté de son
temps, ce qui fait que le consentement de son fils n'est plus
aujourd'hui nécessaire, puisqu'il n'y a rien de changé là-dessus
d'alors. M. d'Antin forme un procès qui même est encore dans tout son
entier\,; on veut son consentement, on le satisfait, il acquiesce, à la
bonne heure. Ne serais-je pas malheureux si, n'y ayant que ces deux
hommes et moi en procès, je me trouve seul traité comme ceux qui n'en
ont point, eux consultés et contentés, moi condamné et pendu, pour ainsi
dire avec ma grâce au cou, moi avec un procès pendant au parlement, avec
une compétence ordonnée par le roi, enregistrée au parlement, deux
préjugés du roi en plein lit de justice, renouvelés tout à l'heure à
l'occasion de MM. de Villars et d'Harcourt, tandis que M. de Luxembourg,
avec un préjugé contraire à lui par la provision de préséance sur lui,
M. d'Antin pas seulement duc, et des plaidoyers seulement préparés et
non commencés, sont ménagés\,; en sorte que l'un reste pair, chose
autrement à lui très-mal sûre, et pair précédant plus de la moitié des
autres\,; et l'autre le devient, l'autre, dis-je, qui avec toute sa
faveur voit son procès perdu, s'il se juge.

«\,Encore une fois, monsieur, au point du sacre près, j'aime mieux
perdre mon affaire, et que le règlement passe\,; mais quelle
impossibilité que le règlement passe, et que je ne la perde pas, votre
cœur et votre esprit m'honorant, l'un de son amitié, l'autre de son
suffrage et de sa persuasion que mon droit est bon\,? Que si malgré
raison on veut que je perde, n'en pourrais-je point être récompensé, et
pour n'avoir ni charge ni gouvernement de province, ni barbe grise comme
M. de Chevreuse, mettez la main à la conscience, n'ai-je pas plus de
droit que lui, par voie d'échange, d'obtenir une grâce pour l'un de mes
fils, en abandonnant le droit de mon rang\,? Permettez-moi de vous
supplier de ne pas regarder comme une extravagance cette pensée qui se
peut tourner de plus d'une manière, et de considérer que, dans toutes
les circonstances présentes, il serait dur d'être regardé à trente-six
ans comme un enfant.

«\,Outre ce que m'a dit M. de Chevreuse, instruit par d'Antin du
règlement, M. le duc d'Orléans m'a dit savoir de d'Antin même qu'il
allait être fait duc et pair. N'en est-ce pas assez pour qu'un homme qui
est sur les lieux puisse être en peine de son autre cause, et s'adresser
pour cela à vous, qu'on sait avoir travaillé insolitement avec le roi,
en le faisant avec toutes les mesures possibles\,? «\,Mais en voilà trop
pour une lettre et assez pour un supplément de mémoire. Trouvez bon que
je vous supplie de le peser avec bonté et réflexion réitérée. Pour le
secret, je le garde tel que, encore que vous m'ayez permis dans tout le
cours de ceci de tout dire à M. d'Harcourt, je l'ai néanmoins traité en
dernier lieu comme les autres, c'est-à-dire comme MM. de Chevreuse et de
Charost, à qui j'ai constamment dit que je n'ai pu rien tirer de vous
sur votre travail avec le roi, et que Sa Majesté vous avait défendu d'en
dire une parole. Ce qui m'a obligé d'en user ainsi avec M. d'Harcourt a
été lé point sensible du sacre, et que je me suis cru plus sûr d'arrêter
M. d'Harcourt, tout mesuré qu'il est, en le lui taisant, et pour le lui
taire en lui taisant tout détail, qu'après le lui avoir dit. Comptez
donc, monsieur, quoi qu'il arrive, sur ma fidélité, sur une inexprimable
reconnaissance et sur un attachement sans mesure.\,»

Il faut maintenant expliquer deux choses\,: ma citation de M. le duc
d'Orléans sur d'Antin et ma pensée pour un de mes fils.

Le roi, comme on l'a vu, avait rejeté toute communication du projet de
règlement à quelques ducs, que le chancelier lui avait proposée, {[}à{]}
moi entre autres, et comptait que nous ignorions ce qui se passait
là-dessus. Ainsi le chancelier m'avait renvoyé cette lettre ostensible
au roi, que je lui avais écrite. La vivacité de son style montre combien
il trouvait impraticable de la lui montrer, parce que c'était lui
montrer en même temps que j'étais dans la bouteille. Tant qu'il
l'ignorait, je ne pouvais me présenter, et il m'importait extrêmement de
le faire pour le contenir entre son penchant pour M. de La
Rochefoucauld, et sur la prévention de son autorité contre ma cause\,;
parce que, tel qu'il était, il ne laissait pas de vouloir garder des
mesures, et d'en être contraint, ce qui fut sa vraie raison de rejeter
la communication à quelques-uns de nous. Or, dès que l'affaire
transpirait, et que je pouvais citer ce que M. le duc d'Orléans m'en
avait dit, je pouvais paraître m'adresser au chancelier, et lui, en
rendre compte au roi sans rien craindre de personnel, puisque c'était
d'Antin qui avait parlé à M. le duc d'Orléans, et ce prince qui me
l'avait rendu. Je mettais donc le chancelier à son aise là-dessus, et en
état de dire au roi sans embarras ce qu'il aurait jugé à propos.

À l'égard de mes enfants, surpris au dernier point de la manière dont le
roi avait répondu au chancelier sur ma question de préséance, je
craignis que cette idée de son autorité ne se pût détruire, parce
qu'elle lui était entrée si avant dans la tête. Il me vint donc en
pensée, lorsque le chancelier me le conta, d'essayer à faire démordre le
roi par un équivalent plus difficile, ou d'obtenir cet équivalent que
j'eusse sans comparaison préféré\,: c'était de faire mon second fils duc
et pair, puisque, sans raison, il était bien question de faire celui de
M. de Chevreuse, et d'Antin, et, moyennant cela, ne contester plus avec
le roi, et lui laisser le plaisir et le repos de faire gagner le procès
à son ami M. de La Rochefoucauld, et à ce qu'il croyait être non de la
justice, à quoi il n'eut jamais que répondre, ni ne s'en mit en fait,
mais de son autorité qu'il mit toujours en avant. Le chancelier ne
répudia pas cette pensée, et je la croyais d'autant meilleure que je
voyais le roi en une veine présente de telles facilités à multiplier ces
dignités, qu'il n'était question que d'en fabriquer le chausse-pied.
D'autre part, je craignais encore le crédit mourant de M. de La
Rochefoucauld. Ses infirmités l'avaient dépris des chasses et des
voyages depuis quelque temps, mais non pas de faire de fois à autre des
incursions dans le cabinet du roi, où il se faisait mener pour l'intérêt
de quelque valet ou de quelque autre rapsodie, où très-souvent il
arrachait, à force d'impétuosité, ce qu'il voulait du roi, et que
souvent aussi le roi ne voulait pas, qui haussait les épaules à l'abri
de son aveuglement, et qui lâchait enfin partie de compassion et
d'ancienne amitié, partie pour s'en défaire. Je redoutais donc la
crainte du roi des clabauderies de ce vieil aveugle, qui ne manquerait
pas de lui venir faire une sortie dès qu'il se saurait condamné, et qui,
à force de gémir, de gronder et de crier, me donnerait peut-être encore
à courre. Tout cela me fit donc juger que ma proposition n'était point
inepte, en soutenant d'ailleurs mon droit, mais dans le génie du roi,
c'est-à-dire en me restreignant à mettre son autorité de mon côté. Mais,
comme cette façon de combattre ne pouvait être de mise que pour lui
seul, ni même imaginée, quoique l'expérience de tous les jours apprît
l'inutilité de toute autre avec lui, en quelque occasion que ce fût, où
il se figurât que son autorité pouvait être le moins du monde
intéressée, j'estime qu'il est à propos de présenter ici l'état de la
question qui était entre M. de La Rochefoucauld et moi, et les
véritables raisons de part et d'autre sur lesquelles tout juge éclairé
et équitable avait uniquement son jugement à fonder. Outre que l'affaire
est déjà ici nécessairement entamée, le récit n'en sera pas assez long
pour le séparer de ce qui en a déjà été dit en le renvoyant aux Pièces,
d'autant qu'il est dans l'ordre des temps de le commencer par celui de
l'anecdote dont le chancelier me demanda, comme on a vu,
l'éclaircissement entier, qui doit par cette raison avoir ici sa place.

En 1622 le comté de La Rochefoucauld fut érigé en duché-pairie par Louis
XIII. Par cette grâce, M. de La Rochefoucauld devint ce qu'on appelle
improprement duc à brevet\footnote{Voy., sur les \emph{ducs, à brevet},
  t. 1er. p.~129, 130. note.}.

Les brouilleries d'État, où les seigneurs de La Rochefoucauld, aînés et
cadets, se sont très-particulièrement signalés contre les rois, depuis
Henri II jusqu'à Louis XIV, et jusqu'à son favori, M. le duc de La
Rochefoucauld inclusivement, avec qui j'avais ce procès à faire
décider\,; les brouilleries, dis-je, qui survinrent dans l'État
entraînèrent celui en faveur de qui l'érection s'était faite contre
celui qui l'en avait honoré, et le mirent hors d'état de la faire
vérifier au parlement. Il était encore dans la même situation,
c'est-à-dire en Poitou, exilé, après s'être engagé contre le roi,
lorsque le cardinal de Richelieu, premier ministre alors, fut fait duc
et pair, et voulut être reçu au parlement en cette qualité le même jour
et tout de suite de l'enregistrement de ses lettres.

Tandis qu'on y procédait, le parlement assemblé et les pairs en place,
le cardinal de Richelieu était à la cheminée de la grand'chambre, comme
on s'y tient d'ordinaire jusqu'à ce que le premier huissier vienne
avertir d'aller prêter le serment. On peut juger qu'il était environné
d'une grande suite et nombreuse compagnie.

M. le Prince cependant était avec les autres pairs en place, avec double
intention. Son dessein était de payer d'un trait aussi hardi
qu'important les services que lui et les siens avaient reçus de M. de La
Rochefoucauld et de ses pères, et s'il eut le don de prophétie, ceux que
MM. ses enfants devaient recevoir du fils et du petit-fils de M. de La
Rochefoucauld. Il y avait non-seulement défaut de permission
d'enregistrer ses lettres, mais une défense expresse du roi, et
réitérée, au parlement de le faire. M. le Prince, de concert avec le
premier président Le Jay et avec Lamoignon, conseiller en la
grand'chambre, père du premier président Lamoignon, complota de saisir
le moment le plus confus et le plus inattendu avec hardiesse pour faire
passer l'enregistrement des lettres de La Rochefoucauld, et choisirent
comme vraiment tel l'instant entre l'enregistrement de celles de
Richelieu et le rapport de la vie et mœurs du cardinal pour sa
réception, comptant bien que, parmi le bruit et la foule qui accompagne
toujours tels actes, on ne se doute-roit et on ne s'apercevrait même pas
du coup qu'ils voulaient faire réussir.

Tout convenu avec un petit nombre de ce qui devait être et se trouva en
séance pour donner branle au reste, M. le Prince, sans attendre que le
second rapporteur, pour l'information de vie et mœurs, eût la bouche
ouverte pour parvenir à la réception du cardinal de Richelieu, et qu'on
montât aux hauts sièges pour ouïr l'avocat et l'avocat général, et y
recevoir le cardinal comme on faisait alors\,; M. le Prince, dis-je,
regarda le premier président, qui, sachant ce qui s'allait faire, ne se
hâtait pas de donner la parole à ce rapporteur, et demanda s'il n'y
avait pas quelque autre enregistrement à faire, parce qu'il lui semblait
qu'il y en avait. Le Jay, effrayé au moment de l'exécution, répondit
fort bas qu'il y avait celui des lettres de La Rochefoucauld, déjà
anciennes, mais qui avaient toujours été arrêtées par le roi. «\,Bon,
reprit M. le Prince, cela est vieux et usé, je vous réponds que le roi
n'y pense plus\,;» et ajouta tout de suite, en se tournant vers
Lamoignon\,: «\, Quelqu'un ne les a-t-il point là\,?» Lamoignon se
découvre et les montre. À l'instant M. le Prince, fortifiant Le Jay de
ses regards\,: «\, Rapportez-les-nous, dit-il à Lamoignon, M. le premier
président le veut.\,» Lamoignon ne se le fit pas dire deux fois. Il
enfile la lecture des lettres, la dépêche le plus vite qu'il peut, et
opine après en deux mots à leur enregistrement. Les magistrats dont les
trois quarts ignoraient la défense du roi de les enregistrer, et dont
presque aucun, parmi ce brouhaha de la foule qui remplissaient la
grand'chambre, n'avait pu entendre le dialogue si court de M. le Prince
avec le premier président, opinèrent du bonnet avec le reste de la
séance, comme c'est l'ordinaire en ces enregistrements, et attribuèrent
la précipitation dont on usait à l'égard d'abréger, tant qu'on pouvait,
l'attente du premier ministre d'être mandé pour être reçu. Ils n'eurent
ni le temps ni l'avisement de faire réflexion que s'il n'y eût pas eu là
quelque chose d'extraordinaire, il eût été de la bienséance de procéder
à la réception du cardinal de Richelieu avant de faire ce second
enregistrement, pour ne pas le faire attendre si longtemps, et pour que,
étant reçu et en place, il en eût aussi été juge. L'arrêt de
vérification des lettres de La Rochefoucauld fut prononcé d'abord après
les opinions prises, et cette grande affaire fut ainsi emportée, pour ne
pas dire dérobée, à la barbe du premier ministre présent dans la
grand'chambre, qui ne pensait à rien moins, et qui, parmi tout ce monde
et ce bruit dont il était environné à cette cheminée, croyait toujours
que c'était son affaire qui se faisait. Aussitôt après l'arrêt de
l'enregistrement de La Rochefoucauld prononcé, on procéda à ce qui
regardait la réception du cardinal, qui prêta son serment, et toute la
cérémonie s'acheva.

Au sortir du palais il apprit ce qu'il s'était passé, et ne put le
croire. Il manda le premier président qui s'excusa sur M. le Prince,
mais qui n'en essuya pas moins une rude réprimande. M. le Prince en fut
brouillé quelque temps, et la disgrâce de M. de La Rochefoucauld
approfondie, mais l'enregistrement n'en demeura pas moins fait et
consommé. C'est ce qui attacha de plus en plus M. de La Rochefoucaud à
M. le Prince, et ses enfants aux siens\,; c'est ce qui forma l'intimité
héréditaire de MM. de La Rochefoucauld avec les Lamoignon\,; c'est ce
qui fit durer l'exil de M. de La Rochefoucauld bien au delà de la fin de
tous les troubles, et de la réconciliation de tous ceux qui y avaient eu
part. Cet exil durait encore lorsqu'en 1634 il y eut de nouvelles
lettres d'érection de Retz en faveur du gendre après le beau-père, avec
rang nouveau, et qu'au commencement de 1635 mon père fut fait duc et
pair, et tous deux vérifiés et reçus au parlement sans la moindre
opposition de la part de M. de La Rochefoucauld, qui apparemment
n'imaginait pas encore de les précéder, et se tenait bien heureux
d'avoir sa dignité assurée. Revenu après en grâce, il se fit recevoir en
1637, et prétendit la préséance sur M. de Retz et mon père. C'est ce qui
forma la question entre la priorité d'enregistrement d'une part, et la
priorité de première réception au parlement de l'autre. Il est temps de
l'expliquer dans tout son jour après avoir raconté les faits, tant
anciens que nouveaux, depuis la naissance de cette dispute. On ne
s'arrêtera point aux écrits trop prolixes de part et d'autre, on se
renfermera dans le pur nécessaire à l'éclaircissement de la question.

\hypertarget{chapitre-xi.}{%
\chapter{CHAPITRE XI.}\label{chapitre-xi.}}

1711

~

{\textsc{Courte et foncière explication de la question de préséance
entre la première réception du pair au parlement, et la date de
l'enregistrement de la pairie.}} {\textsc{- Nature de la dignité.}}
{\textsc{- Ce qui de tout temps fixait l'ancienneté du rang des pairs,
l'a fixée toujours et la fixe encore aujourd'hui.}} {\textsc{- Fausse et
indécente difficulté tombée de la date de chaque réception successive.}}
{\textsc{- Dignité de duc et pair mixte de fief et d'office, et unique
de ce genre.}} {\textsc{- L'impétrant, et sa postérité appelée et
installée avec lui en la dignité de pair, à la différence de tout autre
officier.}} {\textsc{- Reprise de l'édit.}} {\textsc{- Lettre de M. le
duc de Saint-Simon à M. le chancelier.}} {\textsc{- Lettre de M. le
chancelier M. le duc de Saint-Simon.}} {\textsc{- J'apprends du
chancelier les articles de l'édit résolus.}} {\textsc{- Je confie au duc
de Beauvilliers, et au duc et à la duchesse de Chevreuse, que Chaulnes
va être réérigé pour leur second fils.}} {\textsc{- L'édit en gros
s'évente.}} {\textsc{- Mouvements de Matignon et des Rohan\,; leur
intérêt.}} {\textsc{- Lettres de M. le duc de Saint-Simon à M. le
chancelier, de M. le chancelier à M. le duc de Saint-Simon.}} {\textsc{-
L'édit passé, dont j'apprends par le chancelier tous les articles tels
qu'ils y sont.}} {\textsc{- Double séance rejetée et Chaulnes différé,
après avoir été accordés.}} {\textsc{- D'Antin, reçu duc et pair au
parlement, m'invite seul d'étranger au repas.}} {\textsc{- Le roi se
montre content que j'y aie été.}} {\textsc{- Adresse et impudence de
d'Antin.}} {\textsc{- Sagesse et dignité de Boufflers.}} {\textsc{-
Douleur de Matignon et son affaire avec le duc de Chevreuse.}}
{\textsc{- Duc de La Rocheguyon fait au chancelier des plaintes de
l'édit\,; prétend en revenir contre ma préséance, qui le refroidit, et
le duc de Villeroy, entièrement et pour toujours avec moi.}} {\textsc{-
Fâcheux personnage du duc de Luxembourg sur l'édit\,; est à Rouen, et
pourquoi.}}

~

On ne répétera point ce qui a été expliqué dans le précédent mémoire sur
la foi et hommage, qui, n'en déplaise à la première vue de M. le
chancelier, est un moyen sans réplique\,; on ne s'arrêtera pas non plus
aux trois préjugés du roi que chaque partie peut tirer à son avantage,
encore qu'il soit évident que celui qu'en tire M. de Saint-Simon ait
bien plus de force et soit bien plus naturel. On ne s'arrêtera qu'aux
moyens véritables des deux côtés, qui, sans sortir du fond de la
question, doivent être la matière unique du jugement, entre la priorité
d'enregistrement des lettres d'érection soutenue par M. de La
Rochefoucauld, comme règle et fixation de l'ancienneté\,; et la priorité
de la première réception du nouveau pair, érigé en cette qualité de pair
de France au parlement, que M. de Saint-Simon prétend fixer le rang
d'ancienneté parmi les pairs de France.

M. de La Rochefoucauld pose en fait que l'enregistrement des lettres
d'érection forme, constate, opère la dignité qui jusqu'alors n'est que
voulue par le roi, et si peu exécutée que celui qui a des lettres
d'érection non enregistrées n'a que des honneurs sans être, sans rang,
sans succession aux siens, toutes choses qui ne s'acquièrent que par
l'enregistrement des lettres d'érection, qui par la conséquence qu'il en
tire, réalisant la dignité, en fixent en même temps le rang
d'ancienneté.

Il ajoute, pour confirmer cette maxime, que, si on admettait celle de la
fixation du rang d'ancienneté par la première prestation de serment et
réception au parlement du pair nouvellement érigé, les rangs des pairs
entre eux changeraient à chaque réception de pair, d'où il arriverait
que le fils du plus ancien se trouverait le dernier de tous, et un
changement continuel de rang suivant les dates des réceptions dont on
n'a jamais ouï parler parmi les pairs, et qui en cela les égalerait avec
les charges les plus communes et les plus petits offices.

Toutes ces preuves ne sont que des raisonnements diffus et peu
concluants, des déclamations, force sophismes, qui n'ajoutent rien à
l'exposition simple de ces deux propositions telles qu'on vient de les
présenter. Le spécieux en est éblouissant à qui n'approfondit pas\,;
moi-même j'en ai été un temps pris. Je dois à l'abbé Le Vasseur, qui a
longtemps et utilement pris soin des affaires de mon père et des miennes
jusqu'à sa mort, arrivée comme je l'ai dit ailleurs, en 1709, de m'en
avoir fait honte. Je ne voulais point disputer parce que je ne croyais
pas avoir raison, et après avoir étudié la matière je fus honteux de
m'être si lourdement abusé.

Pour réfuter les deux propositions de M. de La Rochefoucauld, il faut
remonter à la nature de la dignité dont il s'agit de fixer l'ancienneté
pour ceux que le roi en honore, et voir ce qui la fixait anciennement.
Qu'on ne s'étonne point d'un principe qui doit être posé, parce qu'il
est de la première certitude. La dignité de pair est une, et la même
qu'elle a été dans tous les temps de la monarchie\,; les possesseurs ne
se ressemblent plus. Sur cette dissemblance on consent d'aller aussi
loin qu'on voudra, sur la mutilation des droits de la pairie, encore.
C'est l'ouvrage des temps et des rois\,; mais les rois ni les temps
n'ont pu l'anéantir, ce qui en reste est toujours la dignité ancienne,
la même qui fut toujours, jusque dans son dépouillement cette vérité
brille. Il faut une injustice connue par une loi nouvelle pour préférer
les princes du sang et les bâtards aux autres pairs dans la fonction du
sacre, sans oser les en exclure, et ces princes du sang et ces bâtards
comme pairs, les uns à titre de naissance par l'édit d'Henri III, les
autres comme ayant des pairies dont ils sont titulaires et revêtus.
Jusque dans sa dernière décadence, sous le plus jaloux et le plus
autorisé des rois, il a fallu, de son aveu même, l'intervention des
pairs invités de sa part chacun chez lui par le grand maître des
cérémonies, au grand regret et dépit de ce bourgeois qui n'oublia rien
pour en être dispensé\,; invités, dis-je, à se trouver au parlement pour
les renonciations respectives aux couronnes de France et d'Espagne des
princes en droit de les recueillir, par l'indispensable nécessité de la
pairie aux grandes sanctions de l'État. On ne parle pour abréger que de
ce qui est si moderne et dans la plus grande décadence de cette
dignité\,; plus on remonterait, plus trouverait-on des preuves augustes
de la vérité que j'avance. Les lettres d'érection y sont en tout
formelles jusque par leurs exceptions, et les évêques-pairs\footnote{Voyez,
  sur les évêques-pairs et en général sur les pairies, les notes à la
  fin du volume.} sont encore aujourd'hui exactement et précisément les
mêmes qu'ils ont été en tout temps pour les possessions et pour la
naissance, et pour le fond et l'essence de la dignité, en sorte que ce
ne sont pas des images parlantes de ce qu'ils furent autrefois, mais des
vérités, des réalités, et la propre existence même\,; égaux en dignités
aux six anciens pairs laïques quoique si disproportionnés d'ailleurs.
Cette vérité admise sur la question présente, et qui se trouvera
peut-être ailleurs démontrée avec plus d'étendue, il faut voir comment
l'ancienneté se réglait parmi ces anciens pairs.

Les douze premiers n'ont point d'érection\,; elle ne fixait donc pas
leur rang. Depuis qu'il y a eu des érections, il n'y avait point de cour
telle qu'est aujourd'hui celle connue sous le nom de parlement, où ces
érections puissent être enregistrées\,; ainsi l'enregistrement, qui
n'existait point, ne fixait point le rang des pairs. Il résulte donc que
ce rang ne se réglait ni par la date de l'érection ni par celle de
l'enregistrement. Il faut donc chercher ailleurs ce qui fixait leur rang
puisqu'il l'a toujours été entre eux\,; et, de ce qui vient d'être
exposé, M. de La Rochefoucauld conclura que ce n'est pas la première
réception du nouveau pair au parlement, puisque le parlement tel qu'il
est maintenant, et qu'il reçoit et enregistre, n'existait pas dans les
temps dont on parle, et cela est aussi très-certain. Mais il est
également certain aussi qu'il y a eu dans tous les temps une formalité
par laquelle tous ont passé et passent encore, dont les accessoires et
l'extérieur a changé avec les temps, mais dont la substance et la
réalité est toujours demeurée la même, et cette formalité est la
manifestation. Avant qu'on écrivît des patentes qui est l'érection,
avant qu'on les présentât à un tribunal certain pour y être admises qui
est l'enregistrement, il fallait bien qu'il y eût une manière ou une
forme de faire des pairs, puisqu'il y a eu dès lors des pairs. Il
fallait encore que ces pairs eussent entre eux un rang fixé puisqu'il
l'a été dès lors parmi eux, et cette manière ou cette forme n'a pu être
que l'action de manifester un seigneur dans l'assemblée des autres de
pareil degré, d'y déclarer l'élévation de celui-ci aux mêmes droits,
fonctions, rangs, honneurs, distinctions, priviléges, etc., que ces
autres\,; de l'y faire seoir parmi eux, c'est-à-dire au-dessous du
dernier, mais en même ligne et niveau\,; de l'y associer aux mêmes
conseils et aux mêmes jugements qui faisaient la matière de leur
assemblée. Ce ne pouvait être que par là, avant les usages postérieurs
des érections et des enregistrements, que les rois pouvaient déclarer
l'élévation d'un de leurs sujets et vassaux à la première dignité de
leur couronne, en manifestant de fait un conseiller né et un assesseur à
la couronne, et à eux un compagnon, et comme on parlait alors, un
compair aux autres pairs, un juge aux grands vassaux, etc., pour être
dès lors et de là en avant reconnu pour tel. Que dans la suite il y ait
eu ce qu'on appelle érection, et postérieurement encore ce qu'on appelle
enregistrement, cela n'a point changé l'ancien usage. Il a toujours
fallu manifester le pair nouvellement érigé et l'installer dans son
office. Qu'on y ait joint ensuite des formalités nouvelles, un serment,
puis le même serment varié, remis après en son premier état, après cela
une information de vie et mœurs préalable, puis un changement dans cette
information sur la religion catholique, etc.\,; tout cela sont les
accessoires, les choses ajoutées, jointes, concomitantes, mais non pas
la chose même, la manifestation, l'installation qui subsiste toujours la
même, et qui n'est autre que ce que l'on connaît maintenant sous le nom
de première réception au parlement. C'est donc à cette première
réception qu'il faut recourir, comme à la suite, jusqu'ici non
interrompue et non contestée, de l'antiquité la plus reculée jusqu'à
nous, de ce qui a perpétuellement et constamment fixé l'ancienneté des
pairs de tous les âges, et non pas à des usages modernes qu'une sage
police peut avoir introduits, mais qu'elle n'a pu substituer à ce qui
est de toute antiquité la règle connue, et l'unique qui la pût être,
jusqu'à ces établissements nouveaux qui ont ajouté simplement des choses
extérieures, mais sans aucun changement, bien moins de destruction, de
la nature essentielle des choses. En voilà assez pour faire entendre
combien la prétention de M. de La Rochefoucauld sur la priorité de
vérification ou d'enregistrement, qui est la même chose, est destituée
de fondement. Il faut montrer ensuite combien l'est, s'il se peut,
moins\footnote{Saint-Simon veut dire que l'objection dont il \emph{va}
  parler est encore \emph{moins fondée} que la prétention dont il a été
  question dans la phrase précédente.} encore son objection du
changement inconnu du rang des pairs par date de chaque réception, en
même pairie, si la fixation du rang d'ancienneté avait lieu de la
première réception au parlement. C'est ce que M. de La Rochefoucauld
prévit qui lui serait répondu là-dessus, qui lui donna tant
d'éloignement de procéder au parlement, et qui par autorité d'âge et de
faveur lui fit emporter une manière de juger qui aurait pu être bonne en
soi, mais qui n'avait point d'exemple, et que l'intérêt du parlement de
juger ces causes majeures aurait certainement rendue caduque.

On ne peut s'empêcher de remarquer l'indécence, dans la bouche d'un pair
de France, de cette proposition que M. de La Rochefoucauld avance en
conséquence du faux principe qu'il avait posé et dont on vient de
démontrer la faiblesse, que, si l'ancienneté parmi les pairs se tirait
de la première réception au parlement, elle changerait à chaque mutation
dans la même pairie par les diverses dates des diverses réceptions. Son
principe de la date de l'enregistrement tombé pour la fixation de
l'ancienneté, la conséquence tombe aussi. On vient de voir que c'est la
manifestation du nouveau pair qui, dès la première antiquité, a toujours
fixé l'ancienneté parmi eux. Cette manifestation n'est qu'une pour
chaque race et filiation de pair, puisque la dignité est héréditaire,
conséquemment les réceptions subséquentes de chaque filiation ne sont
plus la manifestation, mais seulement la succession annoncée et
manifestée dans le premier de la race\,; laquelle ne peut intervertir le
rang établi de la même pairie, qui demeure dans le rang qu'a tenu le
premier de cette filiation. Cela est évident en soi, cela l'est par
l'exécution constante depuis la première antiquité jusqu'à présent\,;
cela l'est encore, parce que, dans ce grand nombre de chimères et de
prétentions mises en avant de temps en temps sur les rangs entre eux des
pairs et la succession à cette dignité, M. de La Rochefoucauld est le
premier et l'unique qui ait imaginé cette intervention des rangs par
chaque réception dans la même pairie, conséquence insoutenable et
monstrueuse d'un principe destitué de tout fondement, de laquelle on va
démontrer l'ineptie encore plus singulièrement, c'est-à-dire par les
principes et par la nature de la dignité de duc et pair de France.

On ne peut lui contester qu'elle ne soit, par sa nature singulière et
unique, une dignité mixte de fief et d'office. Le duc est grand vassal,
le pair est grand officier. L'un a toute la réalité de mouvance nue de
la couronne, de justice directe, etc.\,; l'autre toute la personnalité,
ou les fonctions au sacre, au parlement, etc.\,; tous deux ont un rang,
des honneurs, etc. C'est ce mixte qui constitue une dignité unique, qui
sans l'office ne pourrait être distincte des ducs vérifiés\,; sans le
fief, des officiers de la couronne\,; et qui pour le fief et pour
l'office a ses lois communes avec les autres grands fiefs et grands
offices, et ses lois aussi particulières à elle-même\,; fief et office
également parties intégrantes et constituantes, sans lesquelles la
dignité ne pourrait exister, ni même être conçue, conséquemment de même
essence, qui opèrent en l'un\,; plénitude nécessaire de mouvance, en
l'autre plénitude nécessaire de fonctions. À tous les deux rangs et
honneurs qui en font parties décentes, non intégrantes, suites et
accompagnements qui ont été de tout temps attachés à la dignité, mais
qui ne la constituent pas, si bien que sans cela elle pourrait exister,
et être conçue. Telles sont les lois de la dignité en elle-même, avec
plusieurs autres qui ne font rien à la question dont il s'agit. Ces lois
communes avec les autres grands fiefs sont l'enregistrement depuis qu'il
est établi pour constater la dignité, et en assurer la possession à
l'impétrant et à sa postérité au désir des lettres avec les autres
grands offices, d'être reçu publiquement au serment de l'office, et
d'entreprendre une actuelle possession avec les formalités établies. La
dignité de duc et pair, quelque immense qu'elle soit dans l'État par sa
nature, n'a point de dispense là-dessus pour le fief ni pour l'office,
et M. de La Rochefoucauld, qui le prétendrait en vain, ne peut
disconvenir, à l'égard de l'office, de ce qu'il soutient à l'égard du
fief. De là il résulte qu'ayant accompli la loi quant au fief, il s'est
assuré et à sa postérité la dignité du fief en entier, et la faculté de
l'office\,; mais, quant à celui-ci, il est demeuré à la simple faculté
jusqu'à l'accomplissement par lui de la loi, imposée de tout temps à
tout officier pour tout office, d'y être reçu par le serment, et la
prise de possession personnelle, essentiellement requis, qui l'en
investit, qui le déclare et le manifeste officier. Les formalités plus
ou moins anciennes ou variées qui accompagnent la réception n'en sont
que les concomitances, et n'en changent point la nature\,; et c'est
cette réception qui dans tous les âges a fixé le rang des pairs entre
eux, qui sans interruption, s'y sont accordés depuis les premiers temps
jusqu'aux nôtres. De cette explication il résulte qu'avoir accompli la
loi des fiefs par l'enregistrement\,; et non celle des offices par la
réception, ce n'est point être en possession, ni avoir rendu en soi
entière et complète une dignité mixte de fief et d'office qui tient de
l'un et de l'autre son existence en toute égalité, conséquemment que le
rang de cette dignité, quoique assurée, ne peut être fixé en cet état,
et ne l'est point\,; d'où il se démontre que celui qui, postérieurement
à l'accomplissement de l'une de ces lois, et antérieurement à
l'accomplissement de l'autre\,; les a, lui, accomplies toutes les deux,
que celui-là, dis-je, a rendu sa dignité entière et complète en lui,
qu'il est grand officier avant l'autre, grand vassal même avant l'autre,
puisque tous deux n'ayant point été faits séparément ducs, séparément
pairs, par deux érections différentes et distinctes, mais ducs et pairs
chacun par une seule et même érection, cet autre tout enregistré qu'il
est, ne peut être valablement et réellement grand vassal qu'il n'ait
fait ce qu'il faut pour être aussi grand officier, puisqu'il est fait
l'un et l'autre ensemble par une seule et même dignité mixte de grand
fief et de grand office, dont le fief et l'office ensemble et par
indivis forment ensemblement l'existence, en sont également,
conjointement, concurremment parties intégrantes, tellement que sans ces
deux choses achevées également et accomplies suivant leurs lois, il ne
se peut dire qu'aucune d'elles le soit véritablement et par effet.
Venons maintenant à la prétendue difficulté, proposée par M. de La
Rochefoucauld, du changement de rang d'ancienneté des pairs de même
pairie, suivant la date des réceptions, successives de ces pairs au
parlement\,; et traitons-la expressément, quoique idée toute neuve qui
doit tomber de soi-même par ce qui vient d'être expliqué, et répudiée
par M. de La Rochefoucauld, même avant de l'avoir imaginée, par tout ce
qu'il a énoncé avec nous, contre les duchés-pairies femelles, sur la
manière de succéder à la dignité de duc et pair. Un seul mot tranche la
difficulté. C'est qu'à l'office de pair est appelé non-seulement
l'impétrant, mais avec lui, par une seule et même vocation, tous ses
descendants masculins à l'infini, tant et si longtemps que la race en
subsiste, au lieu qu'à tous autres offices, quels qu'ils soient, une
seule personne est appelée, et nulle autre avec elle\,; et c'est la
distinction essentielle et par nature de l'office de pair de tous les
autres offices de la couronne, et autres tous tels qu'ils soient, en
France sans aucune exception. De la suit invinciblement, par droit tiré
de la nature de la chose et confirmé par l'usage de tous les temps
jusqu'à aujourd'hui, que c'est cette première réception qui fixe le rang
d'ancienneté pour tous ceux qui, par la vocation, y sont successivement
appelés, auquel la réception subséquente de chacun d'eux ne peut
apporter d'interversion. Pour s'en convaincre, il n'est besoin que de se
souvenir de ce qui a été expliqué. La manifestation ou installation des
pairs dans leur office est ce qui a fixé leur ancienneté avant qu'il y
eût érection, enregistrement, tribunal enregistrant. C'est donc, comme
on l'a vu, pour ne rien répéter, ce qui l'a dû fixer depuis, et ce qui
l'a aussi toujours fixée sans aucun exemple ni prétention contraire. La
fixant pour l'impétrant, il la fixe dans lui et par lui à toute sa
postérité appelée avec lui, installée, reconnue, manifestée avec lui
d'une manière également invariable et unique à cet office, à la
différence de tous autres, en sorte que tout est consommé pour tous les
héritiers successifs de la même pairie. Cet essentiel accompli, il reste
des formalités à faire à chaque héritier de la même pairie, mais
formalités simples, qui ne sont rien moins que l'essence de la dignité,
mais des choses uniquement personnelles, ajoutées, changées, variées en
divers temps pour s'assurer si l'héritier, pair de droit et de fait
indépendamment de tout cela, est personnellement capable d'en exercer
les fonctions. Ainsi le serment, l'information de vie et mœurs, et les
autres formalités qui lui sont personnellement imposées, ne peuvent
changer son rang d'ancienneté, puisque aucunes ne lui confèrent rien de
nouveau, que toutes en sont incapables, et qu'elles ne sont ajoutées que
pour s'assurer d'un exercice digne en sa personne de ce qu'il ne reçoit
pas de nouveau, mais de ce qu'il a en lui essentiellement, et d'une
manière inhérente. Telle est donc la nature singulière et unique de la
dignité de pair de France, dont l'office est un et le même dans toute
une postérité appelée, et qui par conséquent ne peut changer de rang
d'ancienneté première de l'impétrant de qui elle sort, à la différence
de tous ceux de la couronne et de tous autres offices et officiers quels
qu'ils soient en France, qui, n'étant appelés qu'un seul à la fois à un
office, changent de rang d'ancienneté à chaque mutation de personne, par
une conséquence nécessaire. Je pense avoir expliqué la question avec une
évidence qui dispense de s'y arrêter davantage. Suivons-en maintenant la
décision en reprenant l'édit.

Quelques jours d'un temps si vif se passèrent en langueur par
l'interruption du travail du roi avec le chancelier. Je tâchai de
profiter de ce loisir auprès de lui\,; et comme la séparation de lieu,
et ses occupations, que j'ai remarquées ailleurs, rendaient le commerce
incommode, je lui écrivis de Marly, le 11 mai, la lettre suivante. Pour
l'entendre, il faut dire que l'anniversaire de Louis XIII se faisait
tous les ans à Saint-Denis, comme il se fait encore, et qu'à l'exemple
de mon père je n'y ai jamais manqué. Il fut avancé au 13 mai cette
année, parce que l'Ascension tombait au 14, son jour naturel.

«\,Jamais, monsieur, l'anniversaire du feu roi ne me vint si mal à
propos, encore qu'il m'ait fait forcer une fois la fièvre actuelle, une
autre le commencement d'une rougeole, et une troisième un bras tout
ouvert. À cette fois, il faut encore que le bienfaiteur l'emporte sur le
bienfait, et je porterai à Saint-Denis un cœur incisé et palpitant.
Cette dernière violence ne me sera pas la moins sensible, mais c'est un
hommage trop justement dû. Si je m'en croyais, je partirais tard demain
et passerais à Versailles\,; mais je me défie de ces hasards qui
découvrent tout, et, en attendant jeudi, j'ose vous demander quatre
lignes de mort ou de vie, demain au soir, pour remercier Dieu ou pour
demander justice à mon maître de son fils. Sauvez-nous le sacre, nos
plus sensibles entrailles, de préférence à tout\,; puis souvenez-vous de
faire passer le projet avec le plus de mes notes qu'il se pourra\,;
\emph{deinde}, du point de la séance des pères et des fils
conjointement, et en l'absence l'un de l'autre\,; enfin de mon fait
particulier, pour lequel vous avez une lettre ostensible, une analyse de
ce mémoire ostensible, enfin des éclaircissements de l'un et de l'autre
encore ostensibles\,; car le mémoire même serait trop long pour être
montré, et une seconde lettre en supplément de mémoire. Souvenez-vous
encore avec bonté que ma cause dépend de l'autorité royale que j'ai mise
de mon côté par un raisonnement en soi véritable, et que le juge ne
considérera pas comme étranger au fait, bien qu'il le soit, mais comme
le seul motif de décision\,; et n'oubliez pas que vous croyez que, si on
s'obstine contre moi, un dédommagement pour moi dans mon second fils
peut ne pas être regardé comme bien solide à espérer, mais ne doit pas
aussi être regardé comme une chimère à n'oser proposer. Après tout cela,
ne serait-ce point outrecuidance de vous remémorer Chaulnes en nouvelle
érection, par amitié vôtre, non par votre propre persuasion\,?
Pardonnez-moi, monsieur, toutes ces redites, vous qui savez et possédez
trop mieux tous les points que je range ici, selon mon désir, les uns de
préférence aux autres, suivant que je les ai mis. L'assignation à demain
(du travail décisif avec le roi) me donne le frisson et la sueur. J'en
dis pour mon âme, avec toute la résignation que je puis, mon \emph{In
manus} à Dieu, et je vous le dis à vous, monsieur, pour cette dignité,
squelette le plus chéri et le plus précieux de tous biens que je tienne
des libéralités royales. Après tout, il n'y a qu'à s'abandonner à la
volonté de Dieu, à vos nerveux et vifs raisonnements, aux effets de la
grâce ou de la nature, et, quoi qu'il en arrive, à une reconnaissance et
un dévouement pour vous, monsieur, que ces occasions uniques me font
sentir qui peuvent s'enfoncer, s'il se pouvait, plus avant que le cœur.
Pour le secret, il est, monsieur, et sera entier.\,»

Au sortir d'avec le roi, le lendemain 12, le chancelier m'écrivit ce
billet\,: «\, Je ne puis encore vous tirer des limbes aujourd'hui,
monsieur. Supportez vos ténèbres encore quelques jours\,; mais
supportez-les avec espérance d'en sortir bientôt avec avantage\,; et, si
le soleil ne vous paraît pas aussi favorable que vous le voudriez, vous
aurez tort, si je ne me trompe, et très-grand tort. Je suis à vous,
monsieur, mais à condition que vous n'aurez aucun tort.\,»

Deux jours après, je retournai à Marly par Versailles, c'est-à-dire le
samedi, où je vis le chancelier à mon aise. Là j'appris que mon mémoire
sur l'autorité du roi l'avait ramené à mon point, et que la fixation du
rang serait réglée à la réception de l'impétrant et non plus à
l'enregistrement des lettres\,; ainsi, après avoir perdu ma cause sur
des raisons invincibles pour moi, qui ne purent ni faire d'impression ni
trouver de réponse, je la gagnai sur d'autres tout à fait ineptes à ce
dont il s'agissait, mais qui remuèrent le premier mobile du juge, et
voilà ce que sert d'être bien averti et servi. Je rendis mille grâces au
chancelier, qui ouvrit la conversation par là, apparemment pour me
calmer sur le reste, et ce ne fut pas sans réflexions sur les motifs des
jugements. Il me dit ensuite que la double séance du père et du fils,
même ensemble, avait enfin passé après de grands débats, en
considération de la nouvelle faveur à la postérité légitimée. Ce point
me fit encore plaisir. Le venin fut à la queue, je veux dire le point du
sacre, sur lequel le chancelier m'assura avoir insisté de toutes ses
forces, mais vainement\,; la considération des bâtards seule ayant fait
tenir ferme au roi. Alors je sentis bien que c'était une affaire conclue
et sans nulle espérance de retour, et, après les premiers élans que je
ne pus arrêter, je contraignis le reste pour éviter des remontrances
là-dessus insupportables. Les articles des femelles, des ayants cause,
etc., ceux de la substitution et du rachat par les mâles tels que nous
les avions projetés, et Chaulnes favorablement résolus, je m'informai
après des raisons pour lesquelles le règlement demeurait encore secret.
Le chancelier m'avoua qu'il n'en devinait aucune, ayant vu la chose dix
fois prête à éclore, sinon que le roi avait peut-être dessein de faire
voir ce projet au duc du Maine, avant qu'il fût déclaré, pour être en
état d'y changer, si ce cher fils y trouvait quelque chose encore à
désirer. Cela même me fit grand'peine pour ce peu qui s'y trouvait de
bon. Je pressai le chancelier de finir cette affaire dès qu'il y verrait
le moindre jour\,; et je regagnai Marly, pénétré du sacre et en grand
soupçon de la double séance, et en repos sur mon affaire particulière
par la raison qui me la faisait gagner après l'avoir perdue.

Arrivé à Marly, je ne pus me contenir de confier au duc de Beauvilliers,
dont je connaissois le profond secret, celui qui lui causerait tant de
joie. Il était déjà couché. J'ouvris son rideau et lui dis, sous le
secret dont j'étais si sûr avec lui, que son neveu allait être fait duc
et pair. Il en tressaillit de joie. Il me parut comblé de la mienne et
de la part que j'avais eue en une affaire qu'il désirait si fort, mais
dont aussi il ne connaissoit pas moins que moi le peu de fondement,
comme il me l'a souvent avoué devant et après. Je ne voulus lui confier
rien du reste qui ne le touchait pas si précisément, et j'allai écrire à
M\textsuperscript{me} de Saint-Simon, qui était encore à Paris. Dès le
lendemain matin, elle envoya prier la duchesse de Chevreuse, notre
très-proche voisine, de venir chez elle. Elle la transporta de la plus
sensible joie et de la plus vive reconnaissance pour moi, en lui
apprenant le comble de ses désirs, sous un secret entier, excepté pour
le duc de Chevreuse, qui ne tarda pas à venir lui en témoigner autant.

Cependant la mine commença à s'éventer sur le règlement. J'en fus en
peine pour la chose en elle-même, et plus encore sur mon compte
particulier avec le chancelier\,; mais le roi avait parlé à d'Antin, et
celui-ci à d'autres, comme nous le vérifiâmes presque aussitôt.
Là-dessus grands mouvements de Matignon et de toute sa séquelle. Le
mariage de son fils unique, infiniment riche, était arrêté avec une
fille du prince de Rohan, moyennant qu'il fut duc d'Estouville, et les
Rohan ne s'y épargnèrent pas. Je craignis d'autant plus ce contre-temps
que, le 17 mai, rien ne se déclara, quoique le chancelier eut encore
travaillé avec le roi, et à ce qu'il m'avait dit pour la dernière fois.
L'inquiétude me fit lui écrire ce mot de Marly à Versailles\,: «\,Vous
êtes demeuré seul, monsieur, un quart d'heure avec le roi après le
conseil, et vous n'êtes pas demeuré pour un autre, cette après-dînée,
qui a duré une heure et demie, et qui a rompu chasse, chiens et vêpres.
Les affaires d'État, je les respecte et m'en distrais\,; les autres qui
se devaient déclarer aujourd'hui me poignent par leur silence.
M\textsuperscript{me} de Ventadour aurait-elle tout troublé hier avec
son inepte Estouteville, ou le roi veut-il que l'enregistrement soit
fait pour le général avant de rien déclarer\,? enfin, monsieur, a-t-on
changé en tout ou en partie, et ces limbes perpétuelles
s'invoqueront-elles toujours successivement\,? Pardonnez-moi, s'il vous
plaît, toutes ces questions\,; mais, sachez, s'il vous plaît, que M. de
La Rocheguyon et MM. de Cheverny et de Gamaches m'ont parlé aujourd'hui
d'un règlement prêt à éclore pour couper court à toute prétention, et
d'Antin à la queue, à quoi j'ai répondu avec une ignorance naturelle.
Cependant il faut bien que quelqu'un ait parlé, et je me flatte que vous
croyez bien que ce n'est pas moi. Personne ne parle du détail, mais
seulement en gros. Je vais demain après dîner à Paris, et je serai à la
torture si vous n'avez pitié de moi par quatre lignes. Je me prépare à
tout, et suis à vous, monsieur, avec tout dévouement possible.\,»

Ce billet me fut renvoyé sur-le-champ avec cette réponse sur la feuille
à côté.

«\,Demeurez en repos, monsieur, tout est remis à mardi. Ce qu'on a
changé aujourd'hui est peu de chose. Les grands principes subsistent
toujours\,: rien de tout ce que vous faites entrer dans le délai n'y
entre. Il faut se déterminer. On veut et on ne veut pas, et voilà tout.
J'ignore le sujet, le détail et le résultat du conseil dont vous me
parlez, monsieur. Je ne m'étonne point que ces messieurs vous aient dit
ce qu'ils vous ont dit. Cela n'est que trop public. L'essentiel est que
le détail s'ignore, car il blesserait sans doute autant que le gros est
indifférent. Je suis tout à vous, monsieur.\,»

Soit dit en parenthèse qu'un courrier d'Angleterre, arrivé pendant le
dîner du roi et après le départ du chancelier, fit rassembler le conseil
sans lui, auquel le roi fit lire au conseil suivant la dépêche et la
réponse. Telle était l'incommodité de Marly.

Ce 17 susdit était un dimanche, jour de conseil d'État. Le lundi se
passa en inquiétude de ma part sur ce peu de chose que le chancelier
m'avait mandé avoir été changé. Son langage m'avait appris que peu de
chose en cette matière était beaucoup. Le mardi 19, jour de conseil de
finances, et le premier après celui du dimanche, un quart d'heure de
tête-à-tête du chancelier avec le roi mit la dernière main à l'édit. Le
chancelier le fît mettre en forme aussitôt après à Versailles, l'y
scella et l'envoya au parlement, où il fut enregistré le surlendemain,
jeudi 21 mai. J'allai trouver le chancelier à Versailles, de qui
j'appris que ce peu de chose qu'il m'avait mandé avoir été retranché
était\,: la double séance des pairs démis et Chaulnes\,; que le roi,
après avoir accordé l'un et l'autre, n'avait pu enfin se résoudre à la
double séance, et que, prêt à lâcher le mot sur Chaulnes, comme il
l'avait résolu avec le chancelier, il avait payé de propos, d'espérance
certaine, mais sans avoir pu être persuadé de passer outre actuellement.
Le dernier billet du chancelier m'avait fait douter de la double
séance\,; j'y étais préparé. Je ne l'étais point au délai en l'air de
Chaulnes, et j'en fus d'autant plus fâché que j'y avais plus compté, et
que j'en avais donné la joie à M. de Beauvilliers, et fait donner par
M\textsuperscript{me} de Saint-Simon à M. et M\textsuperscript{me} de
Chevreuse. Les arrangements de M. de Chevreuse lui ont coûté cher plus
d'une fois. S'il avait été à Marly, son affaire y serait sûrement finie,
comme je sus bien le lui reprocher vivement. Je ne repondrais pas que la
pique du roi sur ses absences ne lui ait valu ce tire-laisse\footnote{Voy.
  cet édit dans la collection des \emph{Anciennes lois françaises}, par
  Isambert, t. XX, p.~565-569.}. Il est certain que, depuis que la chose
fut accordée en travaillant avec le chancelier, elle ne balança plus,
mais le roi se plut à faire durer cette inquiétude, et à la pousser
quelques mois. L'édit fit, à l'ordinaire, le bruit et la matière des
conversations que font les choses nouvelles\,; nous y perdions trop pour
être contents, nous y gagnions trop pour montrer du chagrin, et sur
chose qui touchait si personnellement le roi, et qui était faite, notre
parti fut une sagesse sobre, modeste et peu répandue en propos, ni même
en réponse. Le chancelier content au dernier point de son édit, trouvait
que je le devais être, parce que j'y gagnais deux procès en commun, et
un en particulier\,; mais aucun gain ne pouvait me compenser les deux
premiers articles. L'édit est entre les mains de tout le
monde\footnote{Voy. cet édit dans la collection des \emph{Anciennes lois
  françaises}, par Isambert, t. XX, p.~565-569.}, ainsi je l'ai omis
parmi les Pièces.

J'allai faire mon compliment à d'Antin. Je ne sais si le changement de
la face de la cour, par la mort de Monseigneur, lui fit quelque
impression à mon égard, quoique, dès l'introduction de l'affaire, il
m'eût parlé avec des politesses qui allèrent aux respects, il me les
prodigua en cette visite. Il ne tarda pas à profiter de la grâce qu'il
avait su si habilement se procurer. Il fut enregistré et reçu au
parlement le même jour 5 juin suivant. Il donna ensuite un grand dîner
chez lui, où il n'y eut qu'une quinzaine de personnes d'invitées, hommes
et femmes de sa famille ou de ses plus particuliers amis. Charost et moi
y fûmes les deux seuls étrangers, encore Charost avait-il toujours vécu
avec lui à l'armée. Il s'en fallait tout, comme on l'a vu, que j'en
fusse là avec lui. Non content de m'envoyer prier chez moi, de m'en
prier lui-même dans le salon à Marly, il m'en pressa encore tellement au
parlement\,; pendant la buvette, qu'il n'y eut pas moyen de l'éviter. Il
me fit les honneurs du repas et de sa maison avec une attention
singulière\,; et, de retour à Marly, je m'aperçus aisément, aux
gracieusetés que le roi chercha à me faire, que je lui avais fait ma
cour d'avoir été de ce dîner. Le favori mit son duché-pairie sur sa
terre d'Antin. En courtisan leste et délié, il dit que ce nom lui était
trop heureux pour le changer. Il pouvait ajouter, quoique de bien autre
naissance que le favori d'Henri III, que ce nom d'Épernon qu'il avait
rendu si grand et si célèbre, lui serait et aux siens trop difficile à
soutenir. Il fit un trait d'imprudence au delà de tous les Gascons\,: il
osa prier le maréchal de Boufflers d'être l'un de ses témoins. Le
maréchal en fut piqué, sans oser refuser une chose qui ne se refuse
point, mais il ne voulut point signer le témoignage banal qu'on lui
apporta. Il en fit un qu'il me montra pour lui en dire mon avis. J'y
admirai comment la vertu supplée à tout. Sans rien de grossier, il ne
s'y rendit coupable d'aucun mensonge\,; et j'ai toujours eu envie d'en
avoir une copie, tant il m'avait plu.

Matignon fut au désespoir. Il s'était mis la chimère d'Estouteville dans
la tête, qu'il espérait faire réussir par le mariage de son fils avec
une fille du prince de Rohan\,; il n'y en avait point de si folle, je me
contente de ce mot parce qu'il n'en fut question que dans leur projet.
Cela seul lui avait fait entreprendre un grand succès contre la duchesse
de Luynes. Il le perdit sans perdre son dessein de vue\,; et il était
entré en accommodement pour faire en sorte que la terre d'Estouteville
lui demeurât, en payant cher la connivence. C'était cette affaire prête
à conclure qui avait empêché M. de Chevreuse d'aller à Marly. Il nous
donnait un procès par cet accommodement auquel l'édit coupa pied, mais
il était ami des chimères de cette sorte, et il trouvait un grand profit
dans cet accommodement. Sa lenteur ordinaire, et ses demandes énormes au
gré de Matignon, avaient traîné l'affaire qu'aucun des deux ne voulait
rompre\,; l'un par intérêt pécuniaire, l'autre par intérêt d'ambition\,;
tous deux espéraient de se faire venir l'un et l'autre à son point. Avec
ces pourparlers l'affaire languit jusqu'au temps de l'édit, et ne fut
conclue et signée que la surveille de sa déclaration. M. de Chevreuse
instruit par d'Antin, vit bien alors qu'il n'y avait plus de temps à
perdre\,; et Matignon, ravi d'avoir enfin d'Estouteville, et à meilleur
marché qu'il n'avait espéré, se hâta de finir\,: Trois jours après la
signature, il apprit l'édit et son contenu, qui lui ôtait toute
espérance du seul usage d'Estouteville, pour lequel il s'en était si
chèrement accommodé. Le voilà donc aux hauts cris. Il prétendit que le
duc de Chevreuse ne s'était pressé tout à coup de conclure que de peur
de n'y être plus à temps après l'édit, et qu'il était cruellement lésé
dans une affaire qu'il n'avait terminée que pour un objet connu à M. de
Chevreuse, et connu lors de la conclusion pour ne pouvoir plus être
rempli. M. de Chevreuse, à son ordinaire tranquille, sage et froid,
laissa crier et prétendit de son côté que Matignon y gagnait encore
pécuniairement ce qu'il avait bien voulu donner à la paix et à son
repos. Les Rohan, déçus de leurs espérances, retirèrent leur parole, qui
n'était donnée qu'au cas de succès de la chimère\,; et, honteux d'avoir
porté si publiquement l'intérêt de Matignon contre M. de Chevreuse, dont
ils étaient si proches, dans le procès que Matignon avait perdu, ne se
voulurent pas mêler de ses plaintes. La réputation si bien établie de M.
de Chevreuse énerva tout ce que Matignon voulut dire, et les immenses
richesses que ce dernier avait tirées de l'abandon d'amitié de
Chamillart pour lui rendirent le monde fort dur sur sa mésaventure.

Un mois après l'enregistrement de l'édit, le chancelier me manda qu'il
serait bien aise de m'entretenir sur une visite qu'il avait reçue du duc
de La Rocheguyon. Il s'était plaint à lui amèrement, au nom de M. de La
Rochefoucauld et au sien, de la décision que l'édit faisait en ma faveur
sur notre question de préséance, et lui dit leur dessein d'en parler au
roi. Le chancelier lui objecta les arrêts de Bouillon et de La
Meilleraye en lit de justice, un édit récent, et le dessein du roi d'y
décider ce procès avec tous les autres. La Rocheguyon insista. Le
chancelier se tint couvert, mais sans lui dissimuler qu'il savait l'état
de la question. L'autre, dans le dessein d'en tirer au moins quelque
parti, glissa quelque chose tendant au même règlement qui subsiste entre
les ducs d'Uzès et de La Trémoille, chose inepte parce que nos pères
n'ont pas été séparément faits ducs et après pairs, comme ceux de MM.
d'Uzès et de La Trémoille. Il finit en soutenant sa pointe, et proposant
des écrits qu'il allait faire préparer. Le chancelier lui dit qu'il
était le maître, et reconduisit honnêtement. La chose en demeura là pour
lors. On en verra les suites en leur temps, qui ne réussirent pas à M.
de La Rocheguyon. Mais cette affaire, venue à la suite de la mort de la
duchesse de Villeroy, refroidit tout à fait l'amitié et le commerce
étroit qui avait été jusqu'alors entre les ducs de Villeroy, de La
Rocheguyon et moi. Il se réduisit peu à peu aux bienséances communes, et
en est toujours demeuré là depuis, jusqu'à leur mort longues années
après. M. de Luxembourg fit, à l'occasion de l'édit, un personnage dont
un peu d'esprit ou de mémoire lui aurait épargné la façon. On a vu que
le projet qui servit de base à l'édit avait été fait par le premier
président d'Harlay, de concert avec d'Aguesseau, depuis chancelier, et
avec le chancelier lors secrétaire d'État et contrôleur général\,; que
Harlay était le conseil, l'ami, pour ne pas dire l'âme damnée du
maréchal de Luxembourg, jusqu'à s'être déshonoré par la partialité
criante et publique dont les injustices les plus inconsidérées nous
forcèrent a sa récusation\,; enfin, que ce projet communiqué, par la
permission du roi, au maréchal de Luxembourg pour ce qui le regardait,
et à M. de Chevreuse, il y avait pleinement consenti, et ne l'avait pas
fait sans avoir bien sondé sa cause, et sans le conseil du premier
président d'Harlay. Le maréchal de Luxembourg vivait avec son fils dans
une union et une confiance peu commune, à laquelle ce fils répondait
pleinement, et cette intimité n'était ignorée de personne. Il avait donc
eu connaissance du projet en même temps que son père et que le duc de
Chevreuse son beau-père, dont la liaison avec eux était au plus intime,
et qui était leur conseil. Le fils avait le même intérêt que le père en
ce qui les regardait dans le projet, et son consentement avait été donné
avec le sien. Il était à Rouen lorsque l'édit fut résolu. Il y avait eu
du désordre pour les blés. Courson, intendant de Rouen, fils de Bâville,
en avait toute la hauteur et toute la dureté, mais il n'en avait pas
pris davantage. C'était un butor, brutal, ignorant, paresseux, glorieux,
insolent du crédit et de l'appui de son père, et surtout étrangement
intéressé. Ces qualités, dont il n'avait pas le sens de voiler aucune,
lui avaient révolté la province. La disette de blé, qui se trouva
factice et qui fut découverte, révolta la ville, qui se persuada que
Courson faisait l'extrême cherté pour en profiter, et qui, poussée à
bout par ses manières autant que par ses faits, et ayant manqué tout à
fait de pain plus d'une fois, s'en prit enfin à lui, et l'eût accablé à
coups de pierres s'il ne se fût sauvé de chez lui, et, toujours
poursuivi dans les rues, se sauva enfin chez le premier président.
Voysin et sa femme, amis de M. de Luxembourg dès la Flandre, saisirent
cette occasion de lui procurer l'agrément, devenu si rare à un
gouverneur de province, d'y aller faire sa charge. Voysin, dans la
première fleur de sa place et de sa faveur, l'obtint aisément. M. de
Luxembourg apparemment s'y trouva bien, ou voulut accoutumer le roi à le
voir en Normandie sans nécessité\,; il y demeura donc après que tout fut
apaisé, ce qui ne se put qu'en pourvoyant effectivement aux blés, et en
ôtant à Rouen et à la province un intendant aussi odieux. Un autre
aurait été chassé du moins, depuis que la robe met à couvert de toute
autre punition, mais le fils de Bâville eut un privilége spécial pour
désoler et piller de province en province. On l'envoya à Bordeaux, où il
se retrouvera.

Il faut encore se souvenir que, lorsque d'Antin commença son affaire, M.
de Luxembourg se joignit à nous contre lui, et qu'en même temps il
reprit contre nous la sienne qu'il avait laissée dormir depuis
longtemps, qui fut tout à la fois une bigarrure singulière. L'édit
résolu, le chancelier qui, amoureux de son ouvrage, le voulait rendre
autant qu'il était possible agréable à tout le monde, fit souvenir le
roi du consentement donné par feu M. de Luxembourg au projet, qui, par
rapport à lui, ne contenait que la même disposition de l'édit, et sur ce
principe lui proposa de lui permettre d'en écrire à celui-ci. Il ne se
rebuta point du refus qu'il reçut, et revint quelques jours après à la
charge, et l'emporta. Il écrivit donc à M. de Luxembourg, le plus
poliment du monde, pour lui faire bien recevoir la décision que son père
et lui avaient approuvée autrefois. Il fut huit ou dix jours sans
réponse. Le roi, impatient de savoir comment M. de Luxembourg avait pris
la chose, et qui n'avait permis cette communication qu'à regret, se
piqua du délai de réponse, et commanda au chancelier de récrire, et
sèchement. Celui-ci, fâché du reproche que cela lui attirait du roi,
obéit fort ponctuellement. M. de Luxembourg, que la première lettre
avait fort surpris, et embarrassé sur la réponse au point d'un si long
délai sans la faire, le fut bien plus de la recharge et du style dont il
la trouva. Il fallut pourtant répondre, mais il fut encore cinq ou six
jours à composer une lettre pleine de propos confus et de raisons
frivoles. Le chancelier en fut piqué au vif. Son honnêteté prodiguée, un
succès tout contraire à celui dont il n'avait pas douté, le reproche du
roi qui se fâcha à lui d'une communication inutile et qui tournait si
mal, mirent le maître et le ministre de mauvaise humeur. Le roi voulut
que le chancelier répliquât durement, qui n'eut aucune peine à exécuter
cet ordre. M. de Luxembourg qui, sans aucun esprit, était fort glorieux,
et sensible au dernier point, fut outré\,; il n'osa répondre du même
style. Son dépit redoubla à la vue de l'édit avec son nom dedans, et sa
cause à son gré perdue. Le monde n'en jugea pas de même\,; le
consentement de son père, avec qui sa considération était tombée, excita
un parallèle peu agréable, et on le trouva heureux de sortir de la sorte
d'un méchant procès qui pouvait lui coûter sa dignité de duc et pair de
Piney, et le réduire à la sienne de duc vérifié. La mort de Monseigneur
avait achevé de lui ôter sa considération. On a vu ailleurs à l'occasion
de l'éclat avec lequel M\textsuperscript{lle} Choin fut renvoyée par
M\textsuperscript{me} la princesse de Conti, à quel point de liaison
intime de cabale le père et le fils étaient avec elle, et avec Clermont
son amant qui en fut perdu. Cette liaison, qui avait toujours subsisté,
avait initié M. de Luxembourg dans tout auprès de Monseigneur, sous le
règne duquel il avait lieu de se promettre beaucoup\,; il était encore
dans la première douleur de la perte de toutes ses espérances, lorsque
cet édit acheva de l'affliger.

\hypertarget{chapitre-xii.}{%
\chapter{CHAPITRE XII.}\label{chapitre-xii.}}

1711

~

{\textsc{Grand changement à la cour par la mort de Monseigneur, et ses
impressions différentes.}} {\textsc{- Duc du Maine.}} {\textsc{- Duc du
Maine fort mal à Marly.}} {\textsc{- Princesse de Conti.}} {\textsc{-
Cabale.}} {\textsc{- Duc de Vendôme.}} {\textsc{- Vaudemont et ses
nièces.}} {\textsc{- M\textsuperscript{lle} de Lislebonne abbesse de
Remi-remont.}} {\textsc{- M\textsuperscript{me} la Duchesse.}}
{\textsc{- Prince de Rohan.}} {\textsc{- Princes étrangers.}} {\textsc{-
D'Antin.}} {\textsc{- Huxelles, Beringhen, Harcourt, Boufflers,
Sainte-Maure, Biron, Roucy, La Vallière.}} {\textsc{- Ducs de
Luxembourg, La Rocheguyon, Villeroy.}} {\textsc{- La Feuillade.}}
{\textsc{- Ministres et financiérs.}} {\textsc{- Le chancelier et son
fils.}} {\textsc{- La Vrillière.}} {\textsc{- Voysin.}} {\textsc{-
Torcy.}} {\textsc{- Desmarets.}} {\textsc{- Duc de Beauvilliers.}}
{\textsc{- Fénelon archevêque de Cambrai.}} {\textsc{- Union de M. de
Cambrai et de tout le petit troupeau.}} {\textsc{- Duc de Charost et sa
mère.}} {\textsc{- Duc et duchesse de Saint-Simon.}} {\textsc{- Conduite
des ducs de Chevreuse et de Beauvilliers.}} {\textsc{- Duc de
Chevreuse.}} {\textsc{- Mgr le Dauphin.}} {\textsc{-
M\textsuperscript{me} de Maintenon point aux ministres, tout au
Dauphin.}} {\textsc{- Ministres travaillent chez le Dauphin.}}

~

Jamais changement ne fut plus grand ni plus marqué que celui que fit la
mort de ce prince. Éloigné encore du trône par la ferme santé du roi,
sans aucun crédit, et par soi de nulle espérance, il était devenu le
centre de toutes les espérances et de la crainte par tous les
personnages, par le loisir qu'une formidable cabale avait eu de se
former, de s'affermir, de s'emparer totalement de lui, sans que la
jalousie du roi, devant qui tout tremblait, s'en mît en peine, parce que
son souci ne daignait pas s'étendre par delà sa vie, pendant laquelle il
ne craignait rien avec raison.

On a déjà vu les impressions si différentes qu'elle fit dans l'état et
dans le cœur du nouveau Dauphin et de son épouse, dans le cœur de M. le
duc de Berry et dans l'esprit de la sienne, dans la situation de M.
{[}le duc{]} et de M\textsuperscript{me} la duchesse d'Orléans, et dans
l'âme de M\textsuperscript{me} de Maintenon, délivrée pour le présent de
toute mesure et de toute épine pour l'avenir.

M. du Maine partagea de bon cœur ces mêmes affections avec son ancienne
gouvernante, devenue la plus tendre et sa plus abandonnée protectrice.
Foncièrement mal, de tout temps, comme on l'a dit, avec Monseigneur, il
avait violemment tremblé de la manière dont on a vu que ce prince avait
reçu les divers degrés de son élévation, et en dernier lieu surtout
celle de ses enfants. Il était loin d'être rassuré là-dessus du côté du
nouveau Dauphin et de M\textsuperscript{me} la Dauphine, mais un et un
sont deux. Délivré de tous les princes du sang en âge et en maintien,
dont il avait su sitôt et si grandement profiter, Monseigneur de moins,
et possédé par M\textsuperscript{me} la Duchesse, lui fut un soulagement
dont il ne prit pas même la peine de cacher l'extrême contentement. Il
avait de trop bons yeux pour ne s'être pas aperçu que
M\textsuperscript{me} la Dauphine n'ignorait rien de la protection qu'il
avait prodiguée au duc de Vendôme sur tout ce qui s'était passé en
Flandre, pour ne pas sentir que les maximes du nouveau Dauphin lui
faisaient penser sur la grandeur qu'il s'était formée, et qu'il ne
captiverait pas aisément par ses souplesses ceux qui pouvaient, et qui,
selon toute apparence, pourraient le plus sur lui, mais la santé du roi
lui faisait espérer encore un long terme de son aveuglement pour lui,
pendant lequel il pourrait arriver de ces heureux hasards qui mettent le
comble à la fortune. L'esprit léger de M. le duc d'Orléans lui parut
moins un obstacle qu'une facilité à en tirer parti d'une façon ou d'une
autre. Celui de M. le duc de Berry n'était pas pour l'inquiéter, mais il
résolut de n'oublier rien pour ne trouver pas une ennemie dans
M\textsuperscript{me} la duchesse de Berry, et il la cultiva avec
adresse.

Il commençait à goûter un si doux repos, lorsque, surpris peu de jours
après, à Marly, d'un mal étrange, dans la nuit, son valet de chambre
l'entendit râler et le trouva sans connaissance. Il cria au secours.
M\textsuperscript{me} la duchesse d'Orléans accourut en larmes\,;
M\textsuperscript{me} la Duchesse et M\textsuperscript{lle}s ses filles
par bienséance, et beaucoup de gens pour faire leur cour, dans
l'espérance que le roi saurait leur empressement. M. du Maine fut
saigné, et accablé de remèdes parce qu'aucun ne réussissait. Fagon, à
qui deux heures à peine suffisaient pour s'habiller par degrés, n'y vint
qu'au bout de quatre, à cause de sa sueur toutes les nuits. Il était
celui de tous le plus nécessaire en cette occasion, parce qu'il
connaissoit ce mal par sa propre expérience, quoique jamais si rudement
attaqué. Il gronda fort de la saignée et de la plupart des remèdes.

On tint conseil si on éveillerait le roi, et il passa que non, à la
pluralité des voix. Il apprit à son petit lever toutes les alarmes de la
nuit, qui étaient déjà bien calmées\,; il alla voir ce cher fils dès
qu'il fut habillé, et y fut deux fois le jour pendant, les deux ou trois
premiers, et une ensuite tous les jours, jusqu'à ce qu'il fût tout à
fait bien.

M\textsuperscript{me} du Maine était cependant à Sceaux, au milieu des
fêtes qu'elle se donnait. Elle s'écria qu'elle mourrait, si elle voyait
M. du Maine en cet état, et ne sortit point de son palais enchanté. M.
du Maine, accoutumé à en approuver tout servilement, approuva fort cette
conduite et l'alla voir à Sceaux dès qu'il put marcher.

M\textsuperscript{me} la princesse de Conti fut celle qui regretta le
plus Monseigneur, et qui y perdit le moins. Elle l'avait possédé seule
et avec empire fort longtemps. M\textsuperscript{lle}s de Lislebonne,
qui ne bougeaient de chez elle, l'avaient peu à peu partagé, mais avec
de grandes mesures de déférence. Le règne de M\textsuperscript{lle}
Choin avait tout absorbé ce qui était resté à sa maîtresse, pour qui
Monseigneur ne conserva que de la bienséance accompagnée d'ennui et
souvent de dégoût, que l'amusement qu'il trouva chez
M\textsuperscript{me} la Duchesse ne fit qu'accroître.
M\textsuperscript{me} la princesse de Conti n'était donc de rien depuis
bien des années, avec l'amertume de savoir M\textsuperscript{lle} de
Lislebonne, sa protégée et son amie, en possession des matinées libres
de Monseigneur, chez elle dans un sanctuaire scellé pour tout autre que
M\textsuperscript{me} d'Espinoy, où se traitaient les choses de
confiance\,; M\textsuperscript{lle} Choin, son infidèle domestique,
devenue la reine du cœur et de l'âme de Monseigneur, et
M\textsuperscript{me} la Duchesse intimement liée à elles, en tiers de
tout avec elles et Monseigneur qu'elle possédait chez elle en cour
publique. Il fallait fléchir avec toutes ces personnes, ne rien voir,
leur plaire\,; et malgré ses humeurs, sa hauteur, son aigreur, elle s'y
était ployée, et fut assez bonne pour être si touchée, qu'elle pensa
suffoquer deux ou trois nuits après la mort de Monseigneur, en sorte
qu'elle se confessa au curé de Marly.

Elle logeait en haut au château. Le roi l'alla voir. Le degré était
incommode\,; il le fit rompre pendant Fontainebleau, et en fit un grand
et commode. Il y avait plus de dix ans qu'il n'avait eu occasion de
monter à Marly, et il fallait de ces occasions uniques pour lui faire
faire l'essai de ce nouveau degré.

M\textsuperscript{me} la princesse de Conti guérit à nos dépens. Nous
avions le second pavillon du coté de Marly fixe, le bas pour nous, le
haut pour M. et M\textsuperscript{me} de Lauzun. Il est aussi près du
château que le premier et n'en a pas le bruit\,; on nous y mit pour
donner le second à M\textsuperscript{me} la princesse de Conti seule
avec sa dame d'honneur. Quoique ennemie de l'air et de l'humidité, elle
le préféra à son logement du château pour s'attirer plus de monde par la
commodité de l'abord, et y tint depuis ses grands jours avec la
vieillesse de la cour qu'elle y rassembla, et qui, faute de mieux, et
par la commodité d'un réduit toujours ouvert, s'y adonna toute.

On jugera aisément du désespoir et de la consternation de cette
puissante cabale, si bien organisée, que l'audace avait conduite aux
attentats qu'on a rapportés. Quoique l'héritier de la couronne qu'elle
avait porté par terre se fût enfin relevé, et que son épouse, unie à
M\textsuperscript{me} de Maintenon, se fût vengée de l'acteur principal
d'une scène si incroyable, la cabale se tenait ferme, gouvernait
Monseigneur, ne craignait point qu'il lui échappât, l'entretenait dans
le plus grand éloignement de son fils et de sa belle-fille, dans le
dépit secret de la disgrâce de Vendôme, se promettait bien de monter sur
le trône avec lui, et d'en anéantir l'héritier sous ce règne. Dieu
souffle sur les desseins\,; en un instant il les renverse, et les
asservit sans espérance à celui pour la pente duquel ils n'avaient rien
oublié ni ménagé. Quelle rage, mais quelle dispersion\,! Vendôme en
frémit en Espagne, où il ne s'était jeté qu'en passant. De ce moment il
résolut d'y fixer ses tabernacles, et de renoncer à la France après ce
qu'il avait attenté, et ce qui l'en avait fait sortir. Mais la guerre,
par où il comptait de se rendre nécessaire, n'était pas pour durer
toujours. Le Dauphin et le roi d'Espagne s'étaient toujours tendrement
aimés\,; leur séparation n'y avait rien changé\,; la reine d'Espagne,
qui y pouvait tout, était sœur de son ennemie et intimement unie avec
elle\,; le besoin passé, son état pouvait tristement changer\,; sa
ressource fut de se lier le plus étroitement qu'il put à la princesse
des Ursins et de devenir son courtisan, après avoir donné la loi à nos
ministres et à notre cour. On en verra bientôt les suites.

Le Vaudemont se sentit perdu. Moins bien de beaucoup auprès du roi
depuis la chute de Chamillart, il ne lui restait plus de protecteur.
Torcy ne s'était jamais fié à lui, et Voysin n'avait jamais répondu que
par des politesses crues à toutes les avances qu'il lui avait
prodiguées. Il était sans commerce étroit avec les ministres, et dans la
plus légère bienséance avec les ducs de Chevreuse et de Beauvilliers, si
même il y en avait. Tessé bien traité, mais connu de
M\textsuperscript{me} la Dauphine\,; la maréchale d'Estrées, qu'il
s'était dévouée par d'autres contours, avaient les reins trop faibles
pour le soutenir auprès de M\textsuperscript{me} la Dauphine\,; si
justement irritée contre ses nièces et contre lui, si uni à M. de
Vendôme et à Chamillart. Elle s'était à là fin dégoûtée de la maréchale
d'Estrées. M\textsuperscript{me} de La Vallière, la plus spirituelle et
la plus dangereuse des Noailles, lui avait enlevé la faveur et la
confiance, et n'avait rien de commun avec une cabale qui marchait sous
l'étendard de la Choin, toujours en garde contre tout ce qui tenait à
son ancienne maîtresse. Vaudemont n'avait donc plus de vie effective que
par le tout-puissant crédit de ses nièces sur Monseigneur, qui lui en
donnait un direct avec lui, et un autre par réflexion de l'attente du
futur. Cette corde rompue, il ne savait plus où se reprendre\,; la
conduite tout autrichienne du duc de Lorraine portait un peu sur lui
depuis que Chamillart n'était plus. Bien qu'à l'extérieur on n'eût pas
donné attention aux circonstances si marquées, et qui ont été
rapportées, de la conspiration tramée en Franche-Comté, qui fut
déconcertée par la victoire du comte du Bourg et par la capture de la
cassette de Mercy, cela n'avait pas laissé d'écarter encore plus ce
protée.

M\textsuperscript{lle} de Lislebonne, pénétrée d'une si profonde chute
personnelle et commune, trop sûre de sa situation avec
M\textsuperscript{me} la Dauphine, et avec tout ce qui approchait
intimement le Dauphin, n'était pas pour se pouvoir résoudre, altière
comme elle était, à traîner dans une cour où elle avait régné toute sa
vie. Son oncle et elle prirent donc le parti d'aller passer l'été en
Lorraine, pour se dérober à ces premiers temps de trouble, et se donner
celui de se former un plan de vie tout nouveau.

La fortune secourut cette fée. La petite vérole enleva tout de suite
plusieurs enfants à M. de Lorraine, entre autres une fille de sept ou
huit ans, qu'il avait fait élire abbesse de Remiremont, il y avait deux
ans, après la mort de M\textsuperscript{me} de Salm. Cet établissement
parut à l'oncle et à la nièce une planche après le naufrage, un état
noble et honnête pour une vieille fille, une retraite fort digne et sans
contrainte, une espèce de maison de campagne pour quand elle y voudrait
aller, sans nécessité de résidence assidue, ni d'abdiquer Paris et la
cour, et un prétexte de l'en tirer à sa volonté, avec quarante mille
livres de rente à qui en avait peu et se trouvait privée des voitures de
Monseigneur et de toutes les commodités qu'elle en tirait. Elle n'eut
que la peine de désirer cet établissement\,; tout en arrivant en
Lorraine, son élection se fit aussitôt.

Sa sœur, mère de famille, plus douce et plus flexible, ne se croyait pas
les mêmes raisons d'éloignement\,; son métier d'espionne de
M\textsuperscript{me} de Maintenon, dont on a vu d'avance un étrange
trait, lui donnait de la protection et de la considération, dont le
ressort était inconnu mais qui était marquée. Elle ne songea donc pas à
quitter la cour, ce qui entrait aussi dans la politique de sa sœur et de
son oncle. M\textsuperscript{me} d'Espinoy donna plutôt part qu'elle ne
demanda permission de Remiremont pour sa sœur, laquelle passa avec la
facilité pour eux ordinaire. M\textsuperscript{lle} de Lislebonne prit
le nom de M\textsuperscript{me} de Remiremont, dont je l'appellerai
désormais pour le peu de mention que j'aurai à faire d'elle dans la
suite.

L'affaire de Remiremont se fit si brusquement que j'arrivai le soir de
la permission donnée, sans en rien savoir, dans le salon, après le
souper du roi. Je fus surpris de voir venir à moi, au sortir du cabinet
du roi, M\textsuperscript{me} la Dauphine avec qui je n'avais aucune
privance, m'environner et me rencoigner en riant avec cinq ou six dames
de sa cour plus familières, me donner à deviner qui était abbesse de
Remiremont. Je reculais toujours\,; et le rire augmentait de ma surprise
d'une question qui me paraissait si hors de toute portée, et de ce que
je n'imaginais personne à nommer. Enfin elle m'apprit que c'était
M\textsuperscript{lle} de Lislebonne, et me demanda ce que j'en disais.
«\,Ce que j'en dis\,? madame, lui répondis-je aussi en riant, j'en suis
ravi pouvu que cela nous en délivre ici, et, à cette condition, j'en
souhaiterais autant à sa sœur. --- Je m'en doutais bien, répliqua la
princesse,\,» et s'en alla riant de tout son cœur. Deux mois plus tôt,
outre que l'occasion n'en eût pu être, une telle déclaration n'eût pas
été de saison, quoique mes sentiments ne fussent pas ignorés. Alors,
passé les premiers moments où cette hardiesse ne laissa pas de retentir,
il n'en fut pas seulement question.

M\textsuperscript{me} la Duchesse fut d'abord abîmée dans la douleur.
Tombée de ses plus vastes espérances, et d'une vie brillante et toujours
agréablement occupée, qui lui mettait la cour à ses pieds, mal avec
M\textsuperscript{me} de Maintenon, brouillée sans retour et d'une façon
déclarée avec M\textsuperscript{me} la Dauphine, en haine ouverte avec
M. du Maine, en équivalent avec M\textsuperscript{me} la duchesse
d'Orléans, en procès avec ses belles-sœurs, sans personne de qui
s'appuyer, avec un fils de dix-huit ans, deux filles qui lui échappaient
déjà par le vol qu'elle leur avait laissé prendre, tout le reste enfant,
elle se trouva réduite à regretter M. le Prince, et M. le Duc, dont la
mort l'avait tant soulagée.

Ce fut alors que l'image si chérie de M. le prince de Conti se présenta
sans cesse à sa pensée et à son cœur, qui n'aurait plus trouvé
d'obstacle à son penchant, et ce prince avec tant de talents que l'envie
avait laissés inutiles, réconcilié peu avant sa mort avec
M\textsuperscript{me} de Maintenon, intimement lié avec le Dauphin par
les choses passées, et de toute sa vie avec les ducs de Chevreuse et de
Beauvilliers et l'archevêque de Cambrai, uni à M\textsuperscript{me} la
Dauphine par la haine commune de Vendôme et par la conduite et les
propos qu'il avait tenus pendant la campagne de Lille, aurait été
bientôt le modérateur de la cour, et de l'État dans la suite. C'était le
seul à qui M\textsuperscript{me} la Duchessse eût été fidèle, elle était
l'unique pour qui il n'eût pas été volage\,; il lui aurait fait hommage
de sa grandeur, et elle aurait brillé de son lustre. Quels souvenirs
désespérants, avec Lassai fils pour tout réconfort\,! Faute de mieux
elle s'y attacha sans mesure, et l'attachement dure encore après plus de
trente ans.

Une désolation si bien fondée cessa pourtant bientôt quant à
l'extérieur\,; elle n'était pas faite pour les larmes, elle voulut
s'étourdir, et pour faire diversion elle se jeta dans les amusements, et
bientôt dans les plaisirs, jusqu'à la dernière indécence pour son âge et
son état. Elle chercha à y noyer ses chagrins, et elle y réussit. Le
prince de Rohan, qui avait jeté un million dans l'hôtel de Guise devenu
un admirable palais entre ses mains, lui donna des fêtes sous prétexte
de lui faire voir sa maison.

On a vu ailleurs combien il était uni à M\textsuperscript{me}s de
Remiremont et d'Espinoy\,; cette union l'avait lié à
M\textsuperscript{me} la Duchesse. Sa chute, l'état où le procès de la
succession de M. le Prince mettaient ses affaires, le nombre d'enfants
qu'elle avait, lui fit espérer que le rang et les établissements de son
fils, de son frère, de sa maison, avec ce palais et des biens immenses,
pourraient tenter M\textsuperscript{me} la Duchesse de se défaire pour
peu d'une de ses filles en faveur de son fils, et que le souvenir de sa
mère pourrait encore assez sur le roi, avec la protection de
M\textsuperscript{me} d'Espinoy auprès de M\textsuperscript{me} de
Maintenon, pour lever la moderne difficulté des alliances avec le sang
royal.

Il redoubla donc de jeu, de soins, de fêtes, d'empressement pour
M\textsuperscript{me} la Duchesse. Il s'était servi de sa situation
brillante auprès de Monseigneur, et de ce qui le gouvernait pour
s'approcher de M\textsuperscript{me} la Dauphine par un jeu prodigieux,
une assiduité et des complaisances sans bornes qu'il redoubla en cette
occasion\,; et la grande opinion qu'il avait de sa figure lui avait fait
hasarder des galanteries par la Montauban sa cousine, dont
M\textsuperscript{me} la Dauphine s'était fort moquée, mais fort en
particulier, et l'avait toujours traité avec distinction et familiarité
à cause de Monseigneur et de ses entours. Il songeait par là à donner
une grande et durable protection à son rang de prince étranger. La
consternation était tombée sur toutes ces usurpations étrangères qui
espéraient tout de Monseigneur par ceux des leurs qui l'obsédaient, et
qui se crurent perdues sans ressource par le nouveau Dauphin dont ils
redoutaient les sentiments, et de ce qui pouvait le plus sur lui. On a
vu qu'ils auraient pu se trouver déçus dans leurs idées sur le père,
mais elles étaient justes sur le fils, à qui la lecture avait appris ce
qu'ils savaient faire, et dont l'équité, le jugement solide et le
discernement ne s'accommodaient pas d'un ordre de gens sortis, formés et
soutenus par le désordre.

Le prince de Rohan ne put réussir dans ses vues auprès de
M\textsuperscript{me} la Duchesse\,; il enraya promptement. Il n'eut
garde de se montrer fâché par une conduite trop marquée qui aurait mis
en évidence ce qu'il voulait si soigneusement cacher, mais n'ayant plus
ni vues ni besoin d'elle, il se retira peu à peu sans cesser de la voir,
et M\textsuperscript{me} de Remiremont et M\textsuperscript{me}
d'Espinoy, qui n'avaient plus à compter avec elle, s'en retirèrent aussi
beaucoup peu à peu. On a vu plus haut ce que devint
M\textsuperscript{lle} Choin.

D'Antin mieux que jamais avec le roi, parvenu sitôt après la mort de
Monseigneur au comble de ses désirs et de la fortune, n'eut pas besoin
de grandes réflexions pour se consoler. On a vu, lors de la campagne de
Lille, avec quelle souple adresse il avait su s'initier avec
M\textsuperscript{me} la Dauphine, qu'il n'avait pas négligée depuis, et
dont il espérait un puissant contre-poids aux mœurs du nouveau Dauphin,
et au plus qu'éloignement qui était entre lui et ceux qui pouvaient le
plus sur ce prince. Il comptait que la santé du roi lui donnerait le
temps de rapprocher le Dauphin et de ramener peut-être à lui ceux qu'il
craignait davantage. La mort de Monseigneur l'affranchissait d'une
assiduité auprès de lui fort pénible qui lui ôtait un temps précieux
auprès du roi, et il n'en pouvait rien retrancher comme valet pris à
condition de servir deux maîtres. Il se trouvait délivré de la
domination de M\textsuperscript{me} la Duchesse, par cela même réduite à
compter avec lui, et débarrassé de plus de tous les manéges
indispensables, et souvent très-difficiles, pour demeurer uni avec tous
les personnages de cette cabale qui dominait Monseigneur, dont les
subdivisions donnaient bien de l'exercice aux initiés qui, comme
d'Antin, voulaient aussi figurer avec eux, et qui avait plus d'une fois
tâté de leur jalousie et de leurs hauteurs. Enfin il espéra augmenter sa
faveur par une assiduité sans partage, qui le rendrait considérable à la
nouvelle cour, et lui donnerait les moyens de s'y initier à la longue.
Il songeait toujours à entrer dans le conseil, car a-t-on jamais vu un
heureux se dire\,: C'est assez\,? Des adhérents de la cabale, ou des
gens particulièrement bien avec Monseigneur et qui se croyaient en
situation de figure ou de fortune sous son règne, tous eurent leur part
de la douleur ou de la chute. Le maréchal d'Huxelles fut au désespoir,
et n'osa en faire semblant, mais pour tenir manégea sourdement une
liaison avec M. du Maine. Le premier écuyer, honteux de regarder d'où
son père était sorti, paré de sa mère et de sa femme, avait osé plus
d'une fois aspirer à être duc, et n'espérait rien moins de Monseigneur,
tellement qu'il fut affligé comme un homme qui a perdu sa fortune.
Harcourt plus avant qu'eux tous, se consola plus aisément que pas un. Il
avait M\textsuperscript{me} de Maintenon entièrement à lui, sa fortune
complète, et il avait su se mettre secrètement bien avec la Dauphine, il
y avait longtemps, au lieu que les deux précédents n'y avaient aucune
jointure, ni avec le Dauphin, et se trouvaient fort éloignés de ce qui
l'approchait le plus, pareils en ce dernier article à Harcourt.
Boufflers, assez avant avec Monseigneur pour lui avoir fait ses plaintes
des froideurs, pour ne rien dire de plus, qu'il recevait du roi sans
cesse depuis ses désirs de l'épée de connétable, et qui en était
favorablement écouté, le regretta par amitié en galant homme. Il était
encore plus à portée du nouveau Dauphin qui savait mieux connaître et
goûter la vertu. Je l'avais extrêmement rapproché des ducs de Chevreuse
et de Beauvilliers\,; je m'en étais fait un travail, et j'y avais assez
réussi pour m'en promettre des fruits. Ainsi Boufflers n'avait qu'à
gagner, considéré d'ailleurs de M\textsuperscript{me} la Dauphine, et
toujours très-bien avec M\textsuperscript{me} de Maintenon, et dans un
comble de fortune.

De classe inférieure, Sainte-Maure, qui n'était bon qu'à jouer, perdit
véritablement sa fortune. La Vallière tenait trop de toutes façons à
M\textsuperscript{me} la princesse de Conti pour attendre beaucoup d'un
prince dans la main de M\textsuperscript{lle} Choin\,; il avait épousé
celle des Noailles qui avait le plus d'esprit, de sens, d'adresse, de
vues, de manéges et d'intrigue, qui gouvernait sa tribu, qui était
comptée à la cour, et qui était dans la plus grande confidence de la
nouvelle Dauphine\,; avec cela hardie, entreprenante, mais avec des
boutades et beaucoup d'humeur. Biron et Roucy qui, sans être menins,
étaient de tout temps très-attachés, et de tous les voyages de
Monseigneur, crurent leur fortune perdue. Roucy eut raison\,; il fallait
être Monseigneur pour en faire une espèce de favori. Biron, prisonnier
d'Audenarde, conservait le chemin de la guerre\,; il est aujourd'hui duc
et pair, comme on le verra en son temps, et doyen des maréchaux de
France. Il était frère de M\textsuperscript{me} de Nogaret et de
M\textsuperscript{me} d'Urfé, amies intimes de M\textsuperscript{me} de
Saint-Simon et les miennes, et neveu de M. de Lauzun de chez qui il ne
bougeait. Je l'avais appproché de M. de Beauvilliers, et j'avais réussi
à le bien mettre avec lui\,; par ce côté si important, et par sa sœur
auprès de M\textsuperscript{me} la Dauphine, il eut de quoi espérer de
la nouvelle cour.

Trois hommes à part peuvent tenir encore place ici\,: les ducs de La
Rocheguyon, de Luxembourg et de Villeroy. On a vu les liens par lesquels
M. de Luxembourg tenait à Monseigneur, dont il avait lieu de se
promettre une figure autant qu'il en pouvait être capable. D'ailleurs il
ne tenait à rien\,; car, hors quelques agréments en Normandie, Voysin ne
pouvait le mener plus loin. Le roi ne considérait en lui que son nom. Il
avait conservé des amis de son père, et il était fort du grand monde,
mais c'était tout, malgré l'amitié de M. de Chevreuse, qui sentait bien
qu'il n'y avait point de parti à en tirer. Il était si grand seigneur
qu'il put se consoler dans soi-même. Il en faut dire encore plus des
deux autres, qui par leurs charges existaient d'une façon plus
importante pour eux et plus soutenue. Les mêmes, lettres, dont j'ai
parlé quelque part ici, qui causèrent leur disgrâce, dont ils ne sont
même personnellement jamais bien revenus avec le roi, les avaient bien
mis avec Monseigneur, outre l'habitude et à peu près le même âge\,; mais
ils n'avaient pas auprès de lui les mêmes ailes que M. de Luxembourg, et
comme lui avaient perdu M. le prince de Conti, leur ami intime, qui les
avait laissés à découvert à M. de Vendôme et aux siens. Celui-ci n'y
était plus, mais il y existait par d'autres, et serait sûrement revenu
après le roi. Ce n'était pas qu'ils fussent personnellement mal avec
lui\,; mais les amis intimes de feu M. le prince de Conti ne pouvaient
jamais être les siens. Ces deux beaux-frères, avec de si grands
établissements, ne firent donc pas une si grande perte.

Un quatrième se trouva dans un nouveau désarroi. C'était La Feuillade.
Perdu à son retour de Turin, il avait cherché à s'attacher à
Monseigneur, et à profiter du peu de temps que Chamillart demeura en
place pour s'appuyer de M\textsuperscript{lle} de Lislebonne et de M. de
Vendôme. On a vu ailleurs qu'il avait percé jusqu'à
M\textsuperscript{lle} Choin. Le jeu d'ailleurs le soutenait à Meudon.
Il était de tous les voyages, sans pourtant avoir rien gagné sur
Monseigneur. Néanmoins, avec de si puissants entours, il comptait sous
lui se ramener la fortune. Il en désespérait du reste du règne du roi\,;
et pour celui qui le devait suivre, il avait tout ce qu'il fallait pour
en être encore plus éloigné\,; aussi fut-il fort affligé.

Deux genres d'hommes fort homogènes, quoique fort disproportionnés, le
furent jusqu'au plus profond du cœur, les ministres et les financiers.
On a vu, à l'occasion de l'établissement du dixième, ce que le
nouveau-Dauphin pensait de ces derniers, et avec quelle liberté il s'en
expliquait. Mœurs, conscience, instruction, tout en lui était pour eux
cause très-certaine des plus vives terreurs. Celle des ministres ne fut
guère moindre. Monseigneur était le prince qu'il leur fallait pour
régner en son nom, avec plus, s'il se peut, de pouvoir qu'ils n'en
avaient usurpé, mais avec beaucoup moins de ménagement. En sa place, ils
voyaient arriver un jeune prince instruit, appliqué, accessible, qui
voudrait voir et savoir, et qui avait, avec une volonté déjà soupçonnée,
tout ce qu'il fallait pour les tenir bas, et vraiment ministres,
c'est-à-dire exécuteurs, et plus du tout ordonnateurs, encore moins
dispensateurs. Ils le sentirent, et déjà ils commencèrent un peu à
baisser le ton, on peut juger avec quelle douleur.

Le chancelier perdait tout le fruit d'un attachement qu'il avait su
ménager dès son entrée aux finances, et qu'il avait eu moyen et
attention de cultiver très-soigneusement par Bignon son neveu, par du
Mont qu'il avait rendu son ami par mille services, par
M\textsuperscript{lle} de Lislebonne et M\textsuperscript{me} d'Espinoy
qu'ils s'étaient aussi dévouées, en sorte qu'il avait lieu de se flatter
sous Monseigneur, qui lui marquait amitié et distinction, du premier
personnage dans les affaires, et d'une influence principale à la cour,
que ses talents étaient bastants pour soutenir, et pour porter fort loin
dans la primauté de sa charge. L'échange de ce qui succédait était bien
différent. Rien là ne lui riait. Ennemi réputé des jésuites, et fort
soupçonné de jansénisme, brouillé dès son entrée aux finances avec le
duc de Beauvilliers, et hors de bienséance ensemble par les prises au
conseil, où ils étaient rarement d'accord, et où, sur les matières de
Rome, elles se poussaient quelquefois loin, et sans ménagement de la
part du chancelier, déclaré de plus, même avec feu, contre l'archevêque
de Cambrai dans tout le cours et les suites de son affaire. C'en était
trop, avec un caractère droit, sec, ferme, pour ne pas se croire perdu,
et pour que l'amitié, qui s'était maintenue entre le duc de Chevreuse et
lui, lui pût être une ressource, et il le sentit bien.

Son fils, aussi universellement abhorré qu'il était mathématiquement
détestable, avait encore trouvé le moyen de se faire également craindre
et mépriser, d'user même la bassesse d'une cour la plus servile, et de
se brouiller avec les jésuites, tout en faisant profession d'intimité
avec eux, en les maltraitant en mille choses, jusque-là qu'au lieu de
lui savoir gré de l'inquisition et de la persécution ouverte qu'il
faisait avec une singulière application à tout ce qu'il croyait qui
pouvait sentir le jansénisme, ils l'imputaient à son goût de faire du
mal.

C'était la bête de la nouvelle Dauphine qui ne s'épargnait pas à lui
nuire auprès du roi. J'en dirai un trait entre plusieurs. Un soir que
Pontchartrain sortait de travailler avec le roi, elle entra du grand
cabinet dans la chambre. M\textsuperscript{me} de Saint-Simon la suivait
avec une ou deux dames. Elle avisa, auprès de la place où Pontchartrain
avait été, de gros vilains crachats pleins de tabac\,: «\,Ah\,! voilà
qui est effroyable\,! dit-elle au roi\,; c'est votre vilain borgne\,; il
n'y a que lui qui puisse faire de ces horreurs-là,\,» et de là à lui
tomber dessus de toutes les façons. Le roi la laissa dire, puis lui
montrant M\textsuperscript{me} de Saint-Simon, l'avertit que sa présence
la devait retenir. «\,Bon\,! répondit-elle, elle ne le dira pas comme
moi\,; mais je suis sûre qu'elle en pense tout de même. Eh\,! qui est-ce
qui en pense autrement\,?» Là-dessus le roi sourit, et se leva pour
passer au souper. Le nouveau Dauphin n'en pensait guère mieux, ni tout
ce qui l'approchait. C'était donc une meule de plus attachée au cou du
père, qui en sentait tout le poids, et M\textsuperscript{me} de
Maintenon, de longue main brouillée avec le père comme on l'a vu en son
temps, n'amait pas mieux le fils que la princesse.

La Vrillière était aimé parce qu'il faisait plaisir de bonne grâce aux
rares occasions que sa charge lui en pouvait fournir, mais qui n'avait
que des provinces sans autre département\footnote{Voyez, dans les notes
  à la fin du volume, quels étaient les départements des secrétaires
  d'État dans l'ancienne monarchie.}. Lui et sa femme ensemble, et
chacun à part, étaient très-bien avec Monseigneur\,; amis intimes de du
Mont, et parvenus auprès de M\textsuperscript{lle} Choin à une amitié de
confiance, à quoi le premier écuyer et Bignon encore plus les avaient
fort servis. La perte fut donc extrême. Il ne tenait d'ailleurs qu'au
chancelier, avec qui il vivait comme un fils\,; et cette liaison si
naturelle m'avait été un obstacle à l'approcher du duc de Beauvilliers,
à quoi j'avais vainement travaillé. M\textsuperscript{me} de Mailly, sa
belle-mère, n'avait pas les reins assez forts pour le soutenir. Il avait
un malheur domestique qu'il eut la sagesse d'ignorer seul à la cour, et
ce malheur creusait sa ruine. M\textsuperscript{me} de La Vrillière, en
butte à M\textsuperscript{me} la Dauphine, triomphait d'elle en folle
depuis bien des années sans ménagement. Il y avait eu jusqu'à des
scènes, et M\textsuperscript{me} la Dauphine ne haïssait rien au monde
tant qu'elle. Tout cela présageait un triste avenir.

Voysin, sans nulle autre protection que celle de M\textsuperscript{me}
de Maintenon, sans art, sans tour, sans ménagement pour personne,
enfoncé dans ses papiers, enivré de sa faveur, sec, pour ne pas dire
brutal, en ses réponses, et insolent dans ses lettres, n'avait pour lui
que le manége de sa femme\,; et tous deux nulle liaison avec la nouvelle
cour, trop nouveaux pour s'être fait des amis, et le mari peu propre à
s'en faire, peut-être moins à en conserver, avec une place la plus
enviée de toutes, et la moins difficile à y trouver un successeur.

Torcy, doux et mesuré, avait pour soi la longue expérience des affaires,
et le secret de l'État et des postes, beaucoup d'amis et point d'ennemis
alors. Il était cousin germain des duchesses de Chevreuse et de
Beauvilliers, et gendre de Pomponne, pour qui MM. de Chevreuse et de
Beauvilliers avaient une confiance entière, et une estime qui allait à
la vénération. D'ailleurs, sans liaison avec Monseigneur, ni avec la
cabale frappée. Une telle position semblait heureuse à l'égard de la
nouvelle cour, mais ce n'était qu'une écorce. Au fond, Torcy n'était
qu'en bienséance avec les ducs et les duchesses de Chevreuse et de
Beauvilliers\,; ni la parenté, ni le commerce continuel et indispensable
d'affaires, n'avaient pu fondre les glaces qui s'étaient mises entre
eux. Ils ne se voyaient que par nécessité d'affaires ou de bienséance,
et cette froide bienséance n'était pas même poussée bien loin. Torcy et
sa femme vivaient dans la plus parfaite union. M\textsuperscript{me} de
Torcy, avec de l'humeur et de la hauteur, ne daignait pas voiler assez
ses sentiments. Son nom les rendait encore plus suspects\,; et quelque
chose de plus que du crédit qu'elle avait pris sur son mari le rendait
coupable d'après elle, et conséquemment aux yeux des deux ducs dangereux
dans le ministère. Il ne fléchissait point au conseil sur les matières
de Rome, où tout en douceur il soutenait avec force et capacité les avis
que le chancelier embrassait après, et qui donnaient lieu à ses prises
avec le duc de Beauvilliers, qui y souffrait beaucoup des raisons
détaillées de l'un, soutenues de la force et de l'autorité de l'autre.
M\textsuperscript{me} de Torcy était moins aimée que Torcy, et plutôt
éloignée qu'approchée de la nouvelle Dauphine pour qui elle ne s'était
jamais contrainte, encore moins pour qui que ce fût. Elle ne laissait
pas d'avoir des amis, ainsi que Torcy, mais dont pas un n'était d'aucune
ressource pour le futur que sa sœur par M\textsuperscript{me} la
Duchesse, qui pût leur faire regretter Monseigneur.

Desmarets avait assez longtemps tâté de la plus profonde disgrâce pour
avoir pu faire d'utiles réflexions, et il avait été ramené sur l'eau
avec tant de travail et de peine qu'il devait avoir appris à connaître
les amis de sa personne, et à discerner ceux que les places donnent
toujours, mais qui ne durent qu'autant qu'elles. Il avait assez d'esprit
et de sens pour que rien lui manquât de ce côté pour la conduite, et
cependant il en manqua tout à fait. Le ministère l'enivra. Il se crut
l'Atlas qui soutenait le monde, et dont l'État ne pouvait se passer\,;
il se laissa séduire par les nouveaux amis de cour, et il compta pour
rien ceux de sa disgrâce.

On a vu ailleurs que mon père, et moi à son exemple, avions été des
principaux, et que je l'avais fort servi auprès de Chamillart, et pour
rentrer dans les finances, et pour lui succéder dans la place de
contrôleur général. On a vu qu'il ne l'ignorait pas, et tout ce qui se
passa là-dessus entre lui et moi. Avec la déclaration que je lui avais
faite, et que je tins exactement, il devait donc être doublement à son
aise avec moi. Néanmoins je m'aperçus bientôt qu'il se refroidissait\,;
je suivis d'un œil sa conduite à mon égard pour ne me pas méprendre
entre ce qui pouvait être accidentel dans un homme chargé d'affaires
épineuses, et ce que j'en soupçonnais. Mes soupçons devinrent une
évidence qui me firent retirer de lui tout à fait, sans toutefois faire
semblant de rien. Les ducs de Chevreuse et de Beauvilliers s'aperçurent
de cette retraite\,; ils m'en parlèrent, ils me pressèrent\,; je leur
avouai le fait et la cause. Ils essayèrent de me persuader que Desmarets
était le même pour moi, et qu'il ne fallait pas prendre garde au froid
et à la distinction que lui donnaient ses tristes occupations. Ils
m'exhortèrent souvent d'aller chez lui, je les laissais dire et ne
changeais rien à ce que je m'étais proposé. À la fin, lassés de mon
opiniâtreté, pendant le dernier voyage de Fontainebleau ils me prirent
un matin et me menèrent dîner chez Desmarets. Je résistai\,; ils le
voulurent\,: j'obéis, et leur dis qu'ils auraient donc le plaisir d'être
convaincus par eux-mêmes. En effet, le froid et l'inapplication furent
si marqués pour moi, que les deux ducs piqués me l'avouèrent, et
convinrent que j'avais raison de cesser de le voir.

Eux-mêmes ne tardèrent pas d'éprouver la même chose. L'honneur d'être
leur cousin germain était le plus grand relief de Desmarets, et leur
situation un appui pour lui et une décoration infinie. La relation
nécessaire d'affaires avec eux était un autre lien. Enfin c'étaient ceux
qui, à force de bras par Chamillart et par eux-mêmes, l'avaient tiré
d'opprobre, et remis en honneur et dans le ministère. Malgré tant de
raisons si majeures d'attachement et d'union, il les mit au même point
où j'étais avec lui. Ils ne se voyaient que de loin à loin par une rare
bienséance, et fort peu de communication d'affairés qui ne se pouvait
éviter entièrement avec le duc de Beauvilliers, de qui je sus vers ces
temps-ci que lui ni le duc de Chevreuse ne lui parlaient plus de rien,
et qu'ils étaient hors de toute portée avec lui.

Il alla jusqu'à persécuter ouvertement le vidame d'Amiens, et les
chevau-légers à cause du vidame, qui rompit ouvertement avec lui. Il
n'en usa pas mieux avec Torcy, sa mère et sa sœur, dont il avait été le
commensal, depuis ses premiers retours de Maillebois jusqu'à son entrée
dans le ministère, et il les poussa tous trois à ne le plus voir du
tout. Le chancelier, qui à la vérité n'avait pas été heureux pour lui,
mais qui avait rompu auprès du roi les premières glaces pour le rappeler
aux finances du temps qu'il était contrôleur général, était le seul de
tous les ministres qui ne fût pas payé, en sorte qu'il n'eut rien à se
reprocher du côté de l'ingratitude, dans une place, et avec une humeur
féroce dont il n'était pas maître, qui le rendait redoutable aux femmes
même, et d'une paresse qui ralentissait tout.

Une conduite si dépravée ne lui donnait pas beau jeu pour l'avenir, et
son peu d'accès auprès de Monseigneur et de son intime cour ne lui
faisait rien perdre à ce qui venait de disparaître. Telle était à la
mort de Monseigneur la situation des ministres. Il faut venir maintenant
à celle du duc de Beauvilliers, et de ceux qui trouvèrent leur ressource
dans ce grand changement, et voir après les effets de ces contrastes.

Peu de gens parurent sur la scène du premier coup d'œil. Ceux-là mêmes
ne purent être guère aperçus, hors les principaux ou les plus marqués,
par les mesures politiques dont ils se couvrirent\,; mais on peut juger
qu'il y eut presse d'avoir part avec ces principaux, et avec ceux des
autres qui purent être reconnus. On peut imaginer encore quels furent
les sentiments du duc de Beauvilliers, le seul homme peut-être pour
lequel Monseigneur avait conçu une véritable aversion, jusqu'à ne
l'avoir pu dissimuler, laquelle était sans cesse bien soigneusement
fomentée. En échange, Beauvilliers voyait l'élévation inespérée d'un
pupille qui se faisait un plaisir secret de l'être encore, et un honneur
public de le montrer, sans que rien eût pu le faire changer là-dessus.
L'honnête homme dans l'amour de l'État, l'homme de bien dans le désir du
progrès de la vertu, et sous ce puissant auspice un autre M. de Cambrai
dans Beauvilliers, se voyait à portée de servir utilement l'État et la
vertu, de préparer le retour de ce cher archevêque, et de le faire un
jour son coopérateur en tout. À travers la candeur et la piété la plus
pure, un reste d'humanité inséparable de l'homme faisant goûter à
celui-ci un élargissement de cœur et d'esprit imprévu, un aise pour des
desseins utiles qui désormais se remplissaient comme d'eux-mêmes, une
sorte de dictature enfin d'autant plus savoureuse qu'elle était plus
rare et plus pleine, moins attendue et moins contredite, et qui par lui
se répandait sur les siens, et sur ceux de son choix. Persécuté au
milieu de la plus éclatante fortune, et, comme on l'a vu ici en plus
d'un endroit, poussé quelquefois jusqu'au dernier bord du précipice, il
se trouvait tout d'un coup fondé sur le plus ferme rocher\,; et
peut-être ne regarda-t-il pas sans quelque complaisance ces mêmes
vagues, de la violence desquelles il avait pensé être emporté
quelquefois, ne pouvoir plus que se briser à ses pieds. Son âme
toutefois parut toujours dans la même assiette\,; même sagesse, même
modération, même attention, même douceur, même accès, même politesse,
même tranquillité, sans le moindre élan d'élévation, de distraction,
d'empressement. Une autre cause plus digne de lui le comblait
d'allégresse. Sûr du fond du nouveau Dauphin, il prévit son triomphe sur
les esprits et sur les cœurs dès qu'il serait affranchi et en sa place,
et ce fut sur quoi il s'abandonna secrètement avec nous à sa
sensibilité. Chevreuse, un avec lui dans tous les temps de leur vie,
s'éjouit avec lui de la même joie, et y en trouva les mêmes motifs, et
leurs familles s'applaudirent d'un consolidement de fortune et d'éclat
qui ne tarda pas à paraître. Mais celui de tous à qui cet événement
devint le plus sensible fut Fénelon, archevêque de Cambrai. Quelle
préparation\,! Quelle approche d'un triomphe sûr et complet, et quel
puissant rayon de lumière vint à percer tout à coup une demeure de
ténèbres\,! Confiné depuis douze ans dans son diocèse, ce prélat y
vieillissait sous le poids inutile de ses espérances, et voyait les
années s'écouler dans une égalité qui ne pouvait que le désespérer.
Toujours odieux au roi, à qui personne n'osait prononcer son nom, même
en choses indifférentes\,; plus odieux à M\textsuperscript{me} de
Maintenon, parce qu'elle l'avait perdu\,; plus en butte que nul autre à
la terrible cabale qui disposait de Monseigneur, il n'avait de ressource
qu'en l'inaltérable amitié de son pupille, devenu lui-même victime de
cette cabale, et qui, selon le cours ordinaire de la nature, le devait
être trop longtemps pour que le précepteur pût se flatter d'y survivre,
ni par conséquent de sortir de son état de mort au monde. En un clin
d'œil, ce pupille devient Dauphin\,; en un autre, comme on va le voir,
il parvient à une sorte d'avant-règne. Quelle transition pour un
ambitieux\,! On l'a déjà fait connaître lors de sa disgrâce. Son fameux
\emph{Télémaque}, qui l'approfondit plus que tout et la rendit
incurable, le peint d'après nature. C'étaient les thèmes de son pupille
qu'on déroba, qu'on joignit, qu'on publia à son insu dans la force de
son affaire. M. de Noailles, qui, comme on l'a vu, ne voulait rien moins
que toutes les places du duc de Beauvilliers, disait au roi alors et à
qui voulut l'entendre, qu'il fallait être ennemi de sa personne pour
l'avoir composé. Quoique si avancés ici dans la connaissance d'un prélat
qui a fait, jusque du fond de sa disgrâce, tant de peur, et une figure
en tout état si singulière, il ne sera pas inutile d'en dire encore un
mot ici.

Plus coquet que toutes les femmes, mais en solide et non en misères, sa
passion était de plaire, et il avait autant de soin de captiver les
valets que les maîtres, et les plus petites gens que les personnages. Il
avait pour cela des talents faits exprès, une douceur, une insinuation,
des grâces naturelles et qui coulaient de source, un esprit facile,
ingénieux, fleuri, agréable, dont il tenait, pour ainsi dire, le
robinet, pour en verser la qualité et la quantité exactement convenables
à chaque chose et à chaque personne. Il se proportionnait et se faisait
tout à tous\,; une figure fort singulière, mais noble, frappante,
perçante, attirante\,; un abord facile à tous\,; une conversation aisée,
légère et toujours décente, un commerce enchanteur\,; une piété facile,
égale, qui n'effarouchait point et se faisait respecter\,; une
libéralité bien entendue\,; une magnificence qui n'insulte point, et qui
se versait sur les officiers et les soldats, qui embrassait une vaste
hospitalité, et qui, pour la table, les meubles et les équipages,
demeurait dans les justes bornes de sa place\,; également officieux et
modeste, secret dans les assistances qui se pouvaient cacher et qui
étaient sans nombre, leste et délié sur les autres jusqu'à devenir
l'obligé de ceux à qui il les donnait, et à le persuader\,; jamais
empressé, jamais de compliments, mais une politesse qui, en embrassant
tout, était toujours mesurée et proportionnée, en sorte qu'il semblait à
chacun qu'elle n'était que pour lui, avec cette précision dans laquelle
il excellait singulièrement. Adroit surtout dans l'art de porter les
souffrances, il en usurpait un mérite qui donnait tout l'éclat au sien,
et qui en portait l'admiration et le dévouement pour lui dans le cœur de
tous les habitants des Pays-Bas quels qu'ils fussent, et de toutes les
dominations qui les partageaient, dont il avait l'amour et la
vénération. Il jouissait, en attendant un autre genre de vie, qu'il ne
perdit jamais de vue, de toute la douceur de celle-ci, qu'il eût
peut-être regrettée dans l'éclat après lequel il soupira toujours, et il
en jouissait avec une paix si apparente que qui n'eût su ce qu'il avait
été, et ce qu'il pouvait devenir encore, aucun même de ceux qui
l'approchaient le plus, et qui le voyaient avec le plus de familiarité,
ne s'en serait jamais aperçu.

Parmi tant d'extérieur pour le monde, il n'en était pas moins appliqué à
tous les devoirs d'un évêque qui n'aurait eu que son diocèse à
gouverner, et qui n'en aurait été distrait par aucune autre chose.
Visites d'hôpitaux, dispensation large mais judicieuse d'aumônes,
clergé, communautés, rien ne lui échappait. Il disait tous les jours la
messe dans sa chapelle, officiait souvent, suffisait à toutes ses
fonctions épiscopales sans se faire jamais suppléer, prêchait
quelquefois. Il trouvait du temps pour tout, et n'avait point l'air
occupé. Sa maison ouverte, et sa table de même, avait l'air de celle
d'un gouverneur de Flandre, et tout à la fois d'un palais vraiment
épiscopal\,; et toujours beaucoup de gens de guerre distingués, et
beaucoup d'officiers particuliers, sains, malades, blessés, logés chez
lui, défrayés et servis comme s'il n'y en eût eu qu'un seul\,; et lui
ordinairement présent aux consultations des médecins et des chirurgiens,
faisant d'ailleurs, auprès des malades et des blessés les fonctions de
pasteur le plus charitable, et souvent par les maisons et par les
hôpitaux\,; et tout cela sans oubli, sans petitesse, et toujours
prévenant, avec les mains ouvertes. Aussi était-il adoré de tous. Ce
merveilleux dehors n'était pourtant pas tout lui-même.

Sans entreprendre de le sonder, on peut dire hardiment qu'il n'était pas
sans soins et sans recherche de tout ce qui pouvait le raccrocher et le
conduire aux premières places. Intimement uni à cette partie des
jésuites à la tête desquels était le P. Tellier, qui ne l'avaient jamais
abandonné, et qui l'avaient soutenu jusque par delà leurs forces, il
occupa ses dernières années à faire des écrits qui, vivement relevés par
le P. Quesnel et plusieurs autres, ne firent que serrer les nœuds d'une
union utile par où il espéra d'émousser l'aigreur du roi. Le silence
dans l'Église était le partage naturel d'un évêque dont la doctrine
avait, après tant de bruit et de disputes, été solennellement condamnée.
Il avait trop d'esprit pour ne le pas sentir\,; mais il eut trop
d'ambition pour ne compter pas pour rien tant de voix élevées contre
l'auteur d'un dogme proscrit et ses écrits dogmatiques, et beaucoup
d'autres qui ne l'épargnèrent pas sur le motif que le monde éclairé
entrevoyait assez.

Il marcha vers son but sans se détourner ni à droite ni à gauche\,; il
donna lieu à ses amis d'oser nommer son nom quelquefois, il flatta Rome
pour lui si ingrate, il se fit considérer par toute la société des
jésuites comme un prélat d'un grand usage, en faveur duquel rien ne
devait être épargné. Il vint à bout de se concilier La Chétardie, curé
de Saint-Sulpice, directeur imbécile et même gouverneur de
M\textsuperscript{me} de Maintenon.

Parmi ces combats de plume, Fénelon, uniforme dans la douceur de sa
conduite et dans sa passion de se faire aimer, se garda bien de
s'engager dans une guerre d'action. Les Pays-Bas fourmillaient de
jansénistes ou de gens réputés tels. En particulier son diocèse et
Cambrai même en était plein. L'un et l'autre leur furent des lieux de
constant asile et de paix. Heureux et contents d'y trouver du repos sous
un ennemi de plume, ils ne s'émurent de rien à l'égard de leur
archevêque qui, bien que si contraire à leur doctrine, leur laissait
toute sorte de tranquillité. Ils se reposèrent sur d'autres de leur
défense dogmatique, et ne donnèrent point d'atteinte à l'amour général
que tous portaient à Fénelon. Par une conduite si déliée, il ne perdit
rien du mérite d'un prélat doux et pacifique, ni des espérances d'un
évêque dont l'Église devait tout se promettre\,; et dont l'intérêt était
de tout faire pour lui.

Telle était la position de l'archevêque de Cambrai, lorsqu'il apprit la
mort de Monseigneur, l'essor de son disciple, l'autorité de ses amis.
Jamais liaison ne fut plus forte ni plus inaltérable que celle de ce
petit troupeau à part. Elle était fondée sur une confiance intime et
fidèle, qui elle-même l'était, à leur avis, sur l'amour de Dieu et de
son Église. Ils étaient presque tous gens d'une grande vertu, grands et
petits, à fort peu près qui en avaient l'écorce qui était prise par les
autres pour la vertu même. Tous n'avaient qu'un but qu'aucune disgrâce
ne put déranger, tous qu'une marche compassée et cadencée vers ce but,
qui était le retour de Cambrai leur maître, et cependant de ne vivre et
ne respirer que pour lui, de ne penser et de n'agir que sur ses
principes, et de recevoir ses avis en tout genre comme les oracles de
Dieu même dont il était le canal. Que ne peut point un enchantement de
cette nature, qui ayant saisi le cœur des plus honnêtes gens, l'esprit
de gens qui en avaient beaucoup, le goût et la plus ardente amitié des
personnes les plus fidèles, s'est encore divinisé en eux par l'opinion
ferme, ancienne, constante, qu'en cela consiste piété, vertu, gloire de
Dieu, soutien de l'Église, et le salut particulier de leurs âmes, à quoi
de bonne foi tout était postposé chez eux\,! Par ce développement on
voit sans peine quel puissant ressort était l'archevêque de Cambrai à
l'égard des ducs de Chevreuse et de Beauvilliers et de leurs épouses,
qui tous quatre n'étaient qu'un cœur, une âme, un sentiment, une pensée.
Ce fut peut-être cette considération unique qui empêcha la retraite du
duc de Beauvilliers à la mort de ses enfants, et lorsqu'il eut achevé
l'établissement intérieur de sa famille, enfin aux diverses occasions où
on l'a vu ici si près d'être perdu. Le duc de Chevreuse et lui avaient
un goût et un penchant entier à la retraite. Il y était si entier que
leur vie en tenait une proximité tout à fait indécente à leurs
emplois\,; mais l'ardeur de leurs désirs d'être utiles à la gloire de
Dieu, à l'Église, à leur propre salut, le leur fit croire de la
meilleure foi du monde attaché à demeurer en des places qui pussent ne
rien laisser échapper sur le retour de leur père spirituel. Il ne leur
fallut pas une raison à leur avis moins transcendante pour essayer tout,
glisser sur tout et conjurer les orages, pour n'avoir pas à se reprocher
un jour le crime de s'être rendus inutiles à une œuvre à leurs yeux si
principale, dont les occasions leur pouvaient être présentées par les
ressorts inconnus de la Providence, encore que, depuis si longtemps, ils
n'y eussent pu entrevoir le moindre jour.

Le changement subit arrivé par la mort de Monseigneur leur parut cette
grande opération de la Providence, expresse pour M. de Cambrai, si
persévéramment attendue, sans savoir d'où ni comment elle
s'accomplirait, la récompense du juste qui vit de la foi, qui espère
contre toute espérance, et qui est délivré au moment le plus imprévu. Ce
n'est pas que je leur aie ouï rien dire de tout cela\,; mais qui les
voyait comme moi dans leur intérieur, y voyait une telle conformité dans
tout le tissu de leur vie, de leur conduite, de leurs sentiments que
leur attribuer ceux-là, c'est moins les scruter que les avoir bien
connus. Serrés sur tout ce qui pouvait approcher ces matières, renfermés
entre eux autres anciens disciples, avec une discrétion et une fidélité
merveilleuse, sans faire ni admettre aucuns prosélytes dans la crainte
de s'en repentir, ils ne jouissaient qu'ensemble d'une vraie liberté, et
cette liberté leur était si douce qu'ils la préféraient à tout\,; de là,
plus que de toute autre chose, cette union plus que fraternelle des ducs
et des duchesses de Chevreuse et de Beauvilliers\,; de là le mariage du
duc de Mortemart, fils de la disciple sans peur, sans mesure, sans
contrainte\,; de là les retraites impénétrables de la fin de chaque
semaine à Vaucresson, avec un très-petit nombre de disciples trayés,
obscurs et qui s'y succédaient les uns aux autres\,; de là cette clôture
de monastère qui les suivait au milieu de la cour\,; de là cet
attachement au delà de tout au nouveau Dauphin, soigneusement élevé et
entretenu dans les mêmes sentiments. Ils le regardaient comme un autre
Esdras, comme le restaurateur du temple et du peuple de Dieu après la
captivité.

Dans ce petit troupeau était une disciple des premiers temps formée par
M. Bertau, qui tenait des assemblées à l'abbaye de Montmartre, où elle
ayait été instruite dès sa jeunesse, où elle allait toutes les semaines
avec M. de Noailles qui sut bien s'en retirer à temps\,: c'était la
duchesse de Béthune, qui avait toujours augmenté depuis en vertu, et qui
avait été trouvée digne par M\textsuperscript{me} Guyon d'être sa
favorite. C'était par excellence la grande âme, devant qui M. de Cambrai
même était en respect, et qui n'y était à son tour que par humilité et
par différence de sexe. Cette confraternité avait fait de la fille du
surintendant Fouquet l'amie la plus intime des trois filles de Colbert
et de ses gendres, qui la regardaient avec la plus grande vénération.

Le duc de Béthune, son mari, n'était qu'un frère coupe-choux qu'on
tolérait à cause d'elle\,; mais le duc de Charost, son fils, recueillit
tous les fruits de la béatitude de sa sainte mère. Une probité exacte,
beaucoup d'honneur, et tout ce qu'il y pouvait ajouter de vertu à force
de bras, mais rehaussée de tout l'abandon à M. de Cambrai qui se pouvait
espérer du fils de la disciple mère, faisait le fond du caractère de ce
fils, d'ailleurs incrusté d'une ambition extrême, de jalousie à
proportion, d'un grand amour du monde dans lequel il était fort répandu,
et auquel il était fort propre\,; l'esprit du grand monde, aucun
d'affaires, nulle instruction de quelque genre que ce fût, pas même de
dévotion, excepté celle qui était particulière au petit troupeau, et
d'un mouvement de corps incroyable\,; fidèle à ses amis et fort capable
d'amitié, et secret à surprendre à travers cette insupportable affluence
de paroles, héréditaire chez lui de père en fils. Il a peut-être été le
seul qui ait su joindre une profession publique de dévotion de toute sa
vie avec le commerce étroit des libertins de son temps, et l'amitié de
la plupart, qui tous le recherchaient et l'avaient tant qu'ils pouvaient
dans leurs parties où il n'y avait pas de débauche, et non-seulement
sans se moquer de ses pratiques si contraires aux leurs (je dis la
meilleure compagnie et la plus brillante de la cour et des armées), mais
avec liberté et confiance, retenus même par considération pour lui, et
sans que leur gaieté ni leur liberté en fût altérée. Il était de fort
bonne compagnie et bon convive, avec de la valeur, de la gaieté et des
propos et des expressions souvent fort plaisantes. La vivacité de son
tempérament lui donnait des passions auxquelles sa piété donnait un
frein pénible, mais qui en prenait le dessus à force de bras, et qui
fournissaient souvent avec lui à la plaisanterie.

M. de Beauvilliers avait fort souhaité autrefois que Charost et moi
liassions ensemble\,; et cette liaison qui s'était faite avait réussi
jusqu'à la plus grande intimité, qui a toujours duré depuis entre nous.
Je n'ai jamais connu M. de Cambrai que de visage\,; j'étais à peine
entré dans le monde lors du déclin de sa faveur\,; je ne me suis jamais
présenté aux mystères du petit troupeau. C'était donc être bien
inférieur au duc de Charost à l'égard des ducs de Chevreuse et de
Beauvilliers, dont on lui verra bientôt recueillir le fruit, et
néanmoins il en était demeuré avec eux à la confiance de leur gnose
\footnote{Le mot gnose, tiré du grec, signifie la science par
  excellence\,; de là le nom de \emph{gnostiques} donné à des hérétiques
  qui prétendaient qu'il y avait deux christianismes\,: l'un pour le
  peuple, l'autre pour les initiés.}, tandis que je l'avais entière sur
tout ce qui regardait l'État, la cour et la conduite du Dauphin. Sur
leur gnose, ils ne m'en parlaient pas\,; mais ils étaient à cœur ouvert
avec moi sur leur attachement et leur admiration de M. de Cambrai, sur
les désirs et les mesures de son retour. Dampierre et Vaucresson
m'étaient ouverts en tout temps\,; les condisciples obscurs y
paraissaient librement devant moi, et y conversaient de même\,; et
j'étais l'unique, non initié en leur gnose, dans ce genre de confiance
et de liberté avec eux. Il y avait déjà bien des années que je m'étais
aperçu qu'il s'en fallait tout que Charost ne fût aussi avant que moi
dans leur confiance, par bien des choses dont il se plaignait à moi de
leur réserve, que je lui laissais ignorer qu'ils m'avaient confiées\,;
et je ne vis pas depuis qu'il avançât là-dessus avec eux, tandis qu'ils
me disaient et consultaient avec moi toutes choses.

Dans ma surprise de cette différence d'un homme si fort mon ancien d'âge
et de cette sorte d'amitié si puissante avec eux, j'en ai souvent
cherché les causes. Son activité était toute de corps\,; il était bien
plus répandu que moi dans le monde, mais il savait peu et ne suivait
guère ce qui s'y passait de secret et d'important. Il ignorait donc les
machines de la cour, que me découvraient ma liaison avec les acteurs
principaux des deux sexes, et mon application à démêler, à savoir et à
suivre journellement toutes ces sortes de choses toujours curieuses,
ordinairement utiles, et souvent d'un grand usage.

M\textsuperscript{me} de Saint-Simon était aussi tout à fait dans la
confiance de MM. et de M\textsuperscript{me}s de Chevreuse et de
Beauvilliers, qui avaient une grande opinion de sa vertu, de sa
conduite, du caractère de son esprit. J'avais avec eux la liberté de
leur tout dire, qui n'eût pas sié de même à la dévotion du duc de
Charost\,; enfin j'avais eu les occasions, qu'on a vues ici, de les
avertir de choses fort peu apparentes et de la plus extrême importance,
qu'ils n'avaient même pu croire que par les événements\,; et cela avait
mis le dernier degré à leur ouverture sur tout avec moi, dont ils
avaient de plus éprouvé en tout la plus constante et la plus fidèle
amitié de toute préférence.

Ce fut donc une joie bien douce et bien pure de me trouver le seul homme
de la cour dans l'amitié la plus intime, et dans la plus entière
confiance de ce qui, privativement à tout autre, et sans crainte de
revers, allait figurer si grandement à la cour, et si puissamment sur le
nouveau Dauphin qui allait donner le ton à toutes choses. Plus ma
liaison intime était connue avec les deux ducs, et plus je me tins en
garde contre tout extérieur trop satisfait, et plus encore important, et
plus j'eus soin que ma conduite et ma vie se continssent dans tout leur
ordinaire à tous égards.

Dans ce grand changement de scène il ne parut donc d'abord que deux
personnages en posture d'en profiter\,: le duc de Beauvilliers, et par
lui le duc de Chevreuse, et un troisième en éloignement, l'archevêque de
Cambrai. Tout rit aux deux premiers tout à coup, tout s'empressa autour
d'eux, et chacun avait été de leurs amis dans tous les temps. Mais en
eux les courtisans n'eurent pas affaire à ces champignons de nouveaux
ministres tirés en un moment de la poussière, et placés au timon de
l'État, ignorants également d'affaires et de cour, également
enorgueillis et enivrés, incapables de résister, rarement même de se
défier de ces sortes de souplesses, et qui ont la fatuité d'attribuer à
leur mérite ce qui n'est prostitué qu'à la faveur. Ceux-ci, sans rien
changer à la modestie de leur extérieur, ni à l'arrangement de leur vie,
ne pensèrent qu'à se dérober le plus qu'il leur fut possible aux
bassesses entassées à leurs pieds, à faire usage de leurs amis
d'épreuve, à se fortifier près du roi par une assiduité redoublée, à
s'ancrer de plus en plus près de leur Dauphin, à le conduire à paraître
ce qu'il était, sans avoir surtout l'air de le conduire, et pour faire
que, tant du côté de l'estime et des cœurs que de celui de l'autorité,
il différât entièrement de son père.

Ils n'oublièrent pas de tâcher à s'approcher de la Dauphine, du moins à
ne la pas écarter d'eux. Elle l'était par une grande opposition
d'inclinations et de conduite\,; elle l'était encore par
M\textsuperscript{me} de Maintenon. Leur vertu, austère à son gré parce
qu'elle n'en connaissoit que l'écorce, lui faisait peur par leur
influence sur le Dauphin\,; elle les craignait encore plus directement
par un endroit plus délicat, qui était celui-là même qui la devait
véritablement attacher à eux, si, avec tout son esprit, elle eût su
discerner les effets de la vraie piété, de la vraie vertu, de la vraie
sagesse, qui {[}sont{]} d'étouffer et de cacher, avec le plus grand soin
et les plus extrêmes précautions, dont j'ai vu souvent ces deux ducs
très-occupés, ce qui peut altérer la paix et la tranquillité du mariage.
Ainsi, elle tremblait des avis fâcheux, du lieu même de sa plus ancienne
sûreté. Toutes ces raisons avaient mis un froid et un malaise, que tout
l'esprit et la faveur de M\textsuperscript{me} de Lévi n'avait pu
vaincre, et dont ces deux seigneurs et leurs épouses s'étaient aperçus
de bonne heure, à travers les ménagements et la considération que la
princesse ne pouvait leur refuser, mais dont les sentiments étaient
soigneusement entretenus par les Noailles et par la comtesse de Roucy,
autant que celle-ci le pouvait, qui, en communiant tous les huit jours,
ne pardonna jamais au duc de Beauvilliers ni aux siens d'avoir opiné
contre elle dans ce grand procès qu'elle gagna devant le roi contre M.
d'Ambres, dont j'ai parlé ailleurs, et dans lequel M\textsuperscript{me}
de Maintenon, contre sa coutume, se déclara si puissamment pour elle et
pour la duchesse d'Arpajon, sa mère.

Le printemps, qui est la saison de l'assemblée des armées, fit
apercevoir bien distinctement à Cambrai le changement qui était arrivé à
la cour. Cambrai devint la seule route de toutes les différentes parties
de la Flandre. Tout ce qui y servait de gens de la cour, d'officiers
généraux et même d'officiers moins connus, y passèrent tous et s'y
arrêtèrent le plus qu'il leur fut possible. L'archevêque y eut une telle
cour, et si empressée, qu'à travers sa joie, il en fut peiné, dans la
crainte du retentissement et du mauvais effet qu'il en craignait du côté
du roi. On peut juger avec quelle affabilité, quelle modestie, quel
discernement il reçut tant d'hommages, et le bon gré que se surent les
raffinés qui de longue main l'avaient vu et ménagé dans leurs voyages en
Flandre.

Cela fit grand bruit en effet\,; mais le prélat se conduisit si
dextrement que le roi ni M\textsuperscript{me} de Maintenon ne
témoignèrent rien de ce concours, qu'ils voulurent apparemment ignorer.
À l'égard des ducs de Chevreuse et de Beauvilliers, le roi, accoutumé à
les aimer, à les estimer, à y avoir sa confiance, jusque dans les rudes
traverses qu'ils avaient quelquefois essuyées, ne put s'effaroucher de
leur éclat nouveau, soit qu'il ne perçât pas jusqu'à lui, chose bien
difficile à croire, soit plutôt qu'il ne pût être détourné de ses
sentiments pour eux. M\textsuperscript{me} de Maintenon aussi ne montra
rien là-dessus.

Il y avait déjà des années que le duc de Beauvilliers avait initié le
duc de Chevreuse auprès du Dauphin, et qu'il l'avait accoutumé à le
considérer comme une seule chose avec lui. Le liant naturel et la
douceur de l'esprit de Chevreuse, son savoir et sa manière de savoir et
de s'expliquer, ses vues fleuries quoique sujettes à se perdre, furent
des qualités faites exprès pour plaire à ce jeune prince avec lequel il
avait souvent de longs tête-à-tête, et qui le mirent si avant dans sa
confiance que M. de Beauvilliers s'en servit souvent pour des choses
qu'il crut plus à propos de faire présenter par son beau-frère que par
lui-même. Comme ils n'étaient qu'un, tout entre eux marchait par le même
esprit, coulait des mêmes principes, tendait au même but, et se référait
entre eux deux\,; en sorte que le prince avait un seul conducteur en
deux différentes personnes, et qu'il avait pris beaucoup de goût et de
confiance au duc de Chevreuse, qui depuis longtemps était bien reçu à
lui dire tout ce qu'il pensait de lui et ce qu'il désirait sur sa
conduite, et toujours avec des intermèdes d'histoire, de science et de
piété\,; mais la supériorité en confiance, en amitié, et toute la
déférence, était demeurée entière au duc de Beauvilliers.

On peut croire que ces deux hommes ne laissaient pas refroidir dans le
prince ses vifs sentiments pour l'archevêque de Cambrai. Le confesseur
était d'intelligence avec eux sur cet article, et en totale déférence
sur tous autres\,; et jusqu'alors il n'y avait pas eu de quatrième admis
en cet intime intérieur du prince. Le premier soin des deux ducs fut de
le porter à des mesures encore plus grandes, à un air de respect et de
soumission encore plus marqué, à une assiduité de courtisan à l'égard du
roi si naturellement jaloux, et déjà éprouvé tel en diverses occasions
par son petit-fils.

Secondé à souhait par son adroite épouse, en possession elle-même de
toute privance avec le roi et du cœur de M\textsuperscript{me} de
Maintenon, il redoubla ses soins auprès d'elle, qui, dans le transport
de trouver un Dauphin sur qui sûrement compter, au lieu d'un autre qui
ne l'aimait point, se livra à lui, et par cela même lui livra le roi.
Les premiers quinze jours rendirent sensible à tout ce qui était à Marly
un changement si extraordinaire dans le roi, si réservé pour ses enfants
légitimes, et si fort roi avec eux.

Plus au large par un si grand pas fait, le Dauphin s'enhardit avec le
monde qu'il redoutait du vivant de Monseigneur, parce que, quelque grand
qu'il fût, il en essuyait les brocards applaudis. C'est ce qui lui
donnait cette timidité qui le renfermait dans son cabinet, parce que ce
n'était que là qu'il se trouvait à l'abri et à son aise\,; c'est ce qui
le faisait paraître sauvage et le faisait craindre pour l'avenir, tandis
qu'en butte à son père, peut-être alors au roi même, contraint
d'ailleurs par sa vertu\,; en butte à une cabale audacieuse, ennemie,
intéressée à l'être, et à ses dépendances qui formaient le gros et le
fort de la cour, gens avec qui il avait continuellement à vivre\,; enfin
en butte au monde en général, comme monde, il menait une vie d'autant
plus obscure qu'elle était plus nécessairement éclairée, et d'autant
plus cruelle qu'il n'en envisageait point de fin.

Le roi revenu pleinement à lui, l'insolente cabale tout à fait dissipée
par la mort d'un père presque ennemi dont il prenait la place, le monde
en respect, en attention, en empressement, les personnages les plus
opposés en air de servitude, ce même gros de la cour en soumission et en
crainte, l'enjoué et le frivole, partie non médiocre d'une grande cour,
à ses pieds par son épouse, certain d'ailleurs de ses démarches par
M\textsuperscript{me} de Maintenon, on vit ce prince timide, sauvage,
concentré, cette vertu précise, ce savoir déplacé, cet homme engoncé,
étranger dans sa maison, contraint de tout, embarrassé partout\,; on le
vit, dis-je, se montrer par degrés, se déployer peu à peu, se donner au
monde avec mesure, y être libre, majestueux, gai, agréable, tenir le
salon de Marly dans des temps coupés, présider au cercle rassemblé
autour de lui comme la divinité du temple qui sent et qui reçoit avec
bonté les hommages des mortels auxquels il est accoutumé, et les
récompenser de ses douces influences.

Peu à peu la chasse ne fut plus l'entretien que du laisser-courre, ou du
moment du retour. Une conversation aisée, mais instructive et adressée
avec choix et justesse, charma le sage courtisan et fit admirer les
autres. Des morceaux d'histoire convenables, amenés sans art des
occasions naturelles, des applications désirables, mais toujours
discrètes et simplement présentées sans les faire, des intermèdes aisés,
quelquefois même plaisants, tout de source et sans recherche, des traits
échappés de science mais rarement, et comme dardés de plénitude
involontaire\,; firent tout à la fois ouvrir les yeux, les oreilles et
les cœurs. Le Dauphin devint un autre prince de Conti. La soif de faire
sa cour eut en plusieurs moins de part à l'empressement de l'environner
dès qu'il paraissait, que celle de l'entendre et d'y puiser une
instruction délicieuse par l'agrément et la douceur d'une éloquence
naturelle qui n'avait rien de recherché, la justesse en tout, et plus
que cela la consolation, si nécessaire et si désirée, de se voir un
maître futur si capable de l'être par son fonds, et par l'usage qu'il
montrait qu'il en saurait faire.

Gracieux partout, plein d'attention au rang, à la naissance, à l'âge, à
l'acquit de chacun, choses depuis si longtemps, honnies et confondues
avec le plus vil peuple de la cour, régulier à rendre à chacune de ces
choses ce qui leur était dû de politesse, et ce qui s'y en pouvait
ajouter avec dignité, grave mais sans rides, et en même temps gai et
aisé\,; il est incroyable avec quelle étonnante rapidité l'admiration de
l'esprit, l'estime du sens, l'amour du cœur et toutes les espérances
furent entraînées, avec quelle roideur les fausses idées qu'on s'en
était faites et voulu faire furent précipitées, et quel fut l'impétueux
tourbillon du changement qui se fit généralement à son égard.

La joie publique faisait qu'on ne s'en pouvait taire, et qu'on se
demandait les uns aux autres si c'était bien là le même homme, et si ce
qu'on voyait était songe ou réalité. Cheverny, qui fut un de ceux à qui
la question s'adressa, n'y laissa rien à repartir. Il répondit que la
cause de tant de surprise était de ce qu'on ne connaissoit point ce
prince, qu'on n'avait même pas voulu connaître\,; que pour lui il le
trouvait tel qu'il l'avait toujours connu et vu dans son particulier\,;
que, maintenant que la liberté lui était venue de se montrer dans tout
son naturel, et aux autres de l'y voir, il paraissait ce qu'il avait
toujours été\,; et que cette justice lui serait rendue quand
l'expérience de la continuité apprendrait cette vérité.

De la cour à Paris, et de Paris au fond de toutes les provinces, cette
réputation vola avec tant de promptitude que ce peu de gens anciennement
attachés au Dauphin en étaient à se demander les uns aux autres s'ils
pouvaient en croire ce qui leur revenait de toutes parts. Quelque fondé
que fût un si prodigieux succès, il ne faut pas croire qu'il fût dû tout
entier aux merveilles du jeune prince. Deux choses y contribuèrent
beaucoup\,: les mesures immenses et si étrangement poussées de cette
cabale dont j'ai tant parlé, à décrier ce prince sur toutes sortes de
points, et depuis Lille toujours soutenues pour former contre lui une
voix publique dont ils pussent s'appuyer auprès de Monseigneur, et en
cueillir les fruits qu'ils s'en étaient proposés dès le départ pour
cette campagne, que le complot de l'y perdre avait été fait\,; et le
contraste de l'élastique à la chute du poids qui lui écrasait les
épaules, après lequel on le vit redressé, l'étonnement extrême que
produisit ce même contraste entre l'opinion qu'on en avait conçue et ce
qu'on ne pouvait s'empêcher de voir, et le sentiment de joie intime de
chacun, par son plus sensible intérêt, de voir poindre une aurore qui
d'éjà s'avançait, et qui promettait tant d'ordre et de bonheur après une
si longue confusion et tant de ténèbres.

M\textsuperscript{me} de Maintenon, ravie de ces applaudissements, par
amitié pour sa Dauphine, et par son propre intérêt de pouvait compter
sur un Dauphin qui commençait à faire l'espérance et les délices
publiques, s'appliqua à en presser tout l'usage qu'elle put auprès du
roi. Quelque admiration qu'elle voulût montrer pour tout ce qui était de
son goût et de sa volonté, et quelques mesures qu'elle gardât avec tous
ses ministres, leur despotisme, et leur manière de l'exercer, lui
déplaisait beaucoup. Ses plus familiers avaient découvert en des
occasions rares ses plus secrets sentiments là-dessus (qu'Harcourt avait
beaucoup fortifiés en elle), tantôt par des demi-mots de ridicule bien
assenés où elle excellait, quelquefois par quelques paroles plus
sérieuses, bien qu'également étranglées, sur le mauvais de ce
gouvernement. Elle crut donc se procurer un avantage, à l'État un bien,
au roi un soulagement, de faire en sorte qu'il s'accoutumât à faire
préparer les matières par le Dauphin, à lui en laisser expédier
quelques-unes, et peu à peu ainsi à se décharger sur lui du gros et du
plus pesant des affaires, dont il s'était toujours montré si capable, et
dans lesquelles il était initié, puisqu'il était de tous les conseils,
où il parlait depuis longtemps avec beaucoup de justesse et de
discernement. Elle compta que cette nouveauté rendrait les ministres
plus appliqués, plus laborieux, surtout plus traitables et plus
circonspects. Vouloir et faire, sur les choses intérieures et qui par
leur nature pouvaient s'amener de loin par degrés avec adresse, fut
toujours pour elle une seule et même chose.

Le roi, déjà plus enclin à son petit-fils, était moins en garde des
applaudissements qu'il recevait sous ses yeux, qu'il ne l'avait paru sur
ceux de ses premières campagnes. Bloin et les autres valets intérieurs,
dévoués à M. de Vendôme, n'avaient plus cet objet ni Monseigneur en
croupe. Ils étaient en crainte et en tremblement\,; et M. du Maine,
destitué de leur appui, n'osait plus ouvrir la bouche ni hasarder que
M\textsuperscript{me} de Maintenon le découvrît contraire. Ainsi le roi
était sans ces puissants contre-poids, qui avaient tant manégé
auparavant dans ses heures les plus secrètes et les plus libres.

La sage et flexible conduite de ce respectueux et assidu petit-fils
l'avait préparé à se rendre facile aux insinuations de
M\textsuperscript{me} de Maintenon, tellement que, quelque accoutumé que
l'on commençât d'être à la complaisance que le roi prenait dans le
Dauphin, toute la cour fut étrangement surprise de ce que, l'ayant
retenu un matin seul dans son cabinet assez longtemps, il ordonna le
même jour à ses ministres d'aller travailler chez le Dauphin toutes les
fois qu'il les manderait, et sans être mandés encore, de lui aller
rendre compte de toutes les affaires, dont une fois pour toutes il leur
aurait ordonné de le faire.

Il n'est pas aisé de rendre le mouvement prodigieux que fit à la cour un
ordre si directement opposé au goût, à l'esprit, aux maximes, à l'usage
du roi, si constant jusqu'alors, qui, par cela même, marquait une
confiance pour le Dauphin qui n'allait à rien moins qu'à lui remettre
tacitement une grande partie de la disposition des affaires. Ce fut un
coup de foudre sur les ministres, dont ils se trouvèrent tellement
étourdis qu'ils n'en purent cacher l'étonnement ni le déconcerteraient.

Ce fut un ordre en effet bien amer pour des hommes qui, tirés de la
poussière et tout à coup portés à la plus sûre et à la plus suprême
puissance, étaient si accoutumés à régner en plein sous le nom du roi,
auquel ils osoient même substituer quelquefois le leur, en usage
tranquille et sans contredit de faire et de défaire les fortunes,
d'attaquer avec succès les plus hautes, d'être les maîtres des plus
patrimoniales de tout le monde, de disposer avec toute autorité du
dedans et du dehors de l'État, de dispenser à leur gré toute
considération, tout châtiment, toute récompense, de décider de tout
hardiment par un \emph{le roi le veut}, de sécurité entière même à
l'égard de leurs confrères, desquels qui que ce fût n'osait ouvrir la
bouche au roi de rien qui put regarder leur personne, leur famille ni
leur administration, sous peine d'en devenir aussitôt la victime
exemplaire pour quiconque l'eût hasardé, par conséquent en toute liberté
de taire, de dire, de tourner toutes choses au roi comme il leur
convenait, en un mot, rois d'effet, et presque de représentation. Quelle
chute pour de tels hommes que d'avoir à compter sur tout avec un prince
qui avait M\textsuperscript{me} de Maintenon à lui, et qui auprès du roi
était devenu plus fort qu'eux dans leur propre tripot\,; un prince qui
n'avait plus rien entre lui et le trône\,: qui était capable, laborieux,
éclairé, avec un esprit juste et supérieur\,; qui avait acquis sur un
grand fonds tout fait depuis qu'il était dans le conseil\,; à qui rien
ne manquait pour les éclairer\,; qui, avec ces qualités, avait le cœur
bon, était juste, aimait l'ordre\,; qui avait du discernement, de
l'attention, de l'application à suivre et à démêler\,; qui savait
tourner et approfondir\,; qui ne se payait que de choses et point de
langage\,; qui voulait déterminément le bien pour le bien\,; qui pesait
tout au poids de sa conscience\,; qui, par un accès facile et une
curiosité de dessein et de maximes, serait instruit par force canaux\,;
qui saurait comparer et apprécier les choses, se défier et se confier à
propos par un juste discernement et une application sage, et en garde
contre les surprises de toutes parts\,; qui ayant le cœur du roi, avait
aussi son oreille à toute heure\,; et qui, outre les impressions qu'il
prendrait d'eux pour quand il serait leur maître, se trouvait dès lors
en état de confondre le faux et le double, et de porter une lumière
aussi pénétrante qu'inconnue dans l'épaisseur de ces ténèbres qu'ils
avaient formées et épaissies avec tant d'art, et qu'ils entretenaient de
même.

L'élévation du prince et l'état de la cour ne comportait plus le remède
des cabales\,; et la joie publique d'un ordre qui rendait ces rois à la
condition des sujets, qui donnait un frein à leur pouvoir, et une
ressource à l'abus qu'ils en faisaient, ne leur laissait aucune
ressource. Ils n'eurent donc d'autre parti à prendre que de ployer les
épaules à leur tour, ces épaules roidies à la consistance du fer. Ils
allèrent, tous avec un air de condamnés, protester au Dauphin une
obéissance forcée et une joie feinte de l'ordre qu'ils avaient reçu.

Le prince n'eut pas peine à démêler ce qu'eux-mêmes en avaient tant à
cacher. Il les reçut avec un air de bonté et de considération, il entra
avec eux dans le détail de leurs journées pour leur donner les heures
les moins incommodes à la nécessité du travail et de l'expédition, et
pour cette première soumission n'entra pas avec eux en affaires\,; mais
ne différa pas de commencer à travailler chez lui avec eux.

Torcy, Voysin et Desmarets furent ceux sur qui le poids en tomba, par
l'importance de leurs départements. Le chancelier, qui n'en avait point,
n'y eut que faire. Son fils, voyant les autres y travailler assidûment,
aurait bien voulu y être mandé aussi. Il espérait s'approcher par là du
prince, et il était fort touché de l'air important\,; mais sa marine
était à bas, et les délations du détail de Paris, dont il amusait le roi
tous les lundis aux dépens de tout le monde, et dont Argenson lui avait
adroitement laissé usurper tout l'odieux, n'étaient ni du goût du
Dauphin, ni chose à laquelle il voulût perdre son temps. D'ailleurs la
personne de Pontchartrain lui était désagréable, comme on le verra
bientôt, et il ne put parvenir à être mandé, ni trouver sans cela de
quoi oser aller rendre compte, dont il fut fort mortifié. La Vrillière
n'avait que le détail courant de ses provinces, par conséquent point de
matière pour ce travail\,; le département de sa charge était la religion
prétendue réformée, et tout ce qui regardait les huguenots. Tout cela
était tombé depuis les suites de la révocation de l'édit de Nantes,
tellement qu'il n'avait point de département.

Ce serait ici le lieu de parler de la situation dans laquelle je me
trouvai incontinent avec le Dauphin, et la confiance intime sur le
présent et l'avenir, et toutes les mesures qui y étaient relatives, où
je fus admis entre le duc de Beauvilliers et le Dauphin, et le duc de
Chevreuse. La matière est curieuse et intéressante, mais elle mènerait
trop loin à la suite de la longue parenthèse que la mort de Monseigneur
et ses suites, et que l'affaire de d'Antin et de l'édit qu'elle
produisit, a mis au courant. Il le faut reprendre jusqu'au voyage de
Fontainebleau. Je reviendrai après à ce que, pour le présent, je
diffère.

\hypertarget{chapitre-xiii.}{%
\chapter{CHAPITRE XIII.}\label{chapitre-xiii.}}

1711

~

{\textsc{Voyage des généraux d'armée.}} {\textsc{- Permangle bat et
brûle un grand convoi.}} {\textsc{- Duc de Noailles près du roi
d'Espagne avec ses troupes sous Vendôme.}} {\textsc{- La reine d'Espagne
attaquée d'écrouelles.}} {\textsc{- Bonac relève Blécourt à la cour d
Espagne.}} {\textsc{- Marly en jeu et en sa forme ordinaire\,; cause de
sa singulière prolongation.}} {\textsc{- Premier mariage de Belle-Ile.}}
{\textsc{- Mariage de Montboissier avec M\textsuperscript{lle} de
Maillé.}} {\textsc{- Mariage de Parabère avec M\textsuperscript{lle} de
La Vieuville.}} {\textsc{- Course à Marly de l'électeur de Bavière.}}
{\textsc{- Mort de Langeron, lieutenant général des armées navales,}}
{\textsc{- Mort, caractère, descendance et titres du duc d'Albe,
ambassadeur d'Espagne en France\,; sa succession.}} {\textsc{- Fils
d'Amelot président à mortier.}} {\textsc{- Digne souvenir du roi des
services de Molé, premier président et garde des sceaux.}} {\textsc{-
Bergheyck à Marly, mandé en Espagne.}} {\textsc{- Voyage du roi
d'Angleterre par le royaume.}} {\textsc{- Grand prieur à Soleure.}}
{\textsc{- Deuil de l'empereur suspendu, et sa cause.}} {\textsc{- Le
roi d'Espagne donne ce qui lui reste aux Pays-Bas à l'électeur de
Bavière, qui passe à Marly allant à Namur, et envoie le comte d'Albert
en Espagne\,; comte de La Marck suit l'électeur, de la part du roi, sans
caractère.}} {\textsc{- Gassion bat en Flandre douze bataillons et dix
escadrons\,; son mérite et son extraction.}} {\textsc{- Clôture de
l'assemblée extraordinaire du clergé\,; admirable et hardie harangue au
roi de Nesmond, archevêque d'Alby.}} {\textsc{- Le Dauphin montré au
clergé par le roi.}} {\textsc{- Services de Monseigneur à Saint-Denis et
à Notre-Dame.}} {\textsc{- Merveilles du Dauphin à Paris.}} {\textsc{-
Nul duc ne s'y trouve, quoique le roi l'eût désiré.}} {\textsc{-
Création d'officiers gardes-côtes.}} {\textsc{- Pontchartrain en abuse
et de mon amitié, me trompe, m'usurpe, et je me brouille avec lui.}}
{\textsc{- Usurpation très-attentive des secrétaires d'État.}}
{\textsc{- Sottise d'amitié.}} {\textsc{- Trahison noire de
Pontchartrain.}} {\textsc{- Étrange procédé de Pontchartrain, qui me
veut leurrer par Aubenton.}} {\textsc{- Impudence et embarras de
Pontchartrain.}} {\textsc{- Le chancelier soutient le vol de son fils
contre moi.}} {\textsc{- Peine et proposition des Pontchartrain.}}
{\textsc{- Ma conduite avec eux.}}

~

Le maréchal de Villars était allé de bonne heure en Flandre, dans le
dessein d'y faire le siège de Douai. Le maréchal de Montesquiou avait
fait pour cela les dispositions nécessaires, mais l'exécution ne put
avoir lieu. Villars revint à la cour jusqu'au temps de l'ouverture de la
campagne, qu'il s'en retourna prendre le commandement de l'armée. En
attendant, Permangle, maréchal de camp, qui commandait dans Condé, eut
avis qu'un convoi de vivres des ennemis était sur l'Escaut, prêt à
entrer dans la Scarpe, escorté de deux bataillons avec un officier
général. Permangle y marcha avec huit cents hommes, défit les deux
bataillons, en prit le commandement, et de trente-six
bélandres\footnote{Bateaux plats.}, portant cent milliers chacune, en
brûla vingt-cinq.

M. d'Harcourt partit les premiers jours de mai pour les eaux de
Bourbonne. Le maréchal de Besons était déjà à Strasbourg\,; il commanda
l'armée du Rhin en l'attendant, et le duc de Berwick partit bientôt
après pour le Dauphiné.

On ne laissa que quelques régiments d'infanterie sur le Ter. Le duc de
Noailles était demeuré auprès du roi d'Espagne depuis qu'il y était
passé après la prise de Girone\,; et l'armée qui lui était destinée
passa en Aragon, où il eut ordre de la commander à part, ou jointe à
celle de M. de Vendôme mais à ses ordres, de l'une ou de l'autre
manière, suivant ce que Vendôme jugerait à propos pour le service du roi
d'Espagne.

Il y avait déjà quelques mois que la santé de la reine d'Espagne était
altérée\,: il lui était venu des glandes au cou qui, peu à peu,
dégénérèrent en écrouelles\,; elle eut des rechutes de fièvre
fréquentes, mais elle ne s'appliqua pas moins au rétablissement des
affaires.

Bonac, neveu de Bonrepos, alla relever en Espagne Blécourt dont on a
souvent parlé.

Le 8 mai, le lansquenet et les autres jeux recommencèrent dans le salon
de Marly, qui, faute de ces amusements, avait été fort désert depuis la
mort de Monseigneur. M\textsuperscript{me} la Dauphine s'était mise à
jouer à l'oie ne pouvant mieux, mais en particulier chez elle. Elle fut
encore huit ou dix jours sans jouer dans le salon. À la fin tout prit à
Marly la forme ordinaire. Les petites véroles, qui accablaient
Versailles, retinrent le roi à Marly pendant les fêtes de la Pentecôte,
pour la première fois. Il n'y eut point de cérémonie de l'ordre\,; et la
même raison l'y retint aussi à la Fête-Dieu.

Belle-Ile, qui à travers tant de diverses fortunes en a fait une si
prodigieuse pour le petit-fils du surintendant Fouquet, épousa, ayant de
partir pour l'armée, M\textsuperscript{lle} de Sivrac, de la maison de
Durfort. Elle était riche, extrêmement laide, encore plus folle. Elle
s'en entêta et ne le rendit pas heureux, ni père. Son bonheur l'en
délivra quelques années après, et le malheur de la France le remaria
longtemps après. Montboissier épousa en même temps
M\textsuperscript{lle} de Maillé, belle, riche et de beaucoup d'esprit.
Il a succédé longtemps depuis à Canillac, son cousin, chevalier de
l'ordre en 1728, capitaine de la deuxième compagnie des mousquetaires.

Parabère épousa aussi la fille de M\textsuperscript{me} de La Vieuville,
dame d'atours de M\textsuperscript{me} la duchesse de Berry, qui peu
après son mariage fit parler d'elle, et qui enfin a si publiquement vécu
avec M. le duc d'Orléans, et après lui avec tant d'autres.

L'électeur de Bavière, à qui Torcy avait été par ordre du roi porter, à
Compiègne, la nouvelle de la mort de l'empereur aussitôt qu'il l'eut
reçue, et conférer avec lui, vint quelque temps après passer quelques
jours en une maison de campagne, qu'il emprunta, auprès de Paris. Deux
jours après, il vint à Marly, sur les deux heures et demie (c'était le
26 mai)\,; il fut descendre dans l'appartement que feu Monseigneur
occupait. Au bout d'un quart d'heure il passa dans le cabinet du roi, où
il le trouva avec les deux fils de France, M\textsuperscript{me} la
Dauphine et toutes les dames de cette princesse. La conversation s'y
passa debout, à portes ouvertes, pendant un quart d'heure, après quoi
tout sortit, et le roi demeura seul assez longtemps avec l'électeur, les
portes fermées. Il vint ensuite dans le salon, où M. et
M\textsuperscript{me} la Dauphine l'attendaient. La conversation dura
debout quelque temps, et il s'en retourna à sa petite maison. Le roi lui
avait proposé de revenir le surlendemain à la chasse\,; il y vint, se
déshabilla après dans ce même appartement de descente, et suivit après
le roi dans les jardins, qui le fit monter seul avec lui dans son
chariot\,; ils se promenèrent fort dans les hauts de Marly. Au retour,
il fut assez longtemps seul avec le roi dans son cabinet. Il vint après
dans le salon\,; M\textsuperscript{me} la Dauphine y jouait au
lansquenet, qui le fit asseoir auprès d'elle. Sur les huit heures, il
alla souper chez d'Antin avec compagnie d'élite\,; le repas fut gai et
dura trois heures. Il parut partir fort content pour sa petite maison,
d'où il regagna Compiègne par Liancourt.

Ce même jour Langeron, lieutenant général des armées navales et fort bon
marin, mourut à Sceaux, d'apoplexie, sans être gros ni vieux. Il était
fort attaché à M. et à M\textsuperscript{me} du Maine, et sa famille à
la maison de Condé, sa sœur en particulier à M\textsuperscript{me} la
Princesse. Il était frère de l'abbé de Langeron, mort à Cambrai depuis
peu.

Le duc d'Albe, ambassadeur d'Espagne, était mort la veille après une
assez longue maladie. Il l'était depuis plusieurs années, et y avait
acquis une grande réputation de sagesse, d'esprit, de prudence et de
capacité\,; il avait aussi beaucoup de probité et de piété. Il s'était
acquis l'estime et la confiance du roi et des ministres, et une
considération générale. Il vivait avec la meilleure compagnie et avec
magnificence, et beaucoup de politesse et de dignité. Le roi d'Espagne
fit payer toutes ses dettes, et continua quatre mois durant les
appointements de l'ambassade à la duchesse d'Albe, qui ne partit point
que tout ne fût payé. Le corps fut envoyé en Espagne.

Son nom est Tolède\,; tiré de la ville de Tolède, mais avec celui
d'Alvarez pour distinguer cette maison, l'une des premières d'Espagne,
de quelques autres différentes qui le portent aussi avec d'autres noms.
Jean II, roi de Castille, mit dans cette maison la ville d'Alva par don,
que nous appelons \emph{Albe} et qui est auprès de Salamanque, avec
d'autres adjonctions en titre de comté, en 1430. Le troisième comte
d'Albe fut fait duc d'Albe par Henri IV, en 1469\,; et c'est le
bisaïeul, de mâle en mâle, du fameux duc d'Albe, gouverneur des Pays-Bas
sous Philippe II, qui mourut en 1582, et laissa deux fils. L'aîné, qui
avait été fait duc d'Huesca, mourut sans enfants après son cadet, dont
le fils lui succéda. Il épousa Antoinette Enriquez de Ribera, dont le
frère était mort sans enfants\,; elle fit entrer dans la maison de son
mari ses biens et son nom. Ainsi ce sixième duc d'Albe et d'Huesca par
soi, fut par sa mère, héritière de la maison de Beaumont si célèbre en
Navarre et en Aragon, comte de Lerin, et connétable et chancelier
héréditaire de Navarre, et par sa femme duc de Galisteo, comte d'Osorno,
etc. Il fut grand-père du duc d'Albe qui mourut à Madrid d'une façon si
singulière, et qui a été racontée peu de temps {[}après{]} l'arrivée de
Philippe V à Madrid\,; et c'est le fils de celui-là, ambassadeur en
France, de la mort duquel on parle ici. On a vu ailleurs qui et quelle
était la duchesse d'Albe, et qu'ils avaient perdu leur fils unique à
Paris. Le marquis del Carpio, frère du père du duc d'Albe, lui succéda
en ses grandesses et en ses biens.

Il était grand d'Espagne par sa femme, fille et héritière de don Gaspard
de Haro, marquis del Carpio et d'Eliche, comte-duc d'Olivarès,
ambassadeur à Rome, mort vice-roi de Naples, et fils du célèbre don
Louis de Haro qui traita la paix des Pyrénées avec le cardinal Mazarin,
et qui avait hérité des biens, dignités et premier ministère du
comte-duc d'Olivarès, son oncle maternel. Ce marquis del Carpio, dont la
femme était fille de la sœur de l'amirante de Castille, s'était laissé
entraîner par elle dans le parti de l'archiduc\,; et ils étaient à
Vienne, où ils marièrent leur fille au frère du duc de l'Infantado, qui
avait suivi le même parti.

Ils revinrent longtemps après à Madrid, où ce duc d'Albe aida au duc del
Arco, parrain de mon second fils, à faire les honneurs le jour de sa
couverture. J'aurai alors occasion de parler de plusieurs autres grands
de cette maison de Tolède, dont était ce digne marquis de Mancera dont
il a été mention plusieurs fois.

Amelot à qui ses ambassades, où il avait si bien servi, et surtout celle
d'Espagne qui ne lui avait rien valu après l'avoir mis à portée de tout,
eut enfin pour son fils la charge de président à mortier de
Champlâtreux, qui mourut d'apoplexie en s'habillant pour aller à la
réception de d'Antin, et qui ne laissa personne en état ni en âge de la
recueillir\,; car le roi se souvenait toujours du premier président
Molé, garde des sceaux, et leur conserva cette charge tant qu'il y eut
dans cette famille à qui la donner, qui y est revenu depuis. Bergheyck
vit assez longtemps le roi en particulier, et les ministres séparément,
passant de Flandre en Espagne, où le roi d'Espagne le mandait avec
empressement, et d'où M\textsuperscript{me} des Ursins en eut beaucoup
plus à le renvoyer promptement.

Le roi d'Angleterre partit, en ce même temps, pour aller voyager par le
royaume, ennuyé apparemment de ses tristes campagnes incognito, et plus
encore de demeurer à Saint-Germain pendant la guerre. On soupçonna du
mystère en ce voyage, sans qu'il y en eût aucun. Il alla avec une petite
suite d'abord à Dijon, puis en Franche-Comté, en Alsace, et voir l'armée
d'Allemagne\,; de là par Lyon en Dauphiné, à l'armée du duc de Berwick,
voir les ports de Provence, et revenir par le Languedoc et la Guyenne.

Le grand prieur, gobé comme on l'a remarqué en son temps, obtint enfin
sa liberté, sur sa parole de ne point sortir de Soleure jusqu'à ce qu'il
eût obtenu la liberté de ce brigand de fils de Massenar, prisonnier à
Pierre-Encise, que le roi ne voulut point accorder.

Il avait porté quelques jours de plus le deuil des enfants de
M\textsuperscript{me} de Lorraine, par paresse de changer d'habit, ce
qu'il n'aimait point, comptant à tout moment de le prendre de
l'empereur\,; mais l'impératrice mère, qui gouvernait en attendant
l'archiduc, s'avisa, dans la lettre par laquelle elle lui en donnait
part, de parler fort peu à propos de la joie qu'elle aurait de revoir
son autre fils, le roi d'Espagne, etc., avec tous ses titres. Cela
suspendit le deuil, et lui fit renvoyer sa lettre.

Saint-Frémont mena un gros détachement de l'armée de Flandre en
Allemagne. Les ennemis y en firent un plus gros, et sur le bruit que le
prince Eugène l'y devait mener lui-même, on en fit un autre pour le
devancer. On sut, en même temps, que le roi d'Espagne donnait en toute
souveraineté à l'électeur de Bavière tout ce qui lui restait aux
Pays-Bas. De places, il n'y avait que Luxembourg, Namur, Charleroy et
Nieuport\,; il y avait longtemps que cela lui était promis. Il arriva en
même temps à une petite maison des Moreau, riches marchands de drap au
village de Villiers, près Paris, d'où il vint à Marly descendre à
l'appartement de feu Monseigneur\,; Torcy l'y fut trouver et y conféra
longtemps avec lui. Il le mena ensuite dans le cabinet du roi, où il
demeura jusqu'à cinq heures, et en sortit avec l'air très-satisfait. On
fut de là courre le cerf. L'électeur joua au lansquenet dans le salon
avec M\textsuperscript{me} la Dauphine après la chasse, et à dix heures
fut souper chez d'Antin. Il retourna coucher à Villiers, et partit trois
ou quatre jours après pour Namur.

Il envoya le comte d'Albert faire ses remercîments en Espagne, et y
prendre soin de ses affaires. En même temps le comte de La Marck alla
servir de maréchal de camp, et de ministre sans caractère public, auprès
de l'électeur de Bavière. Fort peu après, Gassion défit douze bataillons
et dix escadrons des ennemis auprès de Douai, sur lesquels il tomba à
deux heures après minuit. Il avait fort bien dérobé sa marche, et ils ne
l'attendaient pas. Il leur tua quatorze ou quinze cents hommes et ramena
douze ou treize chevaux. Ce Gassion était petit-neveu du maréchal de
Gassion, et il avait quitté les gardes du corps, à la tête desquels il
était arrivé, pour servir en liberté et en plein de lieutenant général,
et arriver au bâton de maréchal de France. C'était un excellent officier
général et un très-galant homme.

L'assemblée extraordinaire du clergé, qui finissait, vint haranguer le
roi à Marly. Le cardinal de Noailles, qui en était seul président, était
à la tête. Nesmond, archevêque d'Alby, porta la parole, dont je ne
perdis pas un mot. Son discours, outre l'écueil inévitable de l'encens
répété et prodigué, roula sur la condoléance de la mort de Monseigneur,
et sur la matière qui avait occupé l'assemblée. Sur le premier point, il
dit avec assez d'éloquence ce dont il était susceptible, sans rien
outrer. Sur l'autre il surprit, il étonna, il enleva\,; on ne peut
rendre avec quelle finesse il toucha la violence effective avec laquelle
était extorqué leur don prétendu gratuit, ni avec combien d'adresse il
sut mêler les louanges du roi avec la rigueur déployée à plein des
impôts. Venant après au clergé plus expressément, il osa parcourir, tous
les tristes effets d'une si grande continuité d'exactions sur la partie
sacrée du troupeau de Jésus-Christ qui sert de pasteur à l'autre, et ne
feignit point de dire qu'il se croirait coupable de la prévarication la
plus criminelle, si, au lieu d'imiter la force des évêques qui parlaient
à de mauvais princes et à des empereurs païens, lui, qui se trouvait aux
pieds du meilleur et du plus pieux de tous les rois, il lui dissimulait
que le pain de la parole manquait au peuple, et même le pain de vie, le
pain des anges, faute de moyens de former des pasteurs, dont le nombre
était tellement diminué, que tous les diocèses en manquaient sans savoir
où en faire. Ce trait hardi fut paraphrasé avec force, et avec une
adresse admirable de louanges pour le faire passer. Le roi remercia
d'une manière obligeante pour celui qui avait si bien parlé. Il ne
dédaigna pas de mêler dans sa réponse des espèces d'excuses et
d'honnêtetés pour le clergé. Il finit, en montrant le Dauphin, qui était
près de lui, aux prélats, par dire qu'il espérait que ce prince, par sa
justice et ses talents, ferait tout mieux que lui, mêlant quelque chose
de touchant sur son âge et sa mort peu éloignée. Il ajouta que ce prince
réparerait envers le clergé des choses que le malheur des temps l'avait
obligé d'exiger de son affection et de sa bonne volonté, il en tira pour
cette fois huit millions d'extraordinaire. Toute l'assistance fut
attendrie de la réponse, et ne put se taire sur les louanges de la
liberté si nouvelle de la harangue et l'adresse de l'encens dont il sut
l'envelopper. Le roi n'en parut point choqué, et la loua en gros et en
peu de mots, mais obligeants, à l'archevêque, et le Dauphin parut touché
et peiné de ce que le roi dit de lui. Le roi fit donner un grand dîner à
tous les prélats et députés du second ordre, et de petits chariots
ensuite pour aller voir les jardins et les eaux.

À la harangue de l'ouverture que prononça le cardinal de Noailles, le
roi, en montrant le Dauphin au clergé, avait dit\,: «\,Voilà un prince
qui, par sa vertu et sa piété, rendra l'Église encore plus florissante
et le royaume plus heureux.\,» C'était aussi à Marly.

Le Dauphin fut fort attendri, et s'en alla, aussitôt après la réponse du
roi, recevoir dans la chambre la harangue des mêmes députés par le
cardinal de Noailles, qui le traita de Monseigneur, et sans ajouter,
comme avait fait le premier président à la tête de la députation du
parlement, que c'était par l'ordre exprès du roi. La harangue fut belle,
et la réponse courte, sage, polie, modeste, précise,
M\textsuperscript{me} la Dauphine les reçut ensuite chez elle, le
cardinal de Noailles portant toujours la parole. Revenons aux obsèques
de Monseigneur.

On a vu (p.~153 de ce volume) que le genre de la maladie dont il était
mort n'avait permis aucunes cérémonies, et avait fait tout aussitôt
après brusquer son enterrement. Le 18 juin, qui était un jeudi, fut pris
pour le service de Saint-Denis, où se trouvèrent, à l'ordinaire, le
clergé et les cours supérieures. Le Dauphin, M. le duc de Berry et M. le
duc d'Orléans firent le deuil. Le duc de Beauvilliers, premier
gentilhomme de la chambre unique du Dauphin, assisté de Sainte-Maure, un
des menins de Monseigneur, et de d'O, qui l'était du Dauphin, porta sa
queue. Béthune-Orval, depuis devenu duc de Sully, lors premier
gentilhomme de la chambre de M. le duc de Berry, et Pons, maître de sa
garde-robe, portèrent la sienne. Simiane et Armentières, tous deux
premiers gentilshommes de la chambre de M. le duc d'Orléans, portèrent
la sienne\,; ainsi il en eut deux comme M. le duc de Berry, et cette
égalité parut extraordinaire. Comme il n'y avait point d'enterrement, il
n'y eut point d'honneurs\footnote{Il a été question, t. V. p.~311, note,
  des \emph{honneurs} employés au sacre, au baptême, aux obsèques des
  princes, etc.}, ni personne, par conséquent, pour les porter.
L'archevêque-duc de Reims, depuis cardinal de Mailly, officia, et
Poncet, évêque d'Angers, y fit une très-méchante oraison funèbre.

Le roi eut envie que les ducs y assistassent, et fut sur le point de
l'ordonner. Après, l'embarras des séances le retint\,; mais, désirant
toujours qu'ils y allassent, il s'en laissa entendre. Je contribuai à
les en empêcher, de sorte qu'il ne s'y en trouva aucun autre que le duc
de Beauvilliers, par la nécessité de sa charge. Cela fut trouvé mauvais,
et le roi se montra un peu blessé de ce qu'aucun de ceux qui étaient à
Marly n'avait disparu ce jour-là, et plus encore quand il sut qu'il ne
s'en était trouvé aucun autre à Saint-Denis. Personne ne répondit\,; on
laissa couler la chose, et on tint la même conduite pour le service de
Notre-Dame, où pas un duc ne se trouva.

Ce fut le vendredi 3 juillet. Les trois mêmes princes y lirent le deuil.
M. le duc de Berry et M. le duc d'Orléans eurent les mêmes porte-queues.
Le duc de Beauvilliers porta celle du Dauphin, et y fut assisté par
d'Urfé, menin de Monseigneur, et Gamaches, qui l'était du Dauphin. Le
clergé et les cours supérieures s'y trouvèrent à l'ordinaire. Les trois
princes s'habillèrent à l'archevêché et vinrent à pied en cérémonie de
l'archevêché au grand portail de Notre-Dame, par où ils entrèrent. Le
cardinal de Noailles officia, et le P. La Rue, jésuite, tira d'un si
maigre sujet une oraison funèbre qui acheva d'accabler celle de l'évêque
d'Angers. Le cardinal de Noailles traita ensuite les trois princes à un
dîner magnifique\,; le Dauphin le fit mettre à table et les seigneurs
qui l'avaient suivi. Il se surpassa en attentions et en politesses, mais
mesurées avec discernement. Il voulut que toutes les portes fussent
ouvertes et que la foule même le pressât. Il parla à quelques-uns de ce
peuple avec une affabilité qui ne lui fit rien perdre de la gravité
qu'exigeait la triste écorce de la cérémonie\,; et il acheva de charmer
cette multitude par le soin qu'il fit prendre d'une femme grosse qui s'y
était indiscrètement fourrée, et à qui il envoya d'un plat dont elle
n'avait pu dissimuler l'extrême envie qui lui avait pris d'en manger. Ce
ne furent que cris d'acclamations et d'éloges à son passage à travers
Paris, qui du centre gagnèrent bientôt le sentiment des provinces\,:
tant il est vrai qu'en France il en coûte peu à ses princes pour s'y
faire presque adorer. Le roi remarqua bien la conduite des ducs à ce
second service, mais il n'en témoigna rien. La fin de cette cérémonie
fut l'époque de la mitigation du salon de Marly, qui reprit sa forme
ordinaire, comme on l'a dit d'avance.

Il est temps à présent d'en venir à la situation où je me trouvai avec
le nouveau Dauphin, qui développera bien des grandes parties de ce
prince et de choses curieuses. Mais il faut auparavant essuyer une
bourre que je voudrais pouvoir éviter, mais qu'on verra par une prompte
suite inévitable à faire précéder un récit plus intéressant.

Il faut se souvenir de ce qui se trouve (t. VII, p.~269) des usurpations
sur les droits de gouverneur de Blaye, que le maréchal de Montrevel ne
cessait de faire comme commandant en chef en Guyenne, et qui
m'empêchèrent d'y aller, lorsqu'en 1709, les dégoûts que j'ai détaillés
alors me résolurent à me retirer pour toujours de la cour, et qui
finirent en m'y rattachant plus que jamais à la fin de cette année et au
commencement de la suivante, comme je l'ai raconté sur ces temps-là.
Chamillart, avant de quitter à Desmarets le contrôle général des
finances, avait fait un édit de création jusqu'alors inconnue d'offices
militaires, mais héréditaires, pour commander les gardes-côtes,
c'est-à-dire les paysans dont les paroisses bordent les côtes des deux
mers qui baignent la France, et qui, sans autre enrôlement que le devoir
et la nécessité de leur situation, sont obligés en temps de guerre de
garder leurs côtes, et de se porter où il est besoin. Cette érection fut
assaisonnée, comme toutes les autres de ce genre de finance, de tous les
appât de droits et de prérogatives, propres à en tirer bien de l'argent
des légers et inconsidérés François, qui n'ont pu se guérir de courre
après ces leurres, quoique si continuellement avertis de leur néant par
la dérision que les pourvus essuient sans cesse au conseil, dès qu'ils y
portent des plaintes du trouble qu'ils reçoivent dans leurs priviléges,
et à qui, à la paix, on supprime les titres mêmes qu'ils ont achetés.

Cette drogue bursale fut aussitôt donnée à Pontchartrain pour en tirer
ce qu'il pourrait, en déduction de ce qui était dû à la marine.

Celui-ci, ardent à usurper et à étendre sa domination, trouva cette
affaire fort propre à grossir ses conquêtes. Il prit thèse de ce qu'elle
lui était donnée pour remplacement des fonds très-arriérés de la marine,
et pour cela même, de la raison de l'augmenter et de l'en laisser le
maître\,; il s'en fit donner le projet d'édit, et le changea, le grossit
et le dressa comme il lui plut. Il ne négligea pas d'y couler une
clause, par laquelle ces nouveaux officiers gardes-côtes n'obéiraient
qu'aux seuls gouverneurs, commandants en chef et lieutenants-généraux
des provinces, et seraient sous la charge de l'amiral et du département
de la marine. Il en ôta celle qui restreignait la création aux lieux où
la garde des côtes était seulement en usage de tout temps\,; et non
content d'y comprendre toute la vaste étendue des côtes des deux mers,
il y ajouta les deux bords des rivières qui s'y embouchent, en remontant
fort haut, et y prit la précaution de dénommer les lieux jusqu'où cela
devait s'étendre sur chacune. Il forma ainsi des capitaines
gardes-côtes, non-seulement le long des deux mers, mais fort avant dans
les terres, par le moyen des bords des rivières, et mit tous ces pays en
proie aux avanies et aux vexations de ceux qu'il pourvut de ces charges.

Je ne sus rien de tout cela que lorsque Pontchartrain eut bien consommé
son ouvrage, et qu'il me dit alors, sans aucune explication, que je
ferais bien de chercher quelqu'un qui me convînt pour la garde-côte de
mon gouvernement. Je pris cet avis pour un désir de trouver à débiter sa
marchandise, et je ne m'en inquiétai pas. Assez longtemps après il m'en
reparla, et me pressa de lui trouver quelqu'un, pour éviter qu'un
inconnu venu au hasard ne me fit de la peine. Je lui répondis que qui
que ce fût qui prît cette charge de garde-côte ne pouvait s'empêcher d'y
être sous mes ordres, et qu'ainsi peu m'importait qui le fût. Il ne m'en
dit pas davantage, et la chose en demeura là pour lors.

Dans la suite, je voulus faire régler mon droit et les prétentions du
maréchal de Montrevel par Chamillart, pour sortir d'affaires\,;
Montrevel ne l'osa refuser, et il céda d'abord les milices de Blaye.
Elles avaient dans tous les temps été sous la seule autorité de mon
père, et leurs officiers pourvus par des commissions en son nom. M. de
Louvois, avec qui il n'avait jamais été bien, et qui n'ignorait pas cet
usage, n'avait jamais songé à le contester. Chamillart, tout mon ami
qu'il était, fut plus secrétaire d'État que Louvois. Il me fit entendre
que le roi ne s'accommoderait pas de cet usage, dont toutefois il
s'était toujours accommodé, mais dont, en style de secrétaire d'État, le
pauvre Chamillart ne s'accommodait pas lui-même\,; mais il me dit que je
n'avais qu'à nommer, et que, sur ma nomination, l'expédition se ferait
en ses bureaux.

Alors Pontchartrain, qui suivait sournaisement et avec grande attention
les suites de mes contestations avec le maréchal de Montrevel, et aux
questions duquel je répondais sans défiance, parce que je ne lui voyais
point d'intérêt là dedans, me dit-que, puisqu'il fallait une expédition
au nom du roi sur ma nomination, comme il pensait de même que
Chamillart, et par le même intérêt, c'était aux bureaux de la marine et
non en ceux de la guerre qu'elle devait être faite\,; fondé sur ce que
ces officiers nommés par moi serviraient sous La Motte d'Ayran,
capitaine de vaisseau, qu'il avait désigné garde-côte pour Blaye et tout
ce pays-là, et qu'aux termes de l'édit, ces capitaines gardes-côtes
étaient sous la charge de l'amiral et du département de la marine.
Chamillart, au contraire, regardait ces milices comme troupes de terre,
ainsi qu'elles avaient toujours été, et il s'appuyait sur leur
comparaison avec les milices du Boulonais qui borde la mer, qui avait un
capitaine garde-côte de cette nouvelle création, lesquelles cependant
étaient demeurées troupes de terre, et dont les officiers s'expédiaient
au bureau de la guerre sur la nomination de M. d'Aumont, gouverneur de
Boulogne. Ces deux secrétaires d'État, de longue main aigris et hors de
mesure ensemble, s'opiniâtrèrent dans leurs prétentions, et à en porter
le jugement au roi.

Le plus court et le plus simple était de me laisser suivre l'ancien
usage, qui n'avait point été contredit, et d'éviter cette nouvelle
querelle entre eux, en me laissant donner les commissions en mon nom\,;
mais cette sagesse n'accommodait pas l'usurpation commune de leurs
charges aux dépens de la mienne, quoique si intimement lié avec tous les
deux. Ils l'eussent également mis à couvert en acceptant la proposition
que je leur fis de faire expédier aux bureaux de La Vrillière,
secrétaire d'État ayant la Guyenne dans son département. Aucun des deux
n'y voulut entendre, ni démordre de sa prétention. Chamillart, dans la
faveur où il était alors, et appuyé de l'exemple de Boulogne, l'aurait
emporté, et Pontchartrain en aurait eu tout le dégoût. C'était commettre
mes deux amis, si ennemis, ensemble\,; je crus donc devoir suspendre ma
nomination. Le chancelier et son fils m'en remercièrent, et parurent
sentir l'amitié de ce sacrifice, piqué au point où je l'étais contre
Montrevel, et aussi intéressé à me remettre en possession de mes milices
et dégrossir d'autant les contestations à décider entre nous. Dans cette
situation, le temps s'écoula jusqu'à la chute de Chamillart, comme je
crois l'avoir raconté en son lieu, et Montrevel refusa tout net le
maréchal de Boufflers d'en passer par son avis.

Pendant tout cela, je voulus profiter de la nouveauté de Voysin dans la
charge de Chamillart, qui n'aurait pas l'éveil de cette dispute, et
faire expédier aux bureaux de la marine. La vie coupée de la cour, le
mariage de M\textsuperscript{me} la duchesse de Berry, avec tout ce qui
précéda et suivit cette grande affaire, et mille autres enchaînements,
traînèrent ma nomination jusqu'à l'hiver qui précéda la mort de
Monseigneur. Je voulus donc enfin terminer une chose dont le délai était
indécent, et nuisible même au service. Mais quelle fut ma surprise,
lorsque, sur le point de nommer, Pontchartrain me déclara que c'était un
droit du capitaine garde-côte, ajoutant aussitôt que La Motte d'Ayran ne
l'exercerait qu'avec mon agrément, par où il n'aurait que l'apparence,
dont je conserverais la réalité.

J'eus la sagesse de me contenir, et de descendre jusqu'à plaider ma
cause. J'alléguai les commissions de mon père que j'étais en état de
rapporter\,; le droit immémorial et la clarté de ce droit par la cession
de Montrevel même, qui, si actif et si roide en prétentions, s'était vu
forcé d'abandonner celle-là de lui-même, après l'avoir si vivement
soutenue\,; l'étrange contraste d'être dépouillé d'un droit si certain
par un homme qui m'était nécessairement subordonné, et que j'exerçais
indépendamment du gouverneur de la province représenté en tout par le
commandant en chef. Je ne dédaignai pas de lui dire qu'il était plus
honorable pour lui d'expédier sur ma nomination que sur celle d'un
capitaine garde-côte\,; enfin je le fis souvenir du sacrifice que je lui
avais fait trois ans durant de suspendre ma nomination, que ni lui ni
Chamillart ne me contestaient, mais qui voulaient chacun expédier
dessus\,; des remercîments que le chancelier et lui m'avaient faits de
ne les pas commettre avec ce ministre dans sa faveur si supérieure, et
de l'indigne fruit que j'en retirais par la perte de mon droit, qui
était ce que je pouvais attendre de pis d'un ennemi en sa place, lui si
personnellement engagé, dans ce fait même, et en général par l'alliance
si proche et une si longue et si intime amitié et si éprouvée de sa
part, à chercher à augmenter mon autorité à Blaye, et non pas à me
dépouiller de celle que j'y avais de droit, d'usage, et de tout temps.
Rien de tout cela ne fut contesté\,; j'eus un aveu formel sur chaque
article\,; toutefois je parlais aux rochers.

Pontchartrain se retrancha sur l'attribution formelle de l'édit, et par
cela même se chargeait d'un nouveau crime, puisqu'il l'avait changé et
amplifié à dessein. Je me défendis sur la notoriété publique que ces
édits, uniquement faits pour tirer de l'argent, n'avaient point d'effet
contre des possessions et des titres, souvent même contre ce qui n'en
avait point. J'en donnai l'exemple de M. d'Aumont pour Boulogne, rivage
de la mer vis-à-vis d'Angleterre, moi si loin d'elle et si avancé dans
les terres, et celui des divers édits de création de charges municipales
dont les traitants avaient voulu jouir à Blaye, où j'avais toujours
maintenu les jurats de ma nomination.

Pontchartrain répliqua que les édits ne pouvaient nuire au service\,;
qu'il en était que les milices de Boulogne, si voisines de la frontière,
continuassent d'y servir, ce qui emportait exception de l'édit à leur
égard\,; ce qui n'était point à l'égard de Blaye, nommément compris dans
l'édit pour un capitaine garde-côte, c'est-à-dire dans un supplément
postérieur de l'édit qu'il avait fait ajouter, que ce qui m'était arrivé
pour les jurats de Blaye marquait bien que j'aurais pu avoir le même
succès sur l'édit des gardes-côtes, si je m'en fusse plaint à temps,
mais qu'il était maintenant trop tard. Je répondis que je n'avais parlé
sur les jurats que lorsque les traitants avaient voulu vendre ces
charges à Blaye, et longtemps après les édits rendus\,; que Chamillart,
puis Desmarets, m'avaient, l'un après l'autre, fait justice au moment
que je l'avais demandée, quoiqu'ils n'y fussent pas tenus comme lui
l'était par une obligation réelle et essentielle sur ce même fait,
laquelle il me donnait maintenant pour un obstacle invincible. Ces
derniers mots, prononcés avec feu, coupèrent la parole à Pontchartrain.
Il se jeta dans les protestations que ma satisfaction lui était si chère
qu'il ferait jusqu'à l'impossible pour me la procurer, et que nous en
parlerions une autre fois. L'embarras du procédé et de la misère des
raisons le réduisait à chercher à finir une conversation si difficile
pour lui à soutenir. Le dépit, qui de moment à autre s'augmentait en
moi, d'une tromperie si préparée et si étrangement conduite par une si
noire ingratitude, avait besoin de n'être plus excité. Je ne cherchai
donc aussi qu'à la finir.

J'ai annoncé de la bourre, et je suis obligé d'avertir que ce n'est pas
fait, mais qu'elle est absolument nécessaire aux choses qui la suivront
et qui en dédommageront. Pour la continuer, M\textsuperscript{me} de
Saint-Simon, aussi surprise que moi de ce que je lui racontai, mais
toujours plus sage, m'exhorta à ne rien marquer, à vivre avec
Ponchartrain à l'ordinaire, à laisser reposer cette fantaisie, à la
laisser dissiper et à ne pas croire qu'il pût s'aheurter à une
prétention qui le devait toucher si peu, et sur laquelle il me voyait si
sensible. J'en usai comme elle le désira, accoutumé par amitié et par
une heureuse expérience à déférer à ses avis.

Au bout de quelque temps elle lui parla. Il se confondit en respects,
mais sans rien de plus solide. Peu après, étant à Marly, il me dit qu'il
était résolu à tout faire pour me contenter\,; qu'il croyait néanmoins
qu'il valait mieux ne point traiter l'affaire ensemble\,; et qu'il me
priait de trouver bon d'entendre là-dessus d'Aubanton, un de ses
premiers commis. J'y consentis sans entrer plus avant en matière.

Deux jours après, Aubanton vint un matin chez moi. J'écoutai patiemment
une flatteuse rhétorique pour me faire goûter ce que Pontchartrain
m'avait proposé. Je voulus bien expliquer les mêmes raisons que j'ai
abrégées plus haut. Aubanton n'eut rien à y répondre, sinon d'essayer de
me persuader que, par la nécessité de mon agrément, j'avais le fond de
la chose, et le capitaine garde-côte l'écorce par sa nomination. Je
voulus bien encore parler honnêtement. Je répondis qu'il était du bon
sens de la prudence et de l'usage, de terminer les choses durables d'une
manière qui le fût aussi\,; que je voulais bien ne pas douter qu'aucune
nomination du capitaine garde-côte ne serait expédiée que de mon
agrément, tant que Pontchartrain et moi serions, lui en place
d'expédier, moi d'agréer ou non, mais que cela pouvait changer par la
mutation de toutes les choses de ce monde, qu'alors je serais pris pour
dupe par un autre secrétaire d'État qui ne se croirait pas tenu aux
mêmes égards\,; qu'avec Pontchartrain même ces égards pouvaient devenir
susceptibles de mille queues fâcheuses, lorsque le capitaine garde-côte
et moi ne serions pas d'accord sur les choix, qu'il était donc plus
court et plus simple de me laisser continuer à jouir de mon droit, et
qu'après tout ce qui s'était passé là-dessus de si personnel à
Pontchartrain de ma part, je ne pouvais croire qu'il aimât mieux un
capitaine garde-côte que moi, jusqu'à l'enrichir de ma dépouille.
Honnêtetés de ma part, mais avec grande fermeté, respects et
protestations de celle d'Aubanton, terminèrent cette inutile visite. Il
me pressa de lui accorder encore une audience, et de penser moi-même à
quelque expédient que Pontchartrain embrasserait sûrement avec transport
de joie.

Huit jours après, Aubanton revint avec force compliments pour toutes
choses. J'avais cependant rêvé à quelque expédient pour me tirer
d'embarras sans tout perdre et sans me brouiller. J'en étais retenu par
le respect d'une liaison de vingt ans, de la mémoire de celle dont
l'alliance l'avait formée, de l'intimité du chancelier et de la
chancelière, auxquels je n'avais pas dit un mot de tout cela jusqu'alors
pour en attendre le dénoûment, et ces considérations enchaînèrent ma
colère d'un procédé si double et si indigne. Je les fis donc sentir à
Daubanton, et lui dis qu'elles m'avaient amené à un expédient où je
mettais tant au jeu que j'étais surpris moi-même d'avoir pu m'y
résoudre, mais que l'amitié l'avait emporté\,: c'était d'accepter la
nomination des officiers des milices de Blaye par le capitaine
garde-côte, qui ne serait expédiée que de mon agrément, comme
Pontchartrain le proposait, mais d'y ajouter au moins, pour que cet
agrément demeurât solide et nécessaire, la nécessité de mon attache sur
les expéditions, à l'exemple en très-petit de l'attache du colonel
général de la cavalerie sur les commissions de tous les officiers de la
cavalerie. Aubanton avec esprit me laissa voir qu'il goûtait fort
l'expédient, et en même temps qu'il n'espérait pas qu'il fût accepté. Il
me quitta en prenant jour pour la réponse.

Elle fut telle qu'Aubanton l'avait prévue. Il me dit que Pontchartrain
n'osait expédier en une forme insolite sans permission du roi, à qui il
ne croyait pas qu'il fut à propos pour moi de la demander. Je répondis à
d'Aubanton en remontant mon ton, sans sortir pourtant d'un air de
politesse pour lui, et de modestie pour moi, que je n'étais pas surpris
qu'une telle affaire eût une pareille issue depuis que Pontchartrain en
avait fait la sienne propre\,; que c'était le prix de vingt ans
d'amitié, et de ma complaisance du temps de Chamillart pour n'en pas
dire davantage\,; qu'après ce sacrifice si bien senti alors par lui, et
dans une alliance si proche qu'il pouvait un peu compter, il me faisait
un tour que je ne pourrais attendre d'un autre secrétaire d'État en sa
place avec qui je serais dans la plus parfaite indifférence\,; que
j'entendais bien le nœud de la difficulté, qui était qu'à l'ombre d'une
nomination subalterne et obscure d'un capitaine garde-côte, si fort sous
sa main, il ferait de ces emplois les récompenses de ses laquais\,;
qu'il y avait tant de distance de l'étendue du pouvoir de sa charge aux
bornes si étroites de mon gouvernement que je ne laissais pas d'être
surpris qu'il pût être touché de l'accroître de ma dépouille, jusqu'à
l'avoir si adroitement, si longuement et si ténébreusement ménagée\,;
que, tant que j'avais cru n'avoir affaire qu'à un édit bursal et à un
capitaine garde-côte, l'évidente bonté de mes raisons me les avait fait
soutenir\,; que voyant clair enfin, et ne pouvant plus méconnaître ce
que je m'étais caché à moi-même tant que j'avais pu, je savais trop la
disproportion sans bornes du crédit de la place de Pontchartrain à celui
d'un duc et pair, et d'un homme de ma sorte, pour prendre le parti de
lutter avec lui\,; que je sentais dans toute son étendue la facile
victoire qu'il remportait sur moi, et les moyens obscurs qui pied à pied
la lui acquéraient\,; que je cédais dans la pleine connaissance de mon
impuissance, mais qu'en cédant je cédais tout, et n'entendrais jamais
parler sur quoi que ce pût être des milices de Blaye.

Aubanton effrayé d'une déclaration si compassée, car je me possédais
tout entier, mais si nette et si expressive dans ses termes, dans son
ton, dans toute ma contenance, et peut-être par le feu échappé de mes
regards, déploya pour me ramener le reste de son bien dire. Il m'étala
les respects et les désirs de Pontchartrain\,; il me représenta
adroitement qu'en abandonnant jusqu'à la discipline et au commandement
des milices de Blaye, je me faisais un tort à quoi rien ne m'obligeait,
et qui dans la suite me pourrait sembler trop précipité. Je sentis à son
discours et à son maintien l'extrême honte que lui donnait sa misérable
ambassade, et les suites que, tout premier commis qu'il était d'un
cinquième de roi de France, il n'était pas hors d'état de prévoir. Toute
ma réponse fut un simple sourire, et de me lever. Alors il me conjura de
ne pas regarder l'affaire comme finie, je l'interrompis par des
honnêtetés personnelles, et de la satisfaction de l'avoir connu, et je
l'éconduisis de la sorte.

Outré de colère et d'indignation, je me donnai quelques jours. Mené
après toujours par les mêmes motifs, je voulus abuser de ma patience et
jouir aussi de l'embarras d'un si misérable ravisseur. Il me dit en
paroles entrecoupées qu'il s'estimait bien malheureux que mon amitié fût
au prix de l'impossible. Je répondis d'un air assez ouvert que je la
croyais bien au-dessous\,; qu'apparemment il avait vu Aubanton\,; que
cela étant, la matière était épuisée et inutile à traiter. Il répliqua
d'un air confondu quelques demi-mots sur l'ancienneté de l'amitié. Je
lui dis d'un air simple que je ne demandais jamais ce qu'on ne pouvait
pas\,; que je cédais tout, et qu'après cela il n'y avait plus à en
parler. Là-dessus il me donna carte blanche pour nous en rapporter à qui
je voudrais. Je n'ignorais pas quel jugement je pouvais attendre entre
lui et moi dans une cour aussi servile\,; ainsi je répondis qu'à une
affaire finie il ne fallait point de juge. Alors il me proposa son père,
je n'eus pas la force de le refuser. Jusqu'alors qui que ce soit n'avait
su ce qui se passait entre nous. J'ai dit ci-devant ce qui me retenait
d'éclater, et il n'avait garde aussi de montrer son tissu d'infamie.

Revenus à Versailles (car le chancelier ne paraissait à Marly qu'au
conseil), je lui contai ce qu'il ignorait depuis la chute de Chamillart.
Il ne balança pas à me réitérer ses remercîments de la suspension de ma
nomination avant cette chute\,; fit après une longue préface sur son peu
d'indulgence pour son fils, ses défauts, ses sottises, la parfaite
connaissance et la parfaite douleur qu'il en avait, et de là me répéta
toutes ses raisons entortillées de sophismes qu'il avait excellemment à
la main quand il en avait besoin\,; les entremêla d'autorité, et
prétendit enfin que je réduisais son fils à l'impossible. Mon extrême
surprise m'ôta toute repartie. Je lui dis seulement que je ne me croyais
de tort que de n'avoir pas nommé sans ménagement du temps de
Chamillart\,; mais la parole me rentra tout à fait dans la poitrine par
sa réplique, que j'aurais bien fait d'avoir nommé alors, et je ne
songeai qu'à gagner la porte.

On a vu en différents endroits dans quelle amitié et dans quelle
confiance réciproque je vivais avec le chancelier, et avec quelle
adresse, de concert avec M\textsuperscript{me} de Saint-Simon, il
m'empêcha de quitter la cour à la fin de 1709, où je me trouvais
maintenant dans la situation la plus agréable, et comme on le verra
incontinent, dans les espérances les plus flatteuses et les plus
solidement fondées. Ce contraste avec l'état où je me serais trouvé dans
la retraite que je voulais faire étreignit à son égard la colère de le
voir soutenir la perfidie de son fils, mais à la vérité pour la porter
sur ce fils tout entière, tellement que je finis une seconde
conversation avec le chancelier par lui dire que la matière était
épuisée, que nous ne nous persuaderions pas l'un l'autre, que je ne
répondrais plus un seul mot à tout ce qu'il pourrait m'en dire, mais
qu'il trouverait bon aussi que je demeurasse dans ma résolution de
n'ouïr jamais parler en rien des milices de Blaye, et d'en laisser faire
à son fils et à son capitaine garde-côte tout ce que bon leur
semblerait. Le chancelier entendit ce français\,; il me répondit avec
embarras et quelque honte, que je faisais mal, mais que j'étais le
maître.

Lui, la chancelière et Pontchartrain pressèrent extrêmement
M\textsuperscript{me} de Saint-Simon de m'engager à acheter la
capitainerie garde-côte de Blaye, et il parut bientôt qu'ils n'avaient
pas prévu l'embarras où les jetait ma fermeté, à laquelle ils ne
s'étaient pas attendus, et qu'ils auraient bien voulu ne s'être pas
engagés si avant, c'est-à-dire le fils, dans une si vilaine affaire,
projetée et conduite à son ordinaire sans la participation de son père,
et celui-ci à ne l'y pas soutenir quand il l'eut apprise pour être
arbitre entre nous deux.

Pour se tirer d'un si mauvais pas, ils proposèrent à
M\textsuperscript{me} de Saint-Simon d'emprunter de celui qu'ils lui
nommeraient le prix de cette capitainerie, soit que ce fût un prêteur
effectif, soit qu'il ne donnât que son nom pour couvrir leur bourse avec
stipulation expresse qu'il se contenterait des gages de la charge pour
tout intérêt de la somme, et sans être tenus de les lui faire bons au
cas qu'ils ne fussent point payés\,; de n'avoir que la charge même pour
toute hypothèque, et à sa perte si elle se supprimait et était mal ou
point payée sans pouvoir nous en jamais rien demander, et de porter seul
toutes les taxes, augmentations de gages, et toute autre espèce de
choses dont on accablait tous les jours ces nouvelles créations, sans
que nous y pussions entrer pour rien\,: c'était, en un mot, que je
voulusse bien recevoir la charge sans bourse délier, et sans pouvoir y
courir aucune sorte de risque.

J'étais si aigri, que je fus longtemps sans en vouloir ouïr parler. Je
consentis enfin, par complaisance pour M\textsuperscript{me} de
Saint-Simon, mais à condition que devant ni après la chose faite, et qui
ne se fit point, ils ne m'en parleraient jamais.

Je vis rarement et sérieusement Pontchartrain depuis cette rare affaire,
et c'est où nous en étions à la mort de Monseigneur. Pour le chancelier,
je vécus avec lui tout à mon ordinaire\,; elle n'apporta pas le moindre
refroidissement entre nous, comme on le peut voir par ce qui a été
rapporté sur la prétention d'Épernon et de Chaulnes, et l'édit de 1711,
tant la reconnaissance eut de pouvoir sur moi. On verra bientôt qu'elle
ne se borna pas là.

\hypertarget{chapitre-xiv.}{%
\chapter{CHAPITRE XIV.}\label{chapitre-xiv.}}

1711

~

{\textsc{Splendeur du duc de Beauvilliers.}} {\textsc{- Causes, outre
l'amitié, de sa confiance entière en moi.}} {\textsc{- Discussion de la
cour entre lui et moi.}} {\textsc{- Torcy.}} {\textsc{- Desmarets.}}
{\textsc{- La Vrillière.}} {\textsc{- Voysin.}} {\textsc{- Pontchartrain
père et fils.}} {\textsc{- Caractère de Pontchartrain.}} {\textsc{- Je
sauve Pontchartrain perdu.}} {\textsc{- Je conçois le dessein d'une
réconciliation sincère entre le duc de Beauvilliers et le chancelier.}}
{\textsc{- Singulier hasard sur le jansénisme.}} {\textsc{-
Pontchartrain sauvé par le duc de Beauvilliers.}} {\textsc{-}}
{\textsc{- Conversation sur les Pontchartrain avec Beringhen, premier
écuyer.}} {\textsc{- Son caractère.}} {\textsc{- Union et concert le
plus intime entre les ducs et les duchesses de Beauvilliers, Chevreuse
et Saint-Simon.}} {\textsc{- Conduite du dernier avec le Dauphin, et sa
façon d'y être.}} {\textsc{- Mon sentiment sur le jansénisme, les
jansénistes et les jésuites.}}

~

Le duc de Beauvilliers jouissait avec splendeur de l'état si changé de
son pupille\,; il était affranchi des inquiétudes de la cour de
Monseigneur, et des mesures à l'égard du roi par la confiance que ce
monarque donnait à son petit-fils, et la solidité qu'y ajoutait le goût
et l'intérêt de M\textsuperscript{me} de Maintenon ravie d'aise pour sa
Dauphine, et d'avoir un Dauphin sur lequel elle pouvait sûrement compter
dans tous les temps. Beauvilliers commençait donc à marcher plus tête
levée, à cacher moins que le temps était venu de commencer à compter
avec lui\,; il montrait un maintien plus dégagé et une liberté moins
mesurée\,; ses propos avec moi plus fermes et à lui tout à fait
étrangers. J'aperçus un changement inespéré dont je ne le croyais pas
susceptible\,; je vis un homme consolidé, nerveux, actif, allant droit
au fait et se dépouillant des entraves. Il repassa toute la cour avec
moi sans se hérisser de ma franchise sur les portraits, et sans disputer
avec moi. Il se souvenait que je lui avais toujours parlé juste dans
tous les temps, l'expérience lui avait appris que j'en savais plus que
lui en connaissances de gens, que sa charité et son enfermerie
élaignaient de voir et d'apprendre. Mon avis sur Harcourt\,; ma
prédiction sur l'abbé de Polignac suivie de l'effet si peu croyable\,;
celle de la campagne de Lille, si précisément accomplie en effets
prodigieux, ne lui étaient point sortis de l'esprit, et avaient ployé le
sien à tout mon égard. Il était sûr de mon secret, j'ose dire de ma
vérité et de ma probité\,; il ne pouvait douter de toute ma confiance,
de mon dévouement, de mon attachement pour lui sans réserve et à toute
épreuve, et d'une amitié de toute préférence depuis plus de seize ans
que j'étais à la cour, et que mon désir de son alliance nous avait
étroitement unis. Il me parlait donc sans réserve, et la disproportion
d'âge et de fortune n'en mettait plus dans l'épanchement entier sur
toutes les matières, qui était pleinement réciproque et continuel.

Cet examen entre lui et moi de toute la cour allait à discuter qui il
était bon d'approcher ou d'éloigner du Dauphin. La ville eut aussi son
tour, c'est-à-dire la robe, non pas pour approcher ou écarter des gens
que leur état n'en rendait pas susceptibles, mais pour nous concerter
tous deux, car il m'avait mis à cette portée, et placer au Dauphin du
bien de ceux que nous estimerions propres aux emplois, et au contraire
sur les autres. Quatre ou cinq longues conversations près à près, que
nous eûmes tête à tête, ce que je remarque parce que le duc de Chevreuse
ne s'y trouva pas, achevèrent à peu près cette importante matière.

Suivit un autre tête-à-tête où le duc se déboutonna sur tous ceux qui
avaient part aux affaires. Je l'avais averti il y avait déjà longtemps
de l'intime liaison que je voyais se former entre d'Antin et Torcy. La
Bouzols, sœur du dernier, d'une figure hideuse, mais pleine de charmes,
d'esprit, et forte en intrigue, et de tout temps en toute intimité avec
M\textsuperscript{me} la Duchesse, en était le principal instrument.
Celle qui commençait à se montrer entre d'Antin et
M\textsuperscript{lle} de Tourbes qui ne fit que croître, et qui dura
autant que leur vie, y servit encore puissamment. C'était un autre démon
d'esprit et qui aimait à dominer, amie intime de Torcy, de sa sœur, peu
à ses frères le maréchal et l'abbé d'Estrées, tout à
M\textsuperscript{me} la Duchesse de toute leur vie. Rien n'était plus
opposé au duc de Beauvilliers que cette cabale de M\textsuperscript{me}
la Duchesse qui palpitait encore, et que d'Antin personnellement. Le duc
et Torcy étaient éloignés l'un de l'autre, mais en gens sages et
mesurés\,; l'écorce entre eux était conservée\,; le duc de Chevreuse la
ménageait quoique aussi refroidi que son beau-frère\,; l'idée de la cour
ne s'en apercevait pas, elle était accoutumée à l'union singulière de
toute la famille de Colbert\,; elle avait été témoin de celle des deux
ducs avec Pomponne depuis son retour jusqu'à sa mort, qui était de toute
confiance. La communication d'affaires et les bienséances voiloient au
monde prévenu et jusqu'aux plus éveillés le fond de leur situation
ensemble, et eux-mêmes avaient soin d'entretenir ce voile par le dehors
de leur conduite\,; mais le fond le voici.

On a vu quelle était l'extrême piété du duc de Beauvilliers, et quel
aussi son abandon pour M\textsuperscript{me} Guyon, surtout pour M. de
Cambrai, et pour tout ce petit troupeau, qui l'avait pensé perdre plus
d'une fois sans l'en avoir pu détacher le moins du monde, conséquemment
pour les jésuites et pour la partie sulpicienne qui n'avaient jamais
abandonné M. de Cambrai dans aucun temps. De là un aveuglement sur les
matières de Rome et sur le jansénisme, qui ne lui permettait pas de rien
voir ni de rien entendre. Plus le roi avançait en âge, plus sa
faiblesse, toujours sans contre-poids sur ces matières qu'il ignorait
profondément, se trouvait en proie aux jésuites et aux directeurs de
M\textsuperscript{me} de Maintenon par elle\,; plus donc Rome d'une
part, les jésuites de l'autre, gagnaient de terrain, et plus M. de
Beauvilliers y donnait à bride abattue, et c'était principalement depuis
la mort de Pomponne que le grand cours de ces choses avait commencé, et
sans cesse s'était augmenté. Torcy pensait là-dessus tout différemment.
Il connaissoit l'inestimable prix de la conservation des droits de la
couronne, de celle des libertés de l'école, et de celles de l'Église
gallicane\,; il ne connaissoit pas moins les ruses des jésuites et la
grossièreté des sulpiciens. Il était donc souvent opposé sur ces
matières au duc de Beauvilliers au conseil. Il était extrêmement
instruit, avait beaucoup d'esprit, d'honneur, de probité, de lumières\,;
mais sage, retenu, timide même, il ne disait que ce qu'il fallait dire
avec douceur et mesure, respect même, mais il le disait bien, parce
qu'il avait le don de la parole et celui encore de l'écriture\,; presque
toujours encore la raison était de son côté. M. de Beauvilliers, dont le
rang d'opiner était le pénultième des ministres, suait de l'encre
d'entendre Torcy, et plus encore à réfuter son avis qui entraînait plus
que très-souvent les autres ministres. Il sentait qu'il allait essuyer
le feu du chancelier qui opinait immédiatement après lui, et qui ne le
ménageait pas, quelquefois même jusqu'à l'indécence, tellement qu'il
regardait Torcy comme un avec le chancelier sur ces matières, et qui lui
fournissait des armes dont le chancelier se servait contre lui avec
impétuosité, et en général ajoutait aux raisons de Torcy le poids de son
esprit, de sa liberté, de son autorité. Cela s'appelait chez M. de
Beauvilliers être janséniste, et être janséniste était chez lui quelque
chose de plus odieux et de plus dangereux qu'être protestant.

Torcy avait encore deux crimes envers lui\,: l'un de n'avoir jamais eu
de liaison avec M. de Cambrai\,; l'autre d'être mari de
M\textsuperscript{me} de Torcy, qui avait en effet un véritable pouvoir
sur lui, qui du cœur passait à l'esprit. Elle en avait beaucoup
elle-même, et savait beaucoup aussi. Avec cela, libre et peu capable de
cacher ses sentiments, qui étaient tout à fait conformes à son nom. Ce
n'était pas pourtant qu'elle fût imprudente, encore moins qu'elle
affichât rien, mais on la démêlait. C'était donc aux yeux de M. de
Beauvilliers une manière d'hérétique qui pervertissait son mari, et qui
le tenait de trop près et de trop court pour espérer de le convertir,
même de le rendre moins opposé, ou plus complaisant.

M. de Chevreuse, malgré son abjuration de Port-Royal où il avait été
élevé, n'était pas si outré que son beau-frère. C'était un composé fort
bizarre à cet égard. Non moins abandonné à M\textsuperscript{me} Guyon,
à M. de Cambrai surtout, et à toute sa gnose, il avait retenu de son
éducation une aversion parfaite des jésuites qu'il cachait avec soin, où
je le surpris plus d'une fois, et qu'il ne me désavoua pas avec le
secret et la confiance qui était établie entre nous\,; par conséquent,
toujours en garde contre eux, et comme plus foncier que M. de
Beauvilliers, moins livré aux entreprises de Rome\,; je dis moins parce
qu'il était encore beaucoup. Ces gens de Port-Royal qu'il avait
abdiqués, l'estime et l'affection pour eux n'avait pu s'effacer en lui.
Il me l'a avoué de presque tous, et néanmoins en spéculation à eux, il
leur était contraire en pratique. Ce composé ne peut s'expliquer, mais
il était tel que je le représente. Cette façon d'être, jointe avec sa
douceur naturelle, son esprit compassé et si naturellement tourné à être
amiable compositeur\footnote{On appelait \emph{amiable compositeur}
  l'arbitre qui terminait un différend entre les parties à des
  conditions équitables, sans recourir à la rigueur de la justice.}, le
défaut d'occasion d'opinions contraires au conseil, où il n'entrait pas,
quoique effectivement et véritablement ministre, l'écartaient moins de
Torcy que le duc de Beauvilliers, et l'appliquaient à conserver tous les
dehors entre eux, n'y pouvant davantage.

Torcy, qui sentait parfaitement tout ce que le monde ne voyait pas dans
cet intérieur de famille, n'avait pas tort de vouloir s'appuyer de
d'Antin, et celui-ci, qui frappait en dessous à la porte du conseil,
avait raison de se lier à un homme dont la place lui pouvait donner des
moyens de se la faire ouvrir. En même temps moi, qui connaissois cet
intérieur, je ne fus pas surpris que le duc de Beauvilliers, discutant
les ministres avec moi, mît Torcy le premier sur le tapis, et m'en
parlât comme d'un homme qu'il était absolument nécessaire de remercier.

Lié où il était et dans une place qui ne me donnait ni rapport avec lui
ni aucun besoin de lui, je ne le connaissois alors que comme on connait
tout le monde\,; je n'allais jamais chez lui\,; lui aussi ne m'avait
jamais fait aucune avance, quoique nous eussions des amis communs. Je
n'étais pas content de lui sur M. le duc d'Orléans, et s'il faut tout
dire, son indifférence pour moi m'avait déplu. Je n'entrepris donc pas
sa défense avec M. de Beauvilliers, qui passa outre et me demanda qui je
pensais qu'on put mettre en sa place.

Amelot était bien le meilleur, mais il était trop lié à la princesse des
Ursins, trop bien par conséquent avec M\textsuperscript{me} de Maintenon
pour que ce fût l'homme de M. de Beauvilliers, ni le mien par rapport à
M. le duc d'Orléans, que je voulais unir de plus en plus avec le
Dauphin\,: je proposai donc Saint-Contest qui était fort de mes amis, et
d'amitié de père en fils. C'était un homme de beaucoup d'esprit et du
plus délié, sous un extérieur épais, appliqué, travailleur, et qui, avec
les manières les plus pleinement bourgeoises, connaissoit pourtant le
monde, la cour et les gens extrêmement bien, et qui dans son intendance
de Metz avait toujours réussi dans les affaires ou les négociations
qu'il avait eues fort souvent avec l'électeur palatin, celui de Trêves,
le duc de Lorraine, et plusieurs petits princes de ses environs\,; il
était doux, liant, insinuant, et savait aller à ses fins avec adresse et
en contentant ceux avec qui il avait à traiter, M. de Beauvilliers le
connaissoit et le goûtait assez, et il approuva beaucoup ma pensée, en
sorte que cela demeura comme arrêté entre nous.

Desmarets nous fit disputer. Le duc en était, comme je l'ai remarqué, à
n'oser plus lui parler de rien. Il ne pouvait donc se dissimuler son
humeur intraitable, ni l'excès de son ingratitude, mais ces défauts ne
touchaient point à la religion. Il ne donnait nul soupçon de jansénisme,
et il était bien loin encore de revenir au monde lors de la disgrâce de
l'archevêque de Cambrai\,: net sur des points à l'égard du duc si
capitaux, d'autres le sauvaient. Il était neveu de Colbert, élevé dans
les finances, à son école\,; il en avait pris, à ce que l'on pensait,
les principes et les maximes. Il passait pour l'homme le plus capable en
finances\,; enfin, M. de Beauvilliers l'avait ramené sur l'eau à force
de sueurs, de temps et de rames, et quel qu'il l'éprouvât, il ne put se
résoudre à détruire son ouvrage, et tout ce que j'alléguai ne fit que
blanchir. Il ne trouva jamais mieux à mettre en sa place\,; et il se
ferma à l'y laisser.

Nous fûmes aisément du même avis sur La Vrillière. Il convint avec moi
que pour ce que ce secrétaire d'État faisait, et quand même il serait
chargé de plus, il le faisait très-bien, et qu'il n'y avait point à
chercher mieux.

Voysin nous parut également à tous deux nécessaire a renvoyer\,: nulle
capacité, probité de cour, connaissance de personne, dureté, et
rusticité, créature de M\textsuperscript{me} de Maintenon jusqu'au
dernier abandon. Je voulus sonder le duc sur Chamillart, et je fus
édifié, touché môme de sa réponse\,: il me dit qu'il était son ami
depuis quarante ans, et que cette liaison il l'avait resserrée lui-même
par le mariage de sa nièce avec son fils\,; qu'il connaissoit sa probité
à toute épreuve, et ses lumières fort au-dessus de l'idée qu'on en avait
prise\,; mais qu'il croyait le Dauphin un obstacle invincible à son
retour\,; d'ailleurs que Chamillart avait deux défauts qu'il croyait
incompatibles avec le bien de l'État et dont il le savait incorrigible,
avec lesquels il se ferait un grand scrupule de le replacer\,: une
opiniâtreté invincible dont il me conta des traits qui m'étonnèrent,
quelque connaissance que j'eusse de cette opiniâtreté, dont j'ai
rapporté quelques-uns, et des amis sur lesquels il était incapable de
revenir, et dont l'entêtement était extrêmement dangereux. De ce dernier
j'en avais une parfaite expérience qui se trouve répandue ici en plus
d'un endroit. Je fus affligé avec d'autant plus d'amertume que je fus
convaincu, et qu'il fallut me détacher du plaisir extrême de contribuer
à remettre un ami en selle\,; ce qui, en effet, n'était plus possible
avec ce que j'ai expliqué des choses de Flandre, indépendamment de tout
le reste. Je proposai donc La Houssaye que je ne connaissois point, mais
par ce qu'il m'était revenu de sa conduite dans l'intendance d'Alsace où
il était, et il fallait un intendant de frontières et de troupes, et M.
de Beauvilliers l'approuva.

Je trouvai sur Pontchartrain les dispositions les plus funestes et qui
pouvaient le plus flatter celles qu'il avait méritées de moi, mais qui
m'épouvantèrent parce qu'il avait un père à qui j'étais lié d'amitié, de
reconnaissance et de confiance la plus intime, une mère que j'aimais et
respectais véritablement, et que sa femme si proche de la mienne et si
parfaitement unie avec elle, lui avait laissé des enfants. Je vis leur
sort, je vis le chancelier, ou éconduit, ou retiré de lui-même avec le
poignard dans le cœur, et survivre à sa prodigieuse fortune, en proie à
l'horreur de son fils, et au néant de ses petits-fils. J'avais caché mon
ressentiment et ses causes, et plus au duc de Beauvilliers qu'à
personne, dans la situation où je le connaissois avec le chancelier.

Il s'ouvrit à moi sur le père et sur le fils plus qu'il n'avait fait
encore, car il s'ouvrit tout à fait. Rome, le jansénisme, et plus que
tout, la différence extrême de sentiment sur la personne et la doctrine
de M. de Cambrai, avait achevé de cimenter le mur qui avait commencé à
s'élever entre le duc et lui dès son son arrivée à la tête des finances.
Les escarmouches au conseil étaient continuelles. Outre ce que j'en ai
touché ici, il n'y a pas longtemps, le chancelier s'y aidait souvent
d'une légèreté qui lui était naturelle, et qui mettait les rieurs de son
côté. Il passait quelquefois jusqu'à porter des bottes indécentes et
parfois scandaleuses, qui déconcertaient une gravité qui, sur ces
matières, avait rarement raison. Ailleurs le chancelier n'était pas plus
mesuré\,; ils avaient même été plus d'une fois jusqu'à cesser de se
rendre les devoirs communs de la civilité réciproque, et quoiqu'ils n'en
fussent pas là alors, ils n'en étaient pas mieux ensemble, quoique le
duc de Chevreuse et le chancelier fussent toujours demeurés amis.
L'éclat ancien qui n'avait fait qu'augmenter depuis avait engagé dès
lors le duc de Beauvilliers de retirer de la marine ceux qu'il y
protégait, et qu'il y avait mis du temps de Colbert et de Seignelay. Les
blessures étaient devenues si continuelles et si profondes que ces deux
hommes ne se pouvaient pardonner, et que leur haine était publique. Le
duc, avec toute sa piété et ses mesures, se permettait à cet égard plus
de choses qu'il n'en était naturellement capable. Sûr du roi et de son
pupille dans les matières qui formaient leurs disputes, il se défendait
ordinairement avec hauteur et jetait quelquefois au chancelier des
choses et des faits qui l'embarrassaient, et le poussait alors avec
hardiesse. J'appris alors mille détails là-dessus du duc de
Beauvilliers, que ses mesures si resserrées m'avaient cachées jusque-là,
et que le chancelier n'avait eu garde de me dire par considération pour
moi dans la plus qu'intime liaison où il me savait avec le duc, non par
manque de confiance, car il m'en disait assez tous les jours pour ne me
laisser pas ignorer l'état où ils étaient ensemble. Bien que la
séparation intérieure de Pontchartrain d'avec son père passât souvent
jusqu'à l'extérieur, et que les mesures qu'il gardait avec M. de
Beauvilliers fussent les plus respectueuses, il ne l'en aimait pas mieux
au fond, et ce fond était bien aperçu.

L'entreprise d'Écosse que j'ai racontée en son lieu, et dont la triste
issue lui fut justement imputée, lui était devenue un péché irrémissible
auprès des ducs de Beauvilliers et de Chevreuse qui en avait été
l'auteur et le promoteur\,; d'ailleurs son pernicieux caractère achevait
de le leur rendre odieux. On en a vu quelque chose, t. IV, p.~377,
combien peu la Dauphine le ménageait auprès du roi, et que le roi, si en
garde en faveur de ses ministres, la laissait dire avec complaisance.
Mais il ne sera pas inutile de le faire connaître davantage\,: comme il
est depuis longtemps tout à fait mort au monde, j'en parlerai, quoique
vivant encore, comme d'un homme qui n'est plus.

Sa taille était ordinaire, son visage long, mafflé\footnote{Qui a de
  grosses joues.}, fort lippu, dégoûtant, gâté, de petite vérole qui lui
avait crevé un œil. Celui de verre, dont il l'avait remplacé, était
toujours pleurant, et lui donnait une physionomie fausse, rude,
refrognée, qui faisait peur d'abord, mais pas tant encore qu'il en
devait faire. Il avait de l'esprit mais parfaitement de travers, et avec
quelques lettres et quelque teinture d'histoire\,; appliqué, sachant
bien sa marine, assez travailleur, et le voulait paraître beaucoup plus
qu'il ne l'était. Son naturel pervers, que rien n'avait pu adoucir ni
redresser le moins du monde, perçait partout\,; il aimait le mal pour le
mal, et prenait un plaisir singulier à en faire. Si quelquefois il
faisait du bien, c'était une vanterie qui en faisait perdre tout le
mérite, et qui devenait synonyme au reproche\,; encore l'avait-il fait
acheter chèrement par les refus, les difficultés dont il était hérissé
pour tout, jusque pour les choses les plus communes, et par les manières
de le faire qui piquaient, qui insultaient même, et qui lui faisaient
des ennemis de presque tous ceux qu'il prétendait obliger. Avec cela,
noir, traître, et s'en applaudissait\,; fin à scruter, à suivre, à
apprendre et surtout à nuire. Pédant en régent de collége avec tous les
défauts et tout le dégoût d'un homme né dans le ministère et gâté à
l'excès.

Son commerce était insupportable par l'autorité brutale qu'il y
usurpait, et par ses infatigables questions\,; il se croyait tout dû, et
il exigeait tout avec toute l'insolence d'un maître dur. Il
s'établissait le gouverneur de la conduite de chacun, et il en exigeait
compte\,; malheur à qui l'y avait accoutumé par besoin, par lâcheté\,;
c'était une chaîne qui ne pouvait se rompre qu'en rompant avec lui.
Outre qu'il était méchant, il était malin encore, et persécuteur
jusqu'aux enfers, quand il en voulait aux gens\,; ses propos ne
démentaient point les désagréments dont il était chamarré. Ils étaient
éternellement divisés en trois points, et sans cesse demandait, en
s'applaudissant, s'il se faisait bien entendre\,; avec qui que ce fût,
maître de la conversation, interrompant, questionnant, prenant la parole
et le ton, avec des ris forcés à tous moments qui donnaient envie de
pleurer. Une expression pénible, maussade, pleine de répétitions, avec
un air de supériorité d'état et d'esprit qui faisait vomir et qui
révoltait en même temps. Curieux de savoir le dedans et le dessous de
toutes les familles et des intrigues, envieux et jaloux de tout, et dans
sa marine comme un comité sur ses galériens. Aucun officier, même
général, même pour des riens, n'était à couvert de ses sorties en pleine
audience publique, et nul homme ni femme de la cour de ses airs
d'autorité. Il disait aux gens les choses les plus désagréables avec
volupté, et réprimandait durement en maître d'école sous prétexte
d'amitié et en forme d'avis.

Son délice était de tendre des panneaux, et la joie de son cœur de
rendre de mauvais offices. En garde surtout contre son père et sa mère
et leurs amis, et contre toutes les grâces et tous les plaisirs qu'ils
pouvaient désirer de lui, il s'en piquait même, pour ne pas paraître
sous leur férule, au point que le chancelier et la chancelière s'étaient
fait une règle de ne lui rien demander ni recommander, et ne s'en
cachaient point, parce que la négative était certaine. En général, il
triomphait de refuser et de faire mystère des choses même les plus
futiles, surtout d'être hérissé de difficultés sur les choses qui en
souffraient le moins. L'importance lui tournait la tête, son ver rongeur
était de n'être point ministre\,: d'ailleurs incapable de société,
d'amusement de conversation ordinaire\,; toujours plein de ses
fonctions, de ses occupations, et avec qui que ce fût, homme et femme,
roi de ses moments et de ses heures, et le tyran de sa famille et de ses
familiers. Sa première femme, si parfaite en tout, en mourut à la fin à
force de vertu. La seconde l'a vengée.

On a vu sa conduite avec le comte de Toulouse, d'O et le maréchal
d'Estrées. Les femmes des deux derniers l'avaient perdu auprès de
M\textsuperscript{me} la Dauphine et auprès du Dauphin tout ce qui avait
pu l'approcher. M\textsuperscript{me} de Maintenon, qui aimait fort sa
première femme, et qui a toujours conservé du goût et de la
considération personnelle pour la chancelière, ne le pouvait supporter.
Il ne tenait auprès du roi que par l'amusement malicieux des délations
de Paris, qui était de son département, et qui lui avait causé force
prises avec Argenson, lieutenant de police, qu'il voulait tenir petit
garçon sous lui. Argenson en savait plus que lui\,; il s'était
habilement saisi de la confiance du roi, et par elle du secret de la
Bastille et des choses importantes de Paris\,; il les avait enlevées à
Pontchartrain, à qui en habile homme il n'avait laissé que les délations
des sottises des femmes et des folies des jeunes gens. Il s'était ainsi
déchargé sur lui de l'odieux de sa charge, surtout des lettres courantes
de cachet et se conservait le mérite envers beaucoup de gens
considérables de tous états d'avoir sauvé leurs proches de ses griffes,
soit en faisant en sorte de lui en souffler les aventures, ou en
diminuant et raccommodant auprès du roi ce qu'il y avait gâté. Les
jésuites, sulpiciens, etc., regardaient Argenson comme leur appui
fidèle, et le servaient comme tel auprès du roi et de
M\textsuperscript{me} de Maintenon\,; tandis que, comme on l'a déjà dit,
ils n'avaient que de l'aversion pour Pontchartrain, tant il les servait
de mauvaise grâce, et n'imputaient la chasse qu'il ne cessait de faire
aux moindres soupçons de jansénisme, qu'au plaisir qu'il prenait à faire
du mal. La singularité d'un si détestable caractère m'a engagé à m'y
étendre\,; la suite en fera voir encore davantage la nécessité. Avec
tant de vices et d'insolence, il était d'une vérité à surprendre sur sa
naissance\,; il n'en disait pas le tout, mais bien qu'ils étaient de
petits bourgeois de Montfort-l'Amaury, et assez pour désespérer La
Vrillière, qui était glorieux là-dessus fort mal à propos. J'en ai
quelquefois vu des scènes très-plaisantes entre eux deux. Comme
secrétaire d'État, l'orgueil même.

Le duc de Beauvilliers m'allégua la plupart de ces choses, et j'en
sentais à mesure la vérité. Il m'en fit des plaintes amères\,; et les
parades que j'y donnai ne furent reçues que très faiblement. Je le vis
si arrêté dans sa résolution, que je ne jugeai pas à-propos de heurter
par une résistance opiniâtre\,; je glissai donc, et ne butai qu'à
laisser une queue pour pouvoir traiter encore un chapitre si délicat.
Cela donnait lieu à reposer ses idées, et à moi, qui les avais aisément
prises, du temps pour le tourner et tâcher de les changer\,; nous
parlâmes donc d'autre chose, et Pontchar-train ne revint sur le tapis
entre nous deux de trois à quatre jours.

Ce fut le duc qui m'écarta à une promenade du roi pour en faire une avec
lui tête à tête, et qui reprit aussitôt ce chapitre, et je vis bien
qu'il le faisait à dessein. Le mien était tout préparé\,; le sien était
de m'emporter par une foule de raisons, qui toutes n'étaient que trop
bonnes\,; je lui laissai dire tout ce qu'il voulut. Il me pressa sur
beaucoup de choses et de faits de Pontchartrain\,: son humeur étrange,
sa malice, ses mauvais offices, sa satisfaction à faire du mal, son
plaisir à nuire, sa mauvaise grâce à faire du bien, et sa peine à-bien
faire, sa passion de s'étendre et d'usurper, son attention à tout
abaisser devant lui, l'aversion publique, ses procédés indignes avec un
nombre infini de gens de tous états et des plus considérables. Il ne
m'apprenait rien sur tout cela, et de ce dernier point j'en avais
l'expérience la plus étrange et la plus fraîche. Ce ne fut pas sans
combat intérieur que je l'étouffai dans une crise si décisive.

Quand il en eut bien dit, je lui répondis que n'ayant ni la force de
crédit ni la volonté, quand bien même j'aurais la puissance, de
m'opposer jamais en quoi que ce fût à lui, je ne pouvais pourtant me
résoudre à lui abandonner le fils du chancelier, tout imparfait, et plus
encore, que je le reconnaissois. Je lui parlai d'une manière touchante
de mon attachement plein de reconnaissance pour le père, et de ma
tendresse pour les petits-fils.

Cette manière de résister à un homme naturellement bon et plein de
sentiments le rendit rêveur. Je m'aperçus qu'il commençait à flotter
entre la peine de me voir si ferme et une sorte de satisfaction de la
cause que je lui venais d'avouer et de paraphraser. Il ne laissa pas
d'insister encore, et moi de répondre sur le même ton sans l'aigrir par
des négatives fausses et grossières, mais en lui demandant s'il croyait
Pontchartrain entièrement incorrigible\,; il ne répliqua point, je me
tus, et il demeura un peu de temps en silence, et comme en méditation à
part soi.

Il en sortit par me dire qu'avec toutes mes défenses, et qui n'étaient
d'aloi que pour moi seul, il voulait bien me dire que Pontchartrain
était actuellement en un péril très-grand\,; que pour l'amour de moi,
puisque je m'obstinais si fort à le protéger, il voulait encore bien me
dire que le Dauphin ne le pouvait souffrir\,; que la Dauphine avait juré
sa perte, poussée par tout ce qui l'approchait, par le cri public, par
son propre dégoût, par M\textsuperscript{me} de Maintenon même, qui,
d'ancienneté brouillée avec le père, ne pouvait personnellement
supporter le fils par une aversion particulière que ses manières et tout
ce qui lui en revenait lui avaient donnée\,; que le roi seul paraissait
plus indifférent là-dessus, mais sentir bien tous les défauts de
Pontchartrain, et ne semblait pas préparer une grande résistance à tant
et de telles batteries prêtes à jouer. Le duc ajouta que pour lui, s'il
était sensible à la vengeance, je pouvais bien juger de ce qu'il
penserait et ferait\,; mais qu'au défaut d'une affection que le
christianisme lui défendait, il était poussé par tout ce qu'il voyait,
et par tout ce qu'il lui revenait chaque jour de Pontchartrain\,; que sa
chute, pour laquelle il n'avait seulement qu'à laisser faire, il ne la
pouvait regarder que comme un bien public et avantageux à l'État, que
pensant de la sorte, c'était à Pontchartrain, s'il en avait le loisir, à
changer si promptement de conduite, qu'il le convainquît qu'il était
corrigible, après quoi on verrait ce qu'il serait à propos de faire à
son égard.

Comme nous nous parlions toujours sous le plus sûr secret et sans
mesures, je lui demandai si ce qu'il me disait là était une menace d'une
chose possible par celles qui existaient, ou un orage tout formé, et des
desseins pris et prêts à éclore. Il me répondit nettement que c'était le
dernier. J'en frémis, et n'osant le presser sur le détail de cette
affaire, je me contentai de le conjurer d'accorder un court loisir avant
que de perdre un homme au moins si instruit de sa marine, et que son
successeur encore ferait peut-être regretter.

Je n'ai point su quel il était, mais j'ai cru que Desmarets pouvait(être
le désigné. Il avait très-bien pris avec le roi, mieux encore avec
M\textsuperscript{me} de Maintenon, par les charmes de la finance, et le
goût qu'elle commençait à prendre pour sa femme, quoique revenu en place
malgré la fée qui voulait Voysin, mais dont la place de secrétaire
d'État de Chamillart, qu'elle lui avait fait donner, l'avait dépiquée.
Desmarets avait pour soi M\textsuperscript{me} la Dauphine, par les
manéges de sa femme, et par les soins qu'il avait de plaire
pécuniairement à tout ce qui l'approchait véritablement. On a vu plus
haut que son humeur féroce et son ingratitude n'avait pu déprendre de
lui les ducs de Chevreuse et de Beauvilliers, et les causes de leur
persévérance\,; et c'est ce groupe de choses qui m'a persuadé que
c'était Desmarets qu'ils voulaient porter à la plénitude des charges de
son oncle Colbert.

Sur mes instances que je rendis les plus pressantes, M. de Beauvilliers
me permit d'avertir Pontchartrain de dominer son humeur dans ses
audiences et avec tout le monde, de rapporter devant le roi avec moins
de penchant au mal, de rendre compte au conseil des dépêches des
affaires dont il était chargé avec un goût moins enclin à la sévérité,
de lui en spécifier quelques-unes en particulier, que le duc m'expliqua,
où ses manières dures et enclines au mal, tant en ce conseil qu'en ses
audiences, et même dans son travail tête à tête avec le roi où
M\textsuperscript{me} de Maintenon était toujours présente, avaient fait
de fâcheuses impressions, et étaient vivement revenues\,; mais il me
défendit d'aller plus loin\,; et de lui laisser apercevoir d'où je
pouvais être instruit. Je rendis grâces au duc de Beauvilliers, comme
d'une obligation du premier ordre, de ce qu'il voulait bien que je
fisse, et je le conjurai de nouveau de suspendre l'orage jusqu'à ce
qu'il eût vu le fruit de ces avis. Il ne voulut s'engager à rien\,; je
crus apercevoir qu'il craignait le plaisir de la vengeance, que ce
principe le fit rendre un peu à mes instances, et qu'il résista par là
même et par modestie, à la satisfaction de me laisser voir combien il
influait sur le sort de Pontchartrain. De cela même je m'ouvris à
l'espérance. Ainsi finit cette importante conversation.

Elle me donna lieu à de grandes réflexions. Outre celles que j'ai déjà
expliquées sur l'état du chancelier et de ses petits-fils, son fils
chassé, je sentis encore que ce coup paré, si tant était que j'en pusse
venir à bout, ils ne seraient encore en aucune assurance. Pontchartrain,
fait comme il était, ne pourrait se contenir longtemps\,; ses rechutes
deviendraient mortelles, avec cette horreur générale qu'il avait si
justement encourue, et cet éloignement extrême, pour ne rien dire de
plus, toujours subsistant entre son père et le duc de Beauvilliers, dans
la posture nouvelle et stable où se trouvait alors ce dernier. Toute ma
vie j'avais désiré avec la passion la plus vive de les voir solidement
réconciliés, mais comme on désire quelquefois des choses imaginaires et
impossibles. Deux hommes en tout si dissemblables, excepté en probité et
en amour de l'État, n'avaient rien en quoi ils pussent compatir
ensemble. Leurs liaisons, leurs vues, leurs sentiments, leurs
tempéraments se trouvaient tellement contraires qu'il ne s'y pouvait
rien ajouter, et jusqu'à la religion dans deux très-hommes de bien, de
la façon dont ils la prenaient l'un et l'autre, leur était devenue un
très-puissant motif d'aversion. Cependant, par la face nouvelle que la
cour avait prise, je voyais le chancelier et son fils perdus sans cette
réconciliation sincère, et sa nécessité me parut si démontrée que,
quelque impossible et chimérique quelle me semblât, je me mis dans la
tête d'y oser travailler. Sans ce remède unique, je ne voyais aucun
moyen de subsister pour le chancelier, dans la nouvelle et durable face
que la cour avait prise, et je ne trouvais d'épine dans le riant de ma
situation particulière que la peine extrême, et qui troublait toute ma
joie, de voir mes deux plus intimes amis en état ensemble que l'un
infailliblement serait perdu et anéanti par l'autre. Il ne fallait pas
un motif moins puissant pour me faire entreprendre un ouvrage si voisin
de l'impossible, et que l'extrême nécessité cessa lors, pour la première
fois, de me laisser envisager comme une folie.

Dès le soir même, après que les soupeurs se furent retirés de chez
Pontchartrain, j'entrai chez lui, où je n'allais plus familièrement, et
même très-rarement. L'heure ajouta à sa surprise\,; je lui dis,
d'abordée et d'un air grave et froid, que quoique ma coutume ne fût pas
de lui faire des leçons, et que j'eusse lieu d'en être encore plus
éloigné que jamais, j'avais pourtant des choses à lui dire dont je ne
pouvais me dispenser\,; qu'il ne me demandât ni de mes raisons ni d'où
je prenais ce que j'avais à lui dire\,; qu'il se contentât d'apprendre
qu'il ne pouvait m'écouter avec trop d'attention, ni prendre trop de
soin d'en profiter sans délai. Après une préface si énergique, je lui
dis, comme si j'en avais été l'auteur, tout ce que j'avais permission de
lui dire, et cela tout de suite comme une leçon apprise par cœur. Je fus
écouté avec toute l'attention que demandait ma préface et la matière qui
la suivit. Pontchartrain sentit aisément que les faits singuliers que je
lui spécifiai ne pouvaient m'être venus que d'endroits importants. Il
voulut s'excuser sur certaines choses, sur d'autres il avoua, et accusa
son humeur. Je répondis qu'avec moi tout cela était inutile, que son
affaire était de profiter de ce qu'il venait d'entendre, la mienne de
m'aller coucher, et là-dessus je le quittai aussi brusquement que je
l'avais abordé. Je rendis compte le lendemain de ce que j'avais dit à
Pontchartrain au duc de Beauvilliers. Il augmenta ma frayeur par ce
qu'il me laissa voir de l'imminence de la chute, et néanmoins il convint
d'attendre ce que produirait ma remontrance.

À quelques jours de là, me promenant après minuit en tiers avec le
Dauphin et l'abbé de Polignac, la conversation tomba sur le gouvernement
de Hollande, sur sa tolérance de toutes les sectes, et bientôt sur le
jansénisme. L'adroit abbé n'en perdit pas l'occasion, et dit tout ce
qu'il fallait pour plaire. Le Dauphin me donna lieu d'entrer assez dans
la conversation. Je parlai suivant mes sentiments et sans affectation.
La promenade se poussa tard par le plus beau temps du monde, et je
quittai le Dauphin comme il allait rentrer au château. J'expliquerai
ailleurs ce que je pense sur cette matière, parce qu'elle entrera dans
plus d'une chose dans la suite, et ma façon de voir et d'être avec le
Dauphin. Dès le lendemain matin M. de Beauvilliers me prit dans le
salon, et me conta que le Dauphin venait de lui dire avec beaucoup de
joie que, à des discours qu'il m'avait ouï tenir le soir précédent à sa
promenade, il me croyait éloigné du jansénisme, et tout de suite me
demanda de quoi il avait été question, que le Dauphin n'avait pas eu le
temps de lui expliquer. Il me dit, après lui en avoir rendu compte,
qu'il avait tout à fait confirmé le Dauphin dans cette opinion sur moi,
et cela mit en effet sa confiance pour moi au large sur toutes sortes de
chapitres, et voilà ce que font les hasards.

Il fit encore qu'à ces propos le duc me dit tout de suite que le Dauphin
soupçonnait fort Pontchartrain de jansénisme, lui qui faisait sa cour au
roi du zèle de cette persécution. La délicatesse de M. de Beauvilliers
était là-dessus si étrange, qu'après ce qu'il m'avait dit lui-même que
les jésuites et les sulpiciens imputaient au goût malfaisant de
Pontchartrain la persécution qu'il faisait aux jansénistes, je ne le pus
faire revenir de ses soupçons là-dessus, qu'en lui répondant de
Pontchartrain sur ce chapitre, et que, différent en tout d'avec son
père, ils étaient aussi parfaitement divisés sur les jésuites et
l'Oratoire. La fréquentation de Pontchartrain, lors de la mort de sa
femme, avec le P. de La Tour, général de l'Oratoire, et encore quelque
mois après, avait répandu ces soupçons\,; mais j'assurai le duc, comme
il était vrai, que Pontchartrain avec la dernière indécence avait quitté
le commerce du P. de La Tour, comme une chemise sale, et n'en avait pas
ouï parler depuis. Nous nous revîmes le même jour sur le soir. Dans
l'entre-deux, M. de Beauvilliers, sur ma parole, avait répondu de
Pontchartrain au Dauphin sur le jansénisme. Il me le confia, et ce fut
le premier bon office qu'il lui rendit auprès de ce prince. De là, le
duc me dit qu'il n'entendait pas deux choses, Pontchartrain étant tel
là-dessus que je lui avais si fort assuré\,: l'une qu'il était
très-suspect aux jésuites, l'autre comment l'affaire d'un ecclésiastique
d'Orléans était si mal entre ses mains\,; que les jésuites attribuaient
à son goût de faire du mal sa facilité à maltraiter les jansénistes que
l'on exilait, ou qu'on ôtait de places, et n'en étaient pas moins en
garde contre lui, parce qu'il leur était aussi contraire qu'il lui était
possible\,; et que cet ecclésiastique si opposé aux jansénistes, et qui
tirait de là tout son appui, ne pouvait être plus mal servi qu'il était
de Pontchartrain, pour l'union d'un bénéfice, qui était néanmoins
très-essentielle au bon parti. Il s'échauffa assez là-dessus, et de
lui-même me permit d'avertir Pontchartrain, mais comme de moi-même, de
la disposition des jésuites à son égard\,; qu'il lui importait fort de
la changer par une conduite opposée\,; et sur cet ecclésiastique de lui
dire, non plus comme de moi-même, mais de sa part à lui comme en avis,
de rapporter son affaire au premier conseil des dépêches, d'y donner un
tour favorable, et d'ajouter que cela lui était plus important qu'il ne
pensait.

Je fis ce même soir, vers le minuit, une seconde visite à Pontchartrain,
toute semblable à la première, dont l'heure et le ton ne le surprit pas
moins, et bien plus encore que la première pour les choses. Il s'était
peut-être douté à la première d'où lui venaient mes avis. À cette
seconde, il ne put plus l'ignorer. C'était en insolence le premier homme
du monde, lorsqu'il ne craignait point les gens\,; et le premier aussi
en bassesse, où personne ne le surpassait, à proportion de son besoin et
de sa frayeur. Ainsi on peut juger de tout ce qu'il me pria de dire à M.
de Beauvilliers, de quelle façon il se mit à en user avec les jésuites,
et comment tourna l'affaire de l'ecclésiastique d'Orléans.

M. de Beauvilliers en fut si content, qu'il voulut bien que je lui
disse, mais comme de moi-même, le péril en gros où il était auprès du
Dauphin, et les moyens de le rapprocher peu à peu, tous opposés à son
génie et à ses manières accoutumées. Le duc alla jusqu'à me charger de
lui dire qu'il lui ménagerait des occasions de travailler avec le
Dauphin, qu'il l'en avertirait d'avance et de la façon de s'y conduire.

Je revis donc aussitôt Pontchartrain pour la troisième fois\,; je ne vis
jamais homme si transporté. Il se crut noyé et sauvé au même instant, et
les protestations qu'il me fit, tant pour M. de Beauvilliers que pour
moi, furent infinies. Sur mon compte, je sus bien qu'en penser, puisque
c'était trois semaines après qu'il m'eût envoyé Daubenton\,; aussi les
reçus-je pour moi avec le froid le plus dédaigneux, et je lui fis
sentir, au choix de mon peu de paroles, la nullité de part que sa
personne devait prendre au salut inespéré que je lui procurais.

Le duc tint parole\,; Pontchartrain fut averti et instruit\,; et, comme
M. de Beauvilliers ne voulut pas s'y montrer, je fus toujours le canal
entre eux sous le plus entier secret. Pontchartrain travailla chez le
Dauphin\,; le duc avait préparé les choses. Le prince fut content. Cela
dura le reste du voyage de Marly, qui, d'une tirade, nous conduisit à
Fontainebleau sans retourner à Versailles, à cause du mauvais air.

Dans ces entrefaites et sur la fin de Marly, je pris en particulier le
premier écuyer, non pour lui confier quoi que ce soit de ce qui vient
d'être raconté, mais pour m'en servir à ma manière au dessein de
réconciliation que j'avais conçu.

C'était un grand homme, froid, de peu d'esprit, de beaucoup de sens,
fort sage, fort sûr, fort mesuré, qui, à force d'être né et d'avoir
passé sa vie à la cour, fils d'un homme qui était maître passé et dans
une considération singulière, et lui dans les cabinets les plus secrets
de Le Tellier, Louvois et Barbezieux, dont il était si proche par sa
femme, et qui l'avaient admis à tout avec eux, avait acquis une grande
connaissance de la cour et du monde, y était fort compté, s'y était mêlé
de beaucoup de choses, et y était enfin devenu une espèce de personnage.
Il était de tout temps fort bien avec le roi, il avait des particuliers
quelquefois avec lui\,; et il avait eu l'art d'être fort bien avec tous
les ministres, et intimement avec le chancelier, qui avait beaucoup de
créance en lui. J'ai parlé de lui à l'occasion de la mort de
Monseigneur, duquel il espérait beaucoup, et rien de la cour nouvelle,
avec qui il n'avait nulle liaison, même quelque chose de moins avec les
ducs de Chevreuse et de Beauvilliers, par l'ancien chrême des Louvois,
si opposés à tout ce qui était Colbert, et tous leurs commerces et leurs
allures tout à fait différentes.

Je crus donc que c'était le seul homme dont je pusse m'aider pour
attaquer le chancelier sur sa conduite avec le duc de Beauvilliers. Je
lui dis qu'ami au point où je l'étais de M. de Beauvilliers et du
chancelier, je voyais de tout temps leur éloignement avec une peine
extrême, que jusqu'alors je m'étais contenté de m'en affliger en
moi-même\,; mais que, dans la face nouvelle que la cour venait de
prendre, et qui se fortifiait de jour en jour, je ne pouvais dormir en
repos comme j'avais fait tant que leur inimitié n'avait pu être fatale à
aucun des deux\,; que le Dauphin devenait rapidement le maître des
affaires, et par lui son gouverneur, qui le serait sans mesure lorsque
son pupille aurait succédé au roi\,; que le danger présent était grand
par la haine publique que Pontchartrain avait encourue\,; et s'il
subsistait le reste de ce règne, ce qui me paraissait bien difficile, il
me semblait impossible qu'il pût durer au delà\,; que, tombant, je ne
voyais pas ce que pourrait devenir le père d'un homme chassé dans une
cour où tout le crédit serait contre lui, où il survivrait à sa fortune
et à soi-même, et où la décence ni sa propre humeur ne pourrait lui
permettre d'y rester et d'y hasarder de se voir chasser lui-même sur
quelque aventure de Rome et de jansénisme, et se voir bombarder un garde
des sceaux\,; qu'en vain s'appuyait-il sur l'autorité de sa place, sur
son esprit, sur sa capacité, sur sa réputation, puisque ce ne serait pas
lui qu'on attaquerait, mais son fils qui n'avait aucun de ces boucliers,
qui s'était rendu la bête de tout le monde, et dont la chute aurait les
applaudissements publics.

Beringhen connaissoit parfaitement Pontchartrain\,; il m'avoua la vérité
de ce que je lui représentais, sa crainte extrême de ce que je
prévoyais, et me pressa de travailler à une réconciliation si capitale à
la fortune du père et du fils, comme le seul homme qui la pût
entreprendre par l'amitié et la confiance que le duc et le chancelier
avaient également et entièrement pour moi. Je lui répondis que c'était
toute ma passion, mais que je travaillerais en vain tant que le
chancelier s'escarmoucherait avec le duc sans cesse au conseil, et ne se
mesurerait pas ailleurs à son égard\,; qu'il nourrissait ainsi une
haine, pour parler nettement, de longue main enracinée, qu'il
l'augmentait tous les jours loin de songer à l'émousser, en quoi
pourtant consistait son salut et celui de sa famille\,; que c'était à
lui, Beringhen, son ami, et qui ne lui serait point suspect sur M. de
Beau-villiers avec qui il savait bien qu'il n'avait point de liaison, à
lui ouvrir les yeux sur le danger de voir périr toute la fortune
prodigieuse qu'il avait faite\,; et de lui faire comprendre qu'elle
valait bien la peine de se contraindre, et de ployer à la nécessité des
temps\,; qu'après qu'il l'aurait rendu capable d'un vrai changement de
conduite à cet égard, je verrais à tâcher de le mettre à profit auprès
de M. de Beauvilliers, et peu à peu les rapprocher, et de là les
réconcilier enfin si je pouvais.

Le premier écuyer, ou timide comme il l'était naturellement, ou
désespérant de faire entendre raison au chancelier vif et décidé comme
il le connaissoit, ou véritablement court de temps, me dit qu'il en
aurait peu pour parler suffisamment au chancelier qui n'était point à
Marly, qui n'y venait que pour les conseils, et qui ces jours-là s'en
retournait dîner à Versailles, et les autres jours se tenait à
Pontchartrain\,; qu'il avait demandé congé au roi de s'en aller dans
quelques jours chez lui à Armainvilliers, et qu'il y passerait presque
tout le voyage de Fontainebleau, où la cour allait incessamment. Il
finit par me presser de nouveau de travailler à une aussi bonne œuvre
que nul autre que moi ne pouvait exécuter, et moi par l'exhorter de
parler au moins avant de partir, et de parler sans ménagement. La suite
de ceci se verra bientôt à Fontainebleau\,; avant d'y conduire la cour,
il faut reprendre des choses qui ont précédé ce voyage.

On a pu voir épars en plusieurs endroits de ces Mémoires à quel degré
d'intimité et de toute confiance j'étais arrivé avec le duc de
Beauvilliers, avec le duc de Chevreuse, et avec les duchesses leurs
femmes. Tout cela vivait dans la même amitié avec M\textsuperscript{me}
de Saint-Simon, et ce qui était peut-être unique pour des personnes si
généralement cachées et compassées, dans la confiance et la liberté la
plus entière, fondées sur l'estime de sa vertu, et l'expérience de la
sagesse et de la bonté de son esprit et de sa conduite, plus encore s'il
se peut que sur ce qu'elle m'était, et de ce qu'ils savaient que j'étais
pour elle. Il faut donc comprendre que ces trois couples faisaient un
groupe qui ne se cachait rien, qui se consultait tout, qui en ce genre
était inaccessible à quiconque, et dont le commerce était non-seulement
continuel, mais de tous les jours, et souvent de plus d'une fois par
jour quand nous étions dans les mêmes lieux, et il était fort rare que
nous en fussions séparés, parce que Vaucresson était fort proche, et que
je ne sortais presque point de la cour, ni M\textsuperscript{me} de
Saint-Simon non plus. Cette union anciennement prise, mais liée et
augmentée par degrés, en était à ce dernier bien longtemps avant la mort
de Monseigneur, comme divers traits de ces Mémoires auront pu le faire
remarquer.

Dans cet état, M. de Beauvilliers ne cessait depuis longtemps de faire
naître de l'estime, de l'amitié, du goût pour moi en son pupille, sur
l'esprit et le cœur duquel il pouvait tout. Il n'en perdit aucune
occasion pendant plusieurs années. On a vu que j'en sentis l'effet à
l'occasion de l'ambassade de Rome, et un autre si grandement marqué à
son arrivée de la campagne de Lille. L'état triste où il fut après si
longtemps ajouta aux mesures que le sage gouverneur me prescrivit
toujours. On se souvient de la situation où la cabale de Meudon tenait
ce prince, et combien le roi même demeura aliéné de lui, en sus de ce
qu'il en était auparavant par la vie si recluse et si resserrée de son
petit-fils, qui l'avait dès lors mis fort à gauche avec Monseigneur. On
ne doutait dans aucun de ces temps que le duc de Beauvilliers ne
possédât ce jeune prince\,; on ignorait bien le fond de mon intimité
avec le duc, mais la liaison était trop forte, et le commerce trop
continuel et trop libre avec des gens aussi enfermés, pour n'avoir pas
percé.

Être en mesure et en garde infinie était le caractère dominant du duc.
La haine de M\textsuperscript{me} de Maintenon, et les secousses qu'il
avait éprouvées du roi même, augmentaient encore les entraves de sa
timidité naturelle. Il craignait les soupçons de circonvenir son
pupille, il craignait la jalousie et les regards perçants qui s'étaient
fixés sur moi depuis ce choix pour Rome. Il voulait me mettre peu à peu
dans la confiance du jeune prince, mais il ne voulait pas qu'il en parût
rien. Il redoubla encore de précautions depuis la campagne de Lille où
je m'étais si hautement déclaré et dont je fus perdu un temps. Je
rappelle toutes ces époques et ces faits épars dans ces Mémoires, pour
les remettre tous à la fois sous les yeux, et montrer les raisons de la
conduite que le duc de Beauvilliers me fit observer, de concert avec le
prince.

Je ne le voyais chez lui, aux heures de cour que rarement et courtement,
assez pour qu'il ne parût rien d'affecté, assez peu pour qu'on ne pût
soupçonner non-seulement privance, mais même aucun dessein de
m'approcher de lui\,; en tout plus de négligence que de cour. Par cette
raison le prince me distinguait peu chez lui, et ne me donnait guère au
delà de ce qu'il avait accoutumé aux gens de ma sorte\,; mais souvent un
coup d'œil expressif, un sourire à la dérobée m'en disait tout ce que
j'en désirais savoir.

Outre la transcendance d'être sans cesse porté avec étude par le duc de
Beauvilliers auprès de lui, et encore par le duc de Chevreuse, du
caractère dont était ce prince, ce qu'il paraissait du mien par le tissu
de la conduite ordinaire de toute ma vie était un avantage peu commun
pour lui plaire. Il aimait une vie appliquée, égale, unie, il estimait
l'union dans les familles, il considérait les amitiés qui faisaient
honneur\,; et de celles-là, on a vu que j'y fus toujours heureux. Ma
jeunesse n'avait rien eu de ce qui eût pu l'étranger ou l'arrêter.
Toutes mes liaisons particulières s'étaient trouvées avec des personnes
qui presque toutes lui étaient agréables ou directement ou par quelque
recoin\,; mes inimitiés ou mes éloignements, avec celles qui pour la
plupart étaient en opposition avec lui, et très-ordinairement directe,
ce qui était arrivé naturellement et sans aucun art. J'étais bien de
toute ma vie avec les jésuites, quoique sans liaison qu'avec un seul à
la fois, mais liaison unique jusqu'à la mort du dernier qui survécut le
feu roi\,; ils me comptaient parmi leurs amis, comme on l'a vu du P.
Tellier, et comme on le verra davantage. Je l'avais été intime, comme on
l'a vu aussi, de l'évêque de Chartres, Godet. C'étaient là des boucliers
sûrs contre le dangereux soupçon de jansénisme\,; et ce que j'ai
rapporté de cette conversation avec le Dauphin et l'abbé de Polignac en
tiers, dans les jardins de Marly, mit le sceau à l'assurance. Ma façon
d'être à cet égard reviendra trop souvent dans les suites pour ne
mériter pas d'être expliquée, puisque l'occasion s'en présente si
naturellement.

Le célèbre abbé de la Trappe a été ma boussole là-dessus, comme sur bien
d'autres choses dont je désirerais infiniment avoir eu la pratique comme
la théorie.

Je tiens tout parti détestable dans l'Église et dans l'État. Il n'y a de
parti que celui de Jésus-Christ. Je tiens aussi pour hérétiques les cinq
fameuses propositions directes et indirectes, et pour tel tout livre
sans exception qui les contient. Je crois aussi qu'il y a des personnes
qui les tiennent bonnes et vraies, qui sont unies entre elles et qui
font un parti. Ainsi, de tous les côtés, je ne suis pas janséniste.

D'autre part, je suis attaché intimement, et plus encore par conscience
que par la plus saine politique, à ce que très-mal à propos on connaît
sous le nom de libertés de l'Église gallicane, puisque ces libertés ne
sont ni priviléges, ni concessions, ni usurpations, ni libertés même
d'usage et de tolérance, mais la pratique constante de l'Église
universelle, que celle de France a jalousement conservée et défendue
contre les entreprises et les usurpations de la cour de Rome, qui ont
inondé et asservi toutes les autres et fait par ses prétentions un mal
infini à la religion. Je dis la cour de Rome, par respect pour l'évêque
de Rome, à qui seul le nom de pape est demeuré, qui est de foi le chef
de l'Église, le successeur de saint Pierre, le premier évêque, avec
supériorité et juridiction de droit divin sur tous les autres quels
qu'ils soient, et à qui appartient seul la sollicitude et la
surveillance sur toutes les Églises du monde comme étant le vicaire de
Jésus-Christ par excellence, c'est-à-dire le premier de tous ses
vicaires qui sont les évêques. À quoi j'ajoute que je tiens l'Église de
Rome pour la mère et la maîtresse de toutes les autres, avec laquelle il
faut être en communion\,; maîtresse, \emph{magistra}, et non pas
\emph{domina}\,; ni le pape, le seul évêque, ni l'évêque universel,
ordinaire et diocésain de tous les diocèses, ni ayant seul le pouvoir
épiscopal duquel il émane dans les autres évêques, comme l'inquisition,
que je tiens abominable devant Dieu et exécrable aux hommes, le veut
donner comme de foi.

Je crois la signature du fameux formulaire une très-pernicieuse
invention, tolérable toutefois en s'y tenant exactement suivant la paix
de Clément IX, autrement insoutenable. Il résulte que je suis fort
éloigné de croire le pape infaillible, en quelque sens qu'on le prenne,
ni supérieur, ni même égal aux conciles œcuméniques, auxquels seuls
appartient de définir les articles de foi, et de ne pouvoir errer sur
elle.

Sur Port-Royal, je pense tout comme le feu roi s'en expliqua à Maréchal
en soupirant (t. VI, p.~128), que ce que les derniers siècles ont
produit de plus saint, de plus pur, de plus savant, de plus instructif,
de plus pratique, et néanmoins de plus élevé, mais de plus lumineux et
de plus clair, est sorti de cette école, et de ce qu'on connaît sous le
nom de Port-Royal\,; que le nom de jansénisme et de janséniste est un
pot au noir de l'usage le plus commode pour perdre qui on veut, et que
d'un millier de personnes à qui on le jette, il n'y en a peut-être pas
deux qui le méritent\,; que ne point croire ce qu'il plaît à la cour de
Rome de prétendre sur le spirituel, et même sur le temporel, ou mener
une vie simple, retirée, laborieuse, serrée, ou être uni avec des
personnes de cette sorte, c'en est assez pour encourir la tache de
janséniste\,; et que cette étendue de soupçons mal fondés, mais si
commode et si utile à qui l'inspire et en profite, est une plaie cruelle
à la religion, à la société, à l'État.

Je suis persuadé que les jésuites sont d'un excellent usage en les
tenant à celui que saint Ignace a établi. La compagnie est trop
nombreuse pour ne renfermer pas beaucoup de saints, et de ceux-là j'en
ai connu, mais aussi pour n'en contenir pas bien d'autres. Leur
politique et leur jalousie a causé, et cause encore de grands maux\,;
leur piété, leur application à l'instruction de la jeunesse et l'étendue
de leurs lumières et de leur savoir, fait aussi de grands biens.

C'en est assez pour un homme de mon état, ce serait en sortir, et des
bornes de ce qui est traité ici, que descendre dans plus de détails\,;
mais ce n'est pas trop pour les choses dont les récits nécessaires
s'approchent. Ce que je viens d'expliquer ne contentera pas ceux qui
prétendent que le jansénisme et les jansénistes sont une hérésie et des
hérétiques imaginaires, et satisfera sûrement encore moins ceux à qui la
prévention, l'ignorance ou l'intérêt en font voir partout. Ce qui m'a
infiniment surpris, est comment la prévention qui mettait M. de
Beauvilliers de ce dernier côté lui a pu permettre de s'accommoder de
moi au point qu'il a fait, et sans le moindre nuage, toute sa vie, avec
la franchise entière que j'ai toujours eue avec lui là-dessus, comme sur
tous mes autres sentiments sur toutes autres manières.

\hypertarget{chapitre-xv.}{%
\chapter{CHAPITRE XV.}\label{chapitre-xv.}}

1711

~

{\textsc{Situation personnelle de la duchesse de Saint-Simon à la
cour.}} {\textsc{- Précautions de ma conduite.}} {\textsc{- Je sonde
heureusement le Dauphin.}} {\textsc{- Court entretien dérobé avec le
Dauphin.}} {\textsc{- Tête-à-tête du Dauphin avec moi.}} {\textsc{-
Dignité\,: gouvernement, ministère.}} {\textsc{- Belles et justes
espérances.}} {\textsc{- Conférence entre le duc de Beauvilliers et
moi.}} {\textsc{- Autre tête-à-tête du Dauphin avec moi.}} {\textsc{-
Secret de ces entretiens.}} {\textsc{- Dignité\,: princes, princes du
sang, princes légitimés.}} {\textsc{-Belles paroles du Dauphin sur les
bâtards.}} {\textsc{- Conférence entre le duc de Beauvilliers et moi.}}
{\textsc{- Importance solide du duc de Beauvilliers.}} {\textsc{-
Concert entier entre lui et moi.}} {\textsc{- Contrariété d'avis entre
le duc de Beauvilliers et moi sur la succession de Monseigneur.}}
{\textsc{- Manière dont elle fut traitée\,; extrême indécence qui s'y
commit à Marly.}}

~

Divers endroits de ces Mémoires ont fait voir combien
M\textsuperscript{me} de Saint-Simon pouvait compter sur les bontés de
M\textsuperscript{me} la duchesse de Bourgogne, et le dessein constant
qu'elle eut toujours de la faire succéder à la duchesse du Lude. La
place qu'elle fut forcée de remplir auprès de M\textsuperscript{me} la
duchesse de Berry l'approcha de tous les particuliers\,; plus elle fut
vue de près, plus elle fut goûtée, aimée, et si j'ose parler d'après
toutes ces têtes presque couronnées, même après le roi et
M\textsuperscript{me} de Maintenon, elle fut honorée et respectée\,; et
les écarts de la princesse à qui on l'avait attachée malgré elle ne
firent que plus d'impression en faveur de son grand sens, de la
prudence, de la justesse de son esprit et de sa conduite, de la sagesse,
de l'égalité, de la modestie, de la vertu de tout le tissu de sa vie, et
d'une vertu pure toujours suivie, et qui, austère pour elle-même, était
aimable et bien loin de rebuter par ses rides. {[}Elle{]} se fit
toujours rechercher par celles même dont l'âge et la conduite en étaient
les plus éloignés, qui vinrent plus d'une fois se jeter à elle pour en
être conseillées et tirées par son moyen des dangers et des orages
domestiques où leur conduite les avait livrées. Tant de qualités
aimables et solides lui avaient acquis l'amitié et la confiance de
beaucoup de personnes considérables, et tant de réputation que personne
n'y fut plus heureuse qu'elle, sur quoi on peut se souvenir du conseil
que les trois ministres, sans nul concert entre eux, me donnèrent,
lorsque je fus choisi pour Rome, de lui tout communiquer et de profiter
de ses avis. Le Dauphin, qui la voyait souvent dans les parties
particulières et toujours depuis le mariage de M. le duc de Berry, avait
pris pour elle beaucoup d'estime, d'amitié, même de confiance, qui me
fut un autre appui très-fort près de lui, que le duc de Beauvilliers
fortifia toujours, et par amitié, et plus encore par l'opinion qu'il
avait d'elle. Ainsi tout me portait dans la confiance et dans l'amitié
libre et familière du Dauphin.

La cour changée par la mort de Monseigneur, il fut question pour moi de
changer de conduite à l'égard du nouveau Dauphin. M. de Beauvilliers
m'en parla d'abord, mais il jugea que ce changement ne devait se faire
que fort lentement, et de manière à y accoutumer sans effaroucher.
J'avais en divers temps échappé à d'étranges noirceurs\,; je devais
compter que les regards se fixeraient sur moi à proportion de la
jalousie, et que je n'en pouvais éviter les dangers qu'en voilant ma
situation nouvelle, si fort changée par le changement de toute la scène
de la cour\,; pour cela ne m'approcher à découvert que peu à peu du
prince, à mesure que son asile se fortifierait à mon égard, c'est-à-dire
à mesure qu'il croîtrait auprès du roi en confiance, et en autorité dans
les affaires et dans le monde. Je crus néanmoins à propos de le sonder
dès les premiers jours de son nouvel essor. Un soir que je le joignis
dans les jardins de Marly, où il était peu accompagné, et de personne
qui me tînt de court, je profitai de son accueil gracieux pour lui dire
comme à la dérobée, que bien des raisons qu'il n'ignorait pas m'avaient
retenu jusqu'alors dans un éloignement de lui nécessaire, que maintenant
j'espérais pouvoir suivre avec moins de contrainte mon attachement et
mon inclination, et que je me flattais qu'il l'aurait agréable. Il me
répondit bas aussi qu'il y avait en effet des raisons quelquefois qui
retenaient\,; qu'il croyait qu'elles avaient cessé\,; qu'il savait bien
quel j'étais pour lui\,; et qu'il comptait avec plaisir que nous nous
verrions maintenant plus librement de part et d'autre. J'écris
exactement les paroles de sa réponse pour la singulière politesse de
celles qui la finissent. Je la regardai comme l'engagement heureux d'une
amorce qui avait pris comme je me l'étais proposé. Je me rendis peu à
peu plus assidu à ses promenades, mais sans les suivre entières,
qu'autant que la foule, ou des gens dangereux ne les grossissaient pas,
et j'y pris la parole avec plus de liberté. Je demeurai sobre à le voir
chez lui avec le monde, et je m'approchais de lui dans le salon, suivant
que j'y voyais ma convenance.

Je lui avais présenté notre mémoire contre d'Antin lors du procès, et je
n'avais pas manqué de lui glisser un mot sur notre dignité, à laquelle
je le savais très-favorable, et par principes. II avait lu le mémoire et
avait été fort aise, à cause de quelques-uns d'entre nous, de le trouver
fort bon, et la cause de d'Antin insoutenable. Je n'ignorais pas aussi
ce qu'il pensait sur la forme du gouvernement de l'État, et sur beaucoup
de choses qui y ont rapport\,; et ses sentiments là-dessus étaient les
miens mêmes, et ceux des ducs de Chevreuse et de Beauvilliers, par qui
j'étais bien instruit. C'était l'avoir trop beau pour n'essayer pas à en
tirer grand parti. Je me rendis donc attentif à saisir tout ce qui
pourrait me conduire à entrer naturellement en matière, et je ne fus pas
longtemps à en trouver le moment.

Quelques jours après étant dans le salon, j'y vis entrer le Dauphin et
la Dauphine ensemble se parlant à diverses reprises. Je m'approchai
d'eux, et j'entendis les dernières paroles. Elles m'excitèrent à
demander au prince de quoi il s'agissait, non pas de front, mais avec un
tour de liberté respectueuse, que j'usurpais déjà. Il me répondit qu'ils
allaient à Saint-Germain pour la première fois qu'il était Dauphin,
c'est-à-dire en visite ordinaire, après celle en manteau et en mante\,;
que cela changeait le cérémonial avec la princesse d'Angleterre,
m'expliqua la chose, et appuya avec vivacité sur l'obligation de ne
laisser rien perdre de ses droits légitimes. «\, Que j'ai de joie, lui
répondis-je, de vous voir penser ainsi, et que vous avez raison
d'appuyer sur ces sortes d'attentions dont la négligence ternit toutes
choses\,!» Il reprit avec feu, et j'en saisis le moment le plus actif
pour lui dire que si, lui qui était si grand, et dont le rang était si
décidé avait raison d'y être attentif, combien plus nous autres, à qui
on disputait et souvent on ôtait tout sans qu'à peine nous osassions
nous en plaindre, avions-nous raison de nous affliger de nos pertes, et
de tâcher à nous soutenir. Il entra là-dessus avec moi jusqu'à devenir
l'avocat de notre cause, et finit par me dire qu'il regardait notre
restauration comme une justice importante à l'État\,; qu'il savait que
j'étais bien instruit de ces sortes de choses\,; et que je lui ferais
plaisir de l'en entretenir un jour. Il rejoignit dans ce moment la
Dauphine, et s'en allèrent à Saint-Germain.

Le fait qui avait donné lieu à cette courte mais importante ouverture
était que, du vivant de Monseigneur, M\textsuperscript{me} la duchesse
de Bourgogne cédait partout en lieu tiers à la princesse d'Angleterre\,;
mais que, devenue l'épouse de l'héritier présomptif par la mort de
Monseigneur, elle devait désormais précéder partout en lieu tiers cette
même princesse d'Angleterre, qui n'était pas héritière présomptive d'un
frère qui aurait des enfants, et qui n'était pas même encore marié. À
peu de jours de là, le Dauphin m'envoya chercher. J'entrai par la
garde-robe, où du Chesne, son premier valet de chambre, très-homme de
bien, sûr et qui avait sa confiance, m'attendait pour m'introduire dans
son cabinet, où il était seul. Mon remercîment ne fut pas sans mélange
de ma conduite passée et présente, et de ma joie du changement de son
état. Il entra en matière, en homme qui craint moins de s'ouvrir que de
se laisser aller à la vanité de son nouvel éclat. Il me dit que
jusqu'alors il n'avait cherché qu'à s'occuper et à s'instruire\,; sans
s'ingérer à rien, qu'il n'avait pas cru devoir s'offrir ni se présenter
de lui-même, mais que, depuis que le roi lui avait ordonné de prendre
connaissance de tout, de travailler chez lui avec les ministres, et de
le soulager, il regardait tout son temps comme étant dû à l'État et au
public, et comme un larcin tout ce qu'il en déroberait aux affaires, ou
à ce qui le pourrait conduire à s'en rendre capable\,; qu'aussi ne
prenait d'amusement que par délassement, et pour se rendre l'esprit plus
propre à recommencer utilement après un relâchement nécessaire à la
nature. De là il s'étendit sur le roi, m'en parla avec une extrême
tendresse et une grande reconnaissance et me dit qu'il se croyait obligé
d'une manière très-étroite à contribuer à son soulagement, puisqu'il
avait la confiance en lui de le désirer. J'entrai fort dans des
sentiments si dignes, mais en peine si la tendresse, la reconnaissance
et le respect ne dégénéraient point en une admiration dangereuse. Je
glissai quelques mots sur ce que le roi ignorait bien des choses qu'il
s'était mis en état de ne pouvoir apprendre, et auxquelles sûrement sa
bonté ne demeurerait pas insensible si elles pouvaient arriver jusqu'à
lui.

Cette corde, touchée ainsi légèrement, rendit aussitôt un grand son. Le
prince, après quelques mots de préface sur ce qu'il savait par M. de
Beauvilliers qu'on pouvait sûrement me parler de tout, avoua la vérité
de ce que je disais, et tomba incontinent sur les ministres. Il
s'étendit sur l'autorité sans bornes qu'ils avaient usurpée, sur celle
qu'ils s'étaient acquise sur le roi, sur le dangereux usage qu'ils en
pouvaient faire, sur l'impossibilité de faire rien passer au roi, ni du
roi à personne, sans leur entremise\,; et, sans nommer aucun d'eux, il
me fit bien clairement entendre que cette forme de gouvernement était
entièrement contraire à son goût et à ses maximes. Revenant de là
tendrement au roi, il se plaignit de la mauvaise éducation qu'il avait
eue, et des pernicieuses mains dans lesquelles il était successivement
tombé\,; que par là, sous prétexte de politique et d'autorité dont tout
le pouvoir et tout l'utile n'était que pour les ministres, son cœur,
naturellement bon et juste, avait sans cesse été détourné du droit
chemin, sans s'en apercevoir\,; qu'un long usage l'avait confirmé dans
ces routes une fois prises, et avait rendu le royaume très-malheureux
Puis, se ramenant à soi avec humilité, il me donna de grands sujets de
l'admirer. Il revint après à la conduite des ministres, et j'en pris
occasion de le conduire sur leurs usurpations avec les ducs et avec les
gens de la plus haute qualité. À ce récit, l'indignation échappa à sa
retenue\,; il s'échauffa sur le \emph{monseigneur} qu'ils nous refusent,
et qu'ils exigeaient de tout ce qui n'était point titré, à l'exception
de la robe.

Je ne puis rendre à quel point cette audace le choqua, et cette
distinction si follement favorable à la bourgeoisie sur la plus haute
noblesse. Je le laissai parler, tant pour jouir des dignes sentiments de
celui qui se trouvait si proche d'en pouvoir faire des règles et des
lois, que pour m'instruire moi-même du degré ou l'équité enflammée le
pouvait porter. Je repris ensuite les commencements de
l'intervertissement de tout ordre, et je lui dis que le pur hasard
m'avait conservé trois lettres à mon père de M. Colbert, ministre
contrôleur général des finances et secrétaire d'État, qui lui écrivait
\emph{monseigneur}. Cela parut lui faire autant de plaisir que s'il y
avait été intéressé. Il m'ordonna de les envoyer chercher, et admira la
hardiesse d'un changement si entier. Nous le discutâmes\,; et comme il
aimait à approfondir et à remonter tant qu'il pouvait aux sources, il se
mit sur la naissance des charges de secrétaire d'État, dont la ténuité
de l'origine le surprit de nouveau, quoique lui-même, par l'explication
qu'il se prit à en faire, me montrât qu'il n'avait rien à apprendre
là-dessus\footnote{Voy., à la fin du volume, une note sur l'origine des
  secrétaires d'État.}.

Tout cela fut la matière de plus d'une heure d'entretien\,; elle nous
détourna de celle que nous devions traiter, mais d'une manière plus
importante que cette matière même, à laquelle celle de cet entretien
n'était rien moins qu'étrangère. Le Dauphin m'ordonna de l'avertir
lorsque j'aurais ces trois lettres de M. Colbert à mon père, et me dit
qu'en même temps nous reprendrions la matière qu'il s'était proposé de
traiter, et dont celle-ci l'avait diverti.

Il est difficile d'exprimer ce que je sentis en sortant d'avec le
Dauphin. Un magnifique et prochain avenir s'ouvrait devant moi. Je vis
un prince pieux, juste, débonnaire, éclairé et qui cherchait à le
devenir de plus en plus, et l'inutilité avec lui du futile, pièce
toujours si principale avec ces personnes-là. Je sentis aussi par cette
expérience, une autre merveille auprès d'eux, qui est que l'estime et
l'opinion d'attachement, une fois prise par lui et nourrie de tout
temps, résistait au non-usage et à la séparation entière d'habitude. Je
goûtai délicieusement une confiance si précieuse et si pleine, dès la
première occasion d'un tête-à-tête, sur les matières les plus capitales.
Je connus avec certitude un changement de gouvernement par principes.
J'aperçus sans chimères la chute des marteaux de l'État et des
tout-puissants ennemis des seigneurs et de la noblesse qu'ils avaient
mis en poudre à leurs pieds, et qui, ranimée d'un souffle de la bouche
de ce prince devenu roi, reprendrait son ordre, son état et son rang, et
ferait rentrer les autres dans leur situation naturelle. Ce désir en
général sur le rétablissement de l'ordre et du rang avait été toute ma
vie le principal des miens, et fort supérieur à celui de toute fortune
personnelle. Je sentis donc toute la douceur de cette perspective, et de
la délivrance d'une servitude qui m'était secrètement insupportable, et
dont l'impatience perçait souvent malgré moi.

Je ne pus me refuser la charmante comparaison de ce règne de
Monseigneur, que je n'avais envisagé qu'avec toutes les affres possibles
et générales et particulières, avec les solides douceurs de
l'avant-règne de son fils, et bientôt de son règne effectif, qui
commençait sitôt à m'ouvrir son cœur, et en même temps le chemin de
l'espérance la mieux fondée de tout ce qu'un homme de ma sorte se
pouvait le plus légitimement proposer, en ne voulant que l'ordre, la
justice, la raison du bien de l'État, celui des particuliers, et par des
voies honnêtes, honorables, et où la probité et la vérité se pourraient
montrer. Je résolus en même temps de cacher avec grand soin cette faveur
si propre, si on l'apercevait, à effrayer et à rameuter tout contre moi,
mais de la cultiver sous cette sûreté, et à me procurer avec discrétion
de ces audiences dans lesquelles j'aurais tant à apprendre, à semer, à
inculquer doucement, et à me fortifier\,; mais j'aurais cru faire un
larcin, et payer d'ingratitude, si j'avais manqué de faire hommage
entier de cette faveur à celui duquel je la tenais tout entière. Certain
d'ailleurs, comme je l'étais, que le duc de Beauvilliers avait le
passe-partout du cœur et de l'esprit du Dauphin, je ne crus pas
commettre une infidélité de lui aller raconter tout ce qui venait de se
passer entre ce prince et moi\,; et je me persuadai que la franchise du
tribut en soutiendrait la matière, et me servirait par les conseils à y
bien diriger ma conduite. J'allai donc tout de suite rendre cette
conversation au duc de Beauvilliers. Il n'en fut pas moins ravi que je
l'étais moi-même.

Ce duc, à travers une éminente piété presque de l'autre monde, d'une
timidité qui sentait trop les fers, d'un respect pour le roi trop peu
distant de l'adoration de latrie, n'était pas moins pénétré que moi du
mauvais de la forme du gouvernement, de l'éclat de la puissance et de la
manière de l'exercer des ministres, qui, chacun dans leur département,
et même au dehors, étaient des rois absolus\,; enfin non moins duc et
pair que je l'étais moi-même. Il fut étonné d'une ouvertures si grande
avec moi, et surpris d'un si grand effet de ce que lui-même avait pris
tant de soin de planter et de cultiver en ma faveur dans l'esprit de son
pupille. Sa vertu et ses mesures, qui le contenaient avec lui, l'y
captivaient, en sorte qu'il me parut qu'il ne l'avait guère ouï parler
si clairement. J'en fus surpris au dernier point mais cela me parut à
toute sa contenance, et aux répétitions qu'il exigea de moi sur ce qui
regardait le pouvoir des ministres, et la mauvaise éducation du roi. Il
m'avoua même sa joie sur ces deux chapitres, avec une naïveté qui me fit
comprendre que, encore qu'il n'apprît rien de nouveau sur les
dispositions du Dauphin, les expressions pourtant le lui étaient, et que
ce prince n'avait pas été si net, ni peut-être si loin avec lui. La
suite me le fit encore mieux sentir\,; car, soit que son caractère
personnel lui imposât des mesures qu'il ne se crût pas permis de
franchir, ou qu'il ne voulût franchir que peu à peu, peut-être comme un
maître qui aime mieux suivre son écuyer en de certains passages, il ne
tarda pas à prendre des mesures avec moi pour agir sur plusieurs choses
de concert, puis d'une manière conséquente par lui-même, il me parut
très-sensible à la confiance pleine de dépendance dont j'usais avec lui
là-dessus, et bien déterminé à faire usage de sa situation nouvelle. Peu
de jours après j'eus une autre audience. Il faut dire une fois pour
toutes que du Chesne ordinairement, rarement M. de Beauvilliers,
quelquefois le Dauphin bas, à la promenade, m'avertissait de l'heure de
me trouver chez lui, et que lorsque c'était moi qui voulais une
audience, je le disais à du Chesne, qui en prenait l'ordre aussitôt et
m'en avertissait. Où que ce fût dans la suite, Fontainebleau,
Versailles, Marly, j'entrais toujours à la dérobée par la garde-robe, où
du Chesne avait soin de m'attendre toujours seul pour m'introduire
aussitôt, et de m'attendre à la sortie seul encore, de façon que
personne ne s'en est jamais aperçu, sinon une fois la Dauphine, comme je
le raconterai en son lieu, mais qui en garda parfaitement le secret.

Je présentai au Dauphin ces trois lettres dont j'ai parlé de M. Colbert
à mon père. Il les prit, les regarda fort, les lut toutes trois, et
s'intéressa dans l'heureux hasard qui les avait conservées, et sauvées
du peu d'importance de leur contenu. Il en examina les dates, et retomba
sur l'insolence des ministres (il n'en ménagea pas le terme), et sur le
malheur des seigneurs. Je m'étais principalement proposé de le sonder
sur tout ce qui intéresse notre dignité\,; je m'appliquai donc à rompre
doucement tous les propos qui s'écartaient de ce but, à y ramener la
conversation, et la promener sur tous les différents chapitres. Je le
trouvai très-instruit du fond de notre dignité, de ses rapports à l'État
et à la couronne, de tout ce que l'histoire y fournit, assez sur
plusieurs autres choses qui la concernent, peu ou point sur d'autres,
mais pénétré de l'intérêt sensible de l'État, de la majesté des rois de
France et de la primauté de leur couronne, à soutenir et rétablir cette
première dignité du royaume, et du désir de le faire.

Je le touchai là-dessus par ce que j'avais reconnu de sensible en lui
là-dessus, à l'occasion de sa première visite à Saint-Germain avec
M\textsuperscript{me} la Dauphine, depuis la mort de Monseigneur. Je le
fis souvenir de la nouveauté si étrange des prétentions de l'électeur de
Bavière, tout incognito qu'il était, avec Monseigneur à Meudon. Je les
mis en opposition avec l'usage constant jusqu'alors, et avec ce que
l'histoire nous fournit de rois qui se sont contentés d'égalité avec des
fils de France. Je lui fis faire les réflexions naturelles sur le tort
extrême que la tolérance de ces abus faisait aux rois et à leur
couronne, qui portait après sur les choses les plus solides par
l'affaiblissement de l'idée de leur grandeur. Je lui montrai fort
clairement que les degrés de ces chutes étaient les nôtres, qui, avilis
au dedans et abandonnés au dehors, donnions lieu par nos flétrissures à
celles du trône même, par l'avilissement de ce qui en émane de plus
grand, et le peu de cas qu'on accoutume ainsi les étrangers à en faire.
Je lui exposai la nouveauté des usurpations faites sur nous par les
électeurs ses oncles, par quelle méprise cela était arrivé et demeuré,
d'où bientôt après l'électeur de Bavière s'était porté jusqu'à prétendre
la main de Monseigneur, et à s'y soutenir par des \emph{mezzo-termine},
tout incognito qu'il était, parce qu'il s'était aperçu qu'il n'y avait
qu'à prétendre et entreprendre. Je vins après à la comparaison des
grands d'Espagne avec les ducs pairs et vérifiés, qui me donna un beau
champ, et en même temps à la politique de Charles-Quint, soigneusement
imitée par les rois d'Espagne ses successeurs, qui non content d'avoir
si fort élevé leur dignité dans ses États, s'était servi de leur étendue
et de leur dispersion dans les différentes parties de l'Europe, et de
l'autorité que sa puissance lui avait acquise à Rome et dans d'autres
cours, pour leur y procurer le rang le plus grandement distingué, duquel
ils y jouissent encore, et qui sert infiniment à faire respecter la
couronne d'Espagne au dehors de ses États. Je passai de cet exemple à
celui du vaste usage que les papes ont su tirer, pour leur grandeur
temporelle, de celle où ils ont porté les cardinaux, dont la dignité se
peut appeler littéralement une chimère, puisqu'elle n'a rien de
nécessairement ecclésiastique, qu'elle n'en a ni ordres ni juridiction,
ainsi laïque avec les ecclésiastiques, ecclésiastique avec les laïques,
sans autre solidité que le droit d'élection des papes, et l'usage d'être
ses principaux ministres d'État. Me promenant ensuite en Angleterre,
chez les rois du nord et par toute l'Europe, je démontrai sans peine que
la France seule, entre tous les États qui la composent, souffre en la
personne de ses grands ce que pas un des autres n'a jamais toléré, non
pas même la cour impériale, quoique si fourmillante de tant de
véritables princes, et que la France seule aussi en a pensé périr, et la
maison régnante, dont la Ligue, sur tous exemples, me fournit toutes les
preuves.

Le Dauphin, activement attentif, goûtait toutes mes raisons, les
achevait souvent en ma place, recevait avidement l'impression de toutes
ces vérités. Elles furent discutées d'une manière agréable et
instructive. Outre la Ligue, les dangers que l'État et les rois ont si
souvent courus, jusqu'à Louis XIV inclusivement, par les félonies et les
attentats de princes faux et véritables, et les établissements qu'ils
leur ont valu au lieu de châtiment, ne furent pas oubliés. Le Dauphin,
extrêmement instruit de tous ces faits historiques, prit feu en les
déduisant, et gémit de l'ignorance et du peu de réflexions du roi. De
toutes ces diverses matières, je ne faisais presque que les entamer en
les présentant successivement au Dauphin, et le suivre après pour lui
laisser le plaisir de parler, de me laisser voir qu'il était instruit,
lui donner lieu à se persuader par lui-même, à s'échauffer, à se piquer,
et à moi de voir ses sentiments, sa manière de concevoir et de prendre
des impressions, pour profiter de cette connaissance, et augmenter plus
aisément par les mêmes voies sa conviction et son feu. Mais cela fait
sur chaque chose, je cherchais moins à pousser les raisonnements et les
parenthèses qu'à le conduire sur d'autres objets, afin de lui montrer
une modération qui animât sa raison, sa justice, sa persuasion venue de
lui-même, et sa confiance\,; et pour avoir le temps aussi de le sonder
partout, et de l'imprégner doucement et solidement de mes sentiments et
de mes vues sur chacune de ces matières, toutes distinctes dans la même.
Je n'oubliai pas d'assener sur M. d'Espinoy, en passant, le terme
d'apprenti prince, et sur M. de Talmont et autres pareils, par vérité
d'expression, et pour m'aider d'un ridicule qui sert souvent beaucoup
aux desseins les plus sérieux. Content donc au dernier point de ce que
le Dauphin sentait sur les rangs étrangers, la plume et la robe qui eut
aussi son léger chapitre, je mis en avant le nouvel édit de cette année
1711, fait à l'occasion de d'Antin sur les duchés.

Je discutai avec le Dauphin, naturellement curieux de savoir et
d'apprendre\,; je discutai, dis-je, avec lui les prétentions diverses
qui y avaient donné lieu. Je ne le fis que légèrement pour le
satisfaire, dans le dessein de passer le plus tôt que je le pourrais aux
deux premiers articles de cet édit, et de m'y étendre selon que j'y
trouverais d'ouverture. J'y portai donc le prince. Ma surprise et ma
satisfaction furent grandes, lorsqu'à la simple mention, je le vis
prendre la parole et me déduire lui-même et avec ardeur l'iniquité de
ces deux premiers articles, et de là passer tout de suite aux
usurpations des princes du sang, et s'étendre sur l'énormité du rang
nouveau des bâtards. Les usurpations des princes du sang furent un des
points où je le trouvai le plus au fait de l'état en soi de ces princes,
et de celui de notre dignité, et en même temps parfaitement équitable,
comme il me l'avait paru sur tous les autres. Il me déduisit
très-nettement l'un et l'autre, avec cette éloquence noble, simple et
naturelle qui charmait sur les matières les plus sèches, combien plus
sur celle-ci. Il admettait avec grande justice et raison l'idée qu'avait
eue Henri III, par l'équité, de donner aux héritiers possibles d'une
couronne successive et singulièrement masculine une préséance et une
prééminence sur ceux qui, bien que les plus grands de l'État, ne peuvent
toutefois dépouiller jamais la condition de sujets\,; mais n'oubliant
point aussi qu'avant Henri III nos dignités précédaient le sang royal
qui n'en était pas revêtu, et qui jusqu'alors avait si peu compté ce
beau droit exclusif de succéder à la couronne que les cadets de branches
aînées cédaient partout aux chefs des branches cadettes, qui toutefois
pouvaient devenir sujets de ces cadets qu'ils précédaient, il se souvint
bien, de lui-même, que la préséance et prééminence ne put être établie
qu'en supposant et rendant tout le sang royal masculin pair de droit,
sans terre érigée par droit d'aînesse, et plus ancien que nuls autres,
par lui faire tirer son ancienneté d'Hugues Capet, abolissant en même
temps toute préséance entre les princes du sang par autre titre que
celui de leur aînesse.

Avec ces connaissances exactes et vraies, le Dauphin ne pouvait souffrir
l'avilissement de notre dignité, par ceux-là mêmes qui s'en étaient si
bien servis pour leur élévation quoique si juste. Il se déclara donc
fort contre les usurpations que les princes du sang lui avaient
faites\,; sur toutes il ne put souffrir l'attribution aux princes du
sang, par l'édit, de la représentation des anciens pairs au sacre, à
l'exclusion des pairs. Il sentait parfaitement toute la force
d'expression des diverses figures de cette auguste cérémonie, et il me
laissa bien clairement apercevoir qu'il voulait être couronné comme
l'avaient été ses ancêtres. Moins informé des temps et des occasions des
usurpations des princes du sang sur les pairs, que des usurpations
mêmes, je l'en entretins avec un grand plaisir de sa part, plus soigneux
de le suivre et de satisfaire à ses questions pour entretenir son feu et
sa curiosité, que de lui faire des récits et une suite de discours. En
garde contre l'écoulement du temps, lorsque je le crus pour cette fois
suffisamment instruit sur les princes du sang, je m'aidai de la grandeur
des bâtards, qui avait si fort servi à augmenter celle des princes du
sang, pour amener le Dauphin aux légitimés. C'était une corde que je
voulais lui faire toucher le premier, pour sentir au son qu'il lui
donnerait le ton que je devais prendre à cet égard. Ma sensibilité sur
tout ce qu'ils nous ont enlevé, et le respect du Dauphin pour le roi son
grand-père, m'étaient également suspects, de manière qu'attentif à le
suivre sur les princes du sang, et à ne faire que lui montrer les
autres, je fus longtemps à le faire venir à mon point. Il y tomba enfin
de lui-même. Prenant alors un ton plus bas, des paroles plus mesurées,
mais en échange un visage plus significatif, car mes yeux travaillaient
avec autant d'application que mes oreilles, il se mit sur les excuses du
roi, sur ses louanges, sur le malheur de son éducation, et celui de
l'état où il s'était mis de ne pouvoir entendre personne. Je ne
contredisais que de l'air et de la contenance, pour lui faire sentir
modestement combien ce malheur portait à plein sur nous. Il entendit
bien ce langage muet, et il m'encouragea à parler. Je préludai donc
comme lui par les louanges du roi, par les plaintes que lui-même en
avait faites, et je tombai enfin sur les inconvénients qui en
résultaient.

Je me servis, non sans cause, de la piété, de l'exemple, de la tentation
nouvelle, ajoutée à celle de la chose même, qui précipiterait toutes les
femmes entre les bras des rois, le scandale de l'égalité entière entre
le fils du sacrement et le fils du double adultère, c'est-à-dire après
deux générations, de l'égalité parfaite, de l'égalité de la postérité
des rois légitime et illégitime, comme on le voyait déjà entre M. le duc
de Chartres et les enfants de M. du Maine\,; et ces remarques ne furent
point languissantes.

Le Dauphin, satisfait de son exorde, et peut-être content du mien,
excité après par mes paroles, m'interrompit et s'échauffa. Cette
application présente le frappa vivement. Il se mit sur la différence
d'une extraction qui tire toute celle qui la distingue si grandement de
son habileté innée à la couronne, d'avec une autre qui n'est due qu'à un
crime séducteur et scandaleux qui ne porte avec soi qu'infamie. Il
parcourut les divers et nombreux degrés par lesquels les bâtards (car ce
mot fut souvent employé) étaient montés au niveau des princes du sang,
et qui, pour leur avantage, avaient élevé ce niveau de tant d'autres
degrés à nos dépens. Il traita de nouveau le point du sacre énoncé dans
l'édit\,; et, s'il avait paru intolérable dans les princes du sang, il
lui sembla odieux, et presque sacrilége dans les légitimés. Dans tout
cela, néanmoins, de fréquents retours de respect, d'attendrissement même
et de compassions pour le roi, qui me firent admirer souvent la juste
alliance du bon fils et du bon prince dans ce Dauphin si éclairé. Sur la
fin se concentrant en lui-même\,: «\,C'est un grand malheur, me dit-il,
d'avoir de ces sortes d'enfants. Jusqu'ici Dieu me fait la grâce d'être
éloigné de cette route\,; il ne faut pas s'en élever. Je ne sais ce qui
m'arrivera dans la suite. Je puis tomber dans toutes sortes de
désordres, je prie Dieu de m'en préserver\,! mais je crois que, si
j'avais des bâtards, je me garderais bien de les élever de la sorte, et
même de les reconnaître. Mais c'est un sentiment que j'ai à présent par
la grâce que Dieu me fait\,; comme on n'est pas sûr de la mériter et de
l'avoir toujours, il faut au moins se brider là-dessus de telle sorte
qu'on ne puisse plus tomber dans ces inconvénients.\,»

Un sentiment si humble et en même temps si sage me charma\,; je le louai
de toutes mes forces. Cela attira d'autres témoignages de sa piété et de
son humilité\,; après quoi, la conversation revenue à son sujet, je lui
dis qu'on n'ignorait pas la peine qu'il avait eue des dernières
grandeurs que M. du Maine avait obtenues pour ses enfants. Jamais rien
ne peut être plus expressif que le fut sa réponse muette\,: toute sa
personne prit un renouvellement de vivacité que je vis qu'il eut peine à
contenir. L'air de son visage, quelques gestes échappés à la retenue que
l'improbation précise du roi lui imposait, témoignèrent avec éloquence
combien impatiemment il supportait ces grandeurs monstrueuses, et
combien peu elles dureraient de son règne. J'en vis assez pour en
espérer tout, pour oser même le lui faire entendre\,; et je reconnus
très-bien que je lui plaisais.

Enfin, la conversation ayant duré plus de deux heures, il me remit en
gros sur les pertes de notre dignité, sur l'importance de les réparer,
et me témoigna qu'il serait bien aise d'en être instruit à fond. Dans le
commencement de la conversation, je lui avais dit qu'il serait surpris
du nombre et de l'excès de nos pertes, s'il les voyait toutes d'un coup
d'œil. Je lui proposai ici d'en faire les recherches et de les lui
présenter\,; non-seulement il le voulut bien, mais il me pria avec
ardeur de le faire. Je lui demandai un peu de temps pour ne lui rien
donner que de bien exact, et je lui laissai le choix de l'ordre que j'y
donnerais, par natures de choses et de matières, ou pour dates de
pertes. Il préféra le dernier, quoique moins net pour lui, et plus
pénible pour moi\,; je le lui représentai, même sur-le-champ, mais il
persista dans ce choix, et il m'était trop important de le servir
là-dessus à son gré pour y rien ménager de ma peine. J'omets ici les
remercîments que je lui fis de l'honneur de sa confiance, et tout ce
qu'il eut la bonté de me dire de flatteur. Il me donna, en prenant congé
de lui, la liberté de ne le voir en public qu'autant que je le jugerais
à propos sans inconvénient, et en particulier, toutes les fois que je le
désirerais, pour l'entretenir de ce que j'aurais à lui dire.

Il n'est pas difficile d'imaginer dans quel ravissement je sortis d'un
entretien si intéressant. La confiance d'un Dauphin, juste, éclairé, si
près du trône, et qui y participait déjà, ne laissait rien à désirer
pour la satisfaction présente, ni pour les espérances. Le bonheur et la
règle de l'État, et après, le renouvellement de notre dignité, avaient
été dans tous les temps de ma vie l'objet le plus ardent de mes désirs,
qui laissaient loin derrière celui de ma fortune. Je rencontrais tous
ces objets dans le Dauphin\,; je me voyais en situation de contribuer à
ces grands ouvrages, de m'élever en même temps, et avec un peu de
conduite, en possession tranquille de tant et de si précieux avantages.
Je ne pensai donc plus qu'à me rendre digne de l'une et coopérateur
fidèle des autres.

Je rendis compte le lendemain au duc de Beauvilliers de ce qui s'était
passé entre le Dauphin et moi. Il mêla sa joie à la mienne\,; il ne fut
point surpris de ses sentiments sur notre dignité, en particulier sur
les bâtards. J'avais déjà bien su, comme je l'ai rapporté alors, que le
Dauphin s'était expliqué à lui, lors des grandeurs accordées aux enfants
du duc du Maine\,; je vis encore mieux ici qu'ils s'étaient bien
expliqués ensemble sur les bâtards, et que M. de Beauvilliers l'avait
fort instruit sur notre dignité. Nous convînmes de plus en plus d'un
concert entier sur tout ce qui aurait rapport au Dauphin, et aux
matières qui s'étaient traitées dans mes deux conversations avec lui\,;
que je le verrais plutôt à ses promenades qu'aux heures de cour chez
lui, parce que j'y serais plus libre de les suivre et de les quitter, de
remarquer, de parler ou de me taire, suivant ce qui s'y trouverait\,;
d'avoir attention d'éviter d'aborder et de quitter la promenade du roi
avec le Dauphin, et de lui parler en sa présence\,; enfin, de tout ce
que la prudence peut suggérer pour éviter tout éclat, m'insinuer de plus
en plus, et profiter au mieux de ce qui se présentait à moi de si bonne
grâce. Il m'avertit que je pouvais parler de tout sans aucune sorte de
crainte au Dauphin, et que je devais le faire selon que je le jugerais à
propos, étant bon de l'y accoutumer\,; il finit par m'exhorter au
travail où je m'étais engagé\,: c'étaient les fruits de ce qu'il avait
de longue main préparé, puis fait pour moi auprès du Dauphin. Son amitié
et son estime l'avaient persuadé que la confiance que ce prince pourrait
prendre en moi serait utile à l'état et au prince, et il était si sûr de
moi que c'était initier un autre soi-même.

Il préparait et dirigeait le travail particulier du Dauphin avec les
ministres, eux-mêmes ne le pouvaient guère ignorer. L'ancienne rancune
de M\textsuperscript{me} de Maintenon cédait au besoin présent d'un
homme qu'elle n'avait pu renverser, qui était toujours demeuré avec elle
dans une mesure également ferme et modeste, qui était incapable d'abuser
de ce que le Dauphin lui était, duquel elle ne craignait rien pour
l'avenir, bien assurée de la reconnaissance de ce prince, qui sentait
qu'il lui devait la confiance du roi, et l'autorité où il commençait à
l'élever, d'ailleurs sûre de la Dauphine comme d'elle-même, pour l'amour
de laquelle elle avait ramené le roi jusqu'à ce point. Par conséquent le
roi, qui ne trouvait plus d'aigreur ni de manéges en
M\textsuperscript{me} de Maintenon, contre M. de Beauvilliers, suivait
son penchant d'habitude, d'estime et de confiance, et n'était point
blessé de ce qui était pesant aux ministres, et de ce qui mettait le duc
dans une situation si principale au dedans et si considérable au dehors.
Bien qu'on ignorât à la cour jusqu'où allait mon intérieur avec lui, et
entièrement mes particuliers avec le Dauphin, je ne laissais pas d'être
regardé, examiné, compté tout autrement que je ne l'avais été
jusqu'alors. On me craignit, on me courtisa. Mon application fut de
paraître toujours le même, surtout désoccuppé, et d'être en garde contre
tout air important, et contre tout ce qui pouvait découvrir rien de ce
que tant d'envieux et de curieux cherchaient à pénétrer\,; jusqu'à mes
plus intimes amis, jusqu'au chancelier même, je ne laissai voir que
l'écorce que je ne pouvais cacher.

Le duc de Beauvilliers était presque tous les jours enfermé longtemps
avec le Dauphin et le plus souvent mandé par lui. Ils digéraient
ensemble les matières principales de la cour, celles d'État, et le
travail particulier des ministres. Beaucoup de gens qui n'y pensaient
guère y passaient en revue en bien et en mal, qui presque toujours
avaient été ballottés entre le duc et moi, avant d'être discutés entre
lui et le Dauphin. Il en était de même de quantité de matières
importantes, et de celles surtout qui regardaient la conduite de ce
prince\,; une entre autres tomba fort en dispute entre le duc et moi,
sur laquelle je ne pus céder ni le persuader, et qui regardait la
succession de Monseigneur.

Le roi eut un moment envie d'hériter, mais fit bientôt réflexion que
cela serait trop étrange. Elle fut traitée comme celle du plus simple
particulier, et le chancelier et son fils furent chargés seuls, en
qualité de commissaires, d'y faire ce que les juges ordinaires font à la
mort des particuliers. Meudon et Chaville, qui valaient environ quarante
mille livres de rente, et pour un million cinq cent mille livres de
meubles ou de pierreries, composaient tout ce qui était à partager, sur
quoi il y avait à payer trois cent mille livres de dettes. Le roi
d'Espagne se rapporta au roi de ses intérêts, et témoigna qu'il
préférait des meubles pour ce qui lui devait revenir. Il y avait encore
une infinité de bijoux de toute espèce. Le roi voulut que les pierres de
couleur fussent pour le Dauphin, parce que la couronne en avait peu, et
au contraire beaucoup de diamants. On fit donc un inventaire, une prisée
de tous les effets mobiliers, et trois lots\,: les plus beaux meubles et
les cristaux furent pour le roi d'Espagne, et les diamants pour M. le
duc de Berry avec un meuble. Tous les bijoux et les moindres meubles,
qui à cause de Meudon étaient immenses, se vendirent à l'encan pour
payer les dettes. Du Mont et le bailli de Meudon furent chargés de la
vente, qui se fit à Meudon de ces moindres meubles, et des joyaux les
plus communs. Les principaux bijoux, et qui étaient en assez grand
nombre, se vendirent avec une indécence qui n'a peut-être point eu
d'exemple. Ce fut dans Marly, dans l'appartement de
M\textsuperscript{me} la Dauphine, en sa présence, quelquefois en celle
de M. le Dauphin, par complaisance pour elle, et ce fut pendant la
dernière moitié du voyage de Marly l'amusement des après-dînées. Toute
la cour, princes et princesses du sang, hommes et femmes, y entraient à
portes ouvertes\,; chacun achetait à l'enchère\,; on examinait les
pièces, on riait, on causait, en un mot un franc inventaire, un vrai
encan. Le Dauphin ne prit presque rien, mais il fit quelques présents
aux personnes qui avaient été attachées à Monseigneur, et les confondit,
parce qu'il n'avait pas eu lieu de les aimer du temps de ce prince.
Cette vente causa quelques petites riotes entre la Dauphine et M. le duc
de Berry, poussé quelquefois par M\textsuperscript{me} la duchesse de
Berry, par l'envie des mêmes pièces. Elles furent même poussées assez
loin sur du tabac dont il y avait en grande quantité, et d'excellent,
parce que Monseigneur en prenait beaucoup, pour qu'il fallût que M. de
Beauvilliers et quelques dames des plus familières s'en mêlassent, et
pour le coup la Dauphine avait tort, et en vint même à la fin à quelques
excuses de fort bonne grâce. Le partage de M. le duc de Berry était
tombé en litige, parce qu'il avait eu un apanage dont Monseigneur et lui
avaient signé l'acte, ce qui opérait sa renonciation à la succession du
roi et à celle de Monseigneur, comme en étant déjà rempli d'avance. Cela
fut jugé de la sorte devant le roi, qui en même temps lui donna, par une
augmentation d'apanage, tout ce qui lui serait revenu de son partage
outre le meuble et les diamants. Pendant que tout cela s'agitait, le roi
fit hâter le partage et la vente des meubles, dans la crainte que celui
de ses deux petits-fils à qui Meudon demeurerait n'en voulût faire
usage, et partageât ainsi la cour de nouveau.

Cette inquiétude était vaine. On a vu qu'il devait être pleinement
rassuré là-dessus du côté du Dauphin, et à l'égard de M. le duc de Berry
qui n'aurait osé lui déplaire\,; la suite d'un prince cadet, quand même
il aurait usé de Meudon, n'aurait pas rendu la cour moins grosse,
surtout dès qu'on s'y serait aperçu que ce n'aurait pas été faire la
sienne au roi qu'être de ces voyages. Ce prince, qui dans tout son
apanage n'avait aucune demeure, désirait passionnément Meudon, et
M\textsuperscript{me} la duchesse de Berry encore davantage. Mon
sentiment était que le Dauphin lui fît présent de toute sa part\,; il
vivait de la couronne en attendant qu'elle tombât sur sa tête\,; il ne
perdait donc rien à ce don\,; il y gagnait au contraire le plaisir, la
reconnaissance, la bienséance même, d'un bienfait considérable, et plein
de charmes pour M. son frère, et pour M\textsuperscript{me} la duchesse
de Berry, qui recevrait sûrement un applaudissement universel. M. de
Beauvilliers, à qui je le dis, ne me surprit pas peu par un avis
contraire. Sa raison, qu'il m'expliqua, fut que rien ne serait plus
dangereux que donner occasion et tentation à M. {[}le duc{]} et à
M\textsuperscript{me} la duchesse de Berry d'une cour à part qui
déplairait souverainement au roi, et qui tout au plus différée après
lui, séparerait les deux frères, et deviendrait la source sinon de
discorde, du moins de peu d'union\,; qu'il fallait que l'aîné jouît de
tous ses avantages, que le cadet dépendît toujours de lui\,; qu'il
valait mieux qu'il fût pauvre en attendant que son frère fût roi pour
recevoir alors des marques de sa libéralité, que si, mis prématurément à
son aise, il se trouvait alors en état de se passer, conséquemment de
mériter peu ses bienfaits\,; qu'avoir Meudon et ne donner pas le moindre
signe d'en vouloir user, serait au Dauphin un moyen sûr de plaire
infiniment au roi\,; qu'en un mot Meudon convenait au Dauphin, qu'il y
avait sa part et son préciput, et celle encore du roi d'Espagne en lui
donnant des meubles et d'autres choses en échange, et que, si M. le duc
de Berry se trouvait y avoir quelque chose, il l'en fallait récompenser
en diamants.

Ce raisonnement politique me parut fort tiré et ne put m'entrer dans la
tête. Je soutins au duc la supériorité des bienfaits sur la nécessité à
l'égard d'un fils de France\,; la bienséance d'adoucir par des prémices
solides d'amitié cette grande différence que la mort du père mettait
entre les frères, et la totale dont la perspective commençait à se faire
sentir\,; l'utile sûreté d'émousser les semences d'aigreur entre eux, en
saisissant l'occasion unique de gratifier un frère avant d'être son
roi\,; la disproportion de l'avantage idéal d'un côté, très-effectif de
l'autre, et celle de l'impression que prendrait le monde d'une conduite
sèche, dure, littérale, ou remplie de générosité et de tendresse\,;
l'impuissance de retenir un frère dans sa future cour qu'à faute de
maison ailleurs, que tôt ou tard il lui faudrait bien donner, non comme
grâce, mais comme chose de toute nécessité\,; l'abondance des moyens,
toujours nouveaux, fournis par la couronne, de gratifier un frère qui
même était si mal apanagé, et à qui Meudon augmenterait bien plus qu'il
ne diminuerait le besoin des grâces, comme on avait vu que Saint-Cloud
avait été une source de besoins à Monsieur si prodigieusement apanagé,
et au roi un moyen continuel de le tenir, dont il avait si bien su
profiter\,; enfin indépendamment du sacrifice de l'usage de Meudon, le
Dauphin, établi et soutenu comme il l'était dans l'entière confiance du
roi, et ancré déjà par son grand-père dans l'exercice, et en la
disposition même en partie des affaires, ne manquerait pas d'occasions
et de moyens journaliers de lui plaire, et de s'établir de plus en plus
dans son cœur, dans son esprit, et dans toute l'administration. Il me
semblait et il me semble encore que mon raisonnement là-dessus était
juste et solide. Aussi devint-il celui de tout le monde, mais il ne
persuada point M. de Beauvilliers.

Meudon demeura au Dauphin, et tout ce qui regarda cette succession fut
traité avec la même rigueur. Elle ne fit pas honneur dans le monde, ni
un bon effet en M. {[}le duc{]} et M\textsuperscript{me} la duchesse de
Berry, à qui je me gardai bien de laisser entrevoir quoi que ce soit
là-dessus. Mais il n'était pas indifférent au bien dont il avait peu à
proportion de ses charges, et dont il dépensait avec fort peu de mesure,
et poussé de plus par M\textsuperscript{me} la duchesse de Berry, haute
avec emportement, et déjà si éloignée de cœur du Dauphin, surtout de la
Dauphine. Ils se turent sagement, n'imaginèrent pas que le duc de
Beauvilliers eût aucune part en cette affaire, et ne tardèrent pas à
vendre beaucoup de diamants de leur héritage pour remplir les vides que
leurs fantaisies avaient déjà creusés dans leurs affaires.

\hypertarget{chapitre-xvi.}{%
\chapter{CHAPITRE XVI.}\label{chapitre-xvi.}}

1711

~

{\textsc{Je vois souvent le Dauphin tête à tête.}} {\textsc{- Le
Dauphin, seul avec moi, surpris par la Dauphine.}} {\textsc{- Ma
situation à l'égard de la Dauphine.}} {\textsc{- Mérite de
M\textsuperscript{me} de Saint-Simon m'est très-utile.}} {\textsc{-
Aversion de M\textsuperscript{me} de Maintenon pour moi\,; sur quoi
fondée.}} {\textsc{- Je travaille à unir M. le duc d'Orléans au
Dauphin.}} {\textsc{- Intérieur de la famille royale, et le mien avec
elle.}} {\textsc{- Je donne un étrange avis à M. le duc d'Orléans, qui
en fait un plus étrange usage avec M\textsuperscript{me} sa fille.}}
{\textsc{- Je me brouille et me laisse après raccommoder avec lui, et je
demeure très-froidement avec M\textsuperscript{me} la duchesse de Berry
depuis.}} {\textsc{- Dégoûts du roi de M. le duc d'Orléans.}} {\textsc{-
Dangereux manéges du duc du Maine, qui projette le mariage de son fils
avec une sœur de M\textsuperscript{me} la duchesse de Berry.}}
{\textsc{- Je travaille à unir M. le duc d'Orléans au Dauphin et au duc
de Beauvilliers, {[}union{]} à laquelle je réussis.}}

~

Je voyais souvent le Dauphin en particulier, et je rendais aussitôt
après au duc de Beauvilliers ce qui s'y était passé. Je profitai de son
avis, et je parlai de tout au prince. Sa réserve ni sa charité ne
s'effarouchèrent de rien\,; non-seulement il entra aisément et avec
liberté dans tout ce que je mis sur le tapis de choses et de personnes,
mais il m'encouragea à le faire, et me chargea de lui rendre compte de
beaucoup de choses et de gens. Il me donnait des mémoires, je les lui
rendais avec le compte qu'il m'en avait demandé\,; je lui en donnais
d'autres qu'il gardait et qu'il discutait après avec moi en me les
rendant. Je garnissais toutes mes poches de force papiers toutes les
fois que j'allais à ces audiences, et je riais souvent en moi-même,
passant dans le salon, d'y voir force gens qui se trouvaient
actuellement dans mes poches, et qui étaient bien éloignés de se douter
de l'importante discussion qui allait se faire d'eux.

Le Dauphin logeait alors dans celui des quatre grands appartements de
plain-pied au salon, que la maladie de M\textsuperscript{me} la
princesse de Conti, comme je l'ai remarqué lors de la mort de
Monseigneur, fit rompre pendant le voyage suivant de Fontainebleau, pour
y placer un grand escalier, parce que le roi avait eu peine à monter
chez elle par les petits degrés tortueux, uniques alors. La chambre du
prince était dans cet emplacement\,; le lit avait les pieds aux
fenêtres\,; à la ruelle du côté de la cheminée était la porte de la
garde-robe obscure par où j'entrais\,; entre la cheminée et une des deux
fenêtres, un petit bureau portatif à travailler\,; vis-à-vis la porte
ordinaire d'entrée, et derrière le siége à travailler et le bureau, la
porte d'une autre pièce du côté de la Dauphine\,; entre les deux
fenêtres une commode qui n'était que pour des papiers.

Il y avait toujours quelques moments de conversation avant que le
Dauphin se mît à son bureau, et qu'il m'ordonnât de m'asseoir vis-à-vis
tout contre. Devenu plus libre avec lui, je pris la liberté de lui dire,
dans ces premiers moments de conversation debout, qu'il ferait bien de
pousser le verrou de la porte derrière lui. Il me dit que la Dauphine ne
viendrait pas, et que ce n'étaient pas là ses heures. Je répondis que je
ne craindrais point cette princesse seule, mais beaucoup
l'accompagnement qui la suivait toujours\,; il fut opiniâtre et n'en
voulut rien faire. Je n'osai l'en presser davantage\,; il se mit à son
bureau et m'ordonna de m'y mettre aussi. La séance fut longue, après
laquelle nous triâmes nos papiers. Il me donna des siens à mettre dans
mes poches, il en prit des miens, il en enferma dans sa commode, et, au
lieu d'en enfermer d'autres dans son bureau, il en laissa dessus et se
mit à causer, le dos à la cheminée, des papiers dans une main et ses
clefs dans l'autre. J'étais debout au bureau, y cherchant quelques
papiers d'une main et de l'autre en tenant d'autres, lorsque tout à coup
la porte s'ouvrit vis-à-vis de moi, et la Dauphine entra.

Ce premier coup d'oeil de tous les trois, car Dieu merci elle était
seule, l'étonnement, la contenance de tous les trois ne sont jamais
sortis de ma mémoire. Le fixe des yeux et l'immobilité de statue, le
silence, l'embarras également dans tous trois, dura plus d'un lent
\emph{Pater}. La princesse le rompit la première. Elle dit au prince,
d'une voix très-mal assurée, qu'elle ne le croyait pas en si bonne
compagnie, en souriant à lui et puis à moi. J'eus le temps de sourire
aussi et de baisser les yeux avant que le Dauphin répondît. «\, Puisque
vous m'y trouvez, madame, lui dit-il en souriant de même,
allez-vous-en.\,» Elle fut un instant à le regarder en lui souriant
davantage et lui à elle\,; elle me regarda après toujours souriant avec
plus de liberté que d'abord, fit après la pirouette, sortit et ferma la
porte, dont elle n'avait pas dépassé plus que la profondeur.

Jamais je ne vis femme si étonnée\,; jamais, j'en hasarderai le mauvais
mot, je ne vis homme si penaud que le prince, même après la sortie\,;
jamais homme, car il faut tout dire, n'eut si grand'peur que j'eus
d'abord, mais qui se rassura dès que je ne la vis point suivie. Sitôt
qu'elle eut fermé la porte\,: «\,Eh bien, monsieur, dis-je au Dauphin,
si vous aviez bien voulu tirer le verrou\,? --- Vous aviez raison, me
dit-il, et j'ai eu tort. Mais il n'y a point de mal, elle était seule
heureusement, et je vous réponds de son secret. --- Je n'en suis point
en peine, lui dis-je (si l'étais-je bien toutefois), mais c'est un
miracle de ce qu'elle s'est trouvée seule. Avec sa suite vous en auriez
été quitte pour être peut-être grondé, mais moi, je serais perdu sans
ressource.\,» Il convint encore de son tort, et me rassura de plus en
plus sur le secret. Elle nous avait pris non-seulement tête à tête, ce
dont personne au monde n'avait le moindre soupçon, mais sur le fait,
mais, comme on dit, le larcin à la main. Je compris bien qu'elle ne
voudrait pas exposer le Dauphin, mais je craignais la facilité de
quelque confidence, et de là la révélation après du secret. Toutefois il
fut si bien gardé, ou confié, s'il le fut, à personnes si sûres qu'il
n'en a jamais rien transpiré. Je n'insistai pas davantage. Nous
achevâmes, moi d'empocher, le prince de serrer nos papiers. Le reste de
la conversation fut court, et je me retirai par la garde-robe, comme
j'étais venu, et comme je faisais toujours, où du Chesne seul
m'attendait. M. de Beauvilliers, à qui je contai l'aventure, en lui
rendant compte du travail, en pâlit d'abord, et se remit lorsque je lui
dis que la Dauphine était seule, et blâma fort l'imprudence du verrou\,;
mais il me rassura aussi sur le secret.

Depuis cette découverte la Dauphine me sourit souvent, comme pour m'en
faire souvenir, et prit pour moi un air d'attention marqué. Elle aimait
fort M\textsuperscript{me} de Saint-Simon, et ne lui en a jamais parlé.
Moi, elle me craignait en gros, parce qu'elle craignait fort les ducs de
Chevreuse et de Beauvilliers, dont les allures graves et sérieuses
n'étaient pas les siennes, et qu'elle n'ignorait pas mon intime et
ancienne liaison avec eux. Leurs mœurs et leur influence sur le Dauphin
la gênait\,; l'aversion de M\textsuperscript{me} de Maintenon pour eux
ne l'avait pas rassurée\,; la confiance du roi en eux et leur liberté
avec lui, toute timide qu'elle était, la tenait aussi en presse. Elle
les redoutait, surtout M. de Beauvilliers, sur l'article le plus délicat
auprès de son époux, et peut-être auprès du roi\,; et elle ignorait,
sans qu'on osât le lui apprendre, à quel point il était occupé de la
frayeur de ce qu'elle craignait de lui, et qui lui pouvait arriver par
d'autres, et de toutes les précautions possibles à sagement prendre pour
y barrer tout chemin. Pour moi, qui en étais tout aussi éloigné, et
qu'elle n'avait pas lieu d'appréhender là-dessus, je n'avais jamais été
en aucune familiarité avec elle. Cela ne pouvait guère arriver que par
le jeu, et je ne jouais point, très-difficilement par ailleurs, et je ne
l'avais point même recherché. Cette liaison des deux ducs et ma vie
sérieuse avaient formé en elle, qui était timide, cette appréhension à
laquelle M\textsuperscript{me} de Maintenon, qui ne m'aimait pas, avait
pu contribuer aussi\,; mais cela n'allait pas jusqu'à l'éloignement, par
d'autres liaisons aussi fort étroites que j'avais avec des dames de sa
confiance, comme avait été la duchesse de Villeroy, et comme était
M\textsuperscript{me} la duchesse d'Orléans, M\textsuperscript{me} de
Nogaret et quelques autres\,; outre qu'elle était légère, et qu'un
éloignement effectif pour moi ne lui aurait pas permis de vouloir faire
succéder M\textsuperscript{me} de Saint-Simon à la duchesse du Lude
autant qu'elle le désirait, et de prendre là-dessus tous les devants et
tous les tournants pour l'y conduire. Le Dauphin ne le souhaitait pas
moins. Il ne s'en cacha pas à elle-même, et il y avait pris confiance
par l'estime de sa vertu et de sa conduite égale, et amitié par
l'agrément et la douceur, surtout la sûreté de sa société, qu'il
éprouvait sans cesse dans la familiarité des particuliers et des parties
avec M\textsuperscript{me} la duchesse de Bourgogne, de tout temps,
beaucoup plus encore depuis le mariage de M\textsuperscript{me} la
duchesse de Berry, qui, mettant nécessairement M\textsuperscript{me} de
Saint-Simon de tout dans leur intrinsèque, avait formé plus d'habitude
et leur avait montré un assemblage de vertu, de douceur, de sagesse, de
grand sens et de discrétion, qui les charma, dans l'exercice d'un emploi
que l'humeur de M\textsuperscript{me} la duchesse de Berry ne rendait
pas moins difficile que son tempérament, qui lui conciliait la plus
grande considération de cette princesse, et sans aucun soupçon, en même
temps que toute l'amitié et la confiance de M. le duc de Berry\,; et
tout cela entretenu par l'estime et la considération très-marquée en
tout temps pour elle du roi et de M\textsuperscript{me} de Maintenon,
par l'affection générale et la réputation entière qu'elle s'était
acquise et entretenue à la cour depuis qu'elle y était, et sans soins,
surtout sans bassesses ni rien qui les sentît, et avec beaucoup de
dignité, qui, avec l'opinion que le monde avait prise d'elle, la fit
toujours singulièrement respecter, et qui dans tous les temps de ma vie
m'a été un grand soutien et une puissante ressource.

Je viens de dire que M\textsuperscript{me} de Maintenon ne m'aimait pas.
Je ne faisais alors que m'en douter, et cet article mérite de s'y
étendre un moment, au hasard de quelque répétition. Il y avait longtemps
qu'elle me haïssait, sans que je l'eusse mérité d'elle. Chamillart me
l'apprit après la mort du roi, jusqu'à laquelle il ne m'en avait pas
laissé soupçonner la moindre chose. Il me dit alors que, lorsqu'il
travailla à me raccommoder avec le roi et à me remettre dans le train
ordinaire de Marly, ç'avait été moins lui qu'il avait eu à ramener que
M\textsuperscript{me} de Maintenon qu'il avait eue à combattre,
jusque-là qu'il en avait eu des prises avec elle et même fortes, sans
l'avoir jamais pu faire revenir sur moi, ni tirer d'elle contre moi que
des lieux communs et des choses générales, tellement qu'il avait eu par
là toutes les peines du monde et fort longtemps à travailler du côté du
roi, et à l'emporter enfin et de mauvaise grâce par complaisance pour
lui, parce que M\textsuperscript{me} de Maintenon fut toujours et
constamment contraire. Chamillart n'avait pas voulu me révéler ce secret
par fidélité et par modestie, peut-être aussi pour ne me jeter pas dans
une peine et dans un embarras où il ne voyait point de remède, et me
l'avoua enfin quand il n'y eut plus rien de tout cela à ménager. Cette
tardive découverte, lorsqu'elle ne pouvait plus servir à rien, me fit
voir que mes soupçons ne m'avaient pas trompé, encore qu'ils n'allassent
pas jusqu'à ce que j'appris alors.

Je m'étais douté que M. du Maine, à bout enfin de ses incroyables
avances envers moi, qu'on a vues (t. VIII, p.~156) et outré de n'avoir
pu parvenir à me lier, non pas même à m'apprivoiser avec lui, m'avait
secrètement regardé comme son ennemi et dangereux pour son rang, que
j'avais jugé être l'objet de ses infatigables et incompréhensibles
recherches et de celles de M\textsuperscript{me} la duchesse du Maine\,;
et que dans ce sentiment il avait inspiré à M\textsuperscript{me} de
Maintenon cet éloignement que je sentais, et que Chamillart m'apprit
enfin être une véritable haine. Je n'avais personne auprès d'elle, je
n'avais jamais songé à m'approcher d'elle\,; rien de si difficile que
son accès, nulle occasion ne m'en était née, et pour ne rien retenir, je
ne m'en souciai jamais, parce que ce qu'elle était et force choses
qu'elle faisait me donnaient pour elle un extrême éloignement. Mon
intime liaison avec les ducs de Chevreuse et de Beauvilliers d'une part,
avec M. le duc d'Orléans de l'autre, avec le chancelier encore, ne fit
dans la suite qu'augmenter pour moi les mauvaises dispositions de cette
étrange fée\,; et sûrement ses mauvais offices, auxquels je ne comprends
pas comment j'ai pu échapper, et à ceux de Nyert, de Bloin, et des
valets principaux, tous à M. du Maine, et sur lesquels j'étais averti et
défendu souvent par Maréchal. Je ne puis donc comprendre encore d'où
m'est venue, et moins encore comment a pu subsister constamment la
considération, même personnelle, que le roi m'a toujours montrée, depuis
l'audience que Maréchal m'en procura (t. VII, p.~443) jusqu'à sa mort,
ni comment il a tenu à un intérieur si intime et qui m'était si
contraire, et dans les crises qu'on a vues depuis cette audience, et
dans celles qu'on verra dans la suite. Quelquefois il se piquait de
caprice et de certaines choses contre M\textsuperscript{me} de
Maintenon. M. du Maine, timide et réservé, laissait à elle et aux valets
à me nuire. Je n'ai jamais su qu'il m'eût desservi auprès du roi
expressément et à découvert. Il n'allait jamais qu'entre deux terres, et
on verra qu'il me ménagea toujours personnellement en tout ce qui put me
marquer son extrême envie de me raccrocher, et sa patience sans mesure à
ne se lasser point de son peu de succès avec moi.

Parmi tant de choses générales et particulières qui m'occupaient, je ne
l'étais pas peu d'unir bien M. le duc d'Orléans avec le Dauphin, et pour
cela de le lier avec le duc de Beauvilliers. Tout m'y secondait, excepté
lui et M\textsuperscript{me} sa fille, ce qui est étrange à concevoir,
d'autant plus que ce prince en sentait la convenance et le besoin, et
qu'il le désirait. L'obligation si prodigieuse de ce grand mariage qu'il
avait fait, la liaison qui s'en maintenait entière entre la Dauphine et
la duchesse d'Orléans, celle qui subsistait en leur manière entre M. le
duc d'Orléans et le duc de Chevreuse, la partialité publique et non
interrompue de ce prince pour l'archevêque de Cambrai, et le coin des
jésuites qu'il avait toujours utilement ménagé, tout cela était de
grandes avances vers le but que je me proposais. Leur contredit n'était
guère moindre. Les mœurs de M. le duc d'Orléans, l'affection de se parer
de ses débauches et d'impiété, des indiscrétions là-dessus les plus
déplacées, faisaient fuir le Dauphin et rebroussaient infiniment son
ancien gouverneur.

Il était d'ailleurs en brassière du côté du roi, à qui la conduite de
son neveu était par plus d'un endroit odieuse, et cet autre endroit va
être expliqué, et la brassière était redoublée par la haine de
M\textsuperscript{me} de Maintenon pour M. le duc d'Orléans, que le
mariage de sa fille n'avait point émoussée, dans le temps même qu'elle
le faisait.

Ce mariage, qui aurait dû être un centre de réunion, était devenu entre
eux tous un flambeau de discorde. On a vu ici (p.~149 ci-dessus)
quelques traits du caractère terrible de M\textsuperscript{me} la
duchesse de Berry, dont la galanterie étrangement menée, et plus
singulièrement étendue, n'était pas à beaucoup près le plus mauvais côté
en comparaison des autres. On a vu son ingratitude et la folie de ses
desseins. L'élévation de son beau-frère et de sa belle-sœur, à qui elle
devait tout, n'avait fait qu'exciter sa jalousie, son dépit, sa rage\,;
et le besoin qu'elle avait d'eux portait les élans de ces passions à
l'excès. Nourrie dans l'aversion de M\textsuperscript{me} la duchesse
d'Orléans et dans l'indignation du vice de sa naissance, elle ne s'en
contraignit plus dès qu'elle fut mariée. Quoiqu'elle dût ce qu'elle
était devenue à sa mère et à la naissance de sa mère, quoiqu'elle en eût
sans cesse reçu toute sorte d'amitié et nulle contrainte, cette haine et
ce mépris pour elle éclatait à tous moments par les scènes les plus
scandaleuses, que la mère étouffait encore tant qu'elle pouvait, et qui
ne laissèrent pas souvent d'attirer à la fille de justes et rudes
mercuriales du roi, et même de Madame, qui n'avait pourtant jamais pu
s'accoutumer à la naissance de sa belle-fille\,; et ces mercuriales, qui
contenaient pour un temps, augmentaient encore le dépit et la haine.
Outre un naturel hardi et violent, elle se sentait forte de son mari et
de son père.

M. le duc de Berry, né bon, doux, facile, en était extrêmement amoureux,
et, outre que l'amour l'aveuglait, il était effrayé de ses emportements.
M. le duc d'Orléans, comme on ne le verra que trop dans la suite, était
la faiblesse et la fausseté même. Il avait aimé cette fille dès sa
naissance préférablement à tous ses enfants, et il n'avait cessé de
l'aimer de plus en plus\,; il la craignait aussi\,; et elle, qui sentait
ce double ascendant qu'elle avait sur l'un et sur l'autre, en abusait
continuellement. M. le duc de Berry, droit et vrai, mais qui était fort
amoureux, et dont l'esprit et le bien-dire n'approchait pas de celui de
M\textsuperscript{me} la duchesse de Berry, se laissait aller souvent
contre ce qu'il pensait et voulait, et, s'il osait la contredire, il en
essuyait les plus terribles scènes. M. le duc d'Orléans, qui presque
toujours la désapprouvait, et presque toujours s'en expliquait
très-naturellement à M\textsuperscript{me} la duchesse d'Orléans et à
d'autres, même à M. le duc de Berry, ne tenait pas plus que lui devant
elle, et s'il pensait vouloir lui faire entendre raison, les injures ne
lui coûtaient rien\,: elle le traitait comme un nègre, tellement qu'il
ne songeait après qu'à l'apaiser et à obtenir son pardon, qu'elle lui
faisait bien acheter. Ainsi, pour l'ordinaire, il donnait raison à elle
et à M\textsuperscript{me} la duchesse d'Orléans sur les sujets de leurs
brouilleries, ou sur les choses que l'une faisait et que l'autre
improuvait, et c'était un cercle dont on ne pouvait le sortir. Il
passait beaucoup de temps par jour avec elle, surtout tête à tête dans
son cabinet.

On a vu (t. VIII, p.~278) que le monde s'était noirci de fort bonne
heure d'une amitié de père qui, sans les malheureuses circonstances de
cabales enragées, n'aurait jamais été ramassée de personne. La jalousie
d'un si grand mariage, que ces cabales n'avaient pu empêcher, se tourna
à tâcher de le rendre infructueux\,; et l'assiduité d'un père
malheureusement né désœuvré, et dont l'amitié naturelle et de tout temps
trouvait de l'amusement dans l'esprit et la conversation de sa fille,
donna beau jeu aux langues de Satan. Leur bruit fut porté jusqu'à M. le
duc de Berry, qui, de son côté, voulant jouir en liberté de la société
de M\textsuperscript{me} sa femme, s'importunait d'y avoir presque
toujours son beau-père en tiers, et s'en allait peu content. Ce bruit de
surcroît le frappa fort\,; cela nous revint à M\textsuperscript{me} de
Saint-Simon et à moi (ceci n'arriva qu'au retour de Fontainebleau, pour
ce que je vais raconter qui me regarde, mais je n'ai pas cru devoir y
revenir à deux fois). L'importance d'un éclat qui pouvait arriver entre
le gendre et le beau-père sur un fondement si faux mais si odieux, nous
parut devoir être détourné avec promptitude.

J'avais déjà tâché de détourner M. le duc d'Orléans de cette grande
assiduité chez M\textsuperscript{me} sa fille, qui fatiguait M. le duc
de Berry, et je n'y avais pas réussi. Je crus donc devoir recharger plus
fortement encore\,; et voyant mon peu de succès, je lui fis une préface
convenable, et je lui dis après ce qui m'avait forcé à le presser
là-dessus. Il en fut étourdi, il s'écria sur l'horreur d'une imputation
si noire, et la scélératesse de l'avoir portée jusqu'à M. le duc de
Berry. Il me remercia du service de l'en avoir averti, qu'il n'y avait
guère que moi qui le lui pût rendre. Je le laissai en tirer la
conclusion que la chose présentait d'elle-même sur sa conduite. Cela se
passa entre lui et moi à Versailles, sur les quatre heures après midi.
Il n'y avait que M\textsuperscript{me} la duchesse d'Orléans, outre
M\textsuperscript{me} de Saint-Simon, qui sût ce que je devais faire, et
qui m'en avait extrêmement pressé.

Le lendemain M\textsuperscript{me} de Saint-Simon me conta que, rentrant
la veille du souper et du cabinet du roi chez M\textsuperscript{me} la
duchesse de Berry avec elle, elle avait passé tout droit dans sa
garde-robe, et l'y avait appelée\,; que là, d'un air colère et sec, elle
lui avait dit qu'elle était bien étonnée que je la voulusse brouiller
avec M. le duc d'Orléans\,; et que, sur la surprise que
M\textsuperscript{me} de Saint-Simon avait témoignée, elle lui avait dit
que rien n'était si vrai\,; que je voulais l'éloigner d'elle, mais que
je n'en viendrais pas à bout\,; et tout de suite lui conta ce que
j'avais dit à M. son père, qu'il avait eu la bonté de lui rendre une
heure après. M\textsuperscript{me} de Saint-Simon, encore plus surprise,
l'écouta attentivement jusques au bout, et lui répondit que cet horrible
bruit était public, qu'elle pouvait elle-même, tout faux et abominable
qu'il fût, se douter des conséquences qu'il pouvait avoir, sentir s'il
n'était pas important que M. le duc d'Orléans en fût averti, et que
j'avais rendu de telles preuves de mon attachement pour eux, et de mon
désir de leur union et de leur bonheur à tous, qu'il n'était pas
possible qu'elle pût avoir le moindre soupçon contraire, finit
brusquement par la révérence et sortir pour se venir coucher. Le trait
me parut énorme.

J'allai l'après-dinée le conter à M\textsuperscript{me} la duchesse
d'Orléans. J'ajoutai que, instruit par une si surprenante expérience,
j'aurais l'honneur désormais de voir M. le duc d'Orléans si rarement et
si sobrement, que j'en éviterais les risques les plus impossibles à
prévoir\,; et que, pour M\textsuperscript{me} la duchesse de Berry, je
me tiendrais pour dit, et pour toujours, la rare opinion qu'il lui
plaisait prendre de moi. M\textsuperscript{me} la duchesse d'Orléans fut
outrée. Elle se mit à dire de la chose tout ce qu'elle méritait, mais en
même temps à l'excuser sur la faiblesse du père pour sa fille, et à me
conjurer de n'abandonner point M. le duc d'Orléans, qui ne voyait que
moi d'honnête homme en état de lui parler franc et vrai. La cause de la
rupture lui fit peur. L'utilité journalière dont je lui étais auprès de
lui, et à lui-même, si je l'ose dire, depuis que je les avais
raccommodés, l'effraya encore d'en être privée. Elle ne me dissimula ni
l'un ni l'autre, et déploya toute son éloquence, qui n'était pas
médiocre, pour me persuader que l'amitié devait pardonner cette
légèreté, toute pesante qu'elle fût. J'abrégeai la visite, je ne me
pressai pas de la redoubler, et je cessai de voir M. le duc d'Orléans.
L'un et l'autre en furent bien en peine. Ils en parlèrent à
M\textsuperscript{me} de Saint-Simon. M\textsuperscript{me} la duchesse
de Berry que M. son père avait apparemment grondée, essaya de rhabiller
avec elle ce qu'elle lui avait dit, quoique d'assez mauvaise grâce.
M\textsuperscript{me} la duchesse d'Orléans m'envoya prier d'aller chez
elle. Elle s'y remit sur son bien-dire, M. le duc d'Orléans m'y vint
surprendre. Excuses, propos, tout ce qui se peut dire de plus touchant.
Je demeurai longtemps sur la glace du silence, puis du respect\,; à la
fin je me mis en colère, et m'en expliquai tout au plus librement avec
lui. Ce ton-là leur déplut moins que le premier\,; ils redoublèrent
d'excuses, de prières, de promesses de fidélité et de secret à l'avenir.
L'amitié, je n'oserais dire la compassion de sa faiblesse, me séduisit.
Je me laissai entraîner dans l'espérance que je mis dans la bonté de
cette leçon, et, pour le faire court, nous nous raccommodâmes, mais avec
résolution intérieure et ferme de le laisser vivre avec
M\textsuperscript{me} sa fille sans lui en jamais parler, et d'être
très-sobre avec lui sur tout ce qui la regarderait d'ailleurs.

Depuis que j'avais reconnu M\textsuperscript{me} la duchesse de Berry,
je la voyais fort rarement, et je m'étais défait de tout particulier
avec elle. Mais elle venait quelquefois me trouver dans ma chambre, sous
prétexte d'aller chez M\textsuperscript{me} de Saint-Simon, et m'y
tenait des heures tête à tête quand elle se trouvait dans l'embarras.
Depuis cette aventure je ne remis de longtemps le pied chez elle, et
ailleurs je lui battis si froid que je lui fis perdre l'habitude de me
venir chercher. Dans la suite, pour ne rien trop marquer, j'allais à sa
toilette publique une fois en deux mois et des moments chaque fois\,; et
tant qu'elle a vécu je ne m'en suis pas rapproché davantage, malgré
force agaceries directes et indirectes, qui ont souvent recommencé et
auxquelles j'ai constamment résisté. C'est une fois pour toutes ce qu'il
fallait expliquer de cet intérieur de famille royale, et du mien avec
eux tous. Revenons maintenant d'où je suis parti.

La lueur de raison et de religion qui parut en M. le duc d'Orléans,
après sa séparation d'avec M\textsuperscript{me} d'Argenton, n'avait pas
été de longue durée, quoique de bonne foi pendant quelque temps, et
peut-être allongée de politique jusqu'au mariage de
M\textsuperscript{me} la duchesse de Berry, qui suivit cette rupture de
cinq ou six mois. L'ennui, l'habitude, la mauvaise compagnie qu'il
voyait dans ses voyages de Paris, l'entraînèrent\,; il se rembarqua dans
la débauche et dans l'impiété, quoique sans nouvelle maîtresse en titre,
ni de brouilleries avec M\textsuperscript{me} la duchesse d'Orléans que
par celles de M\textsuperscript{me} la duchesse de Berry. C'était entre
le père et la fille à qui emporterait le plus ridiculement la pièce sur
les mœurs et sur la religion, et souvent devant M. le duc de Berry, qui
en avait beaucoup et qui trouvait ces propos fort étranges, et aussi
mauvais qu'il l'osait, les attaques qu'ils lui donnaient là-dessus et
qui ne réussirent jamais.

Le roi n'ignorait rien de la conduite de son neveu. Il avait été fort
choqué de son retour à la débauche et à ses compagnies de Paris. Son
assiduité chez M\textsuperscript{me} sa fille et son attachement pour
elle firent retomber sur lui des dégoûts continuels qu'il prenait
d'elle, et les déplaisirs souvent éclatants qu'elle donnait à sa mère,
laquelle il aimait en père et en protecteur, et pour l'amour de qui il
avait fait ce mariage, malgré toute la répugnance de Monseigneur. Le
manége de M. du Maine ne laissait rien passer ni refroidir. Il se
montrait peu à découvert, mais il faisait le bon personnage en plaignant
une sœur avec qui la haine de l'autre sœur l'avait étroitement réuni.
Les valets principaux le servaient bien\,; et il disposait d'autant plus
sûrement de M\textsuperscript{me} de Maintenon, qu'on a vu, et qu'on
verra encore mieux dans la suite, à quel point d'aveuglement elle
l'aimait, et combien elle haïssait M. le duc d'Orléans. M. du Maine
avait ses raisons. Il avait travaillé au mariage dans la crainte de
celui de M\textsuperscript{lle} de Bourbon\,; mais, le mariage fait, il
ne voulait pas dans l'intérieur du roi, aussi familier que le sien même
pour les heures libres et les entrées, qu'un prince aussi supérieur à
lui l'égalât dans l'amusement, approchât de lui en amitié, et le
diminuât par une considération à laquelle il n'était pas pour atteindre,
et pour être vis-à-vis de lui. Un autre grand intérêt le portait encore
à éloigner le roi de ce prince le plus qu'il lui serait possible. Un de
ses motifs pour le mariage de M\textsuperscript{me} la duchesse de Berry
était aussi celui d'une sœur de cette princesse avec le prince de
Dombes. Le principal obstacle en était levé par le rang entier de prince
du sang qu'il avait obtenu pour ses enfants. M\textsuperscript{me} la
duchesse d'Orléans, toute bâtarde et uniquement occupée de la grandeur
de ses frères et de ses neveux, le désirait passionnément. Elle s'était
servie de cette vue auprès de M. du Maine pour le faire agir en faveur
du mariage de M\textsuperscript{me} la duchesse de Berry\,; elle ne me
l'avait pas caché, mais toutefois sans m'en parler autrement que comme
d'un coup d'aiguillon à son frère, quoique je visse le fond de ses
désirs.

Je crois aussi que ce dessein entrait pour beaucoup dans l'inconcevable
constance des ménagements si recherchés de M. du Maine pour moi, parce
qu'il ne voyait d'obstacle que M. le duc d'Orléans, et que, comme on
présume toujours de son esprit, de son manége, et de la sottise de ceux
qu'on veut emporter, il ne désespérait peut-être pas de me gagner, et
par moi M. le duc d'Orléans, quelque intérêt de rang que j'eusse à
empêcher de consolider si bien celui de ses enfants. De toutes ces
choses résultait un mécontentement et un éloignement du roi pour M. le
duc d'Orléans, qui augmentait sans cesse, moins peut-être par sa
conduite personnelle que par celle de M\textsuperscript{me} la duchesse
de Berry. Le gros de tout cela n'était pas inconnu au duc de
Beauvilliers, qui l'éloignait encore de la liaison que je voulais former
entre M. le duc d'Orléans et lui. Je voyais le but de M. du Maine. Il
voulait plonger au plus bas M. le duc d'Orléans, pour ne lui laisser de
ressource auprès du roi que le mariage du prince de Bombes\,; et comme
il le connaissoit l'unique obstacle à ce dessein, et en même temps la
faiblesse même, il se dévouait à une route de laquelle il espérait un si
grand succès. Mais plus je voyais ce but et la justesse de cette noire
politique pour y arriver, plus je sentais l'extrême nécessité de
fortifier M. le duc d'Orléans d'une union avec M. de Beauvilliers, qui
opérerait celle du Dauphin avec lui, et qui, étant sincère, contiendrait
M. le duc d'Orléans sur beaucoup de choses, le rendrait considérable, et
à la longue briderait M\textsuperscript{me} la duchesse de Berry moins
supportée de M. son père, et émousserait les choses passées dans cet
intérieur de famille royale, et les disposerait tout autrement à
l'avenir, et dans le crédit que le Dauphin prenait de jour en jour,
surtout pensant comme il faisait sur les bâtards, je regardais cette
union comme un des plus grands renforts que la faiblesse de M. le duc
d'Orléans pût recevoir, et un obstacle dirimant au mariage qui aurait
fait le prince de Dombes beau-frère de M. le duc de Berry, qui par
lui-même n'aurait eu la force ni le crédit de l'empêcher, et beaucoup
moins M\textsuperscript{me} la duchesse de Berry d'en oser seulement
ouvrir la bouche, dans l'état où elle s'était mise avec le roi.

Pressé par ces vues, j'en exposai fortement au duc de Beauvilliers
l'importance, et combien il était nécessaire de ne se rebuter de rien
pour ne laisser pas échapper le fruit si principal qu'on s'était proposé
du mariage de M\textsuperscript{me} la duchesse de Berry, qui était
l'union de la famille royale\,; que plus on s'était trompé dans le
personnel de cette princesse, plus il se fallait roidir pour en
détourner et en corriger les inconvénients, dont le moyen unique était
celui que je lui proposais\,; que je le priais d'examiner s'il en
pouvait trouver un autre, et de comparer l'embarras de l'embrasser avec
le danger de le négliger. Je lui représentai l'ascendant que cette union
pouvait lui faire prendre sur la facilité, la faiblesse, j'ajoutai la
timidité de M. le duc d'Orléans, dont l'esprit et la conduite contenue,
et peu à peu guidée par son influence qui portait quand et soi celle du
Dauphin, et qui par là serait doublement comptée, pouvait prendre tout
un autre tour, et servir alors autant qu'elle nuisait maintenant à cette
union de famille si désirable\,; que tout faible et futile par oisiveté
qu'était à cette heure M. le duc d'Orléans, sa proximité si rapprochée
par l'alliance en faisait toujours un prince qui ne pouvait être dans
l'indifférence, et bien moins encore à l'avenir que pendant la vie du
roi qui retenait tout dans le tremblement devant lui. Qu'outre cette
raison, il ne me pouvait nier celle d'un esprit supérieur en tout genre,
et capable d'atteindre à tout ce qu'il voudrait sitôt qu'il en voudrait
faire usage\,; que ses campagnes avaient manifesté cette vérité, qui se
développerait bien davantage lorsque, délivré du joug du roi, le dégoût
d'une vie ennuyée du néant et de l'inutile à laquelle il était
maintenant réduit, et l'aiguillon de l'humeur et de l'esprit ambitieux
et imaginaire de M\textsuperscript{me} sa fille, lui donneraient envie
de se faire compter sous un nouveau règne, et si alors on ne se
repentirait pas de n'avoir pas, quand on l'avait pu, mis pour soi, et
pour une union si nécessaire, ce qu'on y trouverait alors si opposé, et
toujours, en ce cas, plus ou moins embarrassant. J'assaisonnai la force
de ces considérations de celle de l'opinion qu'il savait que M. de
Chevreuse avait foncièrement de ce prince, qu'il voyait toujours de fois
à autre en particulier de tout temps\,; et je me gardai bien d'omettre
ce qu'il ne pouvait ignorer, que M. le duc d'Orléans avait toujours
pensé, et tout haut, sur M de Cambrai. Enfin, je n'oubliai pas de lui
faire entendre que les faits historiques, les arts, les sciences, dont
le Dauphin aimait à s'entretenir, étaient une matière toujours prête et
jamais épuisée où M. le duc d'Orléans était maître, dont il savait
parler nettement et fort agréablement, et qui serait entre eux un
amusement sérieux qui leur plairait beaucoup à l'un et à l'autre, et qui
ne servirait pas peu au dessein si raisonnable que nous nous proposions.
Tant de raisons ébranlèrent le duc de Beauvilliers qui s'était ému dès
les premiers mots, mais qui à ma prière m'avait laissé tout dire sans
interruption. Il convint de tout, mais en même temps il m'opposa les
mœurs et les propos étranges qui lui échappaient quelquefois devant le
Dauphin, et qui l'aliénaient infiniment\,; et me montra sans peine que
cette indiscrétion était un obstacle qui mettait la plus forte barrière
à leur liaison. Je le sentais trop pour en pouvoir disconvenir, mais je
le pressai en ôtant cet obstacle, et je vis un homme intérieurement
rendu à cette condition. Alors je m'arrêtai, parce que je sentis que
tout dépendait de cela, qu'il s'agissait, par conséquent, d'y travailler
avant toutes choses, et que, connaissant la légèreté de M. le duc
d'Orléans, et ce détestable héroïsme d'impiété, qu'il affectait bien
plus encore qu'il n'en avait le fond, je ne pouvais me répondre de
réussir.

Je ne différai pas à l'attaquer, et je n'eus aucune peine à le faire
sincèrement convenir de tous les solides avantages qu'il trouverait,
outre la considération présente, de son union avec le Dauphin, et ce qui
était inséparable avec le duc de Beauvilliers. De l'aveu, je le
conduisis aisément au désir, que je crus devoir aiguiser par la
difficulté que lui-même sentait bien résulter de ses mœurs et de sa
conduite. Je le ballottai longtemps exprès là-dessus dans la même
conversation. Quand je crus l'avoir assez échauffé et assez embarrassé
pour pouvoir espérer le faire venir à mon point en lui proposant la
solution que j'avais projetée, je lui dis que je m'abstenais de
l'exhorter sur ses mœurs et sur ses opinions prétendues, qu'il ne
pouvait avoir foncièrement, et sur lesquelles il se trompait soi-même\,;
qu'il savait de reste ce que je pensais sur tout cela, et que je
n'ignorais plus aussi combien vainement je le presserais d'en changer\,;
qu'aussi était-ce à moins de frais que je croyais qu'il pourrait réussir
à l'union qu'il avait de si pressantes raisons de désirer\,; que le
moyen en était entre ses mains et facile, mais que, s'il se résolvait à
le prendre, il ne fallait pas s'en lasser\,; et qu'en ce cas, je croyais
qu'il ne tarderait pas à en voir des succès qui, suivis et entretenus
avec attention, le pourraient conduire à tout ce qu'il en pouvait
souhaiter. Je l'avais ainsi excité de plus en plus, en le laissant au
large sur le malheureux fond de sa vie\,; je lui fis dans la même vue
acheter l'explication de ce chemin et du moyen facile que je lui
proposais sans le lui dire. Enfin après lui avoir doucement reproché que
je ne l'en croyais pas capable, je me laissai vaincre, et je lui dis que
tout consistait en deux points\,: le premier d'être en garde continuelle
de tout propos le moins du monde licencieux en présence du Dauphin, et
chez M\textsuperscript{me} la princesse de Conti où le Dauphin allait
quelquefois, et d'où de tels discours lui pourraient revenir\,; que son
indiscrétion là-dessus lui aliénait ce prince plus dangereusement et
plus loin beaucoup qu'il ne pouvait se l'imaginer, et que ce que je lui
disais là-dessus n'était pas opinion, mais science\,; que la discrétion
opposée lui plairait tant, qu'elle le ferait revenir peu à peu, en lui
ôtant l'occasion de l'horreur qu'il concevait de ces choses, et de celui
qui les produisait, par conséquent la crainte et les entraves où sa
présence le mettait, qui se changeraient en aise et en liberté quand
l'expérience lui aurait appris qu'il pouvait l'entendre sans scandale,
et se livrer sans scrupule à sa conversation, dont les arts, les
sciences et des choses historiques entretiendraient la matière entre
eux, et peu à peu en bannirait toute contrainte, et n'y laisserait que
de l'agrément. L'autre point était d'aller moins souvent à Paris, d'y
faire la débauche au moins à huis clos, puisqu'il était assez malheureux
que de la vouloir faire, et d'imposer assez à lui-même, et à ceux qui la
faisaient avec lui, pour qu'il n'en fût pas question le lendemain matin.

Il goûta un expédient qui n'attaquait point ses plaisirs\,; il me promit
de le suivre. Il y fut fidèle, surtout pour les propos en présence du
Dauphin, ou qui lui pouvaient revenir. Je rendis ce que j'avais fait au
duc de Beauvilliers. Le Dauphin s'aperçut bientôt de ce changement, et
le dit au duc, par qui il me revint. Peu à peu ils se rapprochèrent\,;
et comme M. de Beauvilliers craignait toute nouveauté apparente, et
qu'il n'avait pas accoutumé de voir M. le duc d'Orléans, tout entre eux
passa par moi, et après ce Marly, où le duc de Chevreuse n'était point,
par lui et par moi, tantôt l'un tantôt l'autre.

\hypertarget{chapitre-xvii.}{%
\chapter{CHAPITRE XVII.}\label{chapitre-xvii.}}

1711

~

{\textsc{Mémoire des pertes de la dignité de duc et pair, etc.}}
{\textsc{- Tête-à-tête du Dauphin avec moi.}} {\textsc{- Affaire du
cardinal de Noailles remise par le roi au Dauphin.}} {\textsc{- Causes
de ce renvoi.}} {\textsc{- Discussion entre le duc de Beauvilliers et
moi sur un prélat à proposer au Dauphin pour travailler sous lui à
l'affaire du cardinal de Noailles.}} {\textsc{- Voyage de Fontainebleau
par Petit-Bourg.}} {\textsc{- Dureté du roi dans sa famille.}}
{\textsc{- Comte de Toulouse attaqué de la pierre.}} {\textsc{- Musique
du roi à la messe de la Dauphine.}} {\textsc{- Je raccommode sincèrement
et solidement le duc de Beauvilliers et le chancelier.}} {\textsc{-
Famille et mort du prince de Nassau, gouverneur de Frise.}} {\textsc{-
Mort de Penautier\,; quel il était.}} {\textsc{- Mort du duc de
Lesdiguières, qui éteint ce duché-pairie.}} {\textsc{- Neuf mille livres
de pension sur Lyon au duc de Villeroy.}} {\textsc{- Mort de Pelletier,
ci-devant ministre et contrôleur général.}} {\textsc{- Mort de
Phélypeaux, conseiller d'État, frère du chancelier.}} {\textsc{- Mort de
Serrant et du chevalier de Maulevrier\,; leur famille.}} {\textsc{- Mort
de la princesse de Fürstemberg\,; sa famille, son caractère.}}
{\textsc{- Maison de son mari.}} {\textsc{- Le tabouret lui est procuré
tard par adresse.}} {\textsc{- Mariage du chevalier de Luxembourg avec
M\textsuperscript{lle} d'Harlay.}} {\textsc{- Mort du cardinal de
Tournou.}} {\textsc{- Mort et caractère du maréchal de Boufflers.}}
{\textsc{- Danger que j'y cours.}} {\textsc{- Triste fin de vie.}}
{\textsc{- Horreur des médecins.}} {\textsc{- Générosité de la maréchale
de Boufflers, qui accepte à peine une pension du roi de douze mille
livres.}}

~

Parmi tous ces soins et ces affaires, il fallait travailler au mémoire
de nos pertes tel que le Dauphin me l'avait demandé. De tout temps je
les avais rassemblées avec les occasions qui les avaient causées autant
que j'avais pu. J'avais eu cette curiosité dès ma première jeunesse\,;
je l'avais toujours suivie depuis\,; je m'étais continuellement appliqué
à m'en instruire des vieux ducs et duchesses les plus de la cour en leur
temps et les mieux informés\,; à constater par d'autres ce que j'en
apprenais, et surtout à m'en donner à moi-même la dernière certitude par
des gens non titrés, anciens, instruits, versés dans les usages de la
cour et du monde, qui y avaient été beaucoup, qui avaient vu par
eux-mêmes, et par d'anciens valets principaux. Je mettais les uns et les
autres sur les voies, et par conversation je les enfilais doucement à
raconter ce que je m'étais proposé de tirer d'eux. J'avais écrit à
mesure\,; ainsi j'avais tous mes matériaux, où j'avais ajouté à mesure
aussi les pertes depuis mon temps, et dont j'avais été témoin avec toute
la cour. Sans une telle avance, le recueil m'eût été impossible, et les
recherches m'en auraient mené trop loin. Mais l'arrangement tel que le
Dauphin le voulut fut encore un travail long et pénible. Je n'y pouvais
être aidé de personne. M. de Chevreuse encore une fois n'était point à
Marly. M. de Beauvilliers était trop occupé\,; je n'osai même me servir
de secrétaire\,; néanmoins j'en vins à bout vers la fin du voyage. M. de
Beauvilliers ne put repasser ce travail que superficiellement. M. de
Chevreuse à qui je l'envoyai, l'examina à fond. J'allai le trouver après
à Dampierre, de Marly, où je couchai une nuit. Il m'en parut content et
n'y corrigea rien. J'y fis une courte préface adressée au Dauphin. Tout
cet ouvrage se trouvera avec les Pièces. Il s'en peut faire, depuis
qu'il fut achevé, un étrange supplément.

J'ajoutai un mémoire qui eût pu être bien meilleur s'il n'eût pas été
fait si rapidement, mais que je crus devoir présenter au Dauphin dans
tout son naturel, en lui en expliquant l'occasion. Ce fut lors de la
sortie du cardinal de Bouillon du royaume, et de son impudente lettre au
roi, que le maréchal de Boufflers me le demanda sur les maisons de
Lorraine, de Bouillon et de Rohan, et avec tant de précipitation que je
le fis en deux fois dans la même journée. Il croyait pouvoir en faire
usage dans un moment critique\,; il n'en fit aucun, c'est toujours le
sort de ce qui regarde la dignité. J'avertis le Dauphin que l'état des
changements arrivés à notre dignité pendant ce règne était prêt à lui
être présenté. J'y avais joint, en faveur de la haute noblesse, la
lettre que le roi écrivit à ses ambassadeurs et autres ministres dans
les cours étrangères, du 19 décembre 1670, sur la rupture du mariage de
Mademoiselle avec M. de Lauzun, parce que mon dessein, comme on l'a pu
déjà voir, n'était pas moins de la relever que les chutes de notre
dignité.

Quelque occupé que fût le Dauphin de l'affaire qui enfanta depuis la
fameuse bulle \emph{Unigenitus} que le roi lui avait renvoyée en partie,
il me donna heure dans son cabinet. J'eus peine à cacher dans mes
poches, sans en laisser remarquer l'enflure, tout ce que j'avais à lui
porter. Il en serra plusieurs papiers parmi les siens les plus
importants, et les autres avec d'autres qui ne l'étaient pas moins, et
j'admirai cependant l'ordre net et correct dont il les tenait tous,
malgré les changements de lieu si ordinaires de la cour, qui n'était pas
une de ses moindres peines. Avant de les mettre sous la clef, il voulut
passer les yeux sur notre décadence, et fut épouvanté du nombre des
articles. Son étonnement augmenta bien davantage, lorsque je lui fis
entendre en peu de mots le contenu du dernier article, qui comprenait
une infinité de choses qui auraient pu faire autant d'autres articles,
mais que j'avais ramassés ensemble pour le fatiguer moins, et n'avoir
pas l'air d'un juste volume. Je lui lus la préface, et je lui expliquai
les sources d'où j'avais puisé ce qui a précédé mon temps. Il admira la
grandeur du travail, l'ordre et la commodité des deux différentes
tables\,; il me remercia de la peine que j'y avais prise, comme si je
n'y eusse pas été intéressé\,; il me répéta que, puisque je l'avais bien
voulu, il ne pouvait regretter la peine que m'avait donnée l'ordre
chronologique qu'il m'avait demandé, auquel j'avais si nettement suppléé
par l'arrangement des tables, que je ne lui dissimulai pas avoir été ce
qui m'avait le plus coûté. Je lui dis qu'avec un prince superficiel et
moins désireux d'approfondir et de savoir à fond, je me serais bien
gardé de présenter les deux ouvrages ensemble, de peur qu'il ne se
contentât des tables et de leurs extraits\,; mais que ce que j'avais
fait pour son soulagement et pour la satisfaction subite d'une première
curiosité, j'espérais qu'il ne deviendrait pas obstacle à la lecture des
articles entiers, où il trouverait encore toute autre chose que les
extraits ne pouvaient renfermer. Il me donna parole de lire le tout à
Fontainebleau d'un bout à l'autre, de le lire pour s'en meubler la tête,
et de m'en entretenir après. Il ajouta qu'il ne remettait cela à
Fontainebleau, où on allait bientôt, que parce qu'il était accablé,
outre le courant, d'une affaire que le roi lui avait renvoyée presque
tout entière, et qui l'occupait d'autant plus que la religion y était
intéressée.

Je ne jugeai pas à propos de prolonger une audience en laquelle je
n'avais rien à ajouter à la matière qui me la procurait, et où je ne le
voyais pas disposé à me parler d'autre chose. Comme il ne s'ouvrit pas
davantage sur l'affaire qui l'occupait tant, et en effet beaucoup trop,
je me contentai de le louer du temps qu'il y voulait bien donner, et de
lui représenter en gros combien il était désirable qu'elle finît
promptement, et combien dangereuses les passions et les altercations qui
l'allongeraient en l'obscurcissant. Il me répondit là-dessus avec son
humilité ordinaire sur lui-même, et avec bonté pour moi, sur quoi je me
retirai. J'allai aussitôt après rendre compte de cette courte audience
au duc de Beauvilliers\,; il fut ravi de la manière dont elle s'était
passée\,; mais, ainsi que le Dauphin, il était tout absorbé de l'affaire
dont ce prince me venait de légèrement parler.

On entend bien que c'était celle du cardinal de Noailles qui enfanta
depuis la fameuse constitution \emph{Unigenitus}, sur laquelle on se
souviendra ici de ce qui a été ci-devant dit et expliqué (p.~84 et suiv.
de ce volume). Les noirs inventeurs de cette profonde trame, contents au
dernier point de l'avoir si bien conduite, et réduit le cardinal de
Noailles à une défensive de laquelle même ils lui faisaient un crime
auprès du roi, ne laissaient pas d'être en peine d'avoir vu ce cardinal
revenir à la cour, et y avoir une audience du roi passablement
favorable, après en avoir obtenu une défense de s'y présenter, qui fut
ainsi de courte durée. Le roi, tiraillé par les prestiges de son
confesseur appuyés du côté de M\textsuperscript{me} de Maintenon par
ceux de l'évêque de Meaux, et l'ineptie irritée de La Chétardie, curé de
Saint-Sulpice, ne résistait qu'à peine pour son ancien goût pour le
cardinal de Noailles, et à l'estime qui allait jusqu'à la vénération
qu'il avait conçue pour lui. Ils s'aperçurent que, quelques progrès
qu'ils fissent, la présence du cardinal, ou les déconcertait, ou du
moins mettait le roi dans un malaise qui les tenait en échec. Le remède
qu'ils y trouvèrent fut de faire renvoyer l'affaire au Dauphin, puisque
le roi lui en renvoyait tant d'autres, qu'il se mêlait de toutes avec
autorité par la volonté et pour le soulagement du roi, et que tous les
ministres travaillaient chez ce prince. Le roi, fatigué de cette
affaire, prit aisément à cette ouverture. Il ordonna donc au Dauphin de
travailler à la finir, de lui en épargner les détails et de ne lui en
rendre compte qu'en gros, et seulement lorsqu'il serait nécessaire.

Rien n'accommodait mieux les ennemis du cardinal de Noailles. Il était
resté le seul en vie des trois prélats qui avaient lutté contre
l'archevêque de Cambrai lors de l'orage du quiétisme, et qu'il l'avaient
culbuté à la cour et fait condamner à Rome. Ce mot seul explique toute
la convenance de la remise de l'affaire présente au Dauphin, livré
absolument au duc de Beauvilliers, beaucoup aussi au duc de Chevreuse,
toujours également passionné pour son ancien précepteur, élevé dans tous
leurs principes sur la doctrine, et qu'ils espéraient bien rendre pareil
à eux sur Rome, et sur les immenses terreurs du jansénisme et des
jansénistes. Le Dauphin avait pourtant montré plus d'une fois en plein
conseil et avec éclat, sur des affaires très-principales que les
jésuites y avaient en leur nom, que la justice et ses lumières
prévalaient à toute affection, mais ils comptèrent gagner l'une et
l'autre en celle-ci avec les deux ducs si puissamment en croupe et si
unis au P. Tellier. Raisonnant peu de jours après avec le duc de
Beauvilliers, allant avec lui de Marly à Saint-Germain du renvoi de
cette affaire au Dauphin, nous convînmes aisément de la nécessité de lui
proposer un évêque pour y travailler sous lui et y exécuter ses ordres à
l'égard des parties, et nous agitâmes les prélats qui pouvaient y être
propres. Je lui nommai l'ancien évêque de Troyes. Plusieurs raisons me
firent penser à lui. C'était un homme d'esprit et de savoir, qui avait
de plus la science et le langage du monde auquel il était fort rompu\,;
il avait brillé dans toutes les assemblées du clergé, où il avait
souvent réuni les esprits\,; il s'était trouvé à la cour dans des
liaisons importantes et fort opposées, sans soupçon sur sa probité. Dans
les affaires de l'Église, il s'était maintenu bien avec tous et avec les
jésuites\,; il était neuf sur celle-ci, puisqu'il était démis et retiré
à Troyes depuis nombre d'années\,; enfin sa droiture et sa piété ne
pouvaient être suspectes à la vie toute pénitente qu'il avait choisie
très-volontairement, et dans laquelle il persévérait depuis si
longtemps. Toutes ces qualités jointes à un esprit poli, doux, facile,
liant, insinuant, qui était proprement le sien, me paraissaient fort
exprès pour remplir les vues de l'emploi dont il s'agissait. J'expliquai
ces raisons à M. de Beauvilliers, qui n'eut rien à m'opposer, sinon que
M. de Troyes était ami du cardinal de Noailles\,; et de cela je ne l'en
pus tirer, quoi que je lui pusse représenter. Je vins donc à un autre,
et lui parlai de Besons, archevêque de Bordeaux, liant aussi, fort
instruit, estimé, transféré d'Aire à Bordeaux par le P. de La Chaise,
enfin ami des jésuites, et qui ne pouvait être suspect.

Le duc ne rejeta pas la proposition, mais il me parla de Bissy, évêque
de Meaux, comme du plus propre à travailler sous le Dauphin. Celui-ci
n'avait pas encore levé le masque\,; il s'entretenait respectueusement
bien avec le cardinal de Noailles, tandis que, de concert en tout avec
le P. Tellier, il l'égorgeait en secret auprès de M\textsuperscript{me}
de Maintenon. Je m'élevai donc contre ce choix, et lui dis ce que je
savais de l'ambition et des menées de ce prélat à Rome, étant évêque de
Toul, des causes de son refus opiniâtre de l'archevêché de Bordeaux, qui
le dépaysait, et beaucoup d'autres choses que je ne répéterai pas et qui
se trouvent t. II, p.~88, et t. IV, p.~256, pour la plupart. Alors M. de
Beauvilliers m'avoua qu'il en avait déjà parlé au Dauphin, et, sur ce
que je m'écriai encore davantage, et que je lui reprochai ensuite plus
doucement une dissertation inutile, puisque le choix était fait, je
l'ébranlai et je vis jour à joindre le Bordeaux au Meaux, dans ce
travail sous le Dauphin. Il n'est pas temps maintenant d'en dire
davantage sur cette affaire. Le roi était à Marly depuis la mort de
Monseigneur, c'est-à-dire qu'il y était arrivé de Meudon la nuit du 14
au 15 d'avril, et il y avait été retenu, comme je l'ai remarqué, à cause
du mauvais air\,; que Versailles était plein de petites véroles, et par
la considération des princes ses petits-fils. Il fut trois mois pleins à
Marly, et il en partit le mercredi 15 juillet, après y avoir tenu
conseil et dîné, passa à Versailles, où il monta un moment dans son
appartement, et alla coucher à Petit-Bourg, chez d'Antin, et le
lendemain à Fontainebleau, où il demeura jusqu'au 14 septembre. Je
supprimerais cette bagatelle, arrivée à l'occasion de ce voyage, si elle
ne servait de plus en plus à caractériser le roi. M\textsuperscript{me}
la duchesse de Berry était grosse, pour la première fois, de près de
trois mois, fort incommodée et avait la fièvre assez forte. M. Fagon
trouva beaucoup d'inconvénient à ne lui pas faire différer le voyage de
quelques jours. Ni elle ni M. le duc d'Orléans n'osèrent en parler. M.
le duc de Berry en hasarda timidement un mot, et fut mal reçu.
M\textsuperscript{me} la duchesse d'Orléans, plus timide encore,
s'adressa à Madame et à M\textsuperscript{me} de Maintenon, qui, toutes
peu tendres qu'elles fussent pour M\textsuperscript{me} la duchesse de
Berry, trouvèrent si hasardeux de la faire partir que, appuyées de
Fagon, elles en parlèrent au roi. Ce fut inutilement. Elles ne se
rebutèrent pas, et cette dispute dura trois ou quatre jours. La fin en
fut que le roi se fâcha tout de bon, et que, par capitulation, le voyage
se fit en bateau au lieu du carrosse du roi.

Pour l'exécuter, ce fut une autre peine d'obtenir que
M\textsuperscript{me} la duchesse de Berry partirait de Marly le 13,
pour aller coucher au Palais-Royal, s'y reposer le 14, et s'embarquer le
15 pour arriver à Petit-Bourg, où le roi devait coucher ce jour-là, et
arriver comme lui le 16 à Fontainebleau, mais toujours par la rivière.
M. le duc de Berry eut permission d'aller avec M\textsuperscript{me} sa
femme\,; mais le roi lui défendit avec colère de sortir du Palais-Royal
pour aller nulle part, même l'Opéra à l'un et à l'autre, quoiqu'on y
allât du Palais-Royal sans sortir, et de plain-pied des appartements
dans les loges de M. le duc d'Orléans. Le 14, le roi, sous prétexte
d'envoyer savoir de leurs nouvelles, leur fit réitérer les mêmes
défenses, et à M. {[}le duc{]} et à M\textsuperscript{me} la duchesse
d'Orléans, à qui il les avait déjà faites à leur départ de Marly. Il les
poussa jusqu'à les faire à M\textsuperscript{me} de Saint-Simon pour ce
qui regardait M\textsuperscript{me} la duchesse de Berry, et lui
enjoignit de ne la pas perdre de vue, ce qui lui fut encore réitéré à
Paris de sa part. On peut juger que ses ordres furent ponctuellement
exécutés. M\textsuperscript{me} de Saint-Simon ne put se défendre de
demeurer et de coucher au Palais-Royal, où on lui donna l'appartement de
la reine mère. Il y eut grand jeu tant qu'ils y furent pour consoler M.
le duc de Berry de sa prison.

Le prévôt des marchands avait reçu ordre de faire préparer des bateaux
pour le voyage\,; il eut si peu de temps qu'ils furent mal choisis.
M\textsuperscript{me} la duchesse de Berry s'embarqua le 15, et arriva
avec la fièvre, à dix heures du soir, à Petit-Bourg, où le roi parut
épanoui d'une obéissance si exacte.

Le lendemain, M\textsuperscript{me} la Dauphine la vit embarquer. Le
pont de Melun pensa être funeste\,; le bateau de M\textsuperscript{me}
la duchesse de Berry heurta, pensa tourner, et s'ouvrit à grand bruit,
en sorte qu'ils furent en très-grand danger. Ils en furent quittes pour
la peur et pour du retardement\,; ils débarquèrent en grand désordre à
Valvin, où leurs équipages les attendaient, et ils arrivèrent à
Fontainebleau à deux heures après minuit. Le roi, content au possible,
l'alla voir le lendemain matin, dans le bel appartement de la reine mère
que le feu roi et la reine d'Angleterre, et après eux Monseigneur,
avaient toujours occupé. M\textsuperscript{me} la duchesse de Berry, à
qui on avait fait garder le lit depuis son arrivée, se blessa et
accoucha, sur les six heures du matin du mardi 21 juillet, d'une fille.
M\textsuperscript{me} de Saint-Simon l'alla dire au roi à son premier
réveil, avant que les grandes entrées fussent appelées\,; il n'en parut
pas fort ému, et il avait été obéi. La duchesse de Beauvilliers
accompagnée de la marquise de Châtillon nommée par le roi, l'une comme
duchesse, l'autre comme dame de qualité, eurent la corvée de porter
l'embryon à Saint-Denis. Comme ce n'était qu'une fille on s'en consola,
et que la couche n'eut point de mauvaises suites. M. le comte de
Toulouse, attaqué de grandes douleurs de vessie depuis deux mois à
Marly, n'y voyait sur les fins presque plus personne. Le roi l'alla voir
plus d'une fois, mais il voulut aussi qu'il allât à Fontainebleau en
même temps que lui. Quoiqu'il ne pût souffrir de voiture, et encore
moins monter à cheval, il en fit le voyage en bateau, et ne put presque
sortir de sa chambre pour aller seulement chez le roi, très-rarement,
tant qu'on fut à Fontainebleau. C'est ainsi que rien ne pouvait
dispenser des voyages, et que le roi faisait éprouver aux siens qu'il
était au-dessus de tout. Il fit en arrivant la galanterie à la Dauphine
d'envoyer à sa messe toute sa musique, comme elle était auparavant à
celle de Monseigneur. Le Dauphin ne se soucia point de l'avoir à sa
messe, qu'il entendait d'ordinaire de bonne heure, et toujours dans un
recueillement qui ne se serait guère accommodé de musique, d'autant plus
qu'il l'aimait beaucoup. Ce fut une distinction que la Dauphine n'avait
point demandée\,; elle la toucha beaucoup et montra à la cour une grande
considération.

Dès que nous fûmes à Fontainebleau, je songeai de plus en plus comment
je pourrais réussir à une réconciliation sincère du duc de Beauvilliers
et du chancelier\,; je continuais à parler au premier du fils, sans
jamais lui nommer le père, et je lui faisais valoir sa conversion par la
soumission qu'il montrait entière à tout ce que je lui portais de sa
part. J'en vis le duc si satisfait, que je crus qu'il était temps de le
sonder tout à fait, pour m'assurer de voir rester le fils en place, dont
j'avais bien de grandes espérances, mais non encore la pleine certitude
que je désirais. Je l'exécutai dans une conférence, dans la galerie des
Cerfs\,; le duc en avait une clef, on y entrait du bas de son degré, et
c'était là d'ordinaire qu'il aimait à parler tête à tête en se promenant
sans crainte d'être interrompu. Après quelques propos sur Pontchartrain,
j'en tirai ce mot décisif, que, si Pontchartrain devenait praticable, il
opinait à le laisser en place puisqu'il y était, plutôt même qu'y en
mettre un autre meilleur que lui pour éviter un déplacement. Je
remerciai extrêmement M. de Beauvilliers, et je le confirmai de mon
mieux dans une résolution pour laquelle j'avais tant labouré. Sûr alors
que Pontchartrain avait échappé au danger, et qu'en continuant de se
conduire à l'égard du duc comme il faisait, et comme la frayeur
l'empêcherait d'y broncher il n'avait plus à craindre, et devait son
salut au duc de Beauvilliers, je crus que c'était le moment d'essayer de
frapper le grand coup que je méditais\,; mais je compris que si la
réconciliation était possible, ce ne serait qu'en la forçant, et, pour
ainsi dire, malgré l'un et l'autre.

Le duc était trop justement ulcéré, et sentait trop ses forces pour
vouloir ouïr parler du chancelier\,; et celui-ci trop outré de voir
toute la faveur et l'autorité, sur lesquelles il avait si
raisonnablement compté sous Monseigneur, passées par la mort de ce
prince au duc de Beauvilliers, et qu'il jouissait déjà d'avance d'une
grande partie, pour souffrir d'entendre parler de l'humiliation de se
courber devant cet homme qu'il s'était accoutumé à attaquer et à haïr,
et consentir à lui faire des avances.

Plein de mon idée, j'allai une après-dînée à la chancellerie, où il
logeait, à heure de l'y trouver seul et de n'être pas interrompu. Il
avait un petit jardin particulier le long de son appartement et de
plain-pied qu'il appelait sa Chartreuse, et qui y ressemblait en effet,
où il aimait à se promener seul, et souvent avec moi tête à tête. Dès
qu'il me vit entrer dans son cabinet, il me mena dans ce petit jardin,
affamé de causer depuis notre longue séparation de Marly, et qu'il ne
faisait que d'arriver à Fontainebleau où je ne l'avais vu qu'un soir ou
deux avec du monde. Là, après une conversation vague, assez courte, des
gens qui effleurent tout parce qu'ils ont beaucoup à se dire, je lui
demandai, à propos du travail des ministres chez le Dauphin, et de la
grandeur nouvelle du duc de Beauvilliers dont il était fort affecté,
s'il savait tout ce qui s'était passé à Marly, et si son fils lui en
avait rendu compte. Sur ce qu'il m'en dit, et qui n'avait nul trait à
son fait, je regardai le chancelier en lui demandant s'il ne lui avait
rien appris de plus particulier et de plus intéressant. Il m'assura que
non, avec curiosité de ce que je voulais dire. «\,Oh\,! bien donc,
monsieur, repris-je, apprenez donc ce que votre éloignement continuel de
Marly et votre passion pour Pontchartrain, d'où vous voudriez ne bouger,
vous fait ignorer, et à quoi peut-être cette conduite vous expose\,:
c'est que M. votre fils a été au moment d'être chassé. --- Hélas\,! me
répondit-il en haussant les épaules, à la conduite qu'il a, et aux
sottises qu'il fait tous les jours, c'est un malheur auquel je m'attends
à tous instants.\,» Puis se tournant vers moi d'un air fort agité\,:
«\,Mais contez-moi donc cela, ajouta-t-il, et à quoi il en est.\,» Je
lui dis le fait, et tout ce que je crus le plus capable de l'effrayer,
mais en prenant garde de lui rien montrer qui le pût faire douter le
moins du monde du duc de Beauvilliers, et le laissant au contraire dans
l'opinion de l'effet de leur haine et de son nouveau crédit, qu'il
exhala vivement à plus d'une reprise. Je le tins longtemps entre deux
fers, comme en effet son fils y avait été longtemps, et lui dans
l'impatience de la conclusion et de savoir où en était son fils, et je
fis exprès monter cette impatience jusqu'à la dernière frayeur. Alors je
lui dis qu'il était sauvé\,; que pour cette fois il n'avait plus rien à
craindre, et que j'avais même lieu de croire qu'il pouvait être soutenu
par qui l'avait sauvé. Voilà le chancelier qui respire, qui m'embrasse,
et qui me demande avec empressement qui peut être le généreux ami à qui
il doit le salut de sa fortune. Je ne me pressai point de répondre, pour
l'exciter davantage, et revins à l'extrême et imminent péril dont la
délivrance était presque incroyable. Le chancelier à pétiller, et à me
demander coup sur coup le nom de celui à qui il devait tout, et à qui il
voulait être sans mesure toute sa vie. Je le promenai encore sur l'excès
de l'obligation et sur les sentiments qui lui étaient dus par le
chancelier et par toute sa famille\,; et, comme il me demanda de nouveau
qui c'était donc, et si je ne le lui nommerais jamais, je le regardai
fixement et d'un air sévère, qui m'appartenait peu avec lui, mais que je
crus devoir usurper pour cette fois\,: «\,Que vous allez être étonné,
lui dis-je, de l'entendre ce nom que vous devez baiser\,; et que vous
allez être honteux\,! Cet homme que vous haïssez sans cause, que vous ne
cessez d'attaquer partout, M. de Beauvilliers enfin,\,» en haussant la
voix et lui lançant un regard de feu, «\,est celui à qui il n'a tenu, en
laissant faire, que votre fils n'ait été chassé, et qui l'a sauvé et
raffermi de plus dans sa place. Qu'en direz-vous, monsieur\,?»
ajoutai-je tout de suite. «\,Croyez-moi, allez vous cacher. --- Ce que
j'en dirai, répondit le chancelier d'une voix entrecoupée d'émotion,
c'est que je suis son serviteur pour jamais, et qu'il n'y a rien que je
ne fasse pour le lui témoigner\,;» puis, me regardant, et m'embrassant
avec un soupir\,: «\,C'est bien là votre ouvrage, je vous y reconnais\,;
eh\,! combien je le sens\,! mais cela est admirable à M. de Beauvilliers
au point où il est, et au point où nous sommes ensemble. Je vous conjure
de l'aller trouver, de lui dire que je me jette à ses pieds, que
j'embrasse ses genoux, que je suis à lui pour toute ma vie\,; mais
auparavant je vous conjure de me raconter tout ce détail dont vous ne
m'avez dit que le gros.\,»

Alors je n'en fis plus de difficulté\,: je lui fis le récit fort étendu
de ce que j'ai cru devoir resserrer ici sans plus ménager le secret que
M. de Beauvilliers m'avait imposé, et par moi ensuite à Pontchartrain,
lorsqu'il voulut après que je lui parlasse de sa part. Ce récit
très-exact, mais appuyé et circonstancié avec soin, jeta le chancelier
dans une honte, dans une confusion, dans un repentir, dans une
admiration, dans une reconnaissance dignes d'un homme de sa droiture et
de son esprit. Il redoubla les remercîments qu'il me fit d'un service si
signalé que j'avais rendu à lui et à son fils, et lorsque j'en étais si
mécontent, mais qu'il fallait qu'il s'en souvînt toute sa vie, et passât
partout par où je voudrais. Je répondis au chancelier qu'à mon égard ce
n'était là au sien que le payement de mes dettes, mais qu'il devait
porter toute sa gratitude vers le duc de Beauvilliers, qui n'ayant reçu
de lui qu'aigreurs et procédés fâcheux, et souvent même de son fils
encore, le sauvait néanmoins par pure générosité, par effort de
religion, sans y être obligé le moins du monde, n'ayant qu'à se taire
pour le laisser périr, et dans un temps encore où il fallait avouer
qu'il n'avait, et que, selon toute apparence humaine, il n'aurait jamais
aucun besoin de lui ni de son fils.

Le chancelier convint bien franchement qu'il n'aurait jamais pensé
trouver là son salut, se livra de même à toute la honte que je voulus
encore lui faire de ses préventions et de ses manières à l'égard de M.
Beauvilliers, ajouta de nouveau qu'il voulait être pour jamais à lui et
sans mesure, et qu'il lui tardait qu'il le sût par lui-même. Je le priai
de suspendre jusqu'à ce que j'eusse préparé le duc à la révélation de
son secret, et, ce que je ne lui dis pas, à vouloir bien recevoir son
hommage et se raccommoder avec lui. Il me conjura de n'y perdre pas un
moment de protester au duc qu'il était à lui sans réserve\,; qu'il le
suppliait de trouver bon que son opinion au conseil lui demeurât libre
en choses graves, mais qu'à cela près, il se rangerait toujours à son
avis toutes les fois que cela lui serait possible\,; qu'il n'y
manquerait jamais dans les choses qui ne seraient pas vraiment
importantes, et que si, dans celles qui le seraient, il ne pouvait pas
toujours se ranger à son avis, il dirait le sien tout uniment, sans
jamais contester ni disputer avec lui\,; qu'enfin, il verrait, par toute
sa conduite, combien exactement il remplirait ses engagements, et
combien en tout genre son dévouement et sa reconnaissance seraient
fidèles et entiers.

J'allai de ce pas chez le duc de Beauvilliers, à qui je racontai sans
détour toute la conversation que je venais d'avoir. Il rougit, et me
demanda avec quelque petite colère qui m'en avait prié. Je lui repartis
que c'était-moi-même\,; que je ne lui dissimulais pas que mon désir, et
enfin mon dessein, avait toujours été de le raccommoder avec le
chancelier, dont le péril troublait toute la joie de ma vie. Un peu de
courte, mais de vive paraphrase que j'ajoutai en même sens, calma le duc
jusqu'à me savoir bon gré, non de la chose, mais du sentiment qui me
l'avait fait faire. Je lui fis comprendre tout de suite assez aisément
que, bien loin qu'il y allât le moins du monde du sien dans la situation
où il se trouvait, une générosité si gratuite et si peu méritée lui
enchaînait le chancelier et son fils, par une obligation de nature à ne
pouvoir jamais s'en séparer, lui épargnait la peine d'achever de perdre
l'un, et de continuer nécessairement par travailler à la perte de
l'autre, que je ne regardais le fils que comme accessoire\,; mais qu'une
fois sincèrement réuni avec le père, j'étais persuadé qu'il y trouverait
des ressources qui le soulageraient en tous les temps, et qui
deviendraient fort utiles à l'État. Le duc, tout à fait radouci, me
chargea de compliments modestes pour le chancelier, et de lui dire qu'il
était bien aise de montrer à lui et à son fils combien ils s'étaient
mécomptés sur lui\,; que les engagements qu'il voulait prendre pour le
conseil étaient trop forts\,; qu'il était juste que tous deux y
conservassent leur liberté entière\,; que l'aigreur et la chaleur
étaient les seules choses à y retrancher\,; et qui l'assurait aussi
qu'il y serait toujours le plus qu'il le pourrait favorable à ce qu'il
jugerait qui lui pourrait être agréable.

Tout de suite j'exigeai du duc et aussitôt après du chancelier, que
mettant à part toute prévention réciproque sur les affaires concernant
Rome et la matière du jansénisme, ils en parleraient mesurément en
conseil, en y disant néanmoins tout ce qui ferait à l'affaire et à leur
sentiment, mais de façon à se marquer réciproquement leur considération
mutuelle, jusque dans ces choses qui les touchaient si fort tous deux et
d'une manière si opposée. J'en eus parole de tous les deux et de bonne
grâce, et tous deux l'ont toujours depuis tenue fort exactement. Je me
gardai bien de rendre au chancelier la manière dont j'avais été reçu
d'abord du duc de Beauvilliers\,; je lui dit tout le reste. Il petillait
de sceller lui-même cette grande réconciliation avec lui\,; mais le duc,
toujours et quelquefois trop plein de mesures, voulut un délai de dix ou
douze jours sans que j'en visse la raison. Je soupçonnai qu'ayant été
pris au dépourvu, et comme par force, il crut avoir besoin de ce temps
pour se dompter entièrement sur le chancelier, et ne rien faire de
mauvaise grâce. Le chancelier toutefois ne s'en douta point, mais son
impatience le porta à me prier de demander en grâce au duc de trouver
bon qu'au premier conseil il profitât de ce petit passage long et noir
qui avait d'un côté la chambre du premier valet de chambre en quartier,
et de l'autre une vaste armoire, et qui était l'unique entrée de
l'antichambre dans la chambre du roi, et que là, comme passant presque
ensemble, il le serrât, lui prît la main, et lui exprimât au moins par
ce langage muet ce qu'il n'avait pas encore la liberté de lui dire. Le
duc y consentit, et cela fut exécuté de la sorte.

Au bout de dix ou douze jours, M. de Beauvilliers me chargea d'avertir
le chancelier qu'il irait chez lui le lendemain après dîner avec le duc
de Chevreuse qui avait à lui parler, et, ce qui me surprit fort, de le
prier de ne lui rien témoigner devant ce tiers à qui toutefois il ne
cachait rien et qui était ami particulier du chancelier\,; il ne voulut
non plus que je m'y trouvasse. La visite ne se passa que civilement,
quoique avec plus d'onction qu'il n'y en avait eu jusque-là entre eux.
Quand elle fut finie, le duc de Beauvilliers pria le duc de Chevreuse de
le laisser seul avec le chancelier. Alors se firent les remercîments
d'une part, les embrassades et les protestations de toutes les deux
d'une amitié sincère. Le chancelier ne feignit point de s'avouer vaincu
de tous points, et l'obligé de toutes les sortes. Ils se remirent, pour
abréger, à tout ce que je leur avais dit de la part de l'un à l'autre\,;
ils convinrent que leur réconciliation demeurerait secrète pour éviter
les discours et les raisonnements\,; et ils se séparèrent extrêmement
contents l'un de l'autre. Le duc de Chevreuse attendait son beau-frère
avec qui il s'en alla, et le chancelier avait mis ordre à être trouvé
seul, et qu'il ne se trouvât personne chez lui pendant leur visite. Le
duc et le chancelier me rendirent tous deux ce qui s'y était passé, et
tous deux me prièrent que leur commerce continuât à passer par moi. Tous
deux aussi me rendirent longtemps comment les choses se passaient entre
eux en conseil.

Le chancelier et sa femme ne tarissaient point de remercîments avec moi.
Pontchartrain, souple par la nécessité dont je lui étais et par crainte
et par honte, ne me dit pas un mot de la capitainerie garde-côte de
Blaye, ni moi à lui. J'en admirai la ténacité, et j'avais beau jeu alors
de lui faire quitter prise, mais je n'en voulus pas faire la moindre
mention, ni leur laisser croire qu'un si petit objet eût pu entrer pour
rien dans le projet du pénible ouvrage que je venais d'exécuter. Son
succès me donna la joie la plus sensible et la plus pure\,; et j'ai eu
celle, que cette amitié de mes deux plus intimes amis a duré vraie,
fidèle, entière, sans lacune et sans ride tant qu'ils ont vécu.
M\textsuperscript{me} de Beauvilliers en fut enfin fort aise, et me le
témoigna, M. et M\textsuperscript{me} de Chevreuse beaucoup aussi, à qui
M. de Beauvilliers ne le cacha pas. Le monde ignora longtemps cette
réconciliation. Les manières si changées au conseil de ces deux
personnages ouvrirent enfin les yeux aux autres ministres, et lentement
après aux courtisans. L'érection nouvelle de Chaulnes, postérieure à
tout ceci de trois mois, fut prise quelque temps pour la cause du
raccommodement dont ils ne s'aperçurent que longtemps après\,; mais à la
fin, tout se sait en vieillissant, et on découvrit la véritable origine.
Je ne pus en faire un secret au premier écuyer, après ce qui s'était
passé entre lui et moi là-dessus. La réconciliation s'était consommée
dans les quinze premiers jours de Fontainebleau\,; son séjour
d'Armainvilliers lui en différa la joie jusque vers la fin du voyage.

Le prince de Nassau, gouverneur héréditaire des provinces de Frise et de
Groningue, se noya au passage du Mordick. La pluie le rendit paresseux
de sortir de son carrosse, et de passer dans un autre bâtiment que celui
où on l'embarqua. Les chevaux s'effrayèrent et causèrent tout le
désordre. Il n'y périt que deux ou trois personnes avec lui. Il avait
pris le nom de prince d'Orange depuis la mort du roi Guillaume qui
l'avait fait son héritier de tout ce qu'il avait pu. Le pensionnaire
Heinsius, tout puissant en Hollande, et la créature la plus affidée et
dévouée au roi Guillaume, le voulait faire stathouder de la république.
Il était bien fait, spirituel, appliqué, affable, aimé\,; il promettait
infiniment pour son âge\,; il avait épousé la sœur du landgrave
d'Hesse-Cassel, depuis roi de Suède. Il la laissa grosse d'un fils
unique, qui porte aussi le nom de prince d'Orange, qui a épousé une
fille du roi George II d'Angleterre, qui est bossu et fort vilain, mais
qui a beaucoup d'esprit et d'ambition, et qui n'oublie rien pour arriver
au stathoudérat de la république, dont néanmoins il paraît encore assez
éloigné\footnote{Le stathoudérat avait été supprimé à la mort de
  Guillaume III, en 1702\,; il ne fut rétabli qu'en faveur de Guillaume
  IV.}.

Penautier mourut fort vieux en Languedoc. De petit caissier, il était
devenu trésorier du clergé, et trésorier des états du Languedoc, et
prodigieusement riche. C'était un grand homme, très-bien fait, fort
galant et fort magnifique, respectueux et très-obligeant\,; il avait
beaucoup d'esprit et il était fort mêlé dans le monde\,; il le fut aussi
dans l'affaire de la Brinvilliers et des poisons, qui a fait tant de
bruit, et mis en prison avec grand danger de sa vie. Il est incroyable
combien de gens, et des plus considérables, se remuèrent pour lui, le
cardinal Bonzi à la tête, fort en faveur alors, qui le tirèrent
d'affaire. Il conserva longtemps depuis ses emplois et ses amis\,; et
quoique sa réputation eût fort souffert de son affaire, il demeura dans
le monde comme s'il n'en avait point eu. Il est sorti de ses bureaux
force financiers qui ont fait grande fortune. Celle de Crosat, son
caissier, est connue de tout le monde.

Le duc de Lesdiguières mourut à Paris à quatre-vingt-cinq ans sans
enfants, et en lui fut éteint ce duché-pairie. C'était un courtisan
imbécile, frère des duc et maréchal de Créqui, qui n'étaient rien moins.
J'en ai parlé sous le nom de Canaples, qu'il portait lors du voyage de
la maison de M\textsuperscript{me} la duchesse de Bourgogne au-devant
d'elle à Lyon, où il commandait, et à l'occasion de son mariage. Sa
femme, qui tenait beaucoup de l'esprit des Mortemart, eut la sottise de
le pleurer. On se moqua bien d'elle\,: «\,Que voulez-vous, dit-elle, je
le respectais comme mon père et je l'aimais comme mon fils.\,» On s'en
moqua encore davantage\,; elle n'osa plus pleurer. Elle avait passé sa
vie dans une grande contrainte avec M\textsuperscript{me} de
Montespan\,; ce mari la contraignait encore davantage\,; avec tout son
esprit, elle se trouva embarrassée de sa liberté. Il avait neuf mille
livres de la ville de Lyon, que le roi donna au duc de Villeroy.
Canaples, cousin germain des Villeroy, avait eu par eux le commandement
de Lyon après l'archevêque de Lyon, frère du vieux maréchal de Villeroy,
qui lui avait fait donner douze mille livres par la ville. Canaples les
eut en lui succédant. On l'ôta à force d'imbécillités. Le maréchal de
Villeroy fit mettre Rochebonne à sa place avec mille écus, et c'est les
neuf mille livres qui furent laissées à Canaples qu'eut le duc de
Villeroy.

M. Pelletier, qui avait été ministre et contrôleur général des finances,
mourut à Paris à plus de quatre-vingts ans. J'ai suffisamment parlé de
lui lors de sa belle retraite, qu'il soutint admirablement. Il avait une
grosse pension, voyait le roi quelquefois par les derrières, qui le
traitait toujours avec beaucoup d'estime et d'amitié, et dont il a
obtenu tout ce qu'il a voulu depuis sa retraite, et les établissements
les plus considérables dans la robe pour sa famille.

Le chancelier perdit aussi son frère, accablé d'apoplexies, qu'il aimait
fort, quoique ce ne fût pas un grand clerc, mais un fort honnête homme,
extrêmement riche par sa femme. Son frère l'avait fait intendant de
Paris, qu'il n'était plus, et conseiller d'État. Il laissa des enfants
que leur richesse ni leur parenté n'ont pu sauver de leur peu de mérite
et de la dernière obscurité.

Le vieux Serrant mourut aussi extrêmement vieux, dans sa belle maison de
Serrant en Anjou, où il était retiré depuis longues années. Il avait été
maître des requêtes et surintendant de Monsieur. Il était Bautru,
bourgeois de Tours, extrêmement riche, oncle et beau-père de Vaubrun,
grand-père de l'abbé de Vaubrun et de la duchesse d'Estrées. Son
petit-fils, le chevalier de Maulevrier, Colbert par son autre fille,
mourut en même temps de la petite vérole, fort aimé, estimé et regretté
à la guerre, où il s'était fort distingué, et était devenu maréchal de
camp fort jeune. Son père était frère de M. Colbert, mort étrangement,
chevalier de l'ordre, de douleur de n'être pas maréchal de France, qu'il
méritait. M. de Louvois, pour l'en empêcher, ne pouvant pis, lui fit
donner l'ordre en 1688.

En même temps mourut aussi la princesse de Fürstemberg. On a vu (t. II,
p. 400) qui était son mari, qui fut le dernier de sa maison, des
premiers et des plus anciens comtes de l'empire, et dont le père en
avait été fait prince, qui était frère de l'évêque de Strasbourg et du
cardinal de Fürstemberg. La princesse de Fürstemberg était fille unique
et fort riche de Ligny, maître des requêtes, et de la sœur de la vieille
Tambonneau, et de la mère du duc et du cardinal de Noailles. Elle avait
été extrêmement jolie, faite à peindre, et quoique boiteuse, dont elle
ne se cachait point, elle avait été une des meilleures danseuses de son
temps. C'était la meilleure et la plus aimable femme du monde, dont elle
était extrêmement, et d'une naïveté très-plaisante. Elle était amie
intime de la duchesse de Foix, et logeait et couchait à Versailles avec
elle. Un soir que M\textsuperscript{me} de Foix s'était amusée fort tard
à jouer chez M. le Grand, elle trouva la princesse de Fürstemberg
couchée, qui, d'une voix lamentable, lui dit qu'elle se mourait, et que
c'était tout de bon. M\textsuperscript{me} de Foix s'approche, lui
demande ce qu'elle a\,; l'autre dit qu'elle ne sait, mais que depuis
deux heures qu'elle est au lit, les artères lui battent, la tête lui
fend, et qu'elle a une sueur à tout percer, qu'enfin elle se trouve
très-mal et que le cœur lui manque. Voilà M\textsuperscript{me} de Foix
bien en peine, et qui de plus, n'ayant point d'autre lit, va par l'autre
ruelle pour se coucher au petit bord. En se fourrant doucement pour ne
pas incommoder son amie, elle se heurte contre du bois fort chaud\,;
elle s'écrie\,; une femme de chambre accourt avec une bougie\,; elle
trouve un moine dont on avait chauffé le lit, que la Fürstemberg n'avait
point senti, et qui, par sa chaleur, l'avait mise dans l'état où elle
était. M\textsuperscript{me} de Foix se moqua bien d'elle, et toute la
cour le lendemain.

Je ne sais comment un Allemand de la naissance de son mari l'avait
épousée. Il la planta là quelques années après, et s'en retourna en
Allemagne, où il devint le premier ministre de l'électeur de Saxe, et
gouverneur en plein de l'électorat quand ce prince fut en Pologne. Sa
femme n'avait jamais été assise, ni prétendu à l'être. Le cardinal de
Fürstemberg, fort en faveur, prétexta que son neveu la demandait. Elle
fit longtemps ses paquets et ses adieux\,: sur le point de partir, le
cardinal de Fürstemberg témoigna au roi sa douleur de la situation de
son neveu avec sa femme, qu'il n'avait osé mener en Allemagne, à cause
de la mésalliance\,; que ses occupations l'empêchaient de se mêler de
ses affaires domestiques\,; que sa maison s'éteignait\,; que ces raisons
le forçaient de la faire venir auprès de lui pour ne plus revenir en
France\,; que ce lui serait une grande consolation, et à son neveu un
grand moyen de bien faire recevoir sa femme, si, en partant d'ici, le
roi lui voulait faire la grâce de la faire asseoir à son souper\,; qu'il
ne le demandait qu'en prenant congé, et pour une fois unique. Le roi,
accoutumé à ne rien refuser à un homme qui l'avait si bien servi, et
tant et si dangereusement souffert pour lui, l'accorda à cette
condition. Elle s'assit donc, mais se garda bien de prendre congé. Le
voyage parut différé. Incontinent après, Monsieur, qui l'aimait fort,
excusa le délai, et représenta au roi en même temps que de ne pas
continuer ce tabouret jusqu'au départ était pis que de l'avoir refusé.
Le cardinal de Fürstemberg, de son côté, que sa nièce, après avoir eu
cet honneur, ne pouvait plus paraître à la cour sans qu'il lui fût
continué\,; et que si elle n'y venait plus, son mari la croirait
chassée, et que cela les brouillerait. Avec tout ce manége, le tabouret
lui demeura, le voyage s'éloigna, puis s'évanouit par insensible
transpiration. Elle demeura le reste de sa vie à Paris, et à la cour
assise. Elle n'eut point de garçons, ni sa fille aînée d'enfants du
prince d'Isenghien, qu'elle laissa bientôt veuf. Sa seconde fille avait
épousé Seignelay, comme on l'a vu en son temps, dont une fille unique
très-riche, qui a épousé le duc de Luxembourg, petit-fils du maréchal\,;
et sa troisième le comte de Lannoid, en Normandie.

Ce fut en ce même temps que le chevalier de Luxembourg, dernier fils du
maréchal, et maréchal de France lui-même vingt-trois ans depuis, épousa
la fille unique d'Harlay, conseiller d'État, fils unique du feu premier
président Harlay, qui était une riche héritière.

On eut en ce même temps à Rome et ici l'étrange nouvelle de la mort du
cardinal de Tournon, légat \emph{a latere} à la Chine et aux Indes. Elle
fit un prodigieux bruit par toute l'Europe. Sa mission, son succès, sa
sainte mais exécrable catastrophe, sont tellement connus et imprimés
partout, que je m'abstiendrai d'entrer dans cette énorme affaire, qui
aussi bien est tout à fait étrangère aux matières de ces Mémoires, si ce
n'est l'admirable cadence de ce martyr avec la naissance de l'affaire de
la bulle \emph{Unigenitus}. Le maréchal de Boufflers mourut à
Fontainebleau, à soixante-huit ans. Il est si souvent mention de lui
dans ces Mémoires qu'il n'en reste presque rien à dire. Rien de si
surprenant qu'avec aussi peu d'esprit, et un esprit aussi courtisan,
mais non jusqu'aux ministres, avec qui il se savait bien soutenir, il
ait conservé une probité sans la plus légère tache, une générosité aussi
parfaitement pure, une noblesse en tout du premier ordre, et une vertu
vraie et sincère, qui ont continuellement éclaté dans tout le cours de
sa conduite et de sa vie. Il fut exactement juste pour le mérite et les
actions des autres, sans acception ni distinction, et à ses propres
dépens\,; bon et adroit à excuser les fautes\,; hardi à saisir les
occasions de remettre en selle les gens les plus disgraciés. Il eut une
passion extrême pour l'État, son honneur, sa prospérité\,; il n'en eut
pas moins par admiration et par reconnaissance pour la gloire et pour la
personne du roi. Personne n'aima mieux sa famille et ses amis, et ne fut
plus exactement honnête homme, ni plus fidèle à tous ses devoirs. Les
gens d'honneur et les bons officiers lui étaient en singulière estime,
et avec une magnificence de roi, il sut être réglé autant qu'il le put
et singulièrement désintéressé\,; il fut sensible à l'estime, à
l'amitié, à la confiance. Discret et secret au dernier point, et d'une
rare modestie en tout temps, mais qui ne l'empêcha pas de se sentir aux
occasions rares qu'on a vues, et de se faire pesamment sentir aussi à
qui s'outrecuidait à son égard. Il tira tout de son amour du bien, de
l'excellente droiture de ses intentions, et d'un travail en tout genre
au-dessus des forces ordinaires, qui, nonobstant le peu d'étendue de ses
lumières, tira souvent de lui des mémoires, des projets et des lettres
d'affaires très-justes et très-sensées, dont il m'a montré plusieurs. Je
lui en communiquais aussi des miens, et il en avait un fort important
dans sa cassette, lorsque je fus averti de son extrémité, telle qu'il
mourut le lendemain. J'avais espéré jusque-là, et je n'avais pas voulu
lui montrer d'inquiétude. Je courus chez lui dans la frayeur du scellé
et de l'inventaire\,; je lui dis que j'espérais tout de l'état où je le
trouvais\,; mais que cette maladie étant grande, il serait longtemps
sans pouvoir s'appliquer à rien de sérieux, pendant quoi j'aurais besoin
de mon mémoire, qu'il me ferait plaisir de me rendre, et que je lui
redonnerais après quand il voudrait. Il ne fut point ému de ce discours,
appela sa femme, qui était arrivée la surveille, la pria d'aller
chercher sa cassette, l'ouvrit, y prit le papier et me le rendit.

J'ai déjà dit que le service si rare, et qui fut si heureux, qu'il
rendit à la bataille de Malplaquet, lui avait tourné la tête jusqu'à
oser demander l'épée de connétable, et sur le refus, la charge de
colonel général de l'infanterie, supprimée aussi, et encore plus
dangereuse. De celle-là, le refus encore plus sec l'outra\,; il oublia
ses récompenses, il ne vit que les refus, en contraste de tout ce qui
fut prodigué au maréchal de Villars pour prix de la même bataille, et
d'une campagne où tous les genres de mérites étaient de son côté, et de
celui de Villars tous les démérites possibles\,: cela le désespéra. Le
roi se dégoûta de lui comme d'un ambitieux qui était insatiable, et ne
s'en contraignit pas. Boufflers aimait le roi comme on aime un maître\,;
il le craignait, l'admirait, l'adorait presque comme un dieu. Il sentit
que l'impression était faite, et, bientôt après, qu'elle était sans
remède. Il en tomba dans un déplaisir cuisant, amer et sombre, qui lui
fit compter toute sa fortune pour rien, et qui peu à peu le jeta dans
des infirmités où les médecins ne purent rien comprendre. Je perdis mon
temps et mes efforts à le consoler\,; car il ne m'avait caché que ses
demandes avant de les faire, mais non leur triste succès. Il s'en
plaignait quelquefois à Monseigneur, qui le considérait, et qui
cherchait à le consoler\,; souvent à Mgr le duc de Bourgogne, et encore
depuis qu'il fut Dauphin, qui l'aimait et l'estimait, et qui l'alla voir
avec affection dans sa maladie. Il revenait d'un tour à Paris
lorsqu'elle le prit\,; quatre ou cinq jours le conduisirent aux portes
de la mort. Un empirique lui donna un remède qui le mit presque hors de
danger par la sueur, et qui défendit bien tout purgatif. Le lendemain
matin, la Faculté, bien étonnée de le trouver en si bon état, lui
persuada une médecine qui le tua dans la journée, avec des accidents qui
montrèrent bien que c'était un poison après le remède qu'il avait pris,
et qui ne fit pas honneur à ceux qui la lui donnèrent. Il fut
universellement regretté, et ses louanges retentirent dans toutes les
bouches, quoique sa considération fût tout à fait tombée. Le roi en
parla bien, mais peu, et se sentit extrêmement soulagé. On emporta chez
la duchesse de Guiche la maréchale de Boufflers, où le Dauphin et la
Dauphine allèrent la voir. Elle voulut s'en aller aussitôt après à
Paris, et ne permit point qu'on demandât rien pour elle, ce qu'elle
rejeta même avec indignation. Néanmoins leurs affaires étaient fort
embarrassées, et quelques jours après on la força d'accepter une pension
du roi de douze mille livres.

\hypertarget{chapitre-xviii.}{%
\chapter{CHAPITRE XVIII.}\label{chapitre-xviii.}}

1711

~

{\textsc{Charost capitaine des gardes du corps par le Dauphin.}}
{\textsc{- Domingue\,; quel, et son propos sur Charost à la Dauphine.}}
{\textsc{- Cause de la charge de Charost.}} {\textsc{- Fortune des trois
Charost.}} {\textsc{- Cause curieuse du mariage du vieux Charost.}}
{\textsc{- Cause du tabouret de grâce de la princesse d'Espinoy.}}
{\textsc{- Prince d'Espinoy chevalier de l'ordre parmi les gentilshommes
en 1661.}} {\textsc{- Pont d'or fait aux Charost pour leur ôter la
charge de capitaine des gardes, et sa cause.}} {\textsc{- Habileté
importante du vieux Charost.}} {\textsc{- Malice de Lauzun sur le duc de
Charost, et sa cause.}} {\textsc{- Raison qui fit renouveler des ducs
vérifiés sans pairie.}} {\textsc{- Repentir de Louis XIII de l'érection
de Paris en archevêché.}} {\textsc{- Cause qui fit Charost duc et
pair.}} {\textsc{- Raison qui priva Harlay, archevêque de Paris, du
cardinalat, et qui le fit duc et pair.}} {\textsc{- Importance des
entrées.}} {\textsc{- Ruses d'Harlay, archevêque de Paris, démontées par
Charost.}} {\textsc{- Dessein du duc de Beauvilliers et du Dauphin de me
faire gouverneur de Mgr le duc de Bretagne.}} {\textsc{- Fortune de
Charost du tout complète.}} {\textsc{- Campagne d'Allemagne.}}
{\textsc{- Campagne de Savoie.}} {\textsc{- Campagne de Flandre.}}
{\textsc{- Témérité du prince Eugène et de Marlborough.}} {\textsc{-
Fautes énormes de Villars.}} {\textsc{- Impudence de Villars, qui donne
faussement un démenti net et public au maréchal de Montesquiou, qui
l'avale.}} {\textsc{- Course de Contade à la cour\,; son caractère.}}
{\textsc{- Siège de Bouchain\,; Ravignan dedans\,; sa situation
personnelle\,; son caractère.}} {\textsc{- Bouchain rendu\,; la garnison
prisonnière\,; générosité des ennemis à l'égard de Ravignan.}}
{\textsc{- Fin de la campagne en Flandre.}} {\textsc{- Villars assez
bien reçu à la cour, et pourquoi.}}

~

La charge vacante eut plusieurs prétendants. Je hasardai de m'en mettre
par une lettre que je présentai au roi. Il me revint aussitôt qu'elle
lui avait plu assez pour me donner de l'espérance\,; mais M. de
Beauvilliers, sans qui je ne faisais rien d'important, et qui m'y avait
exhorté à tout hasard, me la diminua bientôt. Le maréchal était mort le
22 août. Le vendredi matin, 4 septembre, le roi travailla à l'ordinaire
avec le P. Tellier, puis envoya chercher le Dauphin. Il lui dit qu'en
l'âge où il était, ce n'était plus pour soi qu'il devait faire des choix
de gens qui ne le serviraient guère, mais qui serviraient le Dauphin
toute leur vie\,; qu'ainsi il voulait lui donner un capitaine des gardes
à son gré, et qu'il ordonnait de lui dire franchement à qui des
prétendants il donnait la préférence. Le Dauphin, après lui avoir fait
les réponses convenables, lui nomma le duc de Charost comme celui qui
lui était le plus agréable, et dans l'instant il l'obtint. Le roi passa
ensuite chez M\textsuperscript{me} de Maintenon\,; il y fit appeler
Charost, lui donna la charge avec cinq cent mille livres de brevet de
retenue pour en payer autant qu'en avait le maréchal de Boufflers, lui
dit qu'il devait cette préférence au Dauphin, à qui il avait laissé le
choix, et lui ordonna d'envoyer sur-le-champ cette nouvelle à son père,
à qui elle ferait grand plaisir.

Charost était lieutenant général, mais ne servait plus depuis longtemps.
Il n'était pas même sur un pied avec le roi a se faire craindre aux
prétendants de la charge\,; ce fut donc un étonnement extrême et un
bourdonnement étrange, et en même temps un événement qui imprima à toute
la cour un grand respect pour le Dauphin et une persuasion parfaite de
tout ce qu'il pouvait. Un nommé Domingue, portemanteau de la Dauphine et
fort familier avec elle, courut lui dire la nouvelle. Il osa ajouter
qu'il l'en félicitait avec toute la joie possible, parce qu'au moins M.
de Charost, fait capitaine des gardes, ne serait pas gouverneur de Mgr
le duc de Bretagne. On verra qu'il ne fut pas prophète\,; mais la
Dauphine en rit et y applaudit, et ce qui se trouva là de ses
familières, par qui je le sus. Ce Domingue était un garçon d'esprit et
orné, fort au-dessus de son état, et bien traité et avec distinction de
tout le monde. Il était venu tout enfant d'Espagne, avec son père, à la
suite de la reine, à qui il était, et lui aussi quand il fut plus grand,
puis à la dauphine de Bavière, enfin à celle-ci à son mariage. Elle
avait de la bonté pour lui, qui allait à une vraie confiance. Il lui
parlait pourtant en honnête homme, et très-franchement tête à tête, et
ne laissait pas de lui faire souvent impression. Il s'attacha tellement
à elle qu'il ne voulut point se marier pour ne se point partager, et
elle lui en savait gré\,; enfin, il fut tellement touché de sa mort
qu'il ne put se consoler. Il tomba dans des infirmités qui en moins d'un
an le conduisirent au tombeau sans être sorti presque de sa chambre, ni
avoir voulu voir personne que pour sa conscience.

N'ayant pas la charge, je fus ravi de la voir à un de mes plus intimes
amis. Lui et moi nous l'étions réciproquement souhaitée. Je ne vis
jamais homme si aise, et de la chose et de la manière. Le Dauphin, à
travers toute sa modeste retenue, parut extrêmement content, et la
Dauphine aussi, mais par concomitance\,: on a vu quel rang tenait la
duchesse de Béthune dans le petit troupeau de M. de Cambrai et parmi les
disciples de M\textsuperscript{me} Guyon, et quelle considération il en
revenait au duc de Charost, son fils, auprès du Dauphin par celle de M.
de Cambrai, et par les ducs de Chevreuse et de Beauvilliers, ce qui lui
valut la charge. Quoique cette fortune fût fort peu apparente, et aussi
peu espérée, on lui en verra faire une plus haute et encore moins
attendue de lui ni de personne. C'est ce qui m'engage à un peu de
digression sur la singulière et curieuse fortune de ces MM. de Charost.

Le comte de Charost, grand-père de celui-ci, était quatrième fils, mais
tenant lieu de second fils du frère du premier duc de Sully, ministre
favori d'Henri IV. Ce frère, qui était catholique, fut célèbre par ses
nombreuses et importantes ambassades, par les succès qu'il y eut et par
ses emplois considérables dans les armées\,; chevalier du Saint-Esprit
en 1609, et mort à quatre-vingt-quatre ans, en 1649. Charost, son cadet,
ne pouvait pas espérer grand bien de lui. Le fameux procès que le comte
de Soissons intenta au prince de Condé, duquel M. de Sully avait pris la
défense auprès d'Henri IV, qui le rendit partial, et dont le comte de
Soissons ne pardonna jamais le succès au favori, avait lié une amitié
intime entre ce dernier et L'Escalopier, qu'il avait fait nommer
rapporteur du procès, et qu'il en fit récompenser d'une charge de
président à mortier au parlement de Paris. L'Escalopier avait une fille
fort riche, dont M. de Sully, qui ne mourut qu'à la fin de décembre
1641, fit le mariage avec le comte de Charost, son neveu, en février
1639. Ce comte de Charost se trouva un homme de mérite qui se distingua
fort dans toutes les guerres de son temps, et qui y eut toujours des
emplois considérables. Il s'attacha au cardinal de Richelieu, jusqu'à
s'en faire créature\,; cette protection lui valut la charge de capitaine
des gardes du corps, dont se défit, en 1634, le comte de Charlus,
bisaïeul du duc de Lévi, et deux ans après, Calais.

Le cardinal Mazarin, qui se piqua d'aimer et d'avancer tout ce qui avait
particulièrement été attaché au cardinal de Richelieu, rechercha
l'amitié du comte de Charost, et le mit en grande considération auprès
de la reine mère, et ensuite auprès du roi, qui le regardèrent toujours
comme un homme de tête et de valeur, et d'une fidélité à toute épreuve.
Il se fit un principe de demeurer uni avec tout ce qui avait tenu au
cardinal de Richelieu, qu'il appelait toujours son maître, et dont il
avait force portraits, quoique sa mémoire ne fût pas agréable à la reine
mère. Il avait beaucoup dépensé, il aimait la faveur quoique fort homme
d'honneur. Il maria donc son fils, au commencement de 1657, à la fille
unique du premier lit de M. Fouquet qui était lors dans l'apogée du
ministère et de la faveur. La sienne à lui obtint un tabouret de grâce
en 1662, qui fit le mariage de sa fille avec le prince d'Espinoy qui n'y
songeait pas, et qui avait été avec lui de la promotion de l'ordre de
1661, sans aucune prétention parmi les gentilshommes, et qui n'en a
jamais eu jusqu'à sa mort. Celle du cardinal Mazarin, qui suivit de près
le mariage que Charost avait fait de son fils, la fut de bien plus près
de la disgrâce, ou plutôt de la perte de Fouquet que ce premier ministre
mourant avait conseillée.

Colbert, son intendant, qu'il avait recommandé comme un homme
très-capable, s'éleva bientôt sur les ruines du surintendant. Le Tellier
et lui, qui bien qu'ennemis étaient très-unis pour la perte de Fouquet
qu'ils avaient hâtée et approfondie, le furent toujours à la sceller de
toutes parts. Dans la frayeur de son retour, ils ne voyaient qu'avec la
dernière inquiétude le vif sentiment avec lequel le vieux Charost et son
fils avaient pris les malheurs de Fouquet, combien ils s'étaient peu
embarrassés de garder les moindres mesures dans leurs discours et dans
leurs mouvements en sa faveur. Le fils était capitaine des gardes en
survivance de son père, ils n'en avaient rien perdu de leur familiarité,
ni de leur considération auprès du roi et auprès de la reine, et l'un et
l'autre aimaient, estimaient et distinguaient le père comme un ancien
serviteur de toute épreuve, ce qui influait aussi sur le fils. Les deux
ministres ne purent se croire en sûreté à l'égard de Fouquet, ni sur
eux-mêmes, tant que ces deux hommes conserveraient une charge qui leur
donnait un accès si libre et si continuel. Le roi et la reine sa mère,
tiraillés de part et d'autre, se seraient trouvés soulagés de voir leur
charge en d'autres mains\,; mais trop sûrs de leur fidélité, et trop
accoutumés à une sorte de déférence pour le père, ils ne purent se
résoudre à les en dépouiller\,; Ce fut donc aux deux ministres à
recourir à la voie de la négociation, et ils eurent permission de leur
faire un pont d'or.

Charost, vieux routier de cour, sentit qu'à la longue il ne leur
résisterait pas, deviendrait à la fin à charge au roi et serait forcé de
faire avec dégoût, et pour ce qu'on voudrait bien lui donner, une chose
qu'il pouvait faire alors avec agrément en imposant la loi, et en
conservant et augmentant même sa considération et sa familiarité. Le
traité fut donc que M. de Duras lui rendrait le prix de sa charge, et
qu'il en serait pouvu\,; que M. de Charost aurait pour rien la
lieutenance générale unique de Picardie, Boulonais et pays reconquis,
avec le commandement en chef dans la province\,; que son fils, qui
quitterait sa survivance en faveur de M. de Duras, aurait celle de
ladite lieutenance générale, avec celle du gouvernement de Calais, et
que le père et le fils seraient en même temps faits ducs à brevet l'un
et l'autre\,; mais ce ne fut pas tout\,; le père voulut deux choses du
roi, auquel il s'adressa directement, et les obtint toutes les deux.
L'une fut un billet entièrement écrit et signé de la propre main du roi,
portant parole et promesse expresse de ne point faire de pair de France
pour quelque cause que ce pût être, sans faire Charost père ou fils, et
sans le faire avant tout autre, en sorte qu'il aurait le rang
d'ancienneté sur celui ou ceux que le roi voudrait faire. L'autre chose
fut un brevet d'affaires au père et un au fils, c'est-à-dire de moindres
entrées que celles des premiers gentilshommes de la chambre, et beaucoup
plus grandes que toutes les autres. Cette voie si rare et si précieuse
d'un accès continuel et familier n'était pas le compte des deux
ministres qui l'auraient bien empêché s'ils l'avaient pu, mais Charost
brusqua ce dernier point du roi à lui, comme le vin du marché, sans
lequel il ne pouvait le conclure de bon cœur, ni quitter une charge qui
l'approchait si fort de lui, et sans s'assurer pour soi et pour son fils
de s'en approcher encore davantage. Le billet fut un point capital et un
effort extrême de considération. C'est l'unique promesse que le roi ait
jamais donnée par écrit d'aucune grâce. On verra bientôt de quelle
importance furent les entrées et les promesses, et combien ce trait fut
celui d'un habile homme. Il mourut en 1681, à soixante-dix-sept ans, et
toujours en grande considération.

Il ne faut pas omettre que Calais et la lieutenance générale de Picardie
fut et est encore un morceau de quatre-vingt mille livres de rente,
outre le grand établissement. Charost son fils servit avec distinction,
et se maintint dans la familiarité du roi\,: ce ne fut pas sans une
légère éclipse. Il était à Calais lorsque la reine d'Angleterre y arriva
avec le prince de Galles. M. de Lauzun, qui les avait sauvés
d'Angleterre et conduits, s'était pris à Pignerol d'une aversion extrême
contre le malheureux Fouquet, qu'il y avait trouvé et laissé. Cette
haine s'étendit à sa famille, et il n'en est jamais revenu. Tout occupé
qu'il devait être de son retour à la faveur d'une fortune si unique et
si inimaginable, il ne le fut pas moins de nuire à Charost. Il rendit au
roi un compte si désavantageux en tout de Charost, de sa réception de la
reine d'Angleterre, de l'état de Calais et de la garde de la place, que
Charost eut le dégoût d'y voir arriver Laubanie en qualité de
commandant, le même qui s'acquit longtemps depuis tant de gloire à la
défense de Landau. Charost revint, et lui et Lauzun demeurèrent des
années sans se parler et longtemps sans se saluer.

Laubanie se conduisit en très-galant homme qu'il était à l'égard de
Charost, avec toutes sortes d'égards et de respects, et se fit un point
d'honneur de lui rendre justice, et de détruire les mauvaises
impressions que le roi avait prises. Il y réussit, et Charost revint
auprès du roi comme auparavant. Il avait vu faire en divers temps
plusieurs ducs vérifiés, M. de La Feuillade, M. de Chevreuse, M. de La
Rocheguyon, M. de Duras, le maréchal d'Humières\,: il s'en était plaint.
Le roi, qui ne les faisait point pairs pour éviter de faire Charost, lui
répondait toujours froidement qu'il avait tort de se plaindre, qu'il ne
faisait point de pairs, et Charost en effet n'avait point à répliquer,
mais il voyait que le roi se moquait de lui. À la fin la faveur
d'Harlay, archevêque de Paris, prévalut. Il était duc à brevet depuis le
mois d'avril 1674, et il petillait d'attacher la pairie à son siége. Ce
n'est pas d'aujourd'hui que les rois se laissent entraîner en des
fautes, même en les voyant. Le cardinal Gondi avait arraché le
consentement de Louis XIII à l'érection de son évêché de Paris en
archevêché. Rome, à son ordinaire, avait longtemps balancé, pour mieux
faire acheter une grâce qui lui coûtait si peu. Cependant on ouvrit les
yeux là-dessus à Louis XIII\,: il comprit qu'il n'avait pas intérêt à
augmenter l'autorité du siége de sa capitale, ni de ceux qui le
rempliraient, et il en fut si persuadé, qu'il fit dépêcher un courrier à
Rome pour rompre cette affaire\,: le courrier arriva le lendemain du
consistoire où l'érection avait passé\,; le cardinal Gondi fut
archevêque de Paris, d'évêque qu'il en était auparavant, et on se garda
bien de laisser découvrir que, vingt-quatre heures plus tard, Paris
n'eût jamais été métropole.

C'était ici le même inconvénient dans le genre séculier, et plus grand
encore en tant que ce siége avait déjà tout dans le genre
ecclésiastique. Son prélat, que le roi aimait, était duc à brevet\,;
c'était des honneurs pour sa personne, dont il se devait d'autant mieux
contenter, que ses successeurs ne lui étaient rien, et que leur dignité
ne décorait point sa famille. Le roi pouvait aussi se contenter de cette
distinction unique dans le clergé et personnelle qu'il lui avait donnée,
sans se soucier de ses successeurs, et craindre d'en augmenter
l'autorité, que le cardinal de Retz lui avait assez fait sentir, et de
rendre une septième pairie éternelle\,; néanmoins la faveur l'emporta,
et le roi résolut d'élever le siége de Paris à la pairie\,; en même
temps il ne voulait point faire Charost\,; il recommanda donc fort le
secret à l'archevêque de Paris, dans le dessein qu'il fût enregistré et
reçu en même moment, et que la grâce ne se sût que par là, quitte après
de se défaire comme il pourrait des clameurs de Charost.

L'archevêque eut beau mener son affaire le plus sourdement qu'il fut
possible, et le premier président et le procureur général l'y aider par
ordre du roi, les érections sont sujettes à quantité de formes\,;
Charost était au guet, il eut le vent de ce qu'il se préparait, il en
parla au roi qui biaisa, et se hâta de se défaire de lui. Charost par là
encore, plus certain de la chose, et qu'on lui voulait faire passer la
plume par le bec, ne se rebuta point\,; il attaqua le roi à la fin du
petit coucher, où le peu de ceux qui jouissaient de ces entrées avaient
toujours la considération réciproque de sortir tous, dès que l'un d'eux
se présentait à parler au roi comme il donnait le bonsoir, afin de le
laisser seul en liberté avec lui. Là le roi, prêt à se mettre au lit, ne
pouvait prétexter des affaires ni passer dans une autre pièce\,; il
fallait bien qu'il écoutât jusqu'au bout des gens en très-petit nombre,
la plupart en grande dignité, et distingués tous par leurs privances et
presque tous par leurs charges. Le roi, pris ainsi au trébuchet, se mit
à se promener par sa chambre avec Charost, qui, son billet à la main, le
somma de sa parole comme le plus honnête homme qui fût dans son royaume.
Le roi ne put disconvenir de l'engagement, mais il se tourna à exagérer
les services de l'archevêque dont la nature demandait d'autant plus une
récompense éclatante et immédiate de sa main, qu'ils étaient obstacles
invincibles à celle qu'il lui avait voulu donner par Rome, où les
propositions de l'assemblée du clergé de 1682 où il présidait étaient si
odieuses, que le pape, qui ne pouvait ne pas remplir la nomination qu'il
lui avait donnée pour la promotion des couronnes, s'opiniâtrait depuis
tant d'années à la différer toujours, et aimait mieux ne faire plus de
promotions de son pontificat, que de donner un chapeau à l'archevêque.
Charost trouva ces raisons fort bonnes, mais il ajouta qu'elles ne
concluaient en quoi que ce fût pour son exclusion, et pour que le roi
oubliât les services de son père et les siens, et manquât pour l'unique
fois de sa vie à une promesse solennelle, qu'il lui représentait de sa
propre main, et que lui-même avouait telle.

Le roi prétendit que l'archevêque devait passer seul par les
considérations qu'il venait d'expliquer, mais avec assurance qu'il ne
ferait plus aucun pair sans tenir la parole qu'il avait donnée. Charost
insista et se retira au bout d'une demi-heure, fort mal satisfait du
succès d'une si longue dispute. Il en eut encore trois fort près à près,
toutes à la même heure, toutes autant ou plus longues, toutes en se
promenant. À la dernière il emporta le prix de sa persévérance. Le roi
lui dit qu'il lui aurait fait grand plaisir d'entrer dans ses raisons,
et de se fier à lui pour une autre fois, mais enfin, puisqu'il ne se
voulait point relâcher de sa parole qu'il avait, il la lui voulait
tenir, et qu'il pouvait avertir de sa part le premier président et le
procureur général de prendre ses ordres là-dessus, et qu'il pouvait
aussi prendre ses mesures pour ce qu'il avait à faire de sa part. On
peut juger qu'il n'y perdit pas de temps\,; lui-même m'a conté ce détail
et celui qui va suivre, et m'a dit que sans ses entrées et la facilité
de forcer le roi de l'écouter seul à la fin de son petit coucher tant
qu'il voulait, il n'aurait jamais emporté sa pairie.

L'archevêque de Paris, qui avait compté sur la distinction d'être seul,
voulut au moins être le premier des deux, et prit secrètement toutes ses
mesures. Charost n'y fut pas moins attentif, ni moins bien servi qu'il
l'avait été sur l'érection même. Il retourna au roi toujours au petit
coucher, toujours son billet en main\,; il se plaignit du dessein
avantageux de l'archevêque, et montra au roi que sa parole n'était pas
moins engagée à ce qu'il fût le premier de ceux qu'il ferait, qu'à n'en
faire aucun sans lui. Le principal était accordé, l'accessoire ne tint
pas. Le roi avait bien tacitement consenti à la surprise que
l'archevêque lui voulait faire, mais une fois éventée et portée en
plainte, elle ne tint pas. Le roi promit à Charost d'arrêter
l'archevêque qui, en effet, ne fut enregistré et reçu au parlement que
huit jours après lui. Mais ce fut encore une autre ruse où Charost le
poursuivit jusqu'au bout. L'archevêque, outré de n'avoir pu faire que
Charost ne fût point fait pair en même temps que lui, plus piqué encore
de n'avoir pu réussir à faire passer sa pairie la première, eut la
petitesse d'en vouloir éviter au moins la préséance actuelle, et pour
cela voulut, ce qui ne se fait jamais, être reçu à la dérobée sans
assistance d'aucun pair. Il eut encore l'infortune d'être découvert et
forcé dans ce dernier retranchement. Charost, toujours aux écoutes, fut
encore averti. Il sut le jour que le secret complot se devait
exécuter\,; en vingt-quatre heures il s'assura du plus grand nombre de
pairs qu'il put, qui arrivèrent avec lui à la grand'chambre à sept
heures du matin comme on allait commencer l'affaire de l'archevêque. Ils
l'y trouvèrent lui-même qui attendait à l'ordinaire des pairs qui vont
être reçus, et ils lui firent des compliments dont il se serait bien
passé. Sa surprise et son dépit ne purent se cacher. Ces pairs prirent
aussitôt leurs places, et l'archevêque fut obligé de prendre la sienne
au-dessous du duc de Charost.

Cette aventure fut fort ridicule pour l'archevêque, et Charost eut
complète satisfaction. Il avait été duc à brevet avec son père en 1672,
et il fut pair avec l'archevêque de Paris en 1690. Il était chevalier de
l'ordre de 1688. La teinture que M. de Lauzun lui avait donnée auprès du
roi, et qui n'était pas encore effacée, comme elle la fut depuis, eut
grande part à tout ce qu'il eut à surmonter dans cette occasion pour lui
si capitale. Il maria son fils, cause de cette digression, en 1680, à sa
cousine germaine, fille du prince d'Espinoy et de sa première femme qui
mourut trois ans après, et lui laissa deux fils. Il se remaria huit ans
après à une Lamet, fille unique de Baule, gouverneur de Dourlens, dont
il eut après le gouvernement. Il avait déjà les survivances de son père
de Calais et de Picardie, etc. Il fut lieutenant général des armées du
roi en 1702, et n'a presque pas servi depuis. Son père se démit de son
duché en sa faveur en 1697. Il aimait à aller au parlement, et y
entraînait souvent son cousin le duc d'Estrées. Le cardinal d'Estrées
disait plaisamment qu'il y avait là du Lescalopier. Démis, il continua à
y aller plus d'un an, parce que son fils ne s'y faisait point recevoir.
Le roi à la fin le trouva mauvais, et le duc de Charost fut reçu au
parlement, et son père cessa d'y pouvoir aller, qui, lors de sa
démission avait pris le nom de duc de Béthune. Nous verrons dans la
suite la continuation de cette fortune. M. de Beauvilliers qui ne
jugeait le duc de Charost propre qu'aux choses du dehors, qui en effet
ne lui communiquait jamais rien, et qui l'avait extrêmement approché du
Dauphin sur ce même pied-là de tout temps, le voulut placer de même
auprès de lui, récompenser ainsi la liaison si intime de sa mère,
favoriser tout le petit troupeau, et avoir un homme à eux et à lui dans
cette charge principale, et qui par la singularité de la grâce fit
montre du crédit du Dauphin.

Il avait sur moi d'autres vues qu'il ne tarda pas à m'expliquer, et où
je fus bientôt après confirmé par le Dauphin même. C'était de me faire
gouverneur de Mgr le duc de Bretagne, né le 8 janvier 1707, lorsqu'il
serait en âge de sortir des mains des femmes, place dont il y avait
d'autant plus d'apparence que le roi en laisserait la disposition au
Dauphin, qu'il venait de lui donner celle d'une autre principale, et qui
ne lui était ni si directe ni si intime. Dieu qui souffle sur les
projets des hommes n'a pas permis l'accomplissement de celui-là. On
verra bientôt enterrer ce jeune prince avec toute l'espérance et le
bonheur de la nation, avec toutes les grâces, les charmes et les
plaisirs de la cour. Ainsi Charost, par des événements uniques, eut le
pont d'or que la compagnie des gardes valut à sa famille pour s'en
démettre, rattrapa en sus cette même compagnie, et on verra qu'outre
qu'il la fît passer à fils et à petit-fils, avec les charges qui en
avaient été la récompense et la dignité de duc et pair où elle l'avait
porté, il eut encore la place qui m'avait été destinée, et dont la vue
fit préférer Charost pour la charge de capitaine des gardes du corps.

Les armées du Rhin et des Alpes passèrent de part et d'autre la campagne
à s'observer, et à subsister. Besons, qui soulageait fort d'Harcourt,
vivait aux dépens de l'ennemi au delà du Rhin, tandis qu'Harcourt était
demeuré dans nos lignes de Wessembourg, avec le gros de l'armée, que
Besons rejoignit après avoir consommé tout ce qu'il avait pu de
fourrages. Le reste de la campagne s'y passa dans cette tranquillité
jusqu'à la mi-octobre, qu'Harcourt, ne voyant plus rien à craindre, la
laissa en quartiers de fourrages sous Besons, et s'en alla prendre des
eaux à Bourbonne.

Berwick, toujours sur une assez faible défensive, faute de troupes et de
moyens à pouvoir mieux, ne fut que mollement inquiété\,; M. de Savoie,
qui commandait son armée, aurait pu l'attaquer plus d'une fois avec
beaucoup d'avantage, mais il fut retenu par ses soupçons et plus encore
par son mécontentement. Il prit ombrage du trop grand affaiblissement de
la France, qui faisait trop pencher la balance, et il ne pouvait obtenir
du nouveau gouvernement de Vienne de lui tenir les paroles qu'il avait
tirées du précédent, sur des cessions en Lombardie, ni en tirer les
payements de ce qui lui était dû de subsides.

En Flandre, le prince Eugène et le duc de Marlborough, dans leur union
accoutumée, se contentèrent longtemps de vivre aux dépens des pays du
roi et de resserrer son armée dans des lignes. À ce qui s'y était passé
les années précédentes, c'était pour celle-ci en être quitte à bon
marché, quoique fort honteux. Néanmoins ces avantages des alliés,
quoique très-réels, ne leur parurent pas dignes de leurs campagnes
ordinaires. Marlborough, au faîte de la gloire et de la plus haute
fortune où un capitaine de sa nation pût parvenir, se trouvait menacé
d'un funeste revers qu'il avait un pressant intérêt de parer par quelque
grand coup qui ranimât son parti, et qui pût ébranler celui qui lui
était contraire. Le prince Eugène, personnellement mal avec l'archiduc
successeur de son frère, et fort en brassière avec le nouveau
gouvernement de Vienne, avait le même intérêt que Marlborough. Il leur
était particulier à chacun, et en commun ils avaient celui de la
continuation de la guerre qui maintenait toute leur autorité, leur
puissance et leurs établissements, et qui augmentait journellement leurs
immenses richesses, de Marlborough surtout également avare et avide. De
si pressantes raisons les jetèrent à une entreprise en apparence
insensée, que leur bonheur, leur témérité, et l'incompréhensible
conduite du maréchal de Villars fit réussir. Ce dernier couvrait
Bouchain. Outre le peu de places qui nous restaient de cette frontière
si malmenée, celle-là est un passage fort important, tient la tête des
rivières, ouvre ou ferme un grand pays. Pour en faire le siége il
fallait tourner toute notre armée, et la place par un long détour, et
s'exposer à tout au passage inévitable de l'Escaut. C'est ce que les
deux généraux ennemis osèrent entreprendre au hasard d'une bataille,
demi-passés ou incontinent après. Villars, qui tirait gros de partout où
il pouvait, mais qui payait peu et mal les espions, fut tard averti. Il
voulut les suivre. S'il se fût pressé, il les eût combattus à l'Escaut.
Il montra désir de réparer cette faute qui ne se pouvait dissimuler, et
arriva de fort bonne heure dans une belle plaine, où il voulut camper.
Plusieurs officiers généraux et le maréchal de Montesquiou même lui
rapportèrent des nouvelles des ennemis si proches et en si mauvais
ordre, que personne ne douta qu'elles ne le déterminassent à les aller
attaquer, et à réparer sur-le-champ l'occasion qu'il venait de manquer.
Son froid, ses difficultés, ses lenteurs, surprirent infiniment l'armée,
où les nouvelles des ennemis s'étaient répandues, et avaient inspiré une
ardeur qui éclata par des cris, et qui fit souvenir avec joie de
l'ancien courage français. Les remontrances furent redoublées, pressées,
poussées au delà de la bienséance. Villars fut inflexible\,; pour toutes
raisons il vanta son courage avec audace, on n'en doutait pas, et fit
des rodomontades pour le lendemain. L'armée, en fureur contre lui,
coucha en bataille, et ne s'ébranla qu'assez avant dans la matinée
suivante par les mêmes lenteurs. Elle eut beau marcher, les ennemis
avaient pris les devants, qui furent redevables de leur salut à la rare
retenue du maréchal de Villars, dont le motif n'a pu être pénétré,
puisque en l'état ou les ennemis se trouvèrent, ils ne pouvaient, de
l'aveu des deux armées, éviter d'être battus.

Villars avait annoncé la bataille par un courrier à la cour, qui fut
quatre jours dans la plus vive attente. Enfin un courrier arriva à
Fontainebleau, que Voysin amena au roi, qui venait de donner le
bonsoir\,; le Dauphin, qui se déshabillait, se rhabilla, et tout courut
en un moment chez le roi, pour apprendre le succès de la bataille, et
savoir les morts et les blessés\,; l'antichambre était pleine, qui
croyait que Voysin en lisait le détail au roi, qui attendait qu'il
sortît avec la dernière impatience, et qui sut enfin de lui qu'il n'y
avait point eu d'action. Pour revenir à l'armée, Villars voyant les
ennemis échappés, il se mit à éclater en reproches. Les officiers
généraux, surpris tout ce qu'on peu l'être, se regardèrent les uns les
autres\,; Albergotti et quelque autre avec lui, prirent la parole pour
le faire souvenir qu'il n'avait pas tenu à leurs représentations les
plus vives qu'il n'eût vivement poursuivi sa marche. Montesquiou, qui se
crut plus offensé et plus à l'abri que les autres par son bâton de
maréchal de France, lui répondit plus vertement qu'eux\,; un prompt
démenti net et sec, sans détour ni enveloppe, fut le salaire de cette
vérité\,; Montesquiou frémit, tourna le dos de sa main sur la garde de
son épée et sortit. Villars, fier de ce triomphe, l'unique de sa
campagne, après en avoir coup sur coup manqué deux si beaux, si sûrs, si
nécessaires, se mit à braver de plus belle, d'autant mieux qu'après cet
étrange essai, il ne craignait plus d'être contredit en face\,; mais la
vérité était contre lui, elle demeurait entière, elle était connue de
toute l'armée, et quoique Montesquiou n'en fût pas aimé, il fut visité
de toute l'armée en foule. Villars enfin, un peu revenu à soi, fut fort
embarrassé\,; il fit des pas pour se raccommoder avec Montesquiou. Les
armées, non plus que les cours, ne manquent pas de gens qui aiment à se
faire de fête et à s'empresser\,; il s'en trouva qui volontiers
s'entremirent entre les deux maréchaux. Le second, bien fâché d'avoir à
repousser contre son supérieur une injure si atroce et si publique, ne
fut pas fâché d'en sortir par l'apparente porte de l'amour du bien
public dans des conjonctures fâcheuses, soutenu par une réputation plus
que faite sur la valeur, et par la consolation d'avoir toute l'armée
pour témoin de la vérité qu'il avait soutenue. Pour couper court à une
si étrange affaire, il ne fut pas question d'éclaircissement qui n'eût
pas été possible, ni d'excuse qui n'eût fait qu'aggraver\,; on crut
qu'un air d'oubli ou de chose non avenue était l'unique voie à prendre.
Dès le lendemain Montesquiou parut un moment chez Villars, et peu à peu
ils se revirent à l'ordinaire. Pour achever tout de suite ce qui regarde
cette aventure, elle revint à Paris et à la cour par toutes les lettres
de l'armée. Le roi aimait Montesquiou qu'il voyait depuis longtemps
quelquefois par les derrières, et qui était ami de tous les valets
principaux, mais son démenti le peinait bien moins que la cause et que
les suites qu'il en voyait par le siége de Bouchain, que les ennemis
avaient formé\,; il ordonna donc à Villars de lui envoyer un officier
général bien instruit pour lui rendre compte des mouvements qui avaient
précédé ce siége. Villars, en bon courtisan, choisit Contade, major du
régiment des gardes, fort connu du roi et fort dans le grand et le
meilleur monde, qui était major général de son armée. Contade savait
aller et parler, et se tourner à propos, et fort bien à qui il avait
affaire\,; il s'était fort attaché à Villars, il était fort ami de la
maréchale et plus qu'ami de longtemps de M\textsuperscript{me} de
Maisons, sœur de la maréchale. Contade arriva le 20 août à
Fontainebleau\,; il fut le lendemain matin vendredi conduit après la
messe du roi chez M\textsuperscript{me} de Maintenon, où ils demeurèrent
deux heures avec lui. Ils y retournèrent encore l'après-dînée où Contade
prit congé\,; il fut après assez longtemps seul avec le Dauphin dans son
cabinet, et repartit le 22 pour retourner à l'armée. On peut juger du
compte que rendit Contade, disposé comme il l'était, choisi et instruit
par Villars, en présence de M\textsuperscript{me} de Maintenon, qui lui
fut toujours si favorable, et d'un ministre moins ministre du roi et
d'État que ministre de cette dame.

Marlborough, qui n'avait jamais tenté un si dangereux hasard, se
félicita publiquement d'y être échappé, et ne songea plus qu'à former le
siége de Bouchain, qui était l'objet qui l'avait engagé à s'y exposer,
ce qu'il exécuta incontinent après. Villars espéra d'abord de sauver la
place en s'y entretenant une communication libre par les marécages. La
garnison y était bonne, forte et bien munie et approvisionnée, et
Ravignan y commandait. Il vint concerter avec les maréchaux\,; sa
personne fit un embarras. Il avait été fait prisonnier avec la garnison
de Tournai et renvoyé sur parole. La difficulté des échanges l'empêcha
de servir. Il exposa le malheur de cette situation au duc de
Marlborough, qui eut la générosité, par sa réponse, de lui permettre de
servir, en l'avertissant toutefois qu'il ne lui répondait en cela que
des Anglais, et nullement des Impériaux ni des Hollandais. Cette
restriction n'arrêta point Ravignan. Il avait beaucoup d'ambition, il ne
pouvait la satisfaire que par la guerre. Il l'aimait et il était fort
bon officier, et de même nom que le président de Mesmes, qui prenait
grande part à lui. Il était fort connu du roi, dont il avait été page,
et qui avait ri quelquefois de ses tours de page, et de ce que la
passion de la chasse lui avait fait faire. Il ne balança donc pas à
servir d'inspecteur qu'il était et partout où il put, mais sans être mis
comme officier général sur les états des armées, parce que la permission
seule des Anglais ne suffisait pas pour cela. Il fallait quelqu'un
d'intelligent pour commander l'été dans Bouchain, et on l'y mit parce
qu'on ne crut pas que la place dût craindre d'être assiégée. Le cas
arrivé, il fut question de savoir si Ravignan y demeurerait. C'était
contrevenir très-directement à sa parole à l'égard des Impériaux et des
Hollandais. Il est même si différent de servir en ligne parmi la foule,
ou de se charger de la défense d'une place attaquée, que Marlborough
avait droit de trouver que c'était abuser de la générosité de sa
permission. Les lois de la guerre n'allaient à rien moins qu'à excepter
Ravignan de toute capitulation si la place était prise, et de le faire
pendre haut et court, ce que Marlborough, quelque bonne volonté qu'il
pût lui conserver, n'était pas en état d'empêcher. Cette matière
amplement délibérée au camp, tandis que Ravignan s'y trouvait, il fut
résolu que son honneur ni la bonne foi de la guerre ne devaient pas être
exposés, et on songeait déjà à envoyer dès le soir même un autre
commandant dans Bouchain\,; mais Ravignan mit moins son honneur à garder
sa parole, qu'à sortir d'une place, où il commandait, à la vue des
ennemis qui allaient former le siége. Il pressa Villars de l'y laisser
retourner, et il fit des instances si fortes que Villars, outré d'un
siége formé par ses fautes, et dont les suites étaient si terribles pour
les campagnes suivantes, ne fut peut-être pas fâché d'en laisser la
défense à un officier aussi entendu, et dont l'opiniâtreté serait
assistée de la perspective d'une potence.

Ainsi, contre l'avis universel, Villars prit sur soi d'y renvoyer
Ravignan, qui ne se le fit pas dire deux fois et y retourna aussitôt.

La communication avec la place, entreprise avec de grands travaux, ne
put se soutenir. Albergotti qui la gardait en fut chassé, et l'événement
fut regardé comme décisif pour le siége. Il produisit des accusations
réciproques entre Albergotti et Villars, qui furent fort poussées. Tout
à la fin du siége, l'adroit Italien n'oublia aucune souplesse pour se
raccommoder avec son général. À l'extérieur il ne parut plus rien\,;
personne n'en fut la dupe, et à leur retour, ils se portèrent l'un à
l'autre tous les coups qu'ils purent, mais avec une égale impuissance.
Villars fit toutes les démonstrations de vouloir combattre et secourir
la place. On est encore à savoir s'il en eut effectivement le dessein.
La fanfaronnade fut courte, il s'éloigna pour subsister. Cependant,
après une défense de moins d'un mois, Bouchain battit la chamade le 13
septembre, et la garnison, prisonnière de guerre, fut conduite à
Tournai. Les généraux ennemis ne voulurent pas s'apercevoir de Ravignan
avec toute la générosité possible, et demeurèrent un mois à réparer la
place. Il était lors la mi-octobre.

Marlborough était pressé de passer la mer pour soutenir son parti fort
abandonné, et une fortune chancelante. Le prince Eugène, si
inséparablement uni à lui par les mêmes intérêts, n'était pas lui-même
sans inquiétudes, comme on l'a vu. Il avait à soutenir à la Haye la
bonne volonté d'Heinsius et de leur cabale, à y tout concerter en
l'absence de Marlborough, et la perspective d'un voyage en Allemagne
vers un nouveau maître et une cour nouvelle avec qui il était mal. De si
fortes raisons, et dans une saison si avancée, leur persuadèrent de
finir la campagne. Notre armée, harassée à l'excès et sans utilité,
profita aussitôt de l'exemple\,; chacun de part et d'autre tourna aux
quartiers d'hiver. Villars fut assez bien reçu, parce qu'on n'avait
personne à lui substituer pour la campagne suivante\,; Montesquiou passa
l'hiver sur la frontière comme les précédents, et, par la raison qui
vient d'être expliquée, fut assez peu content d'une course qu'il vint
faire à la cour.

\hypertarget{note-i.-des-anciennes-pairies-pairs-eccluxe9siastiques-et-lauxefques.}{%
\chapter{NOTE I. DES ANCIENNES PAIRIES\,; PAIRS ECCLÉSIASTIQUES ET
LAÏQUES.}\label{note-i.-des-anciennes-pairies-pairs-eccluxe9siastiques-et-lauxefques.}}

À l'époque féodale, et spécialement aux xii\^{}e et xiii\^{}e siècles,
les douze pairs de France étaient en grande renommée. Le poëte Robert
Wace, qui vivait au xii\^{}e siècle, parle, dans son \emph{Roman du
Brut}, de Douze comtes d'aulte puissance, Que l'on clamoit les pairs de
France.

Suivant l'usage de cette époque, les poëtes transportaient l'institution
des douze pairs dans tous les pays, et à la cour de tous les princes
dont ils chantaient les exploits. Ainsi, dans le \emph{Roman
d'Alexandre}, le roi de Macédoine, avant de commencer la guerre contre
les Perses, mande toute sa noblesse et ses chevaliers, puis choisit
douze pairs, dont l'un doit porter l'étendard royal. L'Écosse et
l'Angleterre ont aussi leurs douze pairs dans le \emph{Roman de
Perceforêt}. Ces légendes poétiques constatent la haute renommée dont
jouissaient les douze pairs de France. Mais quels étaient, en réalité,
les personnages qui formaient cette cour féodale des douze pairs\,? Il y
avait six archevêques ou évêques, trois ducs et trois comtes.

Les pairs ecclésiastiques étaient\,: 1° l'archevêque-duc de Reims,
auquel appartenait le droit de sacrer les rois de France\,; en son
absence, c'était l'évêque de Soissons qui remplissait cette fonction\,;
2° l'évêque-duc de Laon, qui portait la sainte ampoule au sacre des
rois\,; 3° l'évêque-duc de Langres, auquel était confiée l'épée royale
dans la même cérémonie\,; 4° l'évêque-comte de Beauvais\,; il présentait
au roi le manteau royal\,; il allait, avec l'évêque-duc de Laon,
chercher le roi au palais de l'archevêque de Reims, et l'amenait à
l'église\,; ces deux prélats se tenaient aux côtés du roi pendant qu'il
recevait les onctions, l'aidaient à se lever de son fauteuil, et
demandaient à l'assemblée, par un souvenir des anciennes élections des
rois barbares, si elle était disposée à reconnaître le prince pour son
souverain\,; 5° l'évêque-comte de Châlons-sur-Marne\,; il portait au
sacre l'anneau royal\,; 6° l'évêque-comte de Noyon\,; la ceinture et le
baudrier royal lui étaient confiés.

À la tête des pairs laïques, on plaçait primitivement le duc de
Normandie. Matthieu Pâris, parlant des douze pairs, dit positivement\,:
«\,Le duc de Normandie est le premier entre les pairs laïques, et le
plus illustre\footnote{«\,Dux Normanniæ primus inter laicos et
  nobilissimus.\,»}.\,» 2° Le duc de Bourgogne. Lorsque Jean le Bon
donna le duché de Bourgogne à son fils Philippe le Hardi, en 1363, il
lui accorda le premier rang entre les pairs de France\,; et depuis cette
époque, les ducs de Bourgogne en restèrent en possession. Au sacre de
Charles VI, en 1380, Philippe le Hardi, duc de Bourgogne, précéda son
frère aîné, Louis d'Anjou, en sa qualité de \emph{doyen des pairs de
France}. Des lettres patentes de Louis XI, en date du 14 octobre 1468,
confirmèrent la prérogative des successeurs de Philippe le Hardi, et
déclarèrent que le duché de Bourgogne était la première pairie. Au sacre
des rois, le duc de Bourgogne portait la couronne. 3° Le duc de Guyenne
ou d'Aquitaine. C'était à lui qu'était remise, dans cette cérémonie, la
première bannière carrée ou étendard royal. 4° Le comte de Flandre\,; il
portait au sacre une des épées du roi. 5° Le comte de Champagne. On lui
donnait le titre de palatin ou comte du palais, parce qu'il exerçait
primitivement la juridiction sur tous les officiers du palais. Il était
chargé de la seconde bannière royale ou étendard de guerre. 6° Le comte
de Toulouse. Il avait aspiré au premier rang entre les pairs laïques,
comme comte de Narbonne\,; mais sa prétention ne fut pas admise. Au
sacre, il portait les éperons du roi.

\hypertarget{note-ii.-des-secruxe9taires-duxe9tat-de-leur-origine-et-de-leurs-duxe9partements-dans-lancienne-monarchie.}{%
\chapter{NOTE II. DES SECRÉTAIRES D'ÉTAT\,; DE LEUR ORIGINE ET DE LEURS
DÉPARTEMENTS DANS L'ANCIENNE
MONARCHIE.}\label{note-ii.-des-secruxe9taires-duxe9tat-de-leur-origine-et-de-leurs-duxe9partements-dans-lancienne-monarchie.}}

Saint-Simon revient souvent sur les ministres secrétaires d'État, sur
leur puissance récente et faible dans l'origine, sur les accroissements
qu'elle prit successivement, et sur les départements attribués à chacun
d'eux. Il ne sera pas inutile de résumer rapidement pour le lecteur
moderne les renseignements propres à éclaircir ces passages de
Saint-Simon.

\emph{La ténuité de l'origine} des secrétaires d'État, comme dit
Saint-Simon (p. 365), ne saurait être contestée. On les appelait
primitivement \emph{clercs du secret}, parce que, depuis la fin du
xiii\^{}e siècle, ils étaient chargés de rédiger les délibérations du
conseil secret du roi. Ce fut seulement au xvi\^{}e siècle qu'ils
sortirent de cette humble condition. Florimond Robertet, qui était
secrétaire d'État sous le règne de Louis XII, fut le premier qui
contre-signa les ordonnances des rois de France. En 1547, Henri II, qui
venait de monter sur le trône, fixa à quatre le nombre des secrétaires
d'État, et augmenta leurs honoraires. La division de leurs attributions
était, à cette époque, purement géographique\,: ainsi Bochetel avait
dans son département la Normandie, la Picardie, l'Angleterre et
l'Écosse\,; Clausse, la Provence, le Languedoc, la Guyenne, la Bretagne,
l'Espagne et le Portugal\,; de L'Aubespine, la Champagne, la Bourgogne,
la Bresse, la Savoie, la Suisse et l'Allemagne\,; du Thier, le Dauphiné,
le Piémont, Rome, Venise et l'Orient. Une pareille division supposait à
chaque ministre une capacité universelle, ou le réduisait au rôle d'un
simple secrétaire de correspondance. Telle était, en effet, la position
des ministres secrétaires d'État, même au xvi\^{}e siècle. Henri III
voulut vainement déterminer leurs fonctions avec plus de netteté, par
des ordonnances rendues à Blois, aux mois de mai et de septembre 1588\,;
les troubles qui suivirent paralysèrent toutes les réformes de ce
prince.

Ce fut seulement au xvii\^{}e siècle que les ministres commencèrent à se
partager les départements de la maison du roi, de la guerre, de la
marine, des affaires étrangères. Déjà, sous Henri IV, nous voyons un des
secrétaires d'État chargé du département de la maison du roi et des
affaires ecclésiastiques. En 1619, un des secrétaires d'État eut la
correspondance avec tous les généraux, et devint un véritable ministre
de la guerre. Le Tellier et son fils Louvois donnèrent à ce département
la plus haute importance. En 1626, toutes les affaires extérieures, qui
jusqu'alors étaient divisées entre les quatre secrétaires d'État, furent
réunies entre les mains d'un seul\,; le ministère des affaires
étrangères fut créé. Richelieu et Mazarin, qui dirigeaient toute la
politique extérieure, n'y mirent que des commis\,; mais après la mort de
Mazarin, de Lyonne donna à ce ministère une importance qui ne fit que
s'accroître sous ses successeurs. La marine ne forma un département
particulier qu'à l'époque où Colbert en fut chargé. Elle resta, jusqu'en
1669, réunie au département des affaires étrangères. Quant aux finances
et à la justice, ils ne dépendaient pas des secrétaires d'État. Les
surintendants des finances, et, après leur suppression en 1661, les
contrôleurs généraux étaient chargés de l'administration du trésor
public. La justice était placée sous la direction du chancelier. Le
commerce, les travaux publics, les postes, les colonies, l'instruction
publique, ne formaient pas des départements ministériels, et ne
dépendaient pas spécialement d'un des secrétaires d'État. Le roi en
disposait comme bon lui semblait. Ainsi les travaux publics, ou
direction générale des bâtiments, qui avaient appartenu à Colbert, à la
fois contrôleur général des finances et secrétaire d'État de la marine,
furent donnés, après sa mort, au ministre de la guerre, Louvois.

Il n'y avait point, dans l'ancienne monarchie, de ministre de
l'intérieur. Les généralités, qui formaient, sous Louis XIV, les
principales circonscriptions administratives de la France, étaient
partagées entre les quatre secrétaires d'État, et on retrouvait dans
cette organisation une partie des divisions géographiques que nous avons
signalées plus haut. Ainsi, du secrétaire d'État des affaires étrangères
dépendaient la haute et basse Guyenne, les intendances de Bayonne, Auch
et Bordeaux, la Normandie, la Champagne, la principauté de Dombes, le
Berry, et la partie de la Brie qui était rattachée à la généralité de
Châlons-sur-Marne. Le ministre secrétaire d'État de la maison du roi
avait dans son département la ville et généralité de Paris, le
Languedoc, la Provence, la Bourgogne, la Bresse, la Bretagne, le comté
de Foix, la Navarre, l'Auvergne, le Nivernais, le Bourbonnais, le
Limousin, l'Angoumois, la Marche, le Poitou, la Saintonge, l'Aunis, la
Touraine, la Picardie, le Boulonaîs, etc. Telles étaient les provinces
de La Vrillière, dont Saint-Simon parle dans ce volume (p.~282). Les
ports de mer et les colonies dépendaient du ministre de la marine. Le
secrétaire d'État de la guerre avait l'Alsace, la Franche-Comté, la
Lorraine, le Dauphiné, l'Artois, la Flandre, le Roussillon, etc.

Les divisions géographiques que je viens de rappeler ont subi de
fréquentes variations\,; mais cette organisation administrative a
existé, sauf quelques modifications, jusqu'à l'époque de la révolution
française. Pour remédier aux inconvénients d'une administration sans
unité, on tenait tous les quinze jours, en présence du roi, le conseil
des dépêches, où l'on réglait tous les détails de l'administration
intérieure du royaume. Les secrétaires d'État expédiaient dans les
provinces qui leur étaient attribuées les règlements et ordonnances
arrêtés dans ce conseil.

\hypertarget{note-iii.-cardinal-de-bouillon-arruxeat-du-parlement-2-janvier-1711.}{%
\chapter{NOTE III. CARDINAL DE BOUILLON\,; ARRÊT DU PARLEMENT (2 janvier
1711).}\label{note-iii.-cardinal-de-bouillon-arruxeat-du-parlement-2-janvier-1711.}}

Le cardinal de Bouillon a joué un grand rôle à la fin du xvii\^{}e
siècle, et pendant plusieurs années on le considéra comme un des chefs
les plus illustres de l'Église de France. Son orgueil finit par lui
attirer une disgrâce dont il ne se releva jamais. Saint-Simon, qui le
traite avec beaucoup de sévérité, a insisté sur les actes déplorables
auxquels il se laissa entraîner par la vanité et l'ambition\footnote{Voy.,
  entre autres, t. II, p.~428 et suiv.\,; V, p.~296 et suiv.\,; VI,
  p.~277 et suiv. de cette édition.}. Les documents officiels confirment
les assertions de l'historien. Je citerai, entre autres, un arrêt du
parlement en date du 2 janvier 1711.

ARRÊT DU PARLEMENT DE PARIS DU 2 JANVIER 1711\footnote{Extrait des
  registres du parlement, Bibl. imp. du Louvre, ms., B, 1253-1.}.

«\,Vu par la cour la requête à elle présentée par le procureur général
du roi contenant que la cour ayant ordonné, par un arrêt du 5 août
dernier, que le lieutenant général en la sénéchaussée de Lyon se
transporteroit dans l'abbaye et dans l'église de Cluny en présence du
substitut du procureur général du roi au même siége, tant pour y dresser
procès-verbal et y faire faire un plan du mausolée que le cardinal de
Bouillon a commandé d'y faire élever dans cette église et des ouvrages
qui en dépendent, que pour tirer des extraits des actes de ce monument
et la sépulture de la maison de La Tour\,; cet arrêt a été pleinement
exécuté dans toutes ses parties, soit par la description exacte que le
lieutenant général de Lyon a faite de ces ouvrages, soit par les dessins
qu'il en a fait tracer, soit par la copie qu'il en a insérée dans son
procès-verbal de tous les actes contenus dans les registres de Cluny qui
pouvoient avoir rapport à la sépulture de la maison de La Tour dans
l'église de cette abbaye\,; que le procureur général n'entrera point
dans un long détail des conséquences que l'on peut tirer de ce
procès-verbal et de tout ce qui l'accompagne\,; il aime mieux s'en
rapporter à l'impression que ces pièces feront sur l'esprit de la cour,
quand elle les examinera, que de prévenir cette impression par des
paroles toujours inutiles, lorsque les choses parlent d'elles-mêmes\,;
qu'il se contentera donc d'observer qu'entre les ornements étrangers
qu'il paroît par le procès-verbal du lieutenant général de Lyon, que le
cardinal de Bouillon a fait mettre sans aucun fondement à plusieurs
endroits de l'église de Cluny, comme le manteau fourré d'hermine et un
bonnet à peu près semblable à celui des princes de l'empire d'Allemagne,
on trouve, soit dans le mausolée, soit dans les actes qui regardent la
sépulture de la maison de La Tour, une vérité de dessins, dont toutes
les parties tendent également à consacrer et immortaliser, par la
religion d'un tombeau toujours durable, les prétentions trop ambitieuses
de son auteur sur l'origine et sur la grandeur de sa maison\,; c'étoit
là ce que les statues, les inscriptions, les ornements et toute la
structure de ce mausolée devoient apprendre à la postérité, et celui qui
en a conçu l'idée s'étant flatté sans doute que l'on s'accoutumeroit
insensiblement aux titres magnifiques que ce monument suppose et dont
quelque jour il deviendroit une preuve, qui, après avoir paru longtemps
aux yeux du public sans être contestée, pourroit enfin être regardée
comme incontestable\,; que le procureur général du roi, qui doit mettre
au nombre de ses principaux devoirs l'honorable nécessité que son
ministère lui impose de réprimer toute grandeur qui s'élève au-dessus de
ses bornes légitimes, est d'autant plus obligé de le faire dans cette
occasion qu'il s'agit ici, non d'un honneur vain et stérile qui ne fait
point d'autre mal que de flatter l'orgueil de celui qui l'usurpe, mais
d'une ambition aussi dangereuse que téméraire qui a jeté dans le cœur du
cardinal de Bouillon ces principes d'indépendance et ces semences de
révolte qu'il a fait enfin éclater par sa sortie du royaume et par cette
lettre criminelle, par lesquelles il a mérité que la cour lui fît son
procès comme à un coupable de lèse-majesté\,; que, dans la nécessité où
le procureur général du roi se trouve de s'élever contre l'ouvrage d'une
vanité, si vaste dans ses vues et si pernicieuse dans ses effets, il
espère au moins qu'il ne sera jamais obligé de l'imputer qu'à celui qui
jusqu'à présent en paroît l'unique auteur, et qu'il présume assez de la
sagesse et de la fidélité du reste de la maison de La Tour pour croire
qu'entre tous ceux de cette maison qui sont dans le royaume, il ne s'en
trouvera aucun qui veuille se rendre coupable de la faute d'autrui en
entreprenant de la soutenir, et qui ne sente que leur véritable honneur
consiste à savoir se renfermer glorieusement dans la solide et réelle
grandeur de leur maison pour la transmettre d'autant plus pure à leurs
descendants qu'ils l'auront déjà dégagée de tout ce qu'une fiction
étrangère a voulu y ajouter de faux, et de chimérique\,; mais que la
justice que le procureur général du roi croit lui rendre en cela ne le
dispense pas de prendre toutes les précautions nécessaires pour empêcher
que, dans des siècles éloignés et peut-être peu instruits de ce qui se
sera passé dans celui-ci, on n'abuse de la sépulture de la maison de La
Tour dans l'église de Cluny et des titres gravés sur les cercueils de
ceux de cette maison qui y sont enterrés, pour faire revivre des
prétentions auxquelles cette procédure et ces titres paroîtroient donner
un nouveau jour à une espèce de prétention que la faveur des
conjonctures fait quelquefois passer en cette matière pour la vérité\,;
que c'est dans toutes ces vues que le procureur général du roi a cru
devoir dresser les conclusions que son ministère l'oblige de prendre sur
le procès-verbal dudit lieutenant général de Lyon\,; et comme cet
officier n'y a point joint de copie de la table généalogique et des cinq
pièces déposées dans les archives de l'abbaye de Cluny pour servir à la
généalogie de la maison de La Tour, le procureur général du roi, auquel
il est important que ces pièces soient communiquées, ne peut que
demander à la cour qu'elles soient apportées au greffe pour faire
ensuite à cet égard les réquisitions qu'il jugera nécessaires. À ces
causes requéroit le procureur général du roi qu'il plût à la cour
ordonner que lesdits monuments, mausolées, statues, ouvrages et
ornements en dépendant, mentionnés dans le procès-verbal dudit
lieutenant général, ensemble les dessins et modèles dudit mausolée qui
sont dans l'église et abbaye de Cluny, et pareillement les titres gravés
sur le cercueil du sieur Louis de La Tour, enterré dans ladite église,
en seront incessamment ôtés, détruits et effacés, comme aussi que la
délibération du chapitre général de Cluny de l'an 1685, transcrite au
commencement de la cinquième page du registre dudit chapitre\,; la
délibération des religieux de ladite abbaye, du 24 octobre 1685,
transcrite au treizième feuillet des registres des actes capitulaires de
la communauté de Cluny commençant le 2 janvier 1682\,; l'acte du 25
octobre 1692 concernant la réception des corps des feu sieur et feue
dame duchesse de Bouillon et du feu sieur Louis de La Tour, leur
petit-fils, dans l'église de Cluny, et pareillement la délibération du
14 octobre 1693, transcrite au vingt-septième feuillet du même registre
des actes capitulaires de ladite communauté, touchant la réception du
cœur du feu sieur maréchal de Turenne, ensemble tous autres actes
semblables, si aucuns y a, concernant ladite sépulture seront rayés et
biffés, à la marge desquels sera fait mention de l'arrêt qui
interviendra sur ladite requête, lequel sera en outre transcrit en
entier dans le registre des délibérations capitulaires de l'abbaye de
Cluny, enjoint au lieutenant général de Lyon de tenir la main à
l'exécution dudit arrêt, à l'effet de quoi il se transportera dans
ladite abbaye en présence du substitut du procureur général du roi en
ladite sénéchaussée de Lyon, et, avant de faire droit sur ce qui regarde
la table et les pièces servant à la généalogie de la maison de La Tour
trouvées dans les archives de ladite abbaye, ordonne que ladite table
généalogique de ladite maison, et la liasse composée de cinq pièces
mentionnées dans le procès-verbal dudit lieutenant général, seront
apportées au greffe de la cour, enjoint aux religieux dépositaires
desdites tables et pièces de les y envoyer après le premier commandement
qui leur en sera fait\,; à quoi faire ils seront contraints par les
voies en tels cas requises et accoutumées\,; quoi faisant déchargés\,;
pour ce fait, rapporté et communiqué au procureur général, par lui pris
telles conclusions qu'il appartiendra, vu aussi le procès-verbal de
transport du lieutenant général en la sénéchaussée de Lyon, en présence
du substitut du procureur général du roi en ladite sénéchaussée dans
l'église et abbaye de Cluny, du 13 août 1710 et jours suivants, fait en
exécution de l'arrêt du 5 du même mois, ensemble les actes insérés dans
le procès-verbal, et les plans et dessins y attachés, faits en exécution
dudit arrêt par le peintre nommé d'office à cet effet par ledit
lieutenant général suivant ledit arrêt attaché à ladite requête du
procureur général du roi\,; ouï le rapport de maître Jean Le Nain,
conseiller\,; tout considéré,

«\,La Cour, ayant égard à la requête dudit procureur général du roi,
ordonne que lesdits monument ou mausolée, statues, ouvrages et monuments
en dépendant, mentionnés dans le procès-verbal dudit lieutenant général
de Lyon, ensemble les dessins et modèles dudit mausolée qui sont dans
l'église et abbaye de Cluny, et pareillement les titres gravés sur le
cercueil de Louis de La Tour enterré dans ladite église, en seront
incessamment ôtés, détruits et effacés, comme aussi que la délibération
du chapitre général de Cluny de l'année 1685, transcrite au commencement
de la cinquième page du registre dudit chapitre\,; la délibération des
religieux de ladite abbaye du 24 octobre 1685, transcrite au treizième
feuillet du registre des actes capitulaires de la communauté de Cluny
commençant le 2 janvier 1682\,; l'acte du 25 octobre 1692 concernant la
réception des corps du feu duc et de la feue duchesse de Bouillon et du
feu Louis de La Tour, leur petit-fils, dans l'église de Cluny, et
pareillement la délibération du 14 octobre 1693, transcrite au
vingt-septième feuillet du même registre des actes capitulaires de
ladite communauté touchant la réception du cœur du feu maréchal de
Turenne, ensemble tous autres actes semblables, si aucuns y a,
concernant ladite sépulture, seront rayés et biffés, à la marge desquels
sera fait mention du présent arrêt, lequel sera en outre transcrit en
entier dans le registre desdites délibérations capitulaires de l'abbaye
de Cluny, enjoint au lieutenant général de Lyon de tenir la main à
l'exécution dudit arrêt, à l'effet de quoi il se transportera dans
ladite abbaye, en présence du substitut du procureur général du roi en
ladite sénéchaussée de Lyon, et avant faire droit sur ce qui regarde la
table et les pièces servant à la généalogie de la maison de La Tour
trouvées dans les archives de ladite abbaye, ordonne que ladite table
généalogique de ladite maison et la liasse composée de cinq pièces
mentionnées dans ledit procès-verbal dudit lieutenant général de Lyon,
seront apportées au greffe de la cour, enjoint aux religieux
dépositaires desdites tables et pièces de les y envoyer après le premier
commandement qui leur en sera fait, à quoi faire ils seront contraints
par les voies en tels cas requises et accoutumées, pour ce fait,
rapporté et communiqué au procureur général du roi, être par lui pris
telles conclusions qu'il appartiendra, et vu par la cour être ordonné ce
que de raison.\,»

\end{document}
